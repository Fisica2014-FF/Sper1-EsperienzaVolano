Qui vengono riportati i grafici di ciascuna sessione di presa dati, con la relativa interpolazione lineare per ricavare $\alpha$ e $\beta$.
Per primi i grafici relativi alle accelerazioni. L'errore è abbastanza buono su questi, come si vede anche dagli stessi.

\begin{grafico}
    \centering
\begin{tikzpicture}[gnuplot]
%% generated with GNUPLOT 4.6p0 (Lua 5.1; terminal rev. 99, script rev. 100)
%% Tue 15 Apr 2014 09:47:06 PM CEST
\path (0.000,0.000) rectangle (12.500,8.750);
\gpcolor{color=gp lt color border}
\gpsetlinetype{gp lt border}
\gpsetlinewidth{1.00}
\draw[gp path] (1.688,0.985)--(1.868,0.985);
\draw[gp path] (11.947,0.985)--(11.767,0.985);
\node[gp node right] at (1.504,0.985) { 0.15};
\draw[gp path] (1.688,1.669)--(1.868,1.669);
\draw[gp path] (11.947,1.669)--(11.767,1.669);
\node[gp node right] at (1.504,1.669) { 0.2};
\draw[gp path] (1.688,2.353)--(1.868,2.353);
\draw[gp path] (11.947,2.353)--(11.767,2.353);
\node[gp node right] at (1.504,2.353) { 0.25};
\draw[gp path] (1.688,3.037)--(1.868,3.037);
\draw[gp path] (11.947,3.037)--(11.767,3.037);
\node[gp node right] at (1.504,3.037) { 0.3};
\draw[gp path] (1.688,3.721)--(1.868,3.721);
\draw[gp path] (11.947,3.721)--(11.767,3.721);
\node[gp node right] at (1.504,3.721) { 0.35};
\draw[gp path] (1.688,4.405)--(1.868,4.405);
\draw[gp path] (11.947,4.405)--(11.767,4.405);
\node[gp node right] at (1.504,4.405) { 0.4};
\draw[gp path] (1.688,5.089)--(1.868,5.089);
\draw[gp path] (11.947,5.089)--(11.767,5.089);
\node[gp node right] at (1.504,5.089) { 0.45};
\draw[gp path] (1.688,5.773)--(1.868,5.773);
\draw[gp path] (11.947,5.773)--(11.767,5.773);
\node[gp node right] at (1.504,5.773) { 0.5};
\draw[gp path] (1.688,6.457)--(1.868,6.457);
\draw[gp path] (11.947,6.457)--(11.767,6.457);
\node[gp node right] at (1.504,6.457) { 0.55};
\draw[gp path] (1.688,7.141)--(1.868,7.141);
\draw[gp path] (11.947,7.141)--(11.767,7.141);
\node[gp node right] at (1.504,7.141) { 0.6};
\draw[gp path] (1.688,7.825)--(1.868,7.825);
\draw[gp path] (11.947,7.825)--(11.767,7.825);
\node[gp node right] at (1.504,7.825) { 0.65};
\draw[gp path] (1.688,0.985)--(1.688,1.165);
\draw[gp path] (1.688,7.825)--(1.688,7.645);
\node[gp node center] at (1.688,0.677) { 5};
\draw[gp path] (3.740,0.985)--(3.740,1.165);
\draw[gp path] (3.740,7.825)--(3.740,7.645);
\node[gp node center] at (3.740,0.677) { 10};
\draw[gp path] (5.792,0.985)--(5.792,1.165);
\draw[gp path] (5.792,7.825)--(5.792,7.645);
\node[gp node center] at (5.792,0.677) { 15};
\draw[gp path] (7.843,0.985)--(7.843,1.165);
\draw[gp path] (7.843,7.825)--(7.843,7.645);
\node[gp node center] at (7.843,0.677) { 20};
\draw[gp path] (9.895,0.985)--(9.895,1.165);
\draw[gp path] (9.895,7.825)--(9.895,7.645);
\node[gp node center] at (9.895,0.677) { 25};
\draw[gp path] (11.947,0.985)--(11.947,1.165);
\draw[gp path] (11.947,7.825)--(11.947,7.645);
\node[gp node center] at (11.947,0.677) { 30};
\draw[gp path] (1.688,7.825)--(1.688,0.985)--(11.947,0.985)--(11.947,7.825)--cycle;
\node[gp node center,rotate=-270] at (0.246,4.405) {Velocità angolare [rad/s]};
\node[gp node center] at (6.817,0.215) {Tempo [s]};
\node[gp node center] at (6.817,8.287) {Velocità angolari in accelerazione [rad/s]};
\node[gp node left] at (7.167,1.627) {Dati};
\gpcolor{color=gp lt color 0}
\gpsetlinetype{gp lt plot 0}
\draw[gp path] (10.663,1.627)--(11.579,1.627);
\draw[gp path] (10.663,1.717)--(10.663,1.537);
\draw[gp path] (11.579,1.717)--(11.579,1.537);
\draw[gp path] (2.858,1.635)--(2.858,1.705);
\draw[gp path] (2.768,1.635)--(2.948,1.635);
\draw[gp path] (2.768,1.705)--(2.948,1.705);
\draw[gp path] (4.314,2.669)--(4.314,2.735);
\draw[gp path] (4.224,2.669)--(4.404,2.669);
\draw[gp path] (4.224,2.735)--(4.404,2.735);
\draw[gp path] (5.443,3.456)--(5.443,3.520);
\draw[gp path] (5.353,3.456)--(5.533,3.456);
\draw[gp path] (5.353,3.520)--(5.533,3.520);
\draw[gp path] (6.387,4.125)--(6.387,4.189);
\draw[gp path] (6.297,4.125)--(6.477,4.125);
\draw[gp path] (6.297,4.189)--(6.477,4.189);
\draw[gp path] (7.248,4.693)--(7.248,4.755);
\draw[gp path] (7.158,4.693)--(7.338,4.693);
\draw[gp path] (7.158,4.755)--(7.338,4.755);
\draw[gp path] (7.967,5.252)--(7.967,5.314);
\draw[gp path] (7.877,5.252)--(8.057,5.252);
\draw[gp path] (7.877,5.314)--(8.057,5.314);
\draw[gp path] (8.685,5.723)--(8.685,5.784);
\draw[gp path] (8.595,5.723)--(8.775,5.723);
\draw[gp path] (8.595,5.784)--(8.775,5.784);
\draw[gp path] (9.321,6.185)--(9.321,6.247);
\draw[gp path] (9.231,6.185)--(9.411,6.185);
\draw[gp path] (9.231,6.247)--(9.411,6.247);
\draw[gp path] (9.957,6.591)--(9.957,6.652);
\draw[gp path] (9.867,6.591)--(10.047,6.591);
\draw[gp path] (9.867,6.652)--(10.047,6.652);
\draw[gp path] (10.511,7.010)--(10.511,7.071);
\draw[gp path] (10.421,7.010)--(10.601,7.010);
\draw[gp path] (10.421,7.071)--(10.601,7.071);
\draw[gp path] (11.065,7.388)--(11.065,7.449);
\draw[gp path] (10.975,7.388)--(11.155,7.388);
\draw[gp path] (10.975,7.449)--(11.155,7.449);
\draw[gp path] (11.619,7.732)--(11.619,7.792);
\draw[gp path] (11.529,7.732)--(11.709,7.732);
\draw[gp path] (11.529,7.792)--(11.709,7.792);
\gpsetpointsize{4.00}
\gppoint{gp mark 1}{(2.858,1.670)}
\gppoint{gp mark 1}{(4.314,2.702)}
\gppoint{gp mark 1}{(5.443,3.488)}
\gppoint{gp mark 1}{(6.387,4.157)}
\gppoint{gp mark 1}{(7.248,4.724)}
\gppoint{gp mark 1}{(7.967,5.283)}
\gppoint{gp mark 1}{(8.685,5.753)}
\gppoint{gp mark 1}{(9.321,6.216)}
\gppoint{gp mark 1}{(9.957,6.621)}
\gppoint{gp mark 1}{(10.511,7.040)}
\gppoint{gp mark 1}{(11.065,7.419)}
\gppoint{gp mark 1}{(11.619,7.762)}
\gppoint{gp mark 1}{(11.121,1.627)}
\gpcolor{color=gp lt color border}
\node[gp node left] at (7.167,1.319) {Retta interpolante};
\gpcolor{color=gp lt color 1}
\gpsetlinetype{gp lt plot 1}
\draw[gp path] (10.663,1.319)--(11.579,1.319);
\draw[gp path] (2.858,1.686)--(2.946,1.748)--(3.035,1.809)--(3.123,1.871)--(3.212,1.933)%
  --(3.300,1.994)--(3.389,2.056)--(3.477,2.118)--(3.566,2.180)--(3.654,2.241)--(3.742,2.303)%
  --(3.831,2.365)--(3.919,2.427)--(4.008,2.488)--(4.096,2.550)--(4.185,2.612)--(4.273,2.673)%
  --(4.362,2.735)--(4.450,2.797)--(4.539,2.859)--(4.627,2.920)--(4.716,2.982)--(4.804,3.044)%
  --(4.893,3.106)--(4.981,3.167)--(5.070,3.229)--(5.158,3.291)--(5.247,3.353)--(5.335,3.414)%
  --(5.424,3.476)--(5.512,3.538)--(5.601,3.599)--(5.689,3.661)--(5.778,3.723)--(5.866,3.785)%
  --(5.955,3.846)--(6.043,3.908)--(6.132,3.970)--(6.220,4.032)--(6.309,4.093)--(6.397,4.155)%
  --(6.486,4.217)--(6.574,4.279)--(6.663,4.340)--(6.751,4.402)--(6.840,4.464)--(6.928,4.525)%
  --(7.017,4.587)--(7.105,4.649)--(7.194,4.711)--(7.282,4.772)--(7.371,4.834)--(7.459,4.896)%
  --(7.548,4.958)--(7.636,5.019)--(7.725,5.081)--(7.813,5.143)--(7.902,5.204)--(7.990,5.266)%
  --(8.079,5.328)--(8.167,5.390)--(8.256,5.451)--(8.344,5.513)--(8.433,5.575)--(8.521,5.637)%
  --(8.610,5.698)--(8.698,5.760)--(8.787,5.822)--(8.875,5.884)--(8.964,5.945)--(9.052,6.007)%
  --(9.141,6.069)--(9.229,6.130)--(9.318,6.192)--(9.406,6.254)--(9.495,6.316)--(9.583,6.377)%
  --(9.672,6.439)--(9.760,6.501)--(9.849,6.563)--(9.937,6.624)--(10.026,6.686)--(10.114,6.748)%
  --(10.203,6.809)--(10.291,6.871)--(10.380,6.933)--(10.468,6.995)--(10.557,7.056)--(10.645,7.118)%
  --(10.734,7.180)--(10.822,7.242)--(10.911,7.303)--(10.999,7.365)--(11.088,7.427)--(11.176,7.489)%
  --(11.265,7.550)--(11.353,7.612)--(11.442,7.674)--(11.530,7.735)--(11.619,7.797);
\gpcolor{color=gp lt color border}
\gpsetlinetype{gp lt border}
\draw[gp path] (1.688,7.825)--(1.688,0.985)--(11.947,0.985)--(11.947,7.825)--cycle;
%% coordinates of the plot area
\gpdefrectangularnode{gp plot 1}{\pgfpoint{1.688cm}{0.985cm}}{\pgfpoint{11.947cm}{7.825cm}}
\end{tikzpicture}
%% gnuplot variables

\caption{Prima serie, accelerazione}
\label{fig:1}
\end{grafico}

\begin{grafico}
    \centering
\begin{tikzpicture}[gnuplot]
%% generated with GNUPLOT 4.6p0 (Lua 5.1; terminal rev. 99, script rev. 100)
%% Mon 14 Apr 2014 11:09:57 PM CEST
\path (0.000,0.000) rectangle (12.500,8.750);
\gpcolor{color=gp lt color border}
\gpsetlinetype{gp lt border}
\gpsetlinewidth{1.00}
\draw[gp path] (1.320,0.985)--(1.500,0.985);
\draw[gp path] (11.947,0.985)--(11.767,0.985);
\node[gp node right] at (1.136,0.985) { 10};
\draw[gp path] (1.320,1.745)--(1.500,1.745);
\draw[gp path] (11.947,1.745)--(11.767,1.745);
\node[gp node right] at (1.136,1.745) { 15};
\draw[gp path] (1.320,2.505)--(1.500,2.505);
\draw[gp path] (11.947,2.505)--(11.767,2.505);
\node[gp node right] at (1.136,2.505) { 20};
\draw[gp path] (1.320,3.265)--(1.500,3.265);
\draw[gp path] (11.947,3.265)--(11.767,3.265);
\node[gp node right] at (1.136,3.265) { 25};
\draw[gp path] (1.320,4.025)--(1.500,4.025);
\draw[gp path] (11.947,4.025)--(11.767,4.025);
\node[gp node right] at (1.136,4.025) { 30};
\draw[gp path] (1.320,4.785)--(1.500,4.785);
\draw[gp path] (11.947,4.785)--(11.767,4.785);
\node[gp node right] at (1.136,4.785) { 35};
\draw[gp path] (1.320,5.545)--(1.500,5.545);
\draw[gp path] (11.947,5.545)--(11.767,5.545);
\node[gp node right] at (1.136,5.545) { 40};
\draw[gp path] (1.320,6.305)--(1.500,6.305);
\draw[gp path] (11.947,6.305)--(11.767,6.305);
\node[gp node right] at (1.136,6.305) { 45};
\draw[gp path] (1.320,7.065)--(1.500,7.065);
\draw[gp path] (11.947,7.065)--(11.767,7.065);
\node[gp node right] at (1.136,7.065) { 50};
\draw[gp path] (1.320,7.825)--(1.500,7.825);
\draw[gp path] (11.947,7.825)--(11.767,7.825);
\node[gp node right] at (1.136,7.825) { 55};
\draw[gp path] (1.320,0.985)--(1.320,1.165);
\draw[gp path] (1.320,7.825)--(1.320,7.645);
\node[gp node center] at (1.320,0.677) {-10};
\draw[gp path] (3.977,0.985)--(3.977,1.165);
\draw[gp path] (3.977,7.825)--(3.977,7.645);
\node[gp node center] at (3.977,0.677) {-5};
\draw[gp path] (6.634,0.985)--(6.634,1.165);
\draw[gp path] (6.634,7.825)--(6.634,7.645);
\node[gp node center] at (6.634,0.677) { 0};
\draw[gp path] (9.290,0.985)--(9.290,1.165);
\draw[gp path] (9.290,7.825)--(9.290,7.645);
\node[gp node center] at (9.290,0.677) { 5};
\draw[gp path] (11.947,0.985)--(11.947,1.165);
\draw[gp path] (11.947,7.825)--(11.947,7.645);
\node[gp node center] at (11.947,0.677) { 10};
\draw[gp path] (1.320,7.825)--(1.320,0.985)--(11.947,0.985)--(11.947,7.825)--cycle;
\node[gp node center,rotate=-270] at (0.246,4.405) {Velocità angolare [rad/s]};
\node[gp node center] at (6.633,0.215) {Tempo [s]};
\node[gp node center] at (6.633,8.287) {Velocità angolare, decelerazione [rad/s]};
\node[gp node left] at (7.167,1.627) {Retta interpolante};
\gpcolor{color=gp lt color 1}
\gpsetlinetype{gp lt plot 1}
\draw[gp path] (10.663,1.627)--(11.579,1.627);
\draw[gp path] (1.320,7.251)--(1.427,7.188)--(1.535,7.125)--(1.642,7.062)--(1.749,6.998)%
  --(1.857,6.935)--(1.964,6.872)--(2.071,6.809)--(2.179,6.746)--(2.286,6.682)--(2.393,6.619)%
  --(2.501,6.556)--(2.608,6.493)--(2.715,6.430)--(2.823,6.366)--(2.930,6.303)--(3.037,6.240)%
  --(3.145,6.177)--(3.252,6.114)--(3.360,6.050)--(3.467,5.987)--(3.574,5.924)--(3.682,5.861)%
  --(3.789,5.798)--(3.896,5.734)--(4.004,5.671)--(4.111,5.608)--(4.218,5.545)--(4.326,5.482)%
  --(4.433,5.418)--(4.540,5.355)--(4.648,5.292)--(4.755,5.229)--(4.862,5.166)--(4.970,5.102)%
  --(5.077,5.039)--(5.184,4.976)--(5.292,4.913)--(5.399,4.850)--(5.506,4.786)--(5.614,4.723)%
  --(5.721,4.660)--(5.828,4.597)--(5.936,4.534)--(6.043,4.470)--(6.150,4.407)--(6.258,4.344)%
  --(6.365,4.281)--(6.472,4.217)--(6.580,4.154)--(6.687,4.091)--(6.795,4.028)--(6.902,3.965)%
  --(7.009,3.901)--(7.117,3.838)--(7.224,3.775)--(7.331,3.712)--(7.439,3.649)--(7.546,3.585)%
  --(7.653,3.522)--(7.761,3.459)--(7.868,3.396)--(7.975,3.333)--(8.083,3.269)--(8.190,3.206)%
  --(8.297,3.143)--(8.405,3.080)--(8.512,3.017)--(8.619,2.953)--(8.727,2.890)--(8.834,2.827)%
  --(8.941,2.764)--(9.049,2.701)--(9.156,2.637)--(9.263,2.574)--(9.371,2.511)--(9.478,2.448)%
  --(9.585,2.385)--(9.693,2.321)--(9.800,2.258)--(9.907,2.195)--(10.015,2.132)--(10.122,2.069)%
  --(10.230,2.005)--(10.337,1.942)--(10.444,1.879)--(10.552,1.816)--(10.659,1.753)--(10.766,1.689)%
  --(10.874,1.626)--(10.981,1.563)--(11.088,1.500)--(11.196,1.437)--(11.303,1.373)--(11.410,1.310)%
  --(11.518,1.247)--(11.625,1.184)--(11.732,1.121)--(11.840,1.057)--(11.947,0.994);
\gpcolor{color=gp lt color border}
\gpsetlinetype{gp lt border}
\draw[gp path] (1.320,7.825)--(1.320,0.985)--(11.947,0.985)--(11.947,7.825)--cycle;
%% coordinates of the plot area
\gpdefrectangularnode{gp plot 1}{\pgfpoint{1.320cm}{0.985cm}}{\pgfpoint{11.947cm}{7.825cm}}
\end{tikzpicture}
%% gnuplot variables

\caption{Seconda serie, accelerazione}
\label{fig:1}
\end{grafico}

\begin{grafico}
    \centering
\begin{tikzpicture}[gnuplot]
%% generated with GNUPLOT 4.6p0 (Lua 5.1; terminal rev. 99, script rev. 100)
%% Mon 14 Apr 2014 11:09:57 PM CEST
\path (0.000,0.000) rectangle (12.500,8.750);
\gpcolor{color=gp lt color border}
\gpsetlinetype{gp lt border}
\gpsetlinewidth{1.00}
\draw[gp path] (1.320,0.985)--(1.500,0.985);
\draw[gp path] (11.947,0.985)--(11.767,0.985);
\node[gp node right] at (1.136,0.985) { 10};
\draw[gp path] (1.320,1.745)--(1.500,1.745);
\draw[gp path] (11.947,1.745)--(11.767,1.745);
\node[gp node right] at (1.136,1.745) { 15};
\draw[gp path] (1.320,2.505)--(1.500,2.505);
\draw[gp path] (11.947,2.505)--(11.767,2.505);
\node[gp node right] at (1.136,2.505) { 20};
\draw[gp path] (1.320,3.265)--(1.500,3.265);
\draw[gp path] (11.947,3.265)--(11.767,3.265);
\node[gp node right] at (1.136,3.265) { 25};
\draw[gp path] (1.320,4.025)--(1.500,4.025);
\draw[gp path] (11.947,4.025)--(11.767,4.025);
\node[gp node right] at (1.136,4.025) { 30};
\draw[gp path] (1.320,4.785)--(1.500,4.785);
\draw[gp path] (11.947,4.785)--(11.767,4.785);
\node[gp node right] at (1.136,4.785) { 35};
\draw[gp path] (1.320,5.545)--(1.500,5.545);
\draw[gp path] (11.947,5.545)--(11.767,5.545);
\node[gp node right] at (1.136,5.545) { 40};
\draw[gp path] (1.320,6.305)--(1.500,6.305);
\draw[gp path] (11.947,6.305)--(11.767,6.305);
\node[gp node right] at (1.136,6.305) { 45};
\draw[gp path] (1.320,7.065)--(1.500,7.065);
\draw[gp path] (11.947,7.065)--(11.767,7.065);
\node[gp node right] at (1.136,7.065) { 50};
\draw[gp path] (1.320,7.825)--(1.500,7.825);
\draw[gp path] (11.947,7.825)--(11.767,7.825);
\node[gp node right] at (1.136,7.825) { 55};
\draw[gp path] (1.320,0.985)--(1.320,1.165);
\draw[gp path] (1.320,7.825)--(1.320,7.645);
\node[gp node center] at (1.320,0.677) {-10};
\draw[gp path] (3.977,0.985)--(3.977,1.165);
\draw[gp path] (3.977,7.825)--(3.977,7.645);
\node[gp node center] at (3.977,0.677) {-5};
\draw[gp path] (6.634,0.985)--(6.634,1.165);
\draw[gp path] (6.634,7.825)--(6.634,7.645);
\node[gp node center] at (6.634,0.677) { 0};
\draw[gp path] (9.290,0.985)--(9.290,1.165);
\draw[gp path] (9.290,7.825)--(9.290,7.645);
\node[gp node center] at (9.290,0.677) { 5};
\draw[gp path] (11.947,0.985)--(11.947,1.165);
\draw[gp path] (11.947,7.825)--(11.947,7.645);
\node[gp node center] at (11.947,0.677) { 10};
\draw[gp path] (1.320,7.825)--(1.320,0.985)--(11.947,0.985)--(11.947,7.825)--cycle;
\node[gp node center,rotate=-270] at (0.246,4.405) {Velocità angolare [rad/s]};
\node[gp node center] at (6.633,0.215) {Tempo [s]};
\node[gp node center] at (6.633,8.287) {Velocità angolare, decelerazione [rad/s]};
\node[gp node left] at (7.167,1.627) {Retta interpolante};
\gpcolor{color=gp lt color 1}
\gpsetlinetype{gp lt plot 1}
\draw[gp path] (10.663,1.627)--(11.579,1.627);
\draw[gp path] (1.320,7.303)--(1.427,7.239)--(1.535,7.176)--(1.642,7.112)--(1.749,7.048)%
  --(1.857,6.985)--(1.964,6.921)--(2.071,6.857)--(2.179,6.794)--(2.286,6.730)--(2.393,6.666)%
  --(2.501,6.603)--(2.608,6.539)--(2.715,6.476)--(2.823,6.412)--(2.930,6.348)--(3.037,6.285)%
  --(3.145,6.221)--(3.252,6.157)--(3.360,6.094)--(3.467,6.030)--(3.574,5.966)--(3.682,5.903)%
  --(3.789,5.839)--(3.896,5.775)--(4.004,5.712)--(4.111,5.648)--(4.218,5.584)--(4.326,5.521)%
  --(4.433,5.457)--(4.540,5.393)--(4.648,5.330)--(4.755,5.266)--(4.862,5.202)--(4.970,5.139)%
  --(5.077,5.075)--(5.184,5.011)--(5.292,4.948)--(5.399,4.884)--(5.506,4.820)--(5.614,4.757)%
  --(5.721,4.693)--(5.828,4.629)--(5.936,4.566)--(6.043,4.502)--(6.150,4.438)--(6.258,4.375)%
  --(6.365,4.311)--(6.472,4.247)--(6.580,4.184)--(6.687,4.120)--(6.795,4.056)--(6.902,3.993)%
  --(7.009,3.929)--(7.117,3.865)--(7.224,3.802)--(7.331,3.738)--(7.439,3.674)--(7.546,3.611)%
  --(7.653,3.547)--(7.761,3.483)--(7.868,3.420)--(7.975,3.356)--(8.083,3.292)--(8.190,3.229)%
  --(8.297,3.165)--(8.405,3.101)--(8.512,3.038)--(8.619,2.974)--(8.727,2.910)--(8.834,2.847)%
  --(8.941,2.783)--(9.049,2.719)--(9.156,2.656)--(9.263,2.592)--(9.371,2.528)--(9.478,2.465)%
  --(9.585,2.401)--(9.693,2.337)--(9.800,2.274)--(9.907,2.210)--(10.015,2.146)--(10.122,2.083)%
  --(10.230,2.019)--(10.337,1.955)--(10.444,1.892)--(10.552,1.828)--(10.659,1.764)--(10.766,1.701)%
  --(10.874,1.637)--(10.981,1.574)--(11.088,1.510)--(11.196,1.446)--(11.303,1.383)--(11.410,1.319)%
  --(11.518,1.255)--(11.625,1.192)--(11.732,1.128)--(11.840,1.064)--(11.947,1.001);
\gpcolor{color=gp lt color border}
\gpsetlinetype{gp lt border}
\draw[gp path] (1.320,7.825)--(1.320,0.985)--(11.947,0.985)--(11.947,7.825)--cycle;
%% coordinates of the plot area
\gpdefrectangularnode{gp plot 1}{\pgfpoint{1.320cm}{0.985cm}}{\pgfpoint{11.947cm}{7.825cm}}
\end{tikzpicture}
%% gnuplot variables

\caption{Terza serie, accelerazione}
\label{fig:1}
\end{grafico}

\begin{grafico}
    \centering
\begin{tikzpicture}[gnuplot]
%% generated with GNUPLOT 4.6p0 (Lua 5.1; terminal rev. 99, script rev. 100)
%% Tue 15 Apr 2014 09:47:06 PM CEST
\path (0.000,0.000) rectangle (12.500,8.750);
\gpcolor{color=gp lt color border}
\gpsetlinetype{gp lt border}
\gpsetlinewidth{1.00}
\draw[gp path] (1.688,0.985)--(1.868,0.985);
\draw[gp path] (11.947,0.985)--(11.767,0.985);
\node[gp node right] at (1.504,0.985) { 0.15};
\draw[gp path] (1.688,1.669)--(1.868,1.669);
\draw[gp path] (11.947,1.669)--(11.767,1.669);
\node[gp node right] at (1.504,1.669) { 0.2};
\draw[gp path] (1.688,2.353)--(1.868,2.353);
\draw[gp path] (11.947,2.353)--(11.767,2.353);
\node[gp node right] at (1.504,2.353) { 0.25};
\draw[gp path] (1.688,3.037)--(1.868,3.037);
\draw[gp path] (11.947,3.037)--(11.767,3.037);
\node[gp node right] at (1.504,3.037) { 0.3};
\draw[gp path] (1.688,3.721)--(1.868,3.721);
\draw[gp path] (11.947,3.721)--(11.767,3.721);
\node[gp node right] at (1.504,3.721) { 0.35};
\draw[gp path] (1.688,4.405)--(1.868,4.405);
\draw[gp path] (11.947,4.405)--(11.767,4.405);
\node[gp node right] at (1.504,4.405) { 0.4};
\draw[gp path] (1.688,5.089)--(1.868,5.089);
\draw[gp path] (11.947,5.089)--(11.767,5.089);
\node[gp node right] at (1.504,5.089) { 0.45};
\draw[gp path] (1.688,5.773)--(1.868,5.773);
\draw[gp path] (11.947,5.773)--(11.767,5.773);
\node[gp node right] at (1.504,5.773) { 0.5};
\draw[gp path] (1.688,6.457)--(1.868,6.457);
\draw[gp path] (11.947,6.457)--(11.767,6.457);
\node[gp node right] at (1.504,6.457) { 0.55};
\draw[gp path] (1.688,7.141)--(1.868,7.141);
\draw[gp path] (11.947,7.141)--(11.767,7.141);
\node[gp node right] at (1.504,7.141) { 0.6};
\draw[gp path] (1.688,7.825)--(1.868,7.825);
\draw[gp path] (11.947,7.825)--(11.767,7.825);
\node[gp node right] at (1.504,7.825) { 0.65};
\draw[gp path] (1.688,0.985)--(1.688,1.165);
\draw[gp path] (1.688,7.825)--(1.688,7.645);
\node[gp node center] at (1.688,0.677) { 5};
\draw[gp path] (3.740,0.985)--(3.740,1.165);
\draw[gp path] (3.740,7.825)--(3.740,7.645);
\node[gp node center] at (3.740,0.677) { 10};
\draw[gp path] (5.792,0.985)--(5.792,1.165);
\draw[gp path] (5.792,7.825)--(5.792,7.645);
\node[gp node center] at (5.792,0.677) { 15};
\draw[gp path] (7.843,0.985)--(7.843,1.165);
\draw[gp path] (7.843,7.825)--(7.843,7.645);
\node[gp node center] at (7.843,0.677) { 20};
\draw[gp path] (9.895,0.985)--(9.895,1.165);
\draw[gp path] (9.895,7.825)--(9.895,7.645);
\node[gp node center] at (9.895,0.677) { 25};
\draw[gp path] (11.947,0.985)--(11.947,1.165);
\draw[gp path] (11.947,7.825)--(11.947,7.645);
\node[gp node center] at (11.947,0.677) { 30};
\draw[gp path] (1.688,7.825)--(1.688,0.985)--(11.947,0.985)--(11.947,7.825)--cycle;
\node[gp node center,rotate=-270] at (0.246,4.405) {Velocità angolare [rad/s]};
\node[gp node center] at (6.817,0.215) {Tempo [s]};
\node[gp node center] at (6.817,8.287) {Velocità angolari in accelerazione [rad/s]};
\node[gp node left] at (7.167,1.627) {Dati};
\gpcolor{color=gp lt color 0}
\gpsetlinetype{gp lt plot 0}
\draw[gp path] (10.663,1.627)--(11.579,1.627);
\draw[gp path] (10.663,1.717)--(10.663,1.537);
\draw[gp path] (11.579,1.717)--(11.579,1.537);
\draw[gp path] (2.899,1.601)--(2.899,1.669);
\draw[gp path] (2.809,1.601)--(2.989,1.601);
\draw[gp path] (2.809,1.669)--(2.989,1.669);
\draw[gp path] (4.437,2.574)--(4.437,2.637);
\draw[gp path] (4.347,2.574)--(4.527,2.574);
\draw[gp path] (4.347,2.637)--(4.527,2.637);
\draw[gp path] (5.545,3.378)--(5.545,3.440);
\draw[gp path] (5.455,3.378)--(5.635,3.378);
\draw[gp path] (5.455,3.440)--(5.635,3.440);
\draw[gp path] (6.510,4.033)--(6.510,4.094);
\draw[gp path] (6.420,4.033)--(6.600,4.033);
\draw[gp path] (6.420,4.094)--(6.600,4.094);
\draw[gp path] (7.371,4.602)--(7.371,4.662);
\draw[gp path] (7.281,4.602)--(7.461,4.602);
\draw[gp path] (7.281,4.662)--(7.461,4.662);
\draw[gp path] (8.131,5.130)--(8.131,5.190);
\draw[gp path] (8.041,5.130)--(8.221,5.130);
\draw[gp path] (8.041,5.190)--(8.221,5.190);
\draw[gp path] (8.828,5.617)--(8.828,5.677);
\draw[gp path] (8.738,5.617)--(8.918,5.617);
\draw[gp path] (8.738,5.677)--(8.918,5.677);
\draw[gp path] (9.464,6.079)--(9.464,6.139);
\draw[gp path] (9.374,6.079)--(9.554,6.079);
\draw[gp path] (9.374,6.139)--(9.554,6.139);
\draw[gp path] (10.100,6.486)--(10.100,6.545);
\draw[gp path] (10.010,6.486)--(10.190,6.486);
\draw[gp path] (10.010,6.545)--(10.190,6.545);
\draw[gp path] (10.634,6.920)--(10.634,6.979);
\draw[gp path] (10.544,6.920)--(10.724,6.920);
\draw[gp path] (10.544,6.979)--(10.724,6.979);
\draw[gp path] (11.208,7.284)--(11.208,7.343);
\draw[gp path] (11.118,7.284)--(11.298,7.284);
\draw[gp path] (11.118,7.343)--(11.298,7.343);
\draw[gp path] (11.742,7.643)--(11.742,7.702);
\draw[gp path] (11.652,7.643)--(11.832,7.643);
\draw[gp path] (11.652,7.702)--(11.832,7.702);
\gpsetpointsize{4.00}
\gppoint{gp mark 1}{(2.899,1.635)}
\gppoint{gp mark 1}{(4.437,2.606)}
\gppoint{gp mark 1}{(5.545,3.409)}
\gppoint{gp mark 1}{(6.510,4.064)}
\gppoint{gp mark 1}{(7.371,4.632)}
\gppoint{gp mark 1}{(8.131,5.160)}
\gppoint{gp mark 1}{(8.828,5.647)}
\gppoint{gp mark 1}{(9.464,6.109)}
\gppoint{gp mark 1}{(10.100,6.516)}
\gppoint{gp mark 1}{(10.634,6.950)}
\gppoint{gp mark 1}{(11.208,7.313)}
\gppoint{gp mark 1}{(11.742,7.672)}
\gppoint{gp mark 1}{(11.121,1.627)}
\gpcolor{color=gp lt color border}
\node[gp node left] at (7.167,1.319) {Retta interpolante};
\gpcolor{color=gp lt color 1}
\gpsetlinetype{gp lt plot 1}
\draw[gp path] (10.663,1.319)--(11.579,1.319);
\draw[gp path] (2.899,1.579)--(2.988,1.640)--(3.077,1.702)--(3.167,1.763)--(3.256,1.825)%
  --(3.345,1.886)--(3.435,1.948)--(3.524,2.009)--(3.613,2.071)--(3.702,2.132)--(3.792,2.194)%
  --(3.881,2.255)--(3.970,2.317)--(4.060,2.378)--(4.149,2.440)--(4.238,2.501)--(4.328,2.563)%
  --(4.417,2.624)--(4.506,2.686)--(4.596,2.748)--(4.685,2.809)--(4.774,2.871)--(4.864,2.932)%
  --(4.953,2.994)--(5.042,3.055)--(5.132,3.117)--(5.221,3.178)--(5.310,3.240)--(5.400,3.301)%
  --(5.489,3.363)--(5.578,3.424)--(5.668,3.486)--(5.757,3.547)--(5.846,3.609)--(5.936,3.670)%
  --(6.025,3.732)--(6.114,3.793)--(6.204,3.855)--(6.293,3.916)--(6.382,3.978)--(6.472,4.039)%
  --(6.561,4.101)--(6.650,4.162)--(6.740,4.224)--(6.829,4.285)--(6.918,4.347)--(7.008,4.409)%
  --(7.097,4.470)--(7.186,4.532)--(7.276,4.593)--(7.365,4.655)--(7.454,4.716)--(7.544,4.778)%
  --(7.633,4.839)--(7.722,4.901)--(7.811,4.962)--(7.901,5.024)--(7.990,5.085)--(8.079,5.147)%
  --(8.169,5.208)--(8.258,5.270)--(8.347,5.331)--(8.437,5.393)--(8.526,5.454)--(8.615,5.516)%
  --(8.705,5.577)--(8.794,5.639)--(8.883,5.700)--(8.973,5.762)--(9.062,5.823)--(9.151,5.885)%
  --(9.241,5.947)--(9.330,6.008)--(9.419,6.070)--(9.509,6.131)--(9.598,6.193)--(9.687,6.254)%
  --(9.777,6.316)--(9.866,6.377)--(9.955,6.439)--(10.045,6.500)--(10.134,6.562)--(10.223,6.623)%
  --(10.313,6.685)--(10.402,6.746)--(10.491,6.808)--(10.581,6.869)--(10.670,6.931)--(10.759,6.992)%
  --(10.849,7.054)--(10.938,7.115)--(11.027,7.177)--(11.117,7.238)--(11.206,7.300)--(11.295,7.361)%
  --(11.385,7.423)--(11.474,7.484)--(11.563,7.546)--(11.652,7.608)--(11.742,7.669);
\gpcolor{color=gp lt color border}
\gpsetlinetype{gp lt border}
\draw[gp path] (1.688,7.825)--(1.688,0.985)--(11.947,0.985)--(11.947,7.825)--cycle;
%% coordinates of the plot area
\gpdefrectangularnode{gp plot 1}{\pgfpoint{1.688cm}{0.985cm}}{\pgfpoint{11.947cm}{7.825cm}}
\end{tikzpicture}
%% gnuplot variables

\caption{Quarta serie, accelerazione}
\label{fig:1}
\end{grafico}

\begin{grafico}
    \centering
\begin{tikzpicture}[gnuplot]
%% generated with GNUPLOT 4.6p0 (Lua 5.1; terminal rev. 99, script rev. 100)
%% Tue 15 Apr 2014 06:32:32 PM CEST
\path (0.000,0.000) rectangle (12.500,8.750);
\gpcolor{color=gp lt color border}
\gpsetlinetype{gp lt border}
\gpsetlinewidth{1.00}
\draw[gp path] (1.688,0.985)--(1.868,0.985);
\draw[gp path] (11.947,0.985)--(11.767,0.985);
\node[gp node right] at (1.504,0.985) { 0.15};
\draw[gp path] (1.688,1.669)--(1.868,1.669);
\draw[gp path] (11.947,1.669)--(11.767,1.669);
\node[gp node right] at (1.504,1.669) { 0.2};
\draw[gp path] (1.688,2.353)--(1.868,2.353);
\draw[gp path] (11.947,2.353)--(11.767,2.353);
\node[gp node right] at (1.504,2.353) { 0.25};
\draw[gp path] (1.688,3.037)--(1.868,3.037);
\draw[gp path] (11.947,3.037)--(11.767,3.037);
\node[gp node right] at (1.504,3.037) { 0.3};
\draw[gp path] (1.688,3.721)--(1.868,3.721);
\draw[gp path] (11.947,3.721)--(11.767,3.721);
\node[gp node right] at (1.504,3.721) { 0.35};
\draw[gp path] (1.688,4.405)--(1.868,4.405);
\draw[gp path] (11.947,4.405)--(11.767,4.405);
\node[gp node right] at (1.504,4.405) { 0.4};
\draw[gp path] (1.688,5.089)--(1.868,5.089);
\draw[gp path] (11.947,5.089)--(11.767,5.089);
\node[gp node right] at (1.504,5.089) { 0.45};
\draw[gp path] (1.688,5.773)--(1.868,5.773);
\draw[gp path] (11.947,5.773)--(11.767,5.773);
\node[gp node right] at (1.504,5.773) { 0.5};
\draw[gp path] (1.688,6.457)--(1.868,6.457);
\draw[gp path] (11.947,6.457)--(11.767,6.457);
\node[gp node right] at (1.504,6.457) { 0.55};
\draw[gp path] (1.688,7.141)--(1.868,7.141);
\draw[gp path] (11.947,7.141)--(11.767,7.141);
\node[gp node right] at (1.504,7.141) { 0.6};
\draw[gp path] (1.688,7.825)--(1.868,7.825);
\draw[gp path] (11.947,7.825)--(11.767,7.825);
\node[gp node right] at (1.504,7.825) { 0.65};
\draw[gp path] (1.688,0.985)--(1.688,1.165);
\draw[gp path] (1.688,7.825)--(1.688,7.645);
\node[gp node center] at (1.688,0.677) { 5};
\draw[gp path] (3.740,0.985)--(3.740,1.165);
\draw[gp path] (3.740,7.825)--(3.740,7.645);
\node[gp node center] at (3.740,0.677) { 10};
\draw[gp path] (5.792,0.985)--(5.792,1.165);
\draw[gp path] (5.792,7.825)--(5.792,7.645);
\node[gp node center] at (5.792,0.677) { 15};
\draw[gp path] (7.843,0.985)--(7.843,1.165);
\draw[gp path] (7.843,7.825)--(7.843,7.645);
\node[gp node center] at (7.843,0.677) { 20};
\draw[gp path] (9.895,0.985)--(9.895,1.165);
\draw[gp path] (9.895,7.825)--(9.895,7.645);
\node[gp node center] at (9.895,0.677) { 25};
\draw[gp path] (11.947,0.985)--(11.947,1.165);
\draw[gp path] (11.947,7.825)--(11.947,7.645);
\node[gp node center] at (11.947,0.677) { 30};
\draw[gp path] (1.688,7.825)--(1.688,0.985)--(11.947,0.985)--(11.947,7.825)--cycle;
\node[gp node center,rotate=-270] at (0.246,4.405) {Velocità angolare [rad/s]};
\node[gp node center] at (6.817,0.215) {Tempo [s]};
\node[gp node center] at (6.817,8.287) {Velocità angolare, decelerazione [rad/s]};
\node[gp node left] at (7.167,1.627) {Dati};
\gpcolor{color=gp lt color 0}
\gpsetlinetype{gp lt plot 0}
\draw[gp path] (10.663,1.627)--(11.579,1.627);
\draw[gp path] (10.663,1.717)--(10.663,1.537);
\draw[gp path] (11.579,1.717)--(11.579,1.537);
\draw[gp path] (2.837,1.652)--(2.837,1.723);
\draw[gp path] (2.747,1.652)--(2.927,1.652);
\draw[gp path] (2.747,1.723)--(2.927,1.723);
\draw[gp path] (4.355,2.637)--(4.355,2.702);
\draw[gp path] (4.265,2.637)--(4.445,2.637);
\draw[gp path] (4.265,2.702)--(4.445,2.702);
\draw[gp path] (5.463,3.440)--(5.463,3.504);
\draw[gp path] (5.373,3.440)--(5.553,3.440);
\draw[gp path] (5.373,3.504)--(5.553,3.504);
\draw[gp path] (6.428,4.094)--(6.428,4.157);
\draw[gp path] (6.338,4.094)--(6.518,4.094);
\draw[gp path] (6.338,4.157)--(6.518,4.157);
\draw[gp path] (7.269,4.677)--(7.269,4.739);
\draw[gp path] (7.179,4.677)--(7.359,4.677);
\draw[gp path] (7.179,4.739)--(7.359,4.739);
\draw[gp path] (8.008,5.221)--(8.008,5.283);
\draw[gp path] (7.918,5.221)--(8.098,5.221);
\draw[gp path] (7.918,5.283)--(8.098,5.283);
\draw[gp path] (8.726,5.692)--(8.726,5.753);
\draw[gp path] (8.636,5.692)--(8.816,5.692);
\draw[gp path] (8.636,5.753)--(8.816,5.753);
\draw[gp path] (9.341,6.170)--(9.341,6.231);
\draw[gp path] (9.251,6.170)--(9.431,6.170);
\draw[gp path] (9.251,6.231)--(9.431,6.231);
\draw[gp path] (9.998,6.561)--(9.998,6.621);
\draw[gp path] (9.908,6.561)--(10.088,6.561);
\draw[gp path] (9.908,6.621)--(10.088,6.621);
\draw[gp path] (10.572,6.964)--(10.572,7.025);
\draw[gp path] (10.482,6.964)--(10.662,6.964);
\draw[gp path] (10.482,7.025)--(10.662,7.025);
\draw[gp path] (11.126,7.343)--(11.126,7.403);
\draw[gp path] (11.036,7.343)--(11.216,7.343);
\draw[gp path] (11.036,7.403)--(11.216,7.403);
\draw[gp path] (11.598,7.747)--(11.598,7.808);
\draw[gp path] (11.508,7.747)--(11.688,7.747);
\draw[gp path] (11.508,7.808)--(11.688,7.808);
\gpsetpointsize{4.00}
\gppoint{gp mark 1}{(2.837,1.687)}
\gppoint{gp mark 1}{(4.355,2.669)}
\gppoint{gp mark 1}{(5.463,3.472)}
\gppoint{gp mark 1}{(6.428,4.126)}
\gppoint{gp mark 1}{(7.269,4.708)}
\gppoint{gp mark 1}{(8.008,5.252)}
\gppoint{gp mark 1}{(8.726,5.723)}
\gppoint{gp mark 1}{(9.341,6.200)}
\gppoint{gp mark 1}{(9.998,6.591)}
\gppoint{gp mark 1}{(10.572,6.995)}
\gppoint{gp mark 1}{(11.126,7.373)}
\gppoint{gp mark 1}{(11.598,7.777)}
\gppoint{gp mark 1}{(11.121,1.627)}
\gpcolor{color=gp lt color border}
\node[gp node left] at (7.167,1.319) {Retta interpolante};
\gpcolor{color=gp lt color 1}
\gpsetlinetype{gp lt plot 1}
\draw[gp path] (10.663,1.319)--(11.579,1.319);
\draw[gp path] (2.837,1.649)--(2.926,1.711)--(3.014,1.772)--(3.102,1.834)--(3.191,1.895)%
  --(3.279,1.956)--(3.368,2.018)--(3.456,2.079)--(3.545,2.140)--(3.633,2.202)--(3.722,2.263)%
  --(3.810,2.325)--(3.899,2.386)--(3.987,2.447)--(4.076,2.509)--(4.164,2.570)--(4.253,2.632)%
  --(4.341,2.693)--(4.430,2.754)--(4.518,2.816)--(4.607,2.877)--(4.695,2.938)--(4.784,3.000)%
  --(4.872,3.061)--(4.961,3.123)--(5.049,3.184)--(5.138,3.245)--(5.226,3.307)--(5.315,3.368)%
  --(5.403,3.430)--(5.492,3.491)--(5.580,3.552)--(5.669,3.614)--(5.757,3.675)--(5.846,3.736)%
  --(5.934,3.798)--(6.023,3.859)--(6.111,3.921)--(6.200,3.982)--(6.288,4.043)--(6.377,4.105)%
  --(6.465,4.166)--(6.554,4.228)--(6.642,4.289)--(6.731,4.350)--(6.819,4.412)--(6.908,4.473)%
  --(6.996,4.534)--(7.085,4.596)--(7.173,4.657)--(7.262,4.719)--(7.350,4.780)--(7.439,4.841)%
  --(7.527,4.903)--(7.616,4.964)--(7.704,5.026)--(7.793,5.087)--(7.881,5.148)--(7.970,5.210)%
  --(8.058,5.271)--(8.147,5.332)--(8.235,5.394)--(8.324,5.455)--(8.412,5.517)--(8.501,5.578)%
  --(8.589,5.639)--(8.678,5.701)--(8.766,5.762)--(8.855,5.824)--(8.943,5.885)--(9.032,5.946)%
  --(9.120,6.008)--(9.209,6.069)--(9.297,6.130)--(9.386,6.192)--(9.474,6.253)--(9.563,6.315)%
  --(9.651,6.376)--(9.740,6.437)--(9.828,6.499)--(9.917,6.560)--(10.005,6.622)--(10.094,6.683)%
  --(10.182,6.744)--(10.271,6.806)--(10.359,6.867)--(10.448,6.928)--(10.536,6.990)--(10.625,7.051)%
  --(10.713,7.113)--(10.802,7.174)--(10.890,7.235)--(10.979,7.297)--(11.067,7.358)--(11.156,7.420)%
  --(11.244,7.481)--(11.333,7.542)--(11.421,7.604)--(11.510,7.665)--(11.598,7.726);
\gpcolor{color=gp lt color border}
\gpsetlinetype{gp lt border}
\draw[gp path] (1.688,7.825)--(1.688,0.985)--(11.947,0.985)--(11.947,7.825)--cycle;
%% coordinates of the plot area
\gpdefrectangularnode{gp plot 1}{\pgfpoint{1.688cm}{0.985cm}}{\pgfpoint{11.947cm}{7.825cm}}
\end{tikzpicture}
%% gnuplot variables

\caption{Quinta serie, accelerazione}
\label{fig:1}
\end{grafico}

\begin{grafico}
    \centering
\begin{tikzpicture}[gnuplot]
%% generated with GNUPLOT 4.6p0 (Lua 5.1; terminal rev. 99, script rev. 100)
%% Mon 14 Apr 2014 11:09:58 PM CEST
\path (0.000,0.000) rectangle (12.500,8.750);
\gpcolor{color=gp lt color border}
\gpsetlinetype{gp lt border}
\gpsetlinewidth{1.00}
\draw[gp path] (1.320,0.985)--(1.500,0.985);
\draw[gp path] (11.947,0.985)--(11.767,0.985);
\node[gp node right] at (1.136,0.985) { 5};
\draw[gp path] (1.320,1.669)--(1.500,1.669);
\draw[gp path] (11.947,1.669)--(11.767,1.669);
\node[gp node right] at (1.136,1.669) { 10};
\draw[gp path] (1.320,2.353)--(1.500,2.353);
\draw[gp path] (11.947,2.353)--(11.767,2.353);
\node[gp node right] at (1.136,2.353) { 15};
\draw[gp path] (1.320,3.037)--(1.500,3.037);
\draw[gp path] (11.947,3.037)--(11.767,3.037);
\node[gp node right] at (1.136,3.037) { 20};
\draw[gp path] (1.320,3.721)--(1.500,3.721);
\draw[gp path] (11.947,3.721)--(11.767,3.721);
\node[gp node right] at (1.136,3.721) { 25};
\draw[gp path] (1.320,4.405)--(1.500,4.405);
\draw[gp path] (11.947,4.405)--(11.767,4.405);
\node[gp node right] at (1.136,4.405) { 30};
\draw[gp path] (1.320,5.089)--(1.500,5.089);
\draw[gp path] (11.947,5.089)--(11.767,5.089);
\node[gp node right] at (1.136,5.089) { 35};
\draw[gp path] (1.320,5.773)--(1.500,5.773);
\draw[gp path] (11.947,5.773)--(11.767,5.773);
\node[gp node right] at (1.136,5.773) { 40};
\draw[gp path] (1.320,6.457)--(1.500,6.457);
\draw[gp path] (11.947,6.457)--(11.767,6.457);
\node[gp node right] at (1.136,6.457) { 45};
\draw[gp path] (1.320,7.141)--(1.500,7.141);
\draw[gp path] (11.947,7.141)--(11.767,7.141);
\node[gp node right] at (1.136,7.141) { 50};
\draw[gp path] (1.320,7.825)--(1.500,7.825);
\draw[gp path] (11.947,7.825)--(11.767,7.825);
\node[gp node right] at (1.136,7.825) { 55};
\draw[gp path] (1.320,0.985)--(1.320,1.165);
\draw[gp path] (1.320,7.825)--(1.320,7.645);
\node[gp node center] at (1.320,0.677) {-10};
\draw[gp path] (3.977,0.985)--(3.977,1.165);
\draw[gp path] (3.977,7.825)--(3.977,7.645);
\node[gp node center] at (3.977,0.677) {-5};
\draw[gp path] (6.634,0.985)--(6.634,1.165);
\draw[gp path] (6.634,7.825)--(6.634,7.645);
\node[gp node center] at (6.634,0.677) { 0};
\draw[gp path] (9.290,0.985)--(9.290,1.165);
\draw[gp path] (9.290,7.825)--(9.290,7.645);
\node[gp node center] at (9.290,0.677) { 5};
\draw[gp path] (11.947,0.985)--(11.947,1.165);
\draw[gp path] (11.947,7.825)--(11.947,7.645);
\node[gp node center] at (11.947,0.677) { 10};
\draw[gp path] (1.320,7.825)--(1.320,0.985)--(11.947,0.985)--(11.947,7.825)--cycle;
\node[gp node center,rotate=-270] at (0.246,4.405) {Velocità angolare [rad/s]};
\node[gp node center] at (6.633,0.215) {Tempo [s]};
\node[gp node center] at (6.633,8.287) {Velocità angolare, decelerazione [rad/s]};
\node[gp node left] at (7.167,1.627) {Retta interpolante};
\gpcolor{color=gp lt color 1}
\gpsetlinetype{gp lt plot 1}
\draw[gp path] (10.663,1.627)--(11.579,1.627);
\draw[gp path] (1.320,7.628)--(1.427,7.566)--(1.535,7.505)--(1.642,7.444)--(1.749,7.382)%
  --(1.857,7.321)--(1.964,7.259)--(2.071,7.198)--(2.179,7.137)--(2.286,7.075)--(2.393,7.014)%
  --(2.501,6.952)--(2.608,6.891)--(2.715,6.830)--(2.823,6.768)--(2.930,6.707)--(3.037,6.645)%
  --(3.145,6.584)--(3.252,6.523)--(3.360,6.461)--(3.467,6.400)--(3.574,6.338)--(3.682,6.277)%
  --(3.789,6.216)--(3.896,6.154)--(4.004,6.093)--(4.111,6.031)--(4.218,5.970)--(4.326,5.909)%
  --(4.433,5.847)--(4.540,5.786)--(4.648,5.724)--(4.755,5.663)--(4.862,5.602)--(4.970,5.540)%
  --(5.077,5.479)--(5.184,5.417)--(5.292,5.356)--(5.399,5.294)--(5.506,5.233)--(5.614,5.172)%
  --(5.721,5.110)--(5.828,5.049)--(5.936,4.987)--(6.043,4.926)--(6.150,4.865)--(6.258,4.803)%
  --(6.365,4.742)--(6.472,4.680)--(6.580,4.619)--(6.687,4.558)--(6.795,4.496)--(6.902,4.435)%
  --(7.009,4.373)--(7.117,4.312)--(7.224,4.251)--(7.331,4.189)--(7.439,4.128)--(7.546,4.066)%
  --(7.653,4.005)--(7.761,3.944)--(7.868,3.882)--(7.975,3.821)--(8.083,3.759)--(8.190,3.698)%
  --(8.297,3.637)--(8.405,3.575)--(8.512,3.514)--(8.619,3.452)--(8.727,3.391)--(8.834,3.330)%
  --(8.941,3.268)--(9.049,3.207)--(9.156,3.145)--(9.263,3.084)--(9.371,3.023)--(9.478,2.961)%
  --(9.585,2.900)--(9.693,2.838)--(9.800,2.777)--(9.907,2.716)--(10.015,2.654)--(10.122,2.593)%
  --(10.230,2.531)--(10.337,2.470)--(10.444,2.409)--(10.552,2.347)--(10.659,2.286)--(10.766,2.224)%
  --(10.874,2.163)--(10.981,2.102)--(11.088,2.040)--(11.196,1.979)--(11.303,1.917)--(11.410,1.856)%
  --(11.518,1.795)--(11.625,1.733)--(11.732,1.672)--(11.840,1.610)--(11.947,1.549);
\gpcolor{color=gp lt color border}
\gpsetlinetype{gp lt border}
\draw[gp path] (1.320,7.825)--(1.320,0.985)--(11.947,0.985)--(11.947,7.825)--cycle;
%% coordinates of the plot area
\gpdefrectangularnode{gp plot 1}{\pgfpoint{1.320cm}{0.985cm}}{\pgfpoint{11.947cm}{7.825cm}}
\end{tikzpicture}
%% gnuplot variables

\caption{Sesta serie, accelerazione}
\label{fig:1}
\end{grafico}

\begin{grafico}
    \centering
\begin{tikzpicture}[gnuplot]
%% generated with GNUPLOT 4.6p0 (Lua 5.1; terminal rev. 99, script rev. 100)
%% Tue 15 Apr 2014 06:32:32 PM CEST
\path (0.000,0.000) rectangle (12.500,8.750);
\gpcolor{color=gp lt color border}
\gpsetlinetype{gp lt border}
\gpsetlinewidth{1.00}
\draw[gp path] (1.688,0.985)--(1.868,0.985);
\draw[gp path] (11.947,0.985)--(11.767,0.985);
\node[gp node right] at (1.504,0.985) { 0.15};
\draw[gp path] (1.688,1.669)--(1.868,1.669);
\draw[gp path] (11.947,1.669)--(11.767,1.669);
\node[gp node right] at (1.504,1.669) { 0.2};
\draw[gp path] (1.688,2.353)--(1.868,2.353);
\draw[gp path] (11.947,2.353)--(11.767,2.353);
\node[gp node right] at (1.504,2.353) { 0.25};
\draw[gp path] (1.688,3.037)--(1.868,3.037);
\draw[gp path] (11.947,3.037)--(11.767,3.037);
\node[gp node right] at (1.504,3.037) { 0.3};
\draw[gp path] (1.688,3.721)--(1.868,3.721);
\draw[gp path] (11.947,3.721)--(11.767,3.721);
\node[gp node right] at (1.504,3.721) { 0.35};
\draw[gp path] (1.688,4.405)--(1.868,4.405);
\draw[gp path] (11.947,4.405)--(11.767,4.405);
\node[gp node right] at (1.504,4.405) { 0.4};
\draw[gp path] (1.688,5.089)--(1.868,5.089);
\draw[gp path] (11.947,5.089)--(11.767,5.089);
\node[gp node right] at (1.504,5.089) { 0.45};
\draw[gp path] (1.688,5.773)--(1.868,5.773);
\draw[gp path] (11.947,5.773)--(11.767,5.773);
\node[gp node right] at (1.504,5.773) { 0.5};
\draw[gp path] (1.688,6.457)--(1.868,6.457);
\draw[gp path] (11.947,6.457)--(11.767,6.457);
\node[gp node right] at (1.504,6.457) { 0.55};
\draw[gp path] (1.688,7.141)--(1.868,7.141);
\draw[gp path] (11.947,7.141)--(11.767,7.141);
\node[gp node right] at (1.504,7.141) { 0.6};
\draw[gp path] (1.688,7.825)--(1.868,7.825);
\draw[gp path] (11.947,7.825)--(11.767,7.825);
\node[gp node right] at (1.504,7.825) { 0.65};
\draw[gp path] (1.688,0.985)--(1.688,1.165);
\draw[gp path] (1.688,7.825)--(1.688,7.645);
\node[gp node center] at (1.688,0.677) { 5};
\draw[gp path] (3.740,0.985)--(3.740,1.165);
\draw[gp path] (3.740,7.825)--(3.740,7.645);
\node[gp node center] at (3.740,0.677) { 10};
\draw[gp path] (5.792,0.985)--(5.792,1.165);
\draw[gp path] (5.792,7.825)--(5.792,7.645);
\node[gp node center] at (5.792,0.677) { 15};
\draw[gp path] (7.843,0.985)--(7.843,1.165);
\draw[gp path] (7.843,7.825)--(7.843,7.645);
\node[gp node center] at (7.843,0.677) { 20};
\draw[gp path] (9.895,0.985)--(9.895,1.165);
\draw[gp path] (9.895,7.825)--(9.895,7.645);
\node[gp node center] at (9.895,0.677) { 25};
\draw[gp path] (11.947,0.985)--(11.947,1.165);
\draw[gp path] (11.947,7.825)--(11.947,7.645);
\node[gp node center] at (11.947,0.677) { 30};
\draw[gp path] (1.688,7.825)--(1.688,0.985)--(11.947,0.985)--(11.947,7.825)--cycle;
\node[gp node center,rotate=-270] at (0.246,4.405) {Velocità angolare [rad/s]};
\node[gp node center] at (6.817,0.215) {Tempo [s]};
\node[gp node center] at (6.817,8.287) {Velocità angolare, decelerazione [rad/s]};
\node[gp node left] at (7.167,1.627) {Dati};
\gpcolor{color=gp lt color 0}
\gpsetlinetype{gp lt plot 0}
\draw[gp path] (10.663,1.627)--(11.579,1.627);
\draw[gp path] (10.663,1.717)--(10.663,1.537);
\draw[gp path] (11.579,1.717)--(11.579,1.537);
\draw[gp path] (3.083,1.460)--(3.083,1.521);
\draw[gp path] (2.993,1.460)--(3.173,1.460);
\draw[gp path] (2.993,1.521)--(3.173,1.521);
\draw[gp path] (4.540,2.499)--(4.540,2.559);
\draw[gp path] (4.450,2.499)--(4.630,2.499);
\draw[gp path] (4.450,2.559)--(4.630,2.559);
\draw[gp path] (5.689,3.273)--(5.689,3.332);
\draw[gp path] (5.599,3.273)--(5.779,3.273);
\draw[gp path] (5.599,3.332)--(5.779,3.332);
\draw[gp path] (6.633,3.944)--(6.633,4.003);
\draw[gp path] (6.543,3.944)--(6.723,3.944);
\draw[gp path] (6.543,4.003)--(6.723,4.003);
\draw[gp path] (7.495,4.513)--(7.495,4.572);
\draw[gp path] (7.405,4.513)--(7.585,4.513);
\draw[gp path] (7.405,4.572)--(7.585,4.572);
\draw[gp path] (8.315,4.999)--(8.315,5.057);
\draw[gp path] (8.225,4.999)--(8.405,4.999);
\draw[gp path] (8.225,5.057)--(8.405,5.057);
\draw[gp path] (8.951,5.529)--(8.951,5.587);
\draw[gp path] (8.861,5.529)--(9.041,5.529);
\draw[gp path] (8.861,5.587)--(9.041,5.587);
\draw[gp path] (9.587,5.991)--(9.587,6.050);
\draw[gp path] (9.497,5.991)--(9.677,5.991);
\draw[gp path] (9.497,6.050)--(9.677,6.050);
\draw[gp path] (10.203,6.413)--(10.203,6.471);
\draw[gp path] (10.113,6.413)--(10.293,6.413);
\draw[gp path] (10.113,6.471)--(10.293,6.471);
\draw[gp path] (10.777,6.817)--(10.777,6.875);
\draw[gp path] (10.687,6.817)--(10.867,6.817);
\draw[gp path] (10.687,6.875)--(10.867,6.875);
\draw[gp path] (11.352,7.182)--(11.352,7.240);
\draw[gp path] (11.262,7.182)--(11.442,7.182);
\draw[gp path] (11.262,7.240)--(11.442,7.240);
\draw[gp path] (11.865,7.555)--(11.865,7.613);
\draw[gp path] (11.775,7.555)--(11.955,7.555);
\draw[gp path] (11.775,7.613)--(11.955,7.613);
\gpsetpointsize{4.00}
\gppoint{gp mark 1}{(3.083,1.491)}
\gppoint{gp mark 1}{(4.540,2.529)}
\gppoint{gp mark 1}{(5.689,3.303)}
\gppoint{gp mark 1}{(6.633,3.973)}
\gppoint{gp mark 1}{(7.495,4.543)}
\gppoint{gp mark 1}{(8.315,5.028)}
\gppoint{gp mark 1}{(8.951,5.558)}
\gppoint{gp mark 1}{(9.587,6.021)}
\gppoint{gp mark 1}{(10.203,6.442)}
\gppoint{gp mark 1}{(10.777,6.846)}
\gppoint{gp mark 1}{(11.352,7.211)}
\gppoint{gp mark 1}{(11.865,7.584)}
\gppoint{gp mark 1}{(11.121,1.627)}
\gpcolor{color=gp lt color border}
\node[gp node left] at (7.167,1.319) {Retta interpolante};
\gpcolor{color=gp lt color 1}
\gpsetlinetype{gp lt plot 1}
\draw[gp path] (10.663,1.319)--(11.579,1.319);
\draw[gp path] (3.083,1.495)--(3.172,1.557)--(3.261,1.618)--(3.349,1.679)--(3.438,1.741)%
  --(3.527,1.802)--(3.615,1.864)--(3.704,1.925)--(3.793,1.987)--(3.882,2.048)--(3.970,2.109)%
  --(4.059,2.171)--(4.148,2.232)--(4.236,2.294)--(4.325,2.355)--(4.414,2.417)--(4.502,2.478)%
  --(4.591,2.539)--(4.680,2.601)--(4.769,2.662)--(4.857,2.724)--(4.946,2.785)--(5.035,2.847)%
  --(5.123,2.908)--(5.212,2.969)--(5.301,3.031)--(5.390,3.092)--(5.478,3.154)--(5.567,3.215)%
  --(5.656,3.277)--(5.744,3.338)--(5.833,3.400)--(5.922,3.461)--(6.010,3.522)--(6.099,3.584)%
  --(6.188,3.645)--(6.277,3.707)--(6.365,3.768)--(6.454,3.830)--(6.543,3.891)--(6.631,3.952)%
  --(6.720,4.014)--(6.809,4.075)--(6.897,4.137)--(6.986,4.198)--(7.075,4.260)--(7.164,4.321)%
  --(7.252,4.382)--(7.341,4.444)--(7.430,4.505)--(7.518,4.567)--(7.607,4.628)--(7.696,4.690)%
  --(7.785,4.751)--(7.873,4.812)--(7.962,4.874)--(8.051,4.935)--(8.139,4.997)--(8.228,5.058)%
  --(8.317,5.120)--(8.405,5.181)--(8.494,5.242)--(8.583,5.304)--(8.672,5.365)--(8.760,5.427)%
  --(8.849,5.488)--(8.938,5.550)--(9.026,5.611)--(9.115,5.673)--(9.204,5.734)--(9.293,5.795)%
  --(9.381,5.857)--(9.470,5.918)--(9.559,5.980)--(9.647,6.041)--(9.736,6.103)--(9.825,6.164)%
  --(9.913,6.225)--(10.002,6.287)--(10.091,6.348)--(10.180,6.410)--(10.268,6.471)--(10.357,6.533)%
  --(10.446,6.594)--(10.534,6.655)--(10.623,6.717)--(10.712,6.778)--(10.800,6.840)--(10.889,6.901)%
  --(10.978,6.963)--(11.067,7.024)--(11.155,7.085)--(11.244,7.147)--(11.333,7.208)--(11.421,7.270)%
  --(11.510,7.331)--(11.599,7.393)--(11.688,7.454)--(11.776,7.516)--(11.865,7.577);
\gpcolor{color=gp lt color border}
\gpsetlinetype{gp lt border}
\draw[gp path] (1.688,7.825)--(1.688,0.985)--(11.947,0.985)--(11.947,7.825)--cycle;
%% coordinates of the plot area
\gpdefrectangularnode{gp plot 1}{\pgfpoint{1.688cm}{0.985cm}}{\pgfpoint{11.947cm}{7.825cm}}
\end{tikzpicture}
%% gnuplot variables

\caption{Settima serie, accelerazione}
\label{fig:1}
\end{grafico}

\begin{grafico}
    \centering
\begin{tikzpicture}[gnuplot]
%% generated with GNUPLOT 4.6p0 (Lua 5.1; terminal rev. 99, script rev. 100)
%% Tue 15 Apr 2014 09:47:07 PM CEST
\path (0.000,0.000) rectangle (12.500,8.750);
\gpcolor{color=gp lt color border}
\gpsetlinetype{gp lt border}
\gpsetlinewidth{1.00}
\draw[gp path] (1.688,0.985)--(1.868,0.985);
\draw[gp path] (11.947,0.985)--(11.767,0.985);
\node[gp node right] at (1.504,0.985) { 0.15};
\draw[gp path] (1.688,1.669)--(1.868,1.669);
\draw[gp path] (11.947,1.669)--(11.767,1.669);
\node[gp node right] at (1.504,1.669) { 0.2};
\draw[gp path] (1.688,2.353)--(1.868,2.353);
\draw[gp path] (11.947,2.353)--(11.767,2.353);
\node[gp node right] at (1.504,2.353) { 0.25};
\draw[gp path] (1.688,3.037)--(1.868,3.037);
\draw[gp path] (11.947,3.037)--(11.767,3.037);
\node[gp node right] at (1.504,3.037) { 0.3};
\draw[gp path] (1.688,3.721)--(1.868,3.721);
\draw[gp path] (11.947,3.721)--(11.767,3.721);
\node[gp node right] at (1.504,3.721) { 0.35};
\draw[gp path] (1.688,4.405)--(1.868,4.405);
\draw[gp path] (11.947,4.405)--(11.767,4.405);
\node[gp node right] at (1.504,4.405) { 0.4};
\draw[gp path] (1.688,5.089)--(1.868,5.089);
\draw[gp path] (11.947,5.089)--(11.767,5.089);
\node[gp node right] at (1.504,5.089) { 0.45};
\draw[gp path] (1.688,5.773)--(1.868,5.773);
\draw[gp path] (11.947,5.773)--(11.767,5.773);
\node[gp node right] at (1.504,5.773) { 0.5};
\draw[gp path] (1.688,6.457)--(1.868,6.457);
\draw[gp path] (11.947,6.457)--(11.767,6.457);
\node[gp node right] at (1.504,6.457) { 0.55};
\draw[gp path] (1.688,7.141)--(1.868,7.141);
\draw[gp path] (11.947,7.141)--(11.767,7.141);
\node[gp node right] at (1.504,7.141) { 0.6};
\draw[gp path] (1.688,7.825)--(1.868,7.825);
\draw[gp path] (11.947,7.825)--(11.767,7.825);
\node[gp node right] at (1.504,7.825) { 0.65};
\draw[gp path] (1.688,0.985)--(1.688,1.165);
\draw[gp path] (1.688,7.825)--(1.688,7.645);
\node[gp node center] at (1.688,0.677) { 5};
\draw[gp path] (3.740,0.985)--(3.740,1.165);
\draw[gp path] (3.740,7.825)--(3.740,7.645);
\node[gp node center] at (3.740,0.677) { 10};
\draw[gp path] (5.792,0.985)--(5.792,1.165);
\draw[gp path] (5.792,7.825)--(5.792,7.645);
\node[gp node center] at (5.792,0.677) { 15};
\draw[gp path] (7.843,0.985)--(7.843,1.165);
\draw[gp path] (7.843,7.825)--(7.843,7.645);
\node[gp node center] at (7.843,0.677) { 20};
\draw[gp path] (9.895,0.985)--(9.895,1.165);
\draw[gp path] (9.895,7.825)--(9.895,7.645);
\node[gp node center] at (9.895,0.677) { 25};
\draw[gp path] (11.947,0.985)--(11.947,1.165);
\draw[gp path] (11.947,7.825)--(11.947,7.645);
\node[gp node center] at (11.947,0.677) { 30};
\draw[gp path] (1.688,7.825)--(1.688,0.985)--(11.947,0.985)--(11.947,7.825)--cycle;
\node[gp node center,rotate=-270] at (0.246,4.405) {Velocità angolare [rad/s]};
\node[gp node center] at (6.817,0.215) {Tempo [s]};
\node[gp node center] at (6.817,8.287) {Velocità angolari in accelerazione [rad/s]};
\node[gp node left] at (7.167,1.627) {Dati};
\gpcolor{color=gp lt color 0}
\gpsetlinetype{gp lt plot 0}
\draw[gp path] (10.663,1.627)--(11.579,1.627);
\draw[gp path] (10.663,1.717)--(10.663,1.537);
\draw[gp path] (11.579,1.717)--(11.579,1.537);
\draw[gp path] (2.940,1.569)--(2.940,1.635);
\draw[gp path] (2.850,1.569)--(3.030,1.569);
\draw[gp path] (2.850,1.635)--(3.030,1.635);
\draw[gp path] (4.396,2.605)--(4.396,2.669);
\draw[gp path] (4.306,2.605)--(4.486,2.605);
\draw[gp path] (4.306,2.669)--(4.486,2.669);
\draw[gp path] (5.525,3.393)--(5.525,3.456);
\draw[gp path] (5.435,3.393)--(5.615,3.393);
\draw[gp path] (5.435,3.456)--(5.615,3.456);
\draw[gp path] (6.469,4.063)--(6.469,4.125);
\draw[gp path] (6.379,4.063)--(6.559,4.063);
\draw[gp path] (6.379,4.125)--(6.559,4.125);
\draw[gp path] (7.330,4.632)--(7.330,4.693);
\draw[gp path] (7.240,4.632)--(7.420,4.632);
\draw[gp path] (7.240,4.693)--(7.420,4.693);
\draw[gp path] (8.069,5.175)--(8.069,5.236);
\draw[gp path] (7.979,5.175)--(8.159,5.175);
\draw[gp path] (7.979,5.236)--(8.159,5.236);
\draw[gp path] (8.726,5.692)--(8.726,5.753);
\draw[gp path] (8.636,5.692)--(8.816,5.692);
\draw[gp path] (8.636,5.753)--(8.816,5.753);
\draw[gp path] (9.157,6.310)--(9.157,6.373);
\draw[gp path] (9.067,6.310)--(9.247,6.310);
\draw[gp path] (9.067,6.373)--(9.247,6.373);
\draw[gp path] (10.039,6.531)--(10.039,6.591);
\draw[gp path] (9.949,6.531)--(10.129,6.531);
\draw[gp path] (9.949,6.591)--(10.129,6.591);
\draw[gp path] (10.613,6.935)--(10.613,6.995);
\draw[gp path] (10.523,6.935)--(10.703,6.935);
\draw[gp path] (10.523,6.995)--(10.703,6.995);
\draw[gp path] (11.188,7.299)--(11.188,7.358);
\draw[gp path] (11.098,7.299)--(11.278,7.299);
\draw[gp path] (11.098,7.358)--(11.278,7.358);
\draw[gp path] (11.701,7.672)--(11.701,7.732);
\draw[gp path] (11.611,7.672)--(11.791,7.672);
\draw[gp path] (11.611,7.732)--(11.791,7.732);
\gpsetpointsize{4.00}
\gppoint{gp mark 1}{(2.940,1.602)}
\gppoint{gp mark 1}{(4.396,2.637)}
\gppoint{gp mark 1}{(5.525,3.425)}
\gppoint{gp mark 1}{(6.469,4.094)}
\gppoint{gp mark 1}{(7.330,4.662)}
\gppoint{gp mark 1}{(8.069,5.206)}
\gppoint{gp mark 1}{(8.726,5.723)}
\gppoint{gp mark 1}{(9.157,6.341)}
\gppoint{gp mark 1}{(10.039,6.561)}
\gppoint{gp mark 1}{(10.613,6.965)}
\gppoint{gp mark 1}{(11.188,7.328)}
\gppoint{gp mark 1}{(11.701,7.702)}
\gppoint{gp mark 1}{(11.121,1.627)}
\gpcolor{color=gp lt color border}
\node[gp node left] at (7.167,1.319) {Retta interpolante};
\gpcolor{color=gp lt color 1}
\gpsetlinetype{gp lt plot 1}
\draw[gp path] (10.663,1.319)--(11.579,1.319);
\draw[gp path] (2.940,1.635)--(3.028,1.697)--(3.117,1.759)--(3.205,1.821)--(3.294,1.883)%
  --(3.382,1.945)--(3.471,2.007)--(3.559,2.069)--(3.648,2.131)--(3.736,2.192)--(3.825,2.254)%
  --(3.913,2.316)--(4.002,2.378)--(4.090,2.440)--(4.179,2.502)--(4.267,2.564)--(4.356,2.626)%
  --(4.444,2.688)--(4.533,2.750)--(4.621,2.812)--(4.710,2.874)--(4.798,2.935)--(4.887,2.997)%
  --(4.975,3.059)--(5.064,3.121)--(5.152,3.183)--(5.241,3.245)--(5.329,3.307)--(5.418,3.369)%
  --(5.506,3.431)--(5.595,3.493)--(5.683,3.555)--(5.771,3.616)--(5.860,3.678)--(5.948,3.740)%
  --(6.037,3.802)--(6.125,3.864)--(6.214,3.926)--(6.302,3.988)--(6.391,4.050)--(6.479,4.112)%
  --(6.568,4.174)--(6.656,4.236)--(6.745,4.298)--(6.833,4.359)--(6.922,4.421)--(7.010,4.483)%
  --(7.099,4.545)--(7.187,4.607)--(7.276,4.669)--(7.364,4.731)--(7.453,4.793)--(7.541,4.855)%
  --(7.630,4.917)--(7.718,4.979)--(7.807,5.041)--(7.895,5.102)--(7.984,5.164)--(8.072,5.226)%
  --(8.161,5.288)--(8.249,5.350)--(8.338,5.412)--(8.426,5.474)--(8.515,5.536)--(8.603,5.598)%
  --(8.692,5.660)--(8.780,5.722)--(8.869,5.783)--(8.957,5.845)--(9.046,5.907)--(9.134,5.969)%
  --(9.223,6.031)--(9.311,6.093)--(9.400,6.155)--(9.488,6.217)--(9.577,6.279)--(9.665,6.341)%
  --(9.754,6.403)--(9.842,6.465)--(9.931,6.526)--(10.019,6.588)--(10.108,6.650)--(10.196,6.712)%
  --(10.285,6.774)--(10.373,6.836)--(10.462,6.898)--(10.550,6.960)--(10.639,7.022)--(10.727,7.084)%
  --(10.816,7.146)--(10.904,7.207)--(10.993,7.269)--(11.081,7.331)--(11.170,7.393)--(11.258,7.455)%
  --(11.347,7.517)--(11.435,7.579)--(11.524,7.641)--(11.612,7.703)--(11.701,7.765);
\gpcolor{color=gp lt color border}
\gpsetlinetype{gp lt border}
\draw[gp path] (1.688,7.825)--(1.688,0.985)--(11.947,0.985)--(11.947,7.825)--cycle;
%% coordinates of the plot area
\gpdefrectangularnode{gp plot 1}{\pgfpoint{1.688cm}{0.985cm}}{\pgfpoint{11.947cm}{7.825cm}}
\end{tikzpicture}
%% gnuplot variables

\caption{Ottava serie, accelerazione}
\label{fig:1}
\end{grafico}

\begin{grafico}
    \centering
\begin{tikzpicture}[gnuplot]
%% generated with GNUPLOT 4.6p0 (Lua 5.1; terminal rev. 99, script rev. 100)
%% Mon 14 Apr 2014 11:09:58 PM CEST
\path (0.000,0.000) rectangle (12.500,8.750);
\gpcolor{color=gp lt color border}
\gpsetlinetype{gp lt border}
\gpsetlinewidth{1.00}
\draw[gp path] (1.320,0.985)--(1.500,0.985);
\draw[gp path] (11.947,0.985)--(11.767,0.985);
\node[gp node right] at (1.136,0.985) { 5};
\draw[gp path] (1.320,1.669)--(1.500,1.669);
\draw[gp path] (11.947,1.669)--(11.767,1.669);
\node[gp node right] at (1.136,1.669) { 10};
\draw[gp path] (1.320,2.353)--(1.500,2.353);
\draw[gp path] (11.947,2.353)--(11.767,2.353);
\node[gp node right] at (1.136,2.353) { 15};
\draw[gp path] (1.320,3.037)--(1.500,3.037);
\draw[gp path] (11.947,3.037)--(11.767,3.037);
\node[gp node right] at (1.136,3.037) { 20};
\draw[gp path] (1.320,3.721)--(1.500,3.721);
\draw[gp path] (11.947,3.721)--(11.767,3.721);
\node[gp node right] at (1.136,3.721) { 25};
\draw[gp path] (1.320,4.405)--(1.500,4.405);
\draw[gp path] (11.947,4.405)--(11.767,4.405);
\node[gp node right] at (1.136,4.405) { 30};
\draw[gp path] (1.320,5.089)--(1.500,5.089);
\draw[gp path] (11.947,5.089)--(11.767,5.089);
\node[gp node right] at (1.136,5.089) { 35};
\draw[gp path] (1.320,5.773)--(1.500,5.773);
\draw[gp path] (11.947,5.773)--(11.767,5.773);
\node[gp node right] at (1.136,5.773) { 40};
\draw[gp path] (1.320,6.457)--(1.500,6.457);
\draw[gp path] (11.947,6.457)--(11.767,6.457);
\node[gp node right] at (1.136,6.457) { 45};
\draw[gp path] (1.320,7.141)--(1.500,7.141);
\draw[gp path] (11.947,7.141)--(11.767,7.141);
\node[gp node right] at (1.136,7.141) { 50};
\draw[gp path] (1.320,7.825)--(1.500,7.825);
\draw[gp path] (11.947,7.825)--(11.767,7.825);
\node[gp node right] at (1.136,7.825) { 55};
\draw[gp path] (1.320,0.985)--(1.320,1.165);
\draw[gp path] (1.320,7.825)--(1.320,7.645);
\node[gp node center] at (1.320,0.677) {-10};
\draw[gp path] (3.977,0.985)--(3.977,1.165);
\draw[gp path] (3.977,7.825)--(3.977,7.645);
\node[gp node center] at (3.977,0.677) {-5};
\draw[gp path] (6.634,0.985)--(6.634,1.165);
\draw[gp path] (6.634,7.825)--(6.634,7.645);
\node[gp node center] at (6.634,0.677) { 0};
\draw[gp path] (9.290,0.985)--(9.290,1.165);
\draw[gp path] (9.290,7.825)--(9.290,7.645);
\node[gp node center] at (9.290,0.677) { 5};
\draw[gp path] (11.947,0.985)--(11.947,1.165);
\draw[gp path] (11.947,7.825)--(11.947,7.645);
\node[gp node center] at (11.947,0.677) { 10};
\draw[gp path] (1.320,7.825)--(1.320,0.985)--(11.947,0.985)--(11.947,7.825)--cycle;
\node[gp node center,rotate=-270] at (0.246,4.405) {Velocità angolare [rad/s]};
\node[gp node center] at (6.633,0.215) {Tempo [s]};
\node[gp node center] at (6.633,8.287) {Velocità angolare, decelerazione [rad/s]};
\node[gp node left] at (7.167,1.627) {Retta interpolante};
\gpcolor{color=gp lt color 1}
\gpsetlinetype{gp lt plot 1}
\draw[gp path] (10.663,1.627)--(11.579,1.627);
\draw[gp path] (1.320,7.708)--(1.427,7.645)--(1.535,7.583)--(1.642,7.520)--(1.749,7.458)%
  --(1.857,7.395)--(1.964,7.333)--(2.071,7.270)--(2.179,7.208)--(2.286,7.145)--(2.393,7.082)%
  --(2.501,7.020)--(2.608,6.957)--(2.715,6.895)--(2.823,6.832)--(2.930,6.770)--(3.037,6.707)%
  --(3.145,6.644)--(3.252,6.582)--(3.360,6.519)--(3.467,6.457)--(3.574,6.394)--(3.682,6.332)%
  --(3.789,6.269)--(3.896,6.207)--(4.004,6.144)--(4.111,6.081)--(4.218,6.019)--(4.326,5.956)%
  --(4.433,5.894)--(4.540,5.831)--(4.648,5.769)--(4.755,5.706)--(4.862,5.644)--(4.970,5.581)%
  --(5.077,5.518)--(5.184,5.456)--(5.292,5.393)--(5.399,5.331)--(5.506,5.268)--(5.614,5.206)%
  --(5.721,5.143)--(5.828,5.081)--(5.936,5.018)--(6.043,4.955)--(6.150,4.893)--(6.258,4.830)%
  --(6.365,4.768)--(6.472,4.705)--(6.580,4.643)--(6.687,4.580)--(6.795,4.517)--(6.902,4.455)%
  --(7.009,4.392)--(7.117,4.330)--(7.224,4.267)--(7.331,4.205)--(7.439,4.142)--(7.546,4.080)%
  --(7.653,4.017)--(7.761,3.954)--(7.868,3.892)--(7.975,3.829)--(8.083,3.767)--(8.190,3.704)%
  --(8.297,3.642)--(8.405,3.579)--(8.512,3.517)--(8.619,3.454)--(8.727,3.391)--(8.834,3.329)%
  --(8.941,3.266)--(9.049,3.204)--(9.156,3.141)--(9.263,3.079)--(9.371,3.016)--(9.478,2.954)%
  --(9.585,2.891)--(9.693,2.828)--(9.800,2.766)--(9.907,2.703)--(10.015,2.641)--(10.122,2.578)%
  --(10.230,2.516)--(10.337,2.453)--(10.444,2.391)--(10.552,2.328)--(10.659,2.265)--(10.766,2.203)%
  --(10.874,2.140)--(10.981,2.078)--(11.088,2.015)--(11.196,1.953)--(11.303,1.890)--(11.410,1.827)%
  --(11.518,1.765)--(11.625,1.702)--(11.732,1.640)--(11.840,1.577)--(11.947,1.515);
\gpcolor{color=gp lt color border}
\gpsetlinetype{gp lt border}
\draw[gp path] (1.320,7.825)--(1.320,0.985)--(11.947,0.985)--(11.947,7.825)--cycle;
%% coordinates of the plot area
\gpdefrectangularnode{gp plot 1}{\pgfpoint{1.320cm}{0.985cm}}{\pgfpoint{11.947cm}{7.825cm}}
\end{tikzpicture}
%% gnuplot variables

\caption{Nona serie, accelerazione}
\label{fig:1}
\end{grafico}

\begin{grafico}
    \centering
\begin{tikzpicture}[gnuplot]
%% generated with GNUPLOT 4.6p0 (Lua 5.1; terminal rev. 99, script rev. 100)
%% Tue 15 Apr 2014 09:47:07 PM CEST
\path (0.000,0.000) rectangle (12.500,8.750);
\gpcolor{color=gp lt color border}
\gpsetlinetype{gp lt border}
\gpsetlinewidth{1.00}
\draw[gp path] (1.688,0.985)--(1.868,0.985);
\draw[gp path] (11.947,0.985)--(11.767,0.985);
\node[gp node right] at (1.504,0.985) { 0.15};
\draw[gp path] (1.688,1.669)--(1.868,1.669);
\draw[gp path] (11.947,1.669)--(11.767,1.669);
\node[gp node right] at (1.504,1.669) { 0.2};
\draw[gp path] (1.688,2.353)--(1.868,2.353);
\draw[gp path] (11.947,2.353)--(11.767,2.353);
\node[gp node right] at (1.504,2.353) { 0.25};
\draw[gp path] (1.688,3.037)--(1.868,3.037);
\draw[gp path] (11.947,3.037)--(11.767,3.037);
\node[gp node right] at (1.504,3.037) { 0.3};
\draw[gp path] (1.688,3.721)--(1.868,3.721);
\draw[gp path] (11.947,3.721)--(11.767,3.721);
\node[gp node right] at (1.504,3.721) { 0.35};
\draw[gp path] (1.688,4.405)--(1.868,4.405);
\draw[gp path] (11.947,4.405)--(11.767,4.405);
\node[gp node right] at (1.504,4.405) { 0.4};
\draw[gp path] (1.688,5.089)--(1.868,5.089);
\draw[gp path] (11.947,5.089)--(11.767,5.089);
\node[gp node right] at (1.504,5.089) { 0.45};
\draw[gp path] (1.688,5.773)--(1.868,5.773);
\draw[gp path] (11.947,5.773)--(11.767,5.773);
\node[gp node right] at (1.504,5.773) { 0.5};
\draw[gp path] (1.688,6.457)--(1.868,6.457);
\draw[gp path] (11.947,6.457)--(11.767,6.457);
\node[gp node right] at (1.504,6.457) { 0.55};
\draw[gp path] (1.688,7.141)--(1.868,7.141);
\draw[gp path] (11.947,7.141)--(11.767,7.141);
\node[gp node right] at (1.504,7.141) { 0.6};
\draw[gp path] (1.688,7.825)--(1.868,7.825);
\draw[gp path] (11.947,7.825)--(11.767,7.825);
\node[gp node right] at (1.504,7.825) { 0.65};
\draw[gp path] (1.688,0.985)--(1.688,1.165);
\draw[gp path] (1.688,7.825)--(1.688,7.645);
\node[gp node center] at (1.688,0.677) { 5};
\draw[gp path] (3.740,0.985)--(3.740,1.165);
\draw[gp path] (3.740,7.825)--(3.740,7.645);
\node[gp node center] at (3.740,0.677) { 10};
\draw[gp path] (5.792,0.985)--(5.792,1.165);
\draw[gp path] (5.792,7.825)--(5.792,7.645);
\node[gp node center] at (5.792,0.677) { 15};
\draw[gp path] (7.843,0.985)--(7.843,1.165);
\draw[gp path] (7.843,7.825)--(7.843,7.645);
\node[gp node center] at (7.843,0.677) { 20};
\draw[gp path] (9.895,0.985)--(9.895,1.165);
\draw[gp path] (9.895,7.825)--(9.895,7.645);
\node[gp node center] at (9.895,0.677) { 25};
\draw[gp path] (11.947,0.985)--(11.947,1.165);
\draw[gp path] (11.947,7.825)--(11.947,7.645);
\node[gp node center] at (11.947,0.677) { 30};
\draw[gp path] (1.688,7.825)--(1.688,0.985)--(11.947,0.985)--(11.947,7.825)--cycle;
\node[gp node center,rotate=-270] at (0.246,4.405) {Velocità angolare [rad/s]};
\node[gp node center] at (6.817,0.215) {Tempo [s]};
\node[gp node center] at (6.817,8.287) {Velocità angolari in accelerazione [rad/s]};
\node[gp node left] at (7.167,1.627) {Dati};
\gpcolor{color=gp lt color 0}
\gpsetlinetype{gp lt plot 0}
\draw[gp path] (10.663,1.627)--(11.579,1.627);
\draw[gp path] (10.663,1.717)--(10.663,1.537);
\draw[gp path] (11.579,1.717)--(11.579,1.537);
\draw[gp path] (2.981,1.537)--(2.981,1.601);
\draw[gp path] (2.891,1.537)--(3.071,1.537);
\draw[gp path] (2.891,1.601)--(3.071,1.601);
\draw[gp path] (4.478,2.544)--(4.478,2.605);
\draw[gp path] (4.388,2.544)--(4.568,2.544);
\draw[gp path] (4.388,2.605)--(4.568,2.605);
\draw[gp path] (5.627,3.317)--(5.627,3.378);
\draw[gp path] (5.537,3.317)--(5.717,3.317);
\draw[gp path] (5.537,3.378)--(5.717,3.378);
\draw[gp path] (6.592,3.973)--(6.592,4.033);
\draw[gp path] (6.502,3.973)--(6.682,3.973);
\draw[gp path] (6.502,4.033)--(6.682,4.033);
\draw[gp path] (7.433,4.557)--(7.433,4.617);
\draw[gp path] (7.343,4.557)--(7.523,4.557);
\draw[gp path] (7.343,4.617)--(7.523,4.617);
\draw[gp path] (8.192,5.086)--(8.192,5.145);
\draw[gp path] (8.102,5.086)--(8.282,5.086);
\draw[gp path] (8.102,5.145)--(8.282,5.145);
\draw[gp path] (8.890,5.573)--(8.890,5.632);
\draw[gp path] (8.800,5.573)--(8.980,5.573);
\draw[gp path] (8.800,5.632)--(8.980,5.632);
\draw[gp path] (9.546,6.021)--(9.546,6.079);
\draw[gp path] (9.456,6.021)--(9.636,6.021);
\draw[gp path] (9.456,6.079)--(9.636,6.079);
\draw[gp path] (10.162,6.442)--(10.162,6.501);
\draw[gp path] (10.072,6.442)--(10.252,6.442);
\draw[gp path] (10.072,6.501)--(10.252,6.501);
\draw[gp path] (10.757,6.832)--(10.757,6.890);
\draw[gp path] (10.667,6.832)--(10.847,6.832);
\draw[gp path] (10.667,6.890)--(10.847,6.890);
\draw[gp path] (11.311,7.211)--(11.311,7.269);
\draw[gp path] (11.221,7.211)--(11.401,7.211);
\draw[gp path] (11.221,7.269)--(11.401,7.269);
\draw[gp path] (11.824,7.584)--(11.824,7.643);
\draw[gp path] (11.734,7.584)--(11.914,7.584);
\draw[gp path] (11.734,7.643)--(11.914,7.643);
\gpsetpointsize{4.00}
\gppoint{gp mark 1}{(2.981,1.569)}
\gppoint{gp mark 1}{(4.478,2.574)}
\gppoint{gp mark 1}{(5.627,3.348)}
\gppoint{gp mark 1}{(6.592,4.003)}
\gppoint{gp mark 1}{(7.433,4.587)}
\gppoint{gp mark 1}{(8.192,5.116)}
\gppoint{gp mark 1}{(8.890,5.602)}
\gppoint{gp mark 1}{(9.546,6.050)}
\gppoint{gp mark 1}{(10.162,6.471)}
\gppoint{gp mark 1}{(10.757,6.861)}
\gppoint{gp mark 1}{(11.311,7.240)}
\gppoint{gp mark 1}{(11.824,7.614)}
\gppoint{gp mark 1}{(11.121,1.627)}
\gpcolor{color=gp lt color border}
\node[gp node left] at (7.167,1.319) {Retta interpolante};
\gpcolor{color=gp lt color 1}
\gpsetlinetype{gp lt plot 1}
\draw[gp path] (10.663,1.319)--(11.579,1.319);
\draw[gp path] (2.981,1.549)--(3.070,1.610)--(3.159,1.671)--(3.249,1.732)--(3.338,1.793)%
  --(3.427,1.854)--(3.517,1.916)--(3.606,1.977)--(3.695,2.038)--(3.785,2.099)--(3.874,2.160)%
  --(3.963,2.221)--(4.053,2.282)--(4.142,2.344)--(4.231,2.405)--(4.321,2.466)--(4.410,2.527)%
  --(4.499,2.588)--(4.588,2.649)--(4.678,2.710)--(4.767,2.771)--(4.856,2.833)--(4.946,2.894)%
  --(5.035,2.955)--(5.124,3.016)--(5.214,3.077)--(5.303,3.138)--(5.392,3.199)--(5.482,3.260)%
  --(5.571,3.322)--(5.660,3.383)--(5.750,3.444)--(5.839,3.505)--(5.928,3.566)--(6.018,3.627)%
  --(6.107,3.688)--(6.196,3.749)--(6.286,3.811)--(6.375,3.872)--(6.464,3.933)--(6.554,3.994)%
  --(6.643,4.055)--(6.732,4.116)--(6.822,4.177)--(6.911,4.238)--(7.000,4.300)--(7.090,4.361)%
  --(7.179,4.422)--(7.268,4.483)--(7.358,4.544)--(7.447,4.605)--(7.536,4.666)--(7.626,4.728)%
  --(7.715,4.789)--(7.804,4.850)--(7.894,4.911)--(7.983,4.972)--(8.072,5.033)--(8.162,5.094)%
  --(8.251,5.155)--(8.340,5.217)--(8.430,5.278)--(8.519,5.339)--(8.608,5.400)--(8.697,5.461)%
  --(8.787,5.522)--(8.876,5.583)--(8.965,5.644)--(9.055,5.706)--(9.144,5.767)--(9.233,5.828)%
  --(9.323,5.889)--(9.412,5.950)--(9.501,6.011)--(9.591,6.072)--(9.680,6.133)--(9.769,6.195)%
  --(9.859,6.256)--(9.948,6.317)--(10.037,6.378)--(10.127,6.439)--(10.216,6.500)--(10.305,6.561)%
  --(10.395,6.622)--(10.484,6.684)--(10.573,6.745)--(10.663,6.806)--(10.752,6.867)--(10.841,6.928)%
  --(10.931,6.989)--(11.020,7.050)--(11.109,7.112)--(11.199,7.173)--(11.288,7.234)--(11.377,7.295)%
  --(11.467,7.356)--(11.556,7.417)--(11.645,7.478)--(11.735,7.539)--(11.824,7.601);
\gpcolor{color=gp lt color border}
\gpsetlinetype{gp lt border}
\draw[gp path] (1.688,7.825)--(1.688,0.985)--(11.947,0.985)--(11.947,7.825)--cycle;
%% coordinates of the plot area
\gpdefrectangularnode{gp plot 1}{\pgfpoint{1.688cm}{0.985cm}}{\pgfpoint{11.947cm}{7.825cm}}
\end{tikzpicture}
%% gnuplot variables

\caption{Decima serie, accelerazione}
\label{fig:1}
\end{grafico}

Qui invece i dati in decelerazione.
L'errore in decelerazione è abbastanza grande, probabilmente perché l'intervallo di tempo era troppo limitato per osservare un effetto significativo (infatti le $\beta$ sono molto piccole rispetto alle $\alpha$)

\begin{grafico}
    \centering
\begin{tikzpicture}[gnuplot]
%% generated with GNUPLOT 4.6p0 (Lua 5.1; terminal rev. 99, script rev. 100)
%% Tue 15 Apr 2014 09:47:08 PM CEST
\path (0.000,0.000) rectangle (12.500,8.750);
\gpcolor{color=gp lt color border}
\gpsetlinetype{gp lt border}
\gpsetlinewidth{1.00}
\draw[gp path] (1.688,0.985)--(1.868,0.985);
\draw[gp path] (11.947,0.985)--(11.767,0.985);
\node[gp node right] at (1.504,0.985) { 1.15};
\draw[gp path] (1.688,2.125)--(1.868,2.125);
\draw[gp path] (11.947,2.125)--(11.767,2.125);
\node[gp node right] at (1.504,2.125) { 1.2};
\draw[gp path] (1.688,3.265)--(1.868,3.265);
\draw[gp path] (11.947,3.265)--(11.767,3.265);
\node[gp node right] at (1.504,3.265) { 1.25};
\draw[gp path] (1.688,4.405)--(1.868,4.405);
\draw[gp path] (11.947,4.405)--(11.767,4.405);
\node[gp node right] at (1.504,4.405) { 1.3};
\draw[gp path] (1.688,5.545)--(1.868,5.545);
\draw[gp path] (11.947,5.545)--(11.767,5.545);
\node[gp node right] at (1.504,5.545) { 1.35};
\draw[gp path] (1.688,6.685)--(1.868,6.685);
\draw[gp path] (11.947,6.685)--(11.767,6.685);
\node[gp node right] at (1.504,6.685) { 1.4};
\draw[gp path] (1.688,7.825)--(1.868,7.825);
\draw[gp path] (11.947,7.825)--(11.767,7.825);
\node[gp node right] at (1.504,7.825) { 1.45};
\draw[gp path] (1.688,0.985)--(1.688,1.165);
\draw[gp path] (1.688,7.825)--(1.688,7.645);
\node[gp node center] at (1.688,0.677) { 0};
\draw[gp path] (3.398,0.985)--(3.398,1.165);
\draw[gp path] (3.398,7.825)--(3.398,7.645);
\node[gp node center] at (3.398,0.677) { 5};
\draw[gp path] (5.108,0.985)--(5.108,1.165);
\draw[gp path] (5.108,7.825)--(5.108,7.645);
\node[gp node center] at (5.108,0.677) { 10};
\draw[gp path] (6.818,0.985)--(6.818,1.165);
\draw[gp path] (6.818,7.825)--(6.818,7.645);
\node[gp node center] at (6.818,0.677) { 15};
\draw[gp path] (8.527,0.985)--(8.527,1.165);
\draw[gp path] (8.527,7.825)--(8.527,7.645);
\node[gp node center] at (8.527,0.677) { 20};
\draw[gp path] (10.237,0.985)--(10.237,1.165);
\draw[gp path] (10.237,7.825)--(10.237,7.645);
\node[gp node center] at (10.237,0.677) { 25};
\draw[gp path] (11.947,0.985)--(11.947,1.165);
\draw[gp path] (11.947,7.825)--(11.947,7.645);
\node[gp node center] at (11.947,0.677) { 30};
\draw[gp path] (1.688,7.825)--(1.688,0.985)--(11.947,0.985)--(11.947,7.825)--cycle;
\node[gp node center,rotate=-270] at (0.246,4.405) {Velocità angolare [rad/s]};
\node[gp node center] at (6.817,0.215) {Tempo [s]};
\node[gp node center] at (6.817,8.287) {Velocità angolari in decelerazione [rad/s]};
\node[gp node left] at (7.167,1.627) {Dati};
\gpcolor{color=gp lt color 0}
\gpsetlinetype{gp lt plot 0}
\draw[gp path] (10.663,1.627)--(11.579,1.627);
\draw[gp path] (10.663,1.717)--(10.663,1.537);
\draw[gp path] (11.579,1.717)--(11.579,1.537);
\draw[gp path] (2.098,2.119)--(2.098,7.090);
\draw[gp path] (2.008,2.119)--(2.188,2.119);
\draw[gp path] (2.008,7.090)--(2.188,7.090);
\draw[gp path] (2.560,1.748)--(2.560,3.950);
\draw[gp path] (2.470,1.748)--(2.650,1.748);
\draw[gp path] (2.470,3.950)--(2.650,3.950);
\draw[gp path] (3.005,1.942)--(3.005,3.391);
\draw[gp path] (2.915,1.942)--(3.095,1.942);
\draw[gp path] (2.915,3.391)--(3.095,3.391);
\draw[gp path] (3.432,2.299)--(3.432,3.400);
\draw[gp path] (3.342,2.299)--(3.522,2.299);
\draw[gp path] (3.342,3.400)--(3.522,3.400);
\draw[gp path] (3.894,2.092)--(3.894,2.953);
\draw[gp path] (3.804,2.092)--(3.984,2.092);
\draw[gp path] (3.804,2.953)--(3.984,2.953);
\draw[gp path] (4.338,2.129)--(4.338,2.844);
\draw[gp path] (4.248,2.129)--(4.428,2.129);
\draw[gp path] (4.248,2.844)--(4.428,2.844);
\draw[gp path] (4.783,2.155)--(4.783,2.767);
\draw[gp path] (4.693,2.155)--(4.873,2.155);
\draw[gp path] (4.693,2.767)--(4.873,2.767);
\draw[gp path] (5.244,2.044)--(5.244,2.574);
\draw[gp path] (5.154,2.044)--(5.334,2.044);
\draw[gp path] (5.154,2.574)--(5.334,2.574);
\draw[gp path] (5.689,2.074)--(5.689,2.545);
\draw[gp path] (5.599,2.074)--(5.779,2.074);
\draw[gp path] (5.599,2.545)--(5.779,2.545);
\draw[gp path] (6.134,2.097)--(6.134,2.521);
\draw[gp path] (6.044,2.097)--(6.224,2.097);
\draw[gp path] (6.044,2.521)--(6.224,2.521);
\draw[gp path] (6.595,2.022)--(6.595,2.404);
\draw[gp path] (6.505,2.022)--(6.685,2.022);
\draw[gp path] (6.505,2.404)--(6.685,2.404);
\draw[gp path] (7.074,1.874)--(7.074,2.220);
\draw[gp path] (6.984,1.874)--(7.164,1.874);
\draw[gp path] (6.984,2.220)--(7.164,2.220);
\draw[gp path] (7.553,1.749)--(7.553,2.066);
\draw[gp path] (7.463,1.749)--(7.643,1.749);
\draw[gp path] (7.463,2.066)--(7.643,2.066);
\draw[gp path] (7.997,1.789)--(7.997,2.083);
\draw[gp path] (7.907,1.789)--(8.087,1.789);
\draw[gp path] (7.907,2.083)--(8.087,2.083);
\draw[gp path] (8.476,1.687)--(8.476,1.960);
\draw[gp path] (8.386,1.687)--(8.566,1.687);
\draw[gp path] (8.386,1.960)--(8.566,1.960);
\draw[gp path] (8.938,1.662)--(8.938,1.917);
\draw[gp path] (8.848,1.662)--(9.028,1.662);
\draw[gp path] (8.848,1.917)--(9.028,1.917);
\draw[gp path] (9.434,1.522)--(9.434,1.759);
\draw[gp path] (9.344,1.522)--(9.524,1.522);
\draw[gp path] (9.344,1.759)--(9.524,1.759);
\draw[gp path] (9.912,1.453)--(9.912,1.676);
\draw[gp path] (9.822,1.453)--(10.002,1.453);
\draw[gp path] (9.822,1.676)--(10.002,1.676);
\draw[gp path] (10.391,1.392)--(10.391,1.602);
\draw[gp path] (10.301,1.392)--(10.481,1.392);
\draw[gp path] (10.301,1.602)--(10.481,1.602);
\draw[gp path] (10.870,1.338)--(10.870,1.536);
\draw[gp path] (10.780,1.338)--(10.960,1.338);
\draw[gp path] (10.780,1.536)--(10.960,1.536);
\draw[gp path] (11.366,1.242)--(11.366,1.430);
\draw[gp path] (11.276,1.242)--(11.456,1.242);
\draw[gp path] (11.276,1.430)--(11.456,1.430);
\draw[gp path] (11.844,1.200)--(11.844,1.378);
\draw[gp path] (11.754,1.200)--(11.934,1.200);
\draw[gp path] (11.754,1.378)--(11.934,1.378);
\gpsetpointsize{4.00}
\gppoint{gp mark 1}{(2.098,4.604)}
\gppoint{gp mark 1}{(2.560,2.849)}
\gppoint{gp mark 1}{(3.005,2.667)}
\gppoint{gp mark 1}{(3.432,2.849)}
\gppoint{gp mark 1}{(3.894,2.523)}
\gppoint{gp mark 1}{(4.338,2.487)}
\gppoint{gp mark 1}{(4.783,2.461)}
\gppoint{gp mark 1}{(5.244,2.309)}
\gppoint{gp mark 1}{(5.689,2.309)}
\gppoint{gp mark 1}{(6.134,2.309)}
\gppoint{gp mark 1}{(6.595,2.213)}
\gppoint{gp mark 1}{(7.074,2.047)}
\gppoint{gp mark 1}{(7.553,1.908)}
\gppoint{gp mark 1}{(7.997,1.936)}
\gppoint{gp mark 1}{(8.476,1.823)}
\gppoint{gp mark 1}{(8.938,1.789)}
\gppoint{gp mark 1}{(9.434,1.640)}
\gppoint{gp mark 1}{(9.912,1.565)}
\gppoint{gp mark 1}{(10.391,1.497)}
\gppoint{gp mark 1}{(10.870,1.437)}
\gppoint{gp mark 1}{(11.366,1.336)}
\gppoint{gp mark 1}{(11.844,1.289)}
\gppoint{gp mark 1}{(11.121,1.627)}
\gpcolor{color=gp lt color border}
\node[gp node left] at (7.167,1.319) {Retta interpolante};
\gpcolor{color=gp lt color 1}
\gpsetlinetype{gp lt plot 1}
\draw[gp path] (10.663,1.319)--(11.579,1.319);
\draw[gp path] (2.098,2.902)--(2.197,2.885)--(2.295,2.869)--(2.394,2.852)--(2.492,2.836)%
  --(2.591,2.819)--(2.689,2.803)--(2.787,2.786)--(2.886,2.770)--(2.984,2.753)--(3.083,2.736)%
  --(3.181,2.720)--(3.280,2.703)--(3.378,2.687)--(3.477,2.670)--(3.575,2.654)--(3.673,2.637)%
  --(3.772,2.621)--(3.870,2.604)--(3.969,2.588)--(4.067,2.571)--(4.166,2.555)--(4.264,2.538)%
  --(4.363,2.522)--(4.461,2.505)--(4.559,2.488)--(4.658,2.472)--(4.756,2.455)--(4.855,2.439)%
  --(4.953,2.422)--(5.052,2.406)--(5.150,2.389)--(5.249,2.373)--(5.347,2.356)--(5.445,2.340)%
  --(5.544,2.323)--(5.642,2.307)--(5.741,2.290)--(5.839,2.273)--(5.938,2.257)--(6.036,2.240)%
  --(6.135,2.224)--(6.233,2.207)--(6.331,2.191)--(6.430,2.174)--(6.528,2.158)--(6.627,2.141)%
  --(6.725,2.125)--(6.824,2.108)--(6.922,2.092)--(7.021,2.075)--(7.119,2.059)--(7.217,2.042)%
  --(7.316,2.025)--(7.414,2.009)--(7.513,1.992)--(7.611,1.976)--(7.710,1.959)--(7.808,1.943)%
  --(7.907,1.926)--(8.005,1.910)--(8.104,1.893)--(8.202,1.877)--(8.300,1.860)--(8.399,1.844)%
  --(8.497,1.827)--(8.596,1.811)--(8.694,1.794)--(8.793,1.777)--(8.891,1.761)--(8.990,1.744)%
  --(9.088,1.728)--(9.186,1.711)--(9.285,1.695)--(9.383,1.678)--(9.482,1.662)--(9.580,1.645)%
  --(9.679,1.629)--(9.777,1.612)--(9.876,1.596)--(9.974,1.579)--(10.072,1.563)--(10.171,1.546)%
  --(10.269,1.529)--(10.368,1.513)--(10.466,1.496)--(10.565,1.480)--(10.663,1.463)--(10.762,1.447)%
  --(10.860,1.430)--(10.958,1.414)--(11.057,1.397)--(11.155,1.381)--(11.254,1.364)--(11.352,1.348)%
  --(11.451,1.331)--(11.549,1.315)--(11.648,1.298)--(11.746,1.281)--(11.844,1.265);
\gpcolor{color=gp lt color border}
\gpsetlinetype{gp lt border}
\draw[gp path] (1.688,7.825)--(1.688,0.985)--(11.947,0.985)--(11.947,7.825)--cycle;
%% coordinates of the plot area
\gpdefrectangularnode{gp plot 1}{\pgfpoint{1.688cm}{0.985cm}}{\pgfpoint{11.947cm}{7.825cm}}
\end{tikzpicture}
%% gnuplot variables

\caption{Prima serie, decelerazione}
\label{fig:1}
\end{grafico}

\begin{grafico}
    \centering
\begin{tikzpicture}[gnuplot]
%% generated with GNUPLOT 4.6p0 (Lua 5.1; terminal rev. 99, script rev. 100)
%% Tue 15 Apr 2014 06:32:33 PM CEST
\path (0.000,0.000) rectangle (12.500,8.750);
\gpcolor{color=gp lt color border}
\gpsetlinetype{gp lt border}
\gpsetlinewidth{1.00}
\draw[gp path] (1.688,0.985)--(1.868,0.985);
\draw[gp path] (11.947,0.985)--(11.767,0.985);
\node[gp node right] at (1.504,0.985) { 1.04};
\draw[gp path] (1.688,1.669)--(1.868,1.669);
\draw[gp path] (11.947,1.669)--(11.767,1.669);
\node[gp node right] at (1.504,1.669) { 1.06};
\draw[gp path] (1.688,2.353)--(1.868,2.353);
\draw[gp path] (11.947,2.353)--(11.767,2.353);
\node[gp node right] at (1.504,2.353) { 1.08};
\draw[gp path] (1.688,3.037)--(1.868,3.037);
\draw[gp path] (11.947,3.037)--(11.767,3.037);
\node[gp node right] at (1.504,3.037) { 1.1};
\draw[gp path] (1.688,3.721)--(1.868,3.721);
\draw[gp path] (11.947,3.721)--(11.767,3.721);
\node[gp node right] at (1.504,3.721) { 1.12};
\draw[gp path] (1.688,4.405)--(1.868,4.405);
\draw[gp path] (11.947,4.405)--(11.767,4.405);
\node[gp node right] at (1.504,4.405) { 1.14};
\draw[gp path] (1.688,5.089)--(1.868,5.089);
\draw[gp path] (11.947,5.089)--(11.767,5.089);
\node[gp node right] at (1.504,5.089) { 1.16};
\draw[gp path] (1.688,5.773)--(1.868,5.773);
\draw[gp path] (11.947,5.773)--(11.767,5.773);
\node[gp node right] at (1.504,5.773) { 1.18};
\draw[gp path] (1.688,6.457)--(1.868,6.457);
\draw[gp path] (11.947,6.457)--(11.767,6.457);
\node[gp node right] at (1.504,6.457) { 1.2};
\draw[gp path] (1.688,7.141)--(1.868,7.141);
\draw[gp path] (11.947,7.141)--(11.767,7.141);
\node[gp node right] at (1.504,7.141) { 1.22};
\draw[gp path] (1.688,7.825)--(1.868,7.825);
\draw[gp path] (11.947,7.825)--(11.767,7.825);
\node[gp node right] at (1.504,7.825) { 1.24};
\draw[gp path] (1.688,0.985)--(1.688,1.165);
\draw[gp path] (1.688,7.825)--(1.688,7.645);
\node[gp node center] at (1.688,0.677) { 0};
\draw[gp path] (2.828,0.985)--(2.828,1.165);
\draw[gp path] (2.828,7.825)--(2.828,7.645);
\node[gp node center] at (2.828,0.677) { 2};
\draw[gp path] (3.968,0.985)--(3.968,1.165);
\draw[gp path] (3.968,7.825)--(3.968,7.645);
\node[gp node center] at (3.968,0.677) { 4};
\draw[gp path] (5.108,0.985)--(5.108,1.165);
\draw[gp path] (5.108,7.825)--(5.108,7.645);
\node[gp node center] at (5.108,0.677) { 6};
\draw[gp path] (6.248,0.985)--(6.248,1.165);
\draw[gp path] (6.248,7.825)--(6.248,7.645);
\node[gp node center] at (6.248,0.677) { 8};
\draw[gp path] (7.387,0.985)--(7.387,1.165);
\draw[gp path] (7.387,7.825)--(7.387,7.645);
\node[gp node center] at (7.387,0.677) { 10};
\draw[gp path] (8.527,0.985)--(8.527,1.165);
\draw[gp path] (8.527,7.825)--(8.527,7.645);
\node[gp node center] at (8.527,0.677) { 12};
\draw[gp path] (9.667,0.985)--(9.667,1.165);
\draw[gp path] (9.667,7.825)--(9.667,7.645);
\node[gp node center] at (9.667,0.677) { 14};
\draw[gp path] (10.807,0.985)--(10.807,1.165);
\draw[gp path] (10.807,7.825)--(10.807,7.645);
\node[gp node center] at (10.807,0.677) { 16};
\draw[gp path] (11.947,0.985)--(11.947,1.165);
\draw[gp path] (11.947,7.825)--(11.947,7.645);
\node[gp node center] at (11.947,0.677) { 18};
\draw[gp path] (1.688,7.825)--(1.688,0.985)--(11.947,0.985)--(11.947,7.825)--cycle;
\node[gp node center,rotate=-270] at (0.246,4.405) {Velocità angolare [rad/s]};
\node[gp node center] at (6.817,0.215) {Tempo [s]};
\node[gp node center] at (6.817,8.287) {Velocità angolare, accelerazione [rad/s]};
\node[gp node left] at (7.167,1.627) {Dati};
\gpcolor{color=gp lt color 0}
\gpsetlinetype{gp lt plot 0}
\draw[gp path] (10.663,1.627)--(11.579,1.627);
\draw[gp path] (10.663,1.717)--(10.663,1.537);
\draw[gp path] (11.579,1.717)--(11.579,1.537);
\draw[gp path] (2.472,1.640)--(2.472,7.320);
\draw[gp path] (2.382,1.640)--(2.562,1.640);
\draw[gp path] (2.382,7.320)--(2.562,7.320);
\draw[gp path] (3.244,3.325)--(3.244,6.207);
\draw[gp path] (3.154,3.325)--(3.334,3.325);
\draw[gp path] (3.154,6.207)--(3.334,6.207);
\draw[gp path] (4.030,3.669)--(4.030,5.576);
\draw[gp path] (3.940,3.669)--(4.120,3.669);
\draw[gp path] (3.940,5.576)--(4.120,5.576);
\draw[gp path] (4.843,3.532)--(4.843,4.934);
\draw[gp path] (4.753,3.532)--(4.933,3.532);
\draw[gp path] (4.753,4.934)--(4.933,4.934);
\draw[gp path] (5.643,3.556)--(5.643,4.671);
\draw[gp path] (5.553,3.556)--(5.733,3.556);
\draw[gp path] (5.553,4.671)--(5.733,4.671);
\draw[gp path] (6.447,3.550)--(6.447,4.474);
\draw[gp path] (6.357,3.550)--(6.537,3.550);
\draw[gp path] (6.357,4.474)--(6.537,4.474);
\draw[gp path] (7.236,3.642)--(7.236,4.435);
\draw[gp path] (7.146,3.642)--(7.326,3.642);
\draw[gp path] (7.146,4.435)--(7.326,4.435);
\draw[gp path] (8.051,3.558)--(8.051,4.247);
\draw[gp path] (7.961,3.558)--(8.141,3.558);
\draw[gp path] (7.961,4.247)--(8.141,4.247);
\draw[gp path] (8.869,3.478)--(8.869,4.086);
\draw[gp path] (8.779,3.478)--(8.959,3.478);
\draw[gp path] (8.779,4.086)--(8.959,4.086);
\draw[gp path] (9.707,3.320)--(9.707,3.862);
\draw[gp path] (9.617,3.320)--(9.797,3.320);
\draw[gp path] (9.617,3.862)--(9.797,3.862);
\draw[gp path] (10.491,3.423)--(10.491,3.918);
\draw[gp path] (10.401,3.423)--(10.581,3.423);
\draw[gp path] (10.401,3.918)--(10.581,3.918);
\draw[gp path] (11.349,3.218)--(11.349,3.667);
\draw[gp path] (11.259,3.218)--(11.439,3.218);
\draw[gp path] (11.259,3.667)--(11.439,3.667);
\gpsetpointsize{4.00}
\gppoint{gp mark 1}{(2.472,4.480)}
\gppoint{gp mark 1}{(3.244,4.766)}
\gppoint{gp mark 1}{(4.030,4.622)}
\gppoint{gp mark 1}{(4.843,4.233)}
\gppoint{gp mark 1}{(5.643,4.114)}
\gppoint{gp mark 1}{(6.447,4.012)}
\gppoint{gp mark 1}{(7.236,4.038)}
\gppoint{gp mark 1}{(8.051,3.902)}
\gppoint{gp mark 1}{(8.869,3.782)}
\gppoint{gp mark 1}{(9.707,3.591)}
\gppoint{gp mark 1}{(10.491,3.670)}
\gppoint{gp mark 1}{(11.349,3.443)}
\gppoint{gp mark 1}{(11.121,1.627)}
\gpcolor{color=gp lt color border}
\node[gp node left] at (7.167,1.319) {Retta interpolante};
\gpcolor{color=gp lt color 1}
\gpsetlinetype{gp lt plot 1}
\draw[gp path] (10.663,1.319)--(11.579,1.319);
\draw[gp path] (2.472,4.652)--(2.561,4.639)--(2.651,4.627)--(2.741,4.615)--(2.830,4.603)%
  --(2.920,4.591)--(3.010,4.578)--(3.099,4.566)--(3.189,4.554)--(3.279,4.542)--(3.368,4.530)%
  --(3.458,4.517)--(3.548,4.505)--(3.637,4.493)--(3.727,4.481)--(3.817,4.469)--(3.906,4.457)%
  --(3.996,4.444)--(4.086,4.432)--(4.175,4.420)--(4.265,4.408)--(4.355,4.396)--(4.444,4.383)%
  --(4.534,4.371)--(4.624,4.359)--(4.713,4.347)--(4.803,4.335)--(4.893,4.322)--(4.982,4.310)%
  --(5.072,4.298)--(5.162,4.286)--(5.251,4.274)--(5.341,4.262)--(5.431,4.249)--(5.520,4.237)%
  --(5.610,4.225)--(5.700,4.213)--(5.789,4.201)--(5.879,4.188)--(5.969,4.176)--(6.058,4.164)%
  --(6.148,4.152)--(6.238,4.140)--(6.327,4.127)--(6.417,4.115)--(6.507,4.103)--(6.596,4.091)%
  --(6.686,4.079)--(6.776,4.067)--(6.865,4.054)--(6.955,4.042)--(7.045,4.030)--(7.134,4.018)%
  --(7.224,4.006)--(7.314,3.993)--(7.403,3.981)--(7.493,3.969)--(7.583,3.957)--(7.672,3.945)%
  --(7.762,3.932)--(7.852,3.920)--(7.941,3.908)--(8.031,3.896)--(8.121,3.884)--(8.210,3.872)%
  --(8.300,3.859)--(8.390,3.847)--(8.479,3.835)--(8.569,3.823)--(8.659,3.811)--(8.748,3.798)%
  --(8.838,3.786)--(8.928,3.774)--(9.017,3.762)--(9.107,3.750)--(9.197,3.737)--(9.286,3.725)%
  --(9.376,3.713)--(9.466,3.701)--(9.555,3.689)--(9.645,3.677)--(9.735,3.664)--(9.824,3.652)%
  --(9.914,3.640)--(10.004,3.628)--(10.093,3.616)--(10.183,3.603)--(10.273,3.591)--(10.362,3.579)%
  --(10.452,3.567)--(10.542,3.555)--(10.631,3.542)--(10.721,3.530)--(10.811,3.518)--(10.900,3.506)%
  --(10.990,3.494)--(11.080,3.482)--(11.169,3.469)--(11.259,3.457)--(11.349,3.445);
\gpcolor{color=gp lt color border}
\gpsetlinetype{gp lt border}
\draw[gp path] (1.688,7.825)--(1.688,0.985)--(11.947,0.985)--(11.947,7.825)--cycle;
%% coordinates of the plot area
\gpdefrectangularnode{gp plot 1}{\pgfpoint{1.688cm}{0.985cm}}{\pgfpoint{11.947cm}{7.825cm}}
\end{tikzpicture}
%% gnuplot variables

\caption{Seconda serie, decelerazione}
\label{fig:1}
\end{grafico}

\begin{grafico}
    \centering
\begin{tikzpicture}[gnuplot]
%% generated with GNUPLOT 4.6p0 (Lua 5.1; terminal rev. 99, script rev. 100)
%% Tue 15 Apr 2014 09:47:08 PM CEST
\path (0.000,0.000) rectangle (12.500,8.750);
\gpcolor{color=gp lt color border}
\gpsetlinetype{gp lt border}
\gpsetlinewidth{1.00}
\draw[gp path] (1.688,0.985)--(1.868,0.985);
\draw[gp path] (11.947,0.985)--(11.767,0.985);
\node[gp node right] at (1.504,0.985) { 1.1};
\draw[gp path] (1.688,1.669)--(1.868,1.669);
\draw[gp path] (11.947,1.669)--(11.767,1.669);
\node[gp node right] at (1.504,1.669) { 1.12};
\draw[gp path] (1.688,2.353)--(1.868,2.353);
\draw[gp path] (11.947,2.353)--(11.767,2.353);
\node[gp node right] at (1.504,2.353) { 1.14};
\draw[gp path] (1.688,3.037)--(1.868,3.037);
\draw[gp path] (11.947,3.037)--(11.767,3.037);
\node[gp node right] at (1.504,3.037) { 1.16};
\draw[gp path] (1.688,3.721)--(1.868,3.721);
\draw[gp path] (11.947,3.721)--(11.767,3.721);
\node[gp node right] at (1.504,3.721) { 1.18};
\draw[gp path] (1.688,4.405)--(1.868,4.405);
\draw[gp path] (11.947,4.405)--(11.767,4.405);
\node[gp node right] at (1.504,4.405) { 1.2};
\draw[gp path] (1.688,5.089)--(1.868,5.089);
\draw[gp path] (11.947,5.089)--(11.767,5.089);
\node[gp node right] at (1.504,5.089) { 1.22};
\draw[gp path] (1.688,5.773)--(1.868,5.773);
\draw[gp path] (11.947,5.773)--(11.767,5.773);
\node[gp node right] at (1.504,5.773) { 1.24};
\draw[gp path] (1.688,6.457)--(1.868,6.457);
\draw[gp path] (11.947,6.457)--(11.767,6.457);
\node[gp node right] at (1.504,6.457) { 1.26};
\draw[gp path] (1.688,7.141)--(1.868,7.141);
\draw[gp path] (11.947,7.141)--(11.767,7.141);
\node[gp node right] at (1.504,7.141) { 1.28};
\draw[gp path] (1.688,7.825)--(1.868,7.825);
\draw[gp path] (11.947,7.825)--(11.767,7.825);
\node[gp node right] at (1.504,7.825) { 1.3};
\draw[gp path] (1.688,0.985)--(1.688,1.165);
\draw[gp path] (1.688,7.825)--(1.688,7.645);
\node[gp node center] at (1.688,0.677) { 0};
\draw[gp path] (3.154,0.985)--(3.154,1.165);
\draw[gp path] (3.154,7.825)--(3.154,7.645);
\node[gp node center] at (3.154,0.677) { 5};
\draw[gp path] (4.619,0.985)--(4.619,1.165);
\draw[gp path] (4.619,7.825)--(4.619,7.645);
\node[gp node center] at (4.619,0.677) { 10};
\draw[gp path] (6.085,0.985)--(6.085,1.165);
\draw[gp path] (6.085,7.825)--(6.085,7.645);
\node[gp node center] at (6.085,0.677) { 15};
\draw[gp path] (7.550,0.985)--(7.550,1.165);
\draw[gp path] (7.550,7.825)--(7.550,7.645);
\node[gp node center] at (7.550,0.677) { 20};
\draw[gp path] (9.016,0.985)--(9.016,1.165);
\draw[gp path] (9.016,7.825)--(9.016,7.645);
\node[gp node center] at (9.016,0.677) { 25};
\draw[gp path] (10.481,0.985)--(10.481,1.165);
\draw[gp path] (10.481,7.825)--(10.481,7.645);
\node[gp node center] at (10.481,0.677) { 30};
\draw[gp path] (11.947,0.985)--(11.947,1.165);
\draw[gp path] (11.947,7.825)--(11.947,7.645);
\node[gp node center] at (11.947,0.677) { 35};
\draw[gp path] (1.688,7.825)--(1.688,0.985)--(11.947,0.985)--(11.947,7.825)--cycle;
\node[gp node center,rotate=-270] at (0.246,4.405) {Velocità angolare [rad/s]};
\node[gp node center] at (6.817,0.215) {Tempo [s]};
\node[gp node center] at (6.817,8.287) {Velocità angolari in decelerazione [rad/s]};
\node[gp node left] at (7.167,1.627) {Dati};
\gpcolor{color=gp lt color 0}
\gpsetlinetype{gp lt plot 0}
\draw[gp path] (10.663,1.627)--(11.579,1.627);
\draw[gp path] (10.663,1.717)--(10.663,1.537);
\draw[gp path] (11.579,1.717)--(11.579,1.537);
\draw[gp path] (2.071,1.370)--(2.071,7.676);
\draw[gp path] (1.981,1.370)--(2.161,1.370);
\draw[gp path] (1.981,7.676)--(2.161,7.676);
\draw[gp path] (2.444,3.388)--(2.444,6.615);
\draw[gp path] (2.354,3.388)--(2.534,3.388);
\draw[gp path] (2.354,6.615)--(2.534,6.615);
\draw[gp path] (2.828,3.723)--(2.828,5.852);
\draw[gp path] (2.738,3.723)--(2.918,3.723);
\draw[gp path] (2.738,5.852)--(2.918,5.852);
\draw[gp path] (3.237,3.248)--(3.237,4.786);
\draw[gp path] (3.147,3.248)--(3.327,3.248);
\draw[gp path] (3.147,4.786)--(3.327,4.786);
\draw[gp path] (3.605,3.801)--(3.605,5.056);
\draw[gp path] (3.515,3.801)--(3.695,3.801);
\draw[gp path] (3.515,5.056)--(3.695,5.056);
\draw[gp path] (3.989,3.895)--(3.989,4.941);
\draw[gp path] (3.899,3.895)--(4.079,3.895);
\draw[gp path] (3.899,4.941)--(4.079,4.941);
\draw[gp path] (4.382,3.832)--(4.382,4.722);
\draw[gp path] (4.292,3.832)--(4.472,3.832);
\draw[gp path] (4.292,4.722)--(4.472,4.722);
\draw[gp path] (4.773,3.803)--(4.773,4.578);
\draw[gp path] (4.683,3.803)--(4.863,3.803);
\draw[gp path] (4.683,4.578)--(4.863,4.578);
\draw[gp path] (5.170,3.713)--(5.170,4.398);
\draw[gp path] (5.080,3.713)--(5.260,3.713);
\draw[gp path] (5.080,4.398)--(5.260,4.398);
\draw[gp path] (5.557,3.747)--(5.557,4.363);
\draw[gp path] (5.467,3.747)--(5.647,3.747);
\draw[gp path] (5.467,4.363)--(5.647,4.363);
\draw[gp path] (5.967,3.555)--(5.967,4.109);
\draw[gp path] (5.877,3.555)--(6.057,3.555);
\draw[gp path] (5.877,4.109)--(6.057,4.109);
\draw[gp path] (6.378,3.397)--(6.378,3.900);
\draw[gp path] (6.288,3.397)--(6.468,3.397);
\draw[gp path] (6.288,3.900)--(6.468,3.900);
\draw[gp path] (6.781,3.321)--(6.781,3.783);
\draw[gp path] (6.691,3.321)--(6.871,3.321);
\draw[gp path] (6.691,3.783)--(6.871,3.783);
\draw[gp path] (7.194,3.182)--(7.194,3.608);
\draw[gp path] (7.104,3.182)--(7.284,3.182);
\draw[gp path] (7.104,3.608)--(7.284,3.608);
\draw[gp path] (7.600,3.111)--(7.600,3.507);
\draw[gp path] (7.510,3.111)--(7.690,3.111);
\draw[gp path] (7.510,3.507)--(7.690,3.507);
\draw[gp path] (8.010,3.022)--(8.010,3.391);
\draw[gp path] (7.920,3.022)--(8.100,3.022);
\draw[gp path] (7.920,3.391)--(8.100,3.391);
\draw[gp path] (8.418,2.961)--(8.418,3.307);
\draw[gp path] (8.328,2.961)--(8.508,2.961);
\draw[gp path] (8.328,3.307)--(8.508,3.307);
\draw[gp path] (8.844,2.801)--(8.844,3.126);
\draw[gp path] (8.754,2.801)--(8.934,2.801);
\draw[gp path] (8.754,3.126)--(8.934,3.126);
\draw[gp path] (9.275,2.638)--(9.275,2.942);
\draw[gp path] (9.185,2.638)--(9.365,2.638);
\draw[gp path] (9.185,2.942)--(9.365,2.942);
\draw[gp path] (9.683,2.606)--(9.683,2.894);
\draw[gp path] (9.593,2.606)--(9.773,2.606);
\draw[gp path] (9.593,2.894)--(9.773,2.894);
\draw[gp path] (10.114,2.468)--(10.114,2.741);
\draw[gp path] (10.024,2.468)--(10.204,2.468);
\draw[gp path] (10.024,2.741)--(10.204,2.741);
\draw[gp path] (10.525,2.427)--(10.525,2.687);
\draw[gp path] (10.435,2.427)--(10.615,2.427);
\draw[gp path] (10.435,2.687)--(10.615,2.687);
\gpsetpointsize{4.00}
\gppoint{gp mark 1}{(2.071,4.523)}
\gppoint{gp mark 1}{(2.444,5.002)}
\gppoint{gp mark 1}{(2.828,4.787)}
\gppoint{gp mark 1}{(3.237,4.017)}
\gppoint{gp mark 1}{(3.605,4.429)}
\gppoint{gp mark 1}{(3.989,4.418)}
\gppoint{gp mark 1}{(4.382,4.277)}
\gppoint{gp mark 1}{(4.773,4.191)}
\gppoint{gp mark 1}{(5.170,4.055)}
\gppoint{gp mark 1}{(5.557,4.055)}
\gppoint{gp mark 1}{(5.967,3.832)}
\gppoint{gp mark 1}{(6.378,3.648)}
\gppoint{gp mark 1}{(6.781,3.552)}
\gppoint{gp mark 1}{(7.194,3.395)}
\gppoint{gp mark 1}{(7.600,3.309)}
\gppoint{gp mark 1}{(8.010,3.206)}
\gppoint{gp mark 1}{(8.418,3.134)}
\gppoint{gp mark 1}{(8.844,2.964)}
\gppoint{gp mark 1}{(9.275,2.790)}
\gppoint{gp mark 1}{(9.683,2.750)}
\gppoint{gp mark 1}{(10.114,2.604)}
\gppoint{gp mark 1}{(10.525,2.557)}
\gppoint{gp mark 1}{(11.121,1.627)}
\gpcolor{color=gp lt color border}
\node[gp node left] at (7.167,1.319) {Retta interpolante};
\gpcolor{color=gp lt color 1}
\gpsetlinetype{gp lt plot 1}
\draw[gp path] (10.663,1.319)--(11.579,1.319);
\draw[gp path] (2.071,4.892)--(2.156,4.868)--(2.241,4.844)--(2.327,4.820)--(2.412,4.795)%
  --(2.498,4.771)--(2.583,4.747)--(2.668,4.723)--(2.754,4.699)--(2.839,4.675)--(2.925,4.651)%
  --(3.010,4.627)--(3.095,4.603)--(3.181,4.578)--(3.266,4.554)--(3.352,4.530)--(3.437,4.506)%
  --(3.522,4.482)--(3.608,4.458)--(3.693,4.434)--(3.779,4.410)--(3.864,4.386)--(3.949,4.361)%
  --(4.035,4.337)--(4.120,4.313)--(4.206,4.289)--(4.291,4.265)--(4.376,4.241)--(4.462,4.217)%
  --(4.547,4.193)--(4.633,4.169)--(4.718,4.144)--(4.803,4.120)--(4.889,4.096)--(4.974,4.072)%
  --(5.060,4.048)--(5.145,4.024)--(5.230,4.000)--(5.316,3.976)--(5.401,3.952)--(5.487,3.927)%
  --(5.572,3.903)--(5.657,3.879)--(5.743,3.855)--(5.828,3.831)--(5.914,3.807)--(5.999,3.783)%
  --(6.084,3.759)--(6.170,3.735)--(6.255,3.710)--(6.341,3.686)--(6.426,3.662)--(6.511,3.638)%
  --(6.597,3.614)--(6.682,3.590)--(6.768,3.566)--(6.853,3.542)--(6.938,3.517)--(7.024,3.493)%
  --(7.109,3.469)--(7.195,3.445)--(7.280,3.421)--(7.365,3.397)--(7.451,3.373)--(7.536,3.349)%
  --(7.622,3.325)--(7.707,3.300)--(7.793,3.276)--(7.878,3.252)--(7.963,3.228)--(8.049,3.204)%
  --(8.134,3.180)--(8.220,3.156)--(8.305,3.132)--(8.390,3.108)--(8.476,3.083)--(8.561,3.059)%
  --(8.647,3.035)--(8.732,3.011)--(8.817,2.987)--(8.903,2.963)--(8.988,2.939)--(9.074,2.915)%
  --(9.159,2.891)--(9.244,2.866)--(9.330,2.842)--(9.415,2.818)--(9.501,2.794)--(9.586,2.770)%
  --(9.671,2.746)--(9.757,2.722)--(9.842,2.698)--(9.928,2.674)--(10.013,2.649)--(10.098,2.625)%
  --(10.184,2.601)--(10.269,2.577)--(10.355,2.553)--(10.440,2.529)--(10.525,2.505);
\gpcolor{color=gp lt color border}
\gpsetlinetype{gp lt border}
\draw[gp path] (1.688,7.825)--(1.688,0.985)--(11.947,0.985)--(11.947,7.825)--cycle;
%% coordinates of the plot area
\gpdefrectangularnode{gp plot 1}{\pgfpoint{1.688cm}{0.985cm}}{\pgfpoint{11.947cm}{7.825cm}}
\end{tikzpicture}
%% gnuplot variables

\caption{Terza serie, decelerazione}
\label{fig:1}
\end{grafico}

\begin{grafico}
    \centering
\begin{tikzpicture}[gnuplot]
%% generated with GNUPLOT 4.6p0 (Lua 5.1; terminal rev. 99, script rev. 100)
%% Tue 15 Apr 2014 09:47:08 PM CEST
\path (0.000,0.000) rectangle (12.500,8.750);
\gpcolor{color=gp lt color border}
\gpsetlinetype{gp lt border}
\gpsetlinewidth{1.00}
\draw[gp path] (1.688,0.985)--(1.868,0.985);
\draw[gp path] (11.947,0.985)--(11.767,0.985);
\node[gp node right] at (1.504,0.985) { 1.15};
\draw[gp path] (1.688,2.353)--(1.868,2.353);
\draw[gp path] (11.947,2.353)--(11.767,2.353);
\node[gp node right] at (1.504,2.353) { 1.2};
\draw[gp path] (1.688,3.721)--(1.868,3.721);
\draw[gp path] (11.947,3.721)--(11.767,3.721);
\node[gp node right] at (1.504,3.721) { 1.25};
\draw[gp path] (1.688,5.089)--(1.868,5.089);
\draw[gp path] (11.947,5.089)--(11.767,5.089);
\node[gp node right] at (1.504,5.089) { 1.3};
\draw[gp path] (1.688,6.457)--(1.868,6.457);
\draw[gp path] (11.947,6.457)--(11.767,6.457);
\node[gp node right] at (1.504,6.457) { 1.35};
\draw[gp path] (1.688,7.825)--(1.868,7.825);
\draw[gp path] (11.947,7.825)--(11.767,7.825);
\node[gp node right] at (1.504,7.825) { 1.4};
\draw[gp path] (1.688,0.985)--(1.688,1.165);
\draw[gp path] (1.688,7.825)--(1.688,7.645);
\node[gp node center] at (1.688,0.677) { 0};
\draw[gp path] (3.398,0.985)--(3.398,1.165);
\draw[gp path] (3.398,7.825)--(3.398,7.645);
\node[gp node center] at (3.398,0.677) { 5};
\draw[gp path] (5.108,0.985)--(5.108,1.165);
\draw[gp path] (5.108,7.825)--(5.108,7.645);
\node[gp node center] at (5.108,0.677) { 10};
\draw[gp path] (6.818,0.985)--(6.818,1.165);
\draw[gp path] (6.818,7.825)--(6.818,7.645);
\node[gp node center] at (6.818,0.677) { 15};
\draw[gp path] (8.527,0.985)--(8.527,1.165);
\draw[gp path] (8.527,7.825)--(8.527,7.645);
\node[gp node center] at (8.527,0.677) { 20};
\draw[gp path] (10.237,0.985)--(10.237,1.165);
\draw[gp path] (10.237,7.825)--(10.237,7.645);
\node[gp node center] at (10.237,0.677) { 25};
\draw[gp path] (11.947,0.985)--(11.947,1.165);
\draw[gp path] (11.947,7.825)--(11.947,7.645);
\node[gp node center] at (11.947,0.677) { 30};
\draw[gp path] (1.688,7.825)--(1.688,0.985)--(11.947,0.985)--(11.947,7.825)--cycle;
\node[gp node center,rotate=-270] at (0.246,4.405) {Velocità angolare [rad/s]};
\node[gp node center] at (6.817,0.215) {Tempo [s]};
\node[gp node center] at (6.817,8.287) {Velocità angolari in decelerazione [rad/s]};
\node[gp node left] at (7.167,1.627) {Dati};
\gpcolor{color=gp lt color 0}
\gpsetlinetype{gp lt plot 0}
\draw[gp path] (10.663,1.627)--(11.579,1.627);
\draw[gp path] (10.663,1.717)--(10.663,1.537);
\draw[gp path] (11.579,1.717)--(11.579,1.537);
\draw[gp path] (2.115,1.147)--(2.115,6.645);
\draw[gp path] (2.025,1.147)--(2.205,1.147);
\draw[gp path] (2.025,6.645)--(2.205,6.645);
\draw[gp path] (2.560,1.901)--(2.560,4.543);
\draw[gp path] (2.470,1.901)--(2.650,1.901);
\draw[gp path] (2.470,4.543)--(2.650,4.543);
\draw[gp path] (3.005,2.134)--(3.005,3.873);
\draw[gp path] (2.915,2.134)--(3.095,2.134);
\draw[gp path] (2.915,3.873)--(3.095,3.873);
\draw[gp path] (3.449,2.247)--(3.449,3.543);
\draw[gp path] (3.359,2.247)--(3.539,2.247);
\draw[gp path] (3.359,3.543)--(3.539,3.543);
\draw[gp path] (3.877,2.566)--(3.877,3.615);
\draw[gp path] (3.787,2.566)--(3.967,2.566);
\draw[gp path] (3.787,3.615)--(3.967,3.615);
\draw[gp path] (4.321,2.569)--(4.321,3.438);
\draw[gp path] (4.231,2.569)--(4.411,2.569);
\draw[gp path] (4.231,3.438)--(4.411,3.438);
\draw[gp path] (4.783,2.389)--(4.783,3.124);
\draw[gp path] (4.693,2.389)--(4.873,2.389);
\draw[gp path] (4.693,3.124)--(4.873,3.124);
\draw[gp path] (5.244,2.256)--(5.244,2.892);
\draw[gp path] (5.154,2.256)--(5.334,2.256);
\draw[gp path] (5.154,2.892)--(5.334,2.892);
\draw[gp path] (5.672,2.431)--(5.672,3.001);
\draw[gp path] (5.582,2.431)--(5.762,2.431);
\draw[gp path] (5.582,3.001)--(5.762,3.001);
\draw[gp path] (6.134,2.320)--(6.134,2.828);
\draw[gp path] (6.044,2.320)--(6.224,2.320);
\draw[gp path] (6.044,2.828)--(6.224,2.828);
\draw[gp path] (6.595,2.229)--(6.595,2.688);
\draw[gp path] (6.505,2.229)--(6.685,2.229);
\draw[gp path] (6.505,2.688)--(6.685,2.688);
\draw[gp path] (7.057,2.154)--(7.057,2.573);
\draw[gp path] (6.967,2.154)--(7.147,2.154);
\draw[gp path] (6.967,2.573)--(7.147,2.573);
\draw[gp path] (7.536,1.996)--(7.536,2.378);
\draw[gp path] (7.446,1.996)--(7.626,1.996);
\draw[gp path] (7.446,2.378)--(7.626,2.378);
\draw[gp path] (7.997,1.949)--(7.997,2.303);
\draw[gp path] (7.907,1.949)--(8.087,1.949);
\draw[gp path] (7.907,2.303)--(8.087,2.303);
\draw[gp path] (8.459,1.909)--(8.459,2.238);
\draw[gp path] (8.369,1.909)--(8.549,1.909);
\draw[gp path] (8.369,2.238)--(8.549,2.238);
\draw[gp path] (8.921,1.873)--(8.921,2.181);
\draw[gp path] (8.831,1.873)--(9.011,1.873);
\draw[gp path] (8.831,2.181)--(9.011,2.181);
\draw[gp path] (9.399,1.771)--(9.399,2.058);
\draw[gp path] (9.309,1.771)--(9.489,1.771);
\draw[gp path] (9.309,2.058)--(9.489,2.058);
\draw[gp path] (9.878,1.680)--(9.878,1.950);
\draw[gp path] (9.788,1.680)--(9.968,1.680);
\draw[gp path] (9.788,1.950)--(9.968,1.950);
\draw[gp path] (10.357,1.599)--(10.357,1.853);
\draw[gp path] (10.267,1.599)--(10.447,1.599);
\draw[gp path] (10.267,1.853)--(10.447,1.853);
\draw[gp path] (10.853,1.468)--(10.853,1.707);
\draw[gp path] (10.763,1.468)--(10.943,1.468);
\draw[gp path] (10.763,1.707)--(10.943,1.707);
\draw[gp path] (11.314,1.462)--(11.314,1.690);
\draw[gp path] (11.224,1.462)--(11.404,1.462);
\draw[gp path] (11.224,1.690)--(11.404,1.690);
\draw[gp path] (11.810,1.349)--(11.810,1.565);
\draw[gp path] (11.720,1.349)--(11.900,1.349);
\draw[gp path] (11.720,1.565)--(11.900,1.565);
\gpsetpointsize{4.00}
\gppoint{gp mark 1}{(2.115,3.896)}
\gppoint{gp mark 1}{(2.560,3.222)}
\gppoint{gp mark 1}{(3.005,3.003)}
\gppoint{gp mark 1}{(3.449,2.895)}
\gppoint{gp mark 1}{(3.877,3.090)}
\gppoint{gp mark 1}{(4.321,3.003)}
\gppoint{gp mark 1}{(4.783,2.757)}
\gppoint{gp mark 1}{(5.244,2.574)}
\gppoint{gp mark 1}{(5.672,2.716)}
\gppoint{gp mark 1}{(6.134,2.574)}
\gppoint{gp mark 1}{(6.595,2.459)}
\gppoint{gp mark 1}{(7.057,2.363)}
\gppoint{gp mark 1}{(7.536,2.187)}
\gppoint{gp mark 1}{(7.997,2.126)}
\gppoint{gp mark 1}{(8.459,2.073)}
\gppoint{gp mark 1}{(8.921,2.027)}
\gppoint{gp mark 1}{(9.399,1.914)}
\gppoint{gp mark 1}{(9.878,1.815)}
\gppoint{gp mark 1}{(10.357,1.726)}
\gppoint{gp mark 1}{(10.853,1.587)}
\gppoint{gp mark 1}{(11.314,1.576)}
\gppoint{gp mark 1}{(11.810,1.457)}
\gppoint{gp mark 1}{(11.121,1.627)}
\gpcolor{color=gp lt color border}
\node[gp node left] at (7.167,1.319) {Retta interpolante};
\gpcolor{color=gp lt color 1}
\gpsetlinetype{gp lt plot 1}
\draw[gp path] (10.663,1.319)--(11.579,1.319);
\draw[gp path] (2.115,3.286)--(2.213,3.268)--(2.311,3.249)--(2.409,3.231)--(2.507,3.212)%
  --(2.605,3.193)--(2.703,3.175)--(2.801,3.156)--(2.899,3.138)--(2.997,3.119)--(3.095,3.101)%
  --(3.193,3.082)--(3.291,3.064)--(3.389,3.045)--(3.486,3.027)--(3.584,3.008)--(3.682,2.989)%
  --(3.780,2.971)--(3.878,2.952)--(3.976,2.934)--(4.074,2.915)--(4.172,2.897)--(4.270,2.878)%
  --(4.368,2.860)--(4.466,2.841)--(4.564,2.823)--(4.662,2.804)--(4.759,2.785)--(4.857,2.767)%
  --(4.955,2.748)--(5.053,2.730)--(5.151,2.711)--(5.249,2.693)--(5.347,2.674)--(5.445,2.656)%
  --(5.543,2.637)--(5.641,2.618)--(5.739,2.600)--(5.837,2.581)--(5.935,2.563)--(6.033,2.544)%
  --(6.130,2.526)--(6.228,2.507)--(6.326,2.489)--(6.424,2.470)--(6.522,2.452)--(6.620,2.433)%
  --(6.718,2.414)--(6.816,2.396)--(6.914,2.377)--(7.012,2.359)--(7.110,2.340)--(7.208,2.322)%
  --(7.306,2.303)--(7.404,2.285)--(7.501,2.266)--(7.599,2.248)--(7.697,2.229)--(7.795,2.210)%
  --(7.893,2.192)--(7.991,2.173)--(8.089,2.155)--(8.187,2.136)--(8.285,2.118)--(8.383,2.099)%
  --(8.481,2.081)--(8.579,2.062)--(8.677,2.043)--(8.774,2.025)--(8.872,2.006)--(8.970,1.988)%
  --(9.068,1.969)--(9.166,1.951)--(9.264,1.932)--(9.362,1.914)--(9.460,1.895)--(9.558,1.877)%
  --(9.656,1.858)--(9.754,1.839)--(9.852,1.821)--(9.950,1.802)--(10.048,1.784)--(10.145,1.765)%
  --(10.243,1.747)--(10.341,1.728)--(10.439,1.710)--(10.537,1.691)--(10.635,1.673)--(10.733,1.654)%
  --(10.831,1.635)--(10.929,1.617)--(11.027,1.598)--(11.125,1.580)--(11.223,1.561)--(11.321,1.543)%
  --(11.419,1.524)--(11.516,1.506)--(11.614,1.487)--(11.712,1.469)--(11.810,1.450);
\gpcolor{color=gp lt color border}
\gpsetlinetype{gp lt border}
\draw[gp path] (1.688,7.825)--(1.688,0.985)--(11.947,0.985)--(11.947,7.825)--cycle;
%% coordinates of the plot area
\gpdefrectangularnode{gp plot 1}{\pgfpoint{1.688cm}{0.985cm}}{\pgfpoint{11.947cm}{7.825cm}}
\end{tikzpicture}
%% gnuplot variables

\caption{Quarta serie, decelerazione}
\label{fig:1}
\end{grafico}

\begin{grafico}
    \centering
\begin{tikzpicture}[gnuplot]
%% generated with GNUPLOT 4.6p0 (Lua 5.1; terminal rev. 99, script rev. 100)
%% Mon 14 Apr 2014 11:09:59 PM CEST
\path (0.000,0.000) rectangle (12.500,8.750);
\gpcolor{color=gp lt color border}
\gpsetlinetype{gp lt border}
\gpsetlinewidth{1.00}
\draw[gp path] (1.688,0.985)--(1.868,0.985);
\draw[gp path] (11.947,0.985)--(11.767,0.985);
\node[gp node right] at (1.504,0.985) {-1000};
\draw[gp path] (1.688,1.840)--(1.868,1.840);
\draw[gp path] (11.947,1.840)--(11.767,1.840);
\node[gp node right] at (1.504,1.840) {-800};
\draw[gp path] (1.688,2.695)--(1.868,2.695);
\draw[gp path] (11.947,2.695)--(11.767,2.695);
\node[gp node right] at (1.504,2.695) {-600};
\draw[gp path] (1.688,3.550)--(1.868,3.550);
\draw[gp path] (11.947,3.550)--(11.767,3.550);
\node[gp node right] at (1.504,3.550) {-400};
\draw[gp path] (1.688,4.405)--(1.868,4.405);
\draw[gp path] (11.947,4.405)--(11.767,4.405);
\node[gp node right] at (1.504,4.405) {-200};
\draw[gp path] (1.688,5.260)--(1.868,5.260);
\draw[gp path] (11.947,5.260)--(11.767,5.260);
\node[gp node right] at (1.504,5.260) { 0};
\draw[gp path] (1.688,6.115)--(1.868,6.115);
\draw[gp path] (11.947,6.115)--(11.767,6.115);
\node[gp node right] at (1.504,6.115) { 200};
\draw[gp path] (1.688,6.970)--(1.868,6.970);
\draw[gp path] (11.947,6.970)--(11.767,6.970);
\node[gp node right] at (1.504,6.970) { 400};
\draw[gp path] (1.688,7.825)--(1.868,7.825);
\draw[gp path] (11.947,7.825)--(11.767,7.825);
\node[gp node right] at (1.504,7.825) { 600};
\draw[gp path] (1.688,0.985)--(1.688,1.165);
\draw[gp path] (1.688,7.825)--(1.688,7.645);
\node[gp node center] at (1.688,0.677) {-10};
\draw[gp path] (4.253,0.985)--(4.253,1.165);
\draw[gp path] (4.253,7.825)--(4.253,7.645);
\node[gp node center] at (4.253,0.677) {-5};
\draw[gp path] (6.818,0.985)--(6.818,1.165);
\draw[gp path] (6.818,7.825)--(6.818,7.645);
\node[gp node center] at (6.818,0.677) { 0};
\draw[gp path] (9.382,0.985)--(9.382,1.165);
\draw[gp path] (9.382,7.825)--(9.382,7.645);
\node[gp node center] at (9.382,0.677) { 5};
\draw[gp path] (11.947,0.985)--(11.947,1.165);
\draw[gp path] (11.947,7.825)--(11.947,7.645);
\node[gp node center] at (11.947,0.677) { 10};
\draw[gp path] (1.688,7.825)--(1.688,0.985)--(11.947,0.985)--(11.947,7.825)--cycle;
\node[gp node center,rotate=-270] at (0.246,4.405) {Velocità angolare [rad/s]};
\node[gp node center] at (6.817,0.215) {Tempo [s]};
\node[gp node center] at (6.817,8.287) {Velocità angolare, decelerazione [rad/s]};
\node[gp node left] at (7.167,1.627) {Retta interpolante};
\gpcolor{color=gp lt color 1}
\gpsetlinetype{gp lt plot 1}
\draw[gp path] (10.663,1.627)--(11.579,1.627);
\draw[gp path] (1.688,1.316)--(1.792,1.381)--(1.895,1.445)--(1.999,1.510)--(2.103,1.575)%
  --(2.206,1.639)--(2.310,1.704)--(2.413,1.768)--(2.517,1.833)--(2.621,1.897)--(2.724,1.962)%
  --(2.828,2.026)--(2.932,2.091)--(3.035,2.155)--(3.139,2.220)--(3.242,2.285)--(3.346,2.349)%
  --(3.450,2.414)--(3.553,2.478)--(3.657,2.543)--(3.761,2.607)--(3.864,2.672)--(3.968,2.736)%
  --(4.071,2.801)--(4.175,2.865)--(4.279,2.930)--(4.382,2.995)--(4.486,3.059)--(4.590,3.124)%
  --(4.693,3.188)--(4.797,3.253)--(4.900,3.317)--(5.004,3.382)--(5.108,3.446)--(5.211,3.511)%
  --(5.315,3.575)--(5.419,3.640)--(5.522,3.704)--(5.626,3.769)--(5.729,3.834)--(5.833,3.898)%
  --(5.937,3.963)--(6.040,4.027)--(6.144,4.092)--(6.248,4.156)--(6.351,4.221)--(6.455,4.285)%
  --(6.558,4.350)--(6.662,4.414)--(6.766,4.479)--(6.869,4.544)--(6.973,4.608)--(7.077,4.673)%
  --(7.180,4.737)--(7.284,4.802)--(7.387,4.866)--(7.491,4.931)--(7.595,4.995)--(7.698,5.060)%
  --(7.802,5.124)--(7.906,5.189)--(8.009,5.254)--(8.113,5.318)--(8.216,5.383)--(8.320,5.447)%
  --(8.424,5.512)--(8.527,5.576)--(8.631,5.641)--(8.735,5.705)--(8.838,5.770)--(8.942,5.834)%
  --(9.045,5.899)--(9.149,5.964)--(9.253,6.028)--(9.356,6.093)--(9.460,6.157)--(9.564,6.222)%
  --(9.667,6.286)--(9.771,6.351)--(9.874,6.415)--(9.978,6.480)--(10.082,6.544)--(10.185,6.609)%
  --(10.289,6.673)--(10.393,6.738)--(10.496,6.803)--(10.600,6.867)--(10.703,6.932)--(10.807,6.996)%
  --(10.911,7.061)--(11.014,7.125)--(11.118,7.190)--(11.222,7.254)--(11.325,7.319)--(11.429,7.383)%
  --(11.532,7.448)--(11.636,7.513)--(11.740,7.577)--(11.843,7.642)--(11.947,7.706);
\gpcolor{color=gp lt color border}
\gpsetlinetype{gp lt border}
\draw[gp path] (1.688,7.825)--(1.688,0.985)--(11.947,0.985)--(11.947,7.825)--cycle;
%% coordinates of the plot area
\gpdefrectangularnode{gp plot 1}{\pgfpoint{1.688cm}{0.985cm}}{\pgfpoint{11.947cm}{7.825cm}}
\end{tikzpicture}
%% gnuplot variables

\caption{Quinta serie, decelerazione}
\label{fig:1}
\end{grafico}

\begin{grafico}
    \centering
\begin{tikzpicture}[gnuplot]
%% generated with GNUPLOT 4.6p0 (Lua 5.1; terminal rev. 99, script rev. 100)
%% Mon 14 Apr 2014 11:09:59 PM CEST
\path (0.000,0.000) rectangle (12.500,8.750);
\gpcolor{color=gp lt color border}
\gpsetlinetype{gp lt border}
\gpsetlinewidth{1.00}
\draw[gp path] (1.688,0.985)--(1.868,0.985);
\draw[gp path] (11.947,0.985)--(11.767,0.985);
\node[gp node right] at (1.504,0.985) { 20.2};
\draw[gp path] (1.688,2.125)--(1.868,2.125);
\draw[gp path] (11.947,2.125)--(11.767,2.125);
\node[gp node right] at (1.504,2.125) { 20.3};
\draw[gp path] (1.688,3.265)--(1.868,3.265);
\draw[gp path] (11.947,3.265)--(11.767,3.265);
\node[gp node right] at (1.504,3.265) { 20.4};
\draw[gp path] (1.688,4.405)--(1.868,4.405);
\draw[gp path] (11.947,4.405)--(11.767,4.405);
\node[gp node right] at (1.504,4.405) { 20.5};
\draw[gp path] (1.688,5.545)--(1.868,5.545);
\draw[gp path] (11.947,5.545)--(11.767,5.545);
\node[gp node right] at (1.504,5.545) { 20.6};
\draw[gp path] (1.688,6.685)--(1.868,6.685);
\draw[gp path] (11.947,6.685)--(11.767,6.685);
\node[gp node right] at (1.504,6.685) { 20.7};
\draw[gp path] (1.688,7.825)--(1.868,7.825);
\draw[gp path] (11.947,7.825)--(11.767,7.825);
\node[gp node right] at (1.504,7.825) { 20.8};
\draw[gp path] (1.688,0.985)--(1.688,1.165);
\draw[gp path] (1.688,7.825)--(1.688,7.645);
\node[gp node center] at (1.688,0.677) {-10};
\draw[gp path] (4.253,0.985)--(4.253,1.165);
\draw[gp path] (4.253,7.825)--(4.253,7.645);
\node[gp node center] at (4.253,0.677) {-5};
\draw[gp path] (6.818,0.985)--(6.818,1.165);
\draw[gp path] (6.818,7.825)--(6.818,7.645);
\node[gp node center] at (6.818,0.677) { 0};
\draw[gp path] (9.382,0.985)--(9.382,1.165);
\draw[gp path] (9.382,7.825)--(9.382,7.645);
\node[gp node center] at (9.382,0.677) { 5};
\draw[gp path] (11.947,0.985)--(11.947,1.165);
\draw[gp path] (11.947,7.825)--(11.947,7.645);
\node[gp node center] at (11.947,0.677) { 10};
\draw[gp path] (1.688,7.825)--(1.688,0.985)--(11.947,0.985)--(11.947,7.825)--cycle;
\node[gp node center,rotate=-270] at (0.246,4.405) {Velocità angolare [rad/s]};
\node[gp node center] at (6.817,0.215) {Tempo [s]};
\node[gp node center] at (6.817,8.287) {Velocità angolare, decelerazione [rad/s]};
\node[gp node left] at (7.167,1.627) {Retta interpolante};
\gpcolor{color=gp lt color 1}
\gpsetlinetype{gp lt plot 1}
\draw[gp path] (10.663,1.627)--(11.579,1.627);
\draw[gp path] (1.688,7.117)--(1.792,7.059)--(1.895,7.000)--(1.999,6.941)--(2.103,6.883)%
  --(2.206,6.824)--(2.310,6.765)--(2.413,6.707)--(2.517,6.648)--(2.621,6.589)--(2.724,6.531)%
  --(2.828,6.472)--(2.932,6.413)--(3.035,6.355)--(3.139,6.296)--(3.242,6.237)--(3.346,6.179)%
  --(3.450,6.120)--(3.553,6.061)--(3.657,6.003)--(3.761,5.944)--(3.864,5.885)--(3.968,5.826)%
  --(4.071,5.768)--(4.175,5.709)--(4.279,5.650)--(4.382,5.592)--(4.486,5.533)--(4.590,5.474)%
  --(4.693,5.416)--(4.797,5.357)--(4.900,5.298)--(5.004,5.240)--(5.108,5.181)--(5.211,5.122)%
  --(5.315,5.064)--(5.419,5.005)--(5.522,4.946)--(5.626,4.888)--(5.729,4.829)--(5.833,4.770)%
  --(5.937,4.712)--(6.040,4.653)--(6.144,4.594)--(6.248,4.536)--(6.351,4.477)--(6.455,4.418)%
  --(6.558,4.360)--(6.662,4.301)--(6.766,4.242)--(6.869,4.184)--(6.973,4.125)--(7.077,4.066)%
  --(7.180,4.008)--(7.284,3.949)--(7.387,3.890)--(7.491,3.832)--(7.595,3.773)--(7.698,3.714)%
  --(7.802,3.656)--(7.906,3.597)--(8.009,3.538)--(8.113,3.480)--(8.216,3.421)--(8.320,3.362)%
  --(8.424,3.304)--(8.527,3.245)--(8.631,3.186)--(8.735,3.128)--(8.838,3.069)--(8.942,3.010)%
  --(9.045,2.952)--(9.149,2.893)--(9.253,2.834)--(9.356,2.776)--(9.460,2.717)--(9.564,2.658)%
  --(9.667,2.600)--(9.771,2.541)--(9.874,2.482)--(9.978,2.424)--(10.082,2.365)--(10.185,2.306)%
  --(10.289,2.248)--(10.393,2.189)--(10.496,2.130)--(10.600,2.072)--(10.703,2.013)--(10.807,1.954)%
  --(10.911,1.896)--(11.014,1.837)--(11.118,1.778)--(11.222,1.720)--(11.325,1.661)--(11.429,1.602)%
  --(11.532,1.543)--(11.636,1.485)--(11.740,1.426)--(11.843,1.367)--(11.947,1.309);
\gpcolor{color=gp lt color border}
\gpsetlinetype{gp lt border}
\draw[gp path] (1.688,7.825)--(1.688,0.985)--(11.947,0.985)--(11.947,7.825)--cycle;
%% coordinates of the plot area
\gpdefrectangularnode{gp plot 1}{\pgfpoint{1.688cm}{0.985cm}}{\pgfpoint{11.947cm}{7.825cm}}
\end{tikzpicture}
%% gnuplot variables

\caption{Sesta serie, decelerazione}
\label{fig:1}
\end{grafico}

\begin{grafico}
    \centering
\begin{tikzpicture}[gnuplot]
%% generated with GNUPLOT 4.6p0 (Lua 5.1; terminal rev. 99, script rev. 100)
%% Mon 14 Apr 2014 11:10:00 PM CEST
\path (0.000,0.000) rectangle (12.500,8.750);
\gpcolor{color=gp lt color border}
\gpsetlinetype{gp lt border}
\gpsetlinewidth{1.00}
\draw[gp path] (1.688,0.985)--(1.868,0.985);
\draw[gp path] (11.947,0.985)--(11.767,0.985);
\node[gp node right] at (1.504,0.985) {-1500};
\draw[gp path] (1.688,2.353)--(1.868,2.353);
\draw[gp path] (11.947,2.353)--(11.767,2.353);
\node[gp node right] at (1.504,2.353) {-1000};
\draw[gp path] (1.688,3.721)--(1.868,3.721);
\draw[gp path] (11.947,3.721)--(11.767,3.721);
\node[gp node right] at (1.504,3.721) {-500};
\draw[gp path] (1.688,5.089)--(1.868,5.089);
\draw[gp path] (11.947,5.089)--(11.767,5.089);
\node[gp node right] at (1.504,5.089) { 0};
\draw[gp path] (1.688,6.457)--(1.868,6.457);
\draw[gp path] (11.947,6.457)--(11.767,6.457);
\node[gp node right] at (1.504,6.457) { 500};
\draw[gp path] (1.688,7.825)--(1.868,7.825);
\draw[gp path] (11.947,7.825)--(11.767,7.825);
\node[gp node right] at (1.504,7.825) { 1000};
\draw[gp path] (1.688,0.985)--(1.688,1.165);
\draw[gp path] (1.688,7.825)--(1.688,7.645);
\node[gp node center] at (1.688,0.677) {-10};
\draw[gp path] (4.253,0.985)--(4.253,1.165);
\draw[gp path] (4.253,7.825)--(4.253,7.645);
\node[gp node center] at (4.253,0.677) {-5};
\draw[gp path] (6.818,0.985)--(6.818,1.165);
\draw[gp path] (6.818,7.825)--(6.818,7.645);
\node[gp node center] at (6.818,0.677) { 0};
\draw[gp path] (9.382,0.985)--(9.382,1.165);
\draw[gp path] (9.382,7.825)--(9.382,7.645);
\node[gp node center] at (9.382,0.677) { 5};
\draw[gp path] (11.947,0.985)--(11.947,1.165);
\draw[gp path] (11.947,7.825)--(11.947,7.645);
\node[gp node center] at (11.947,0.677) { 10};
\draw[gp path] (1.688,7.825)--(1.688,0.985)--(11.947,0.985)--(11.947,7.825)--cycle;
\node[gp node center,rotate=-270] at (0.246,4.405) {Velocità angolare [rad/s]};
\node[gp node center] at (6.817,0.215) {Tempo [s]};
\node[gp node center] at (6.817,8.287) {Velocità angolare, decelerazione [rad/s]};
\node[gp node left] at (7.167,1.627) {Retta interpolante};
\gpcolor{color=gp lt color 1}
\gpsetlinetype{gp lt plot 1}
\draw[gp path] (10.663,1.627)--(11.579,1.627);
\draw[gp path] (1.688,1.443)--(1.792,1.502)--(1.895,1.562)--(1.999,1.621)--(2.103,1.680)%
  --(2.206,1.739)--(2.310,1.799)--(2.413,1.858)--(2.517,1.917)--(2.621,1.976)--(2.724,2.036)%
  --(2.828,2.095)--(2.932,2.154)--(3.035,2.213)--(3.139,2.272)--(3.242,2.332)--(3.346,2.391)%
  --(3.450,2.450)--(3.553,2.509)--(3.657,2.569)--(3.761,2.628)--(3.864,2.687)--(3.968,2.746)%
  --(4.071,2.806)--(4.175,2.865)--(4.279,2.924)--(4.382,2.983)--(4.486,3.042)--(4.590,3.102)%
  --(4.693,3.161)--(4.797,3.220)--(4.900,3.279)--(5.004,3.339)--(5.108,3.398)--(5.211,3.457)%
  --(5.315,3.516)--(5.419,3.576)--(5.522,3.635)--(5.626,3.694)--(5.729,3.753)--(5.833,3.813)%
  --(5.937,3.872)--(6.040,3.931)--(6.144,3.990)--(6.248,4.049)--(6.351,4.109)--(6.455,4.168)%
  --(6.558,4.227)--(6.662,4.286)--(6.766,4.346)--(6.869,4.405)--(6.973,4.464)--(7.077,4.523)%
  --(7.180,4.583)--(7.284,4.642)--(7.387,4.701)--(7.491,4.760)--(7.595,4.819)--(7.698,4.879)%
  --(7.802,4.938)--(7.906,4.997)--(8.009,5.056)--(8.113,5.116)--(8.216,5.175)--(8.320,5.234)%
  --(8.424,5.293)--(8.527,5.353)--(8.631,5.412)--(8.735,5.471)--(8.838,5.530)--(8.942,5.589)%
  --(9.045,5.649)--(9.149,5.708)--(9.253,5.767)--(9.356,5.826)--(9.460,5.886)--(9.564,5.945)%
  --(9.667,6.004)--(9.771,6.063)--(9.874,6.123)--(9.978,6.182)--(10.082,6.241)--(10.185,6.300)%
  --(10.289,6.359)--(10.393,6.419)--(10.496,6.478)--(10.600,6.537)--(10.703,6.596)--(10.807,6.656)%
  --(10.911,6.715)--(11.014,6.774)--(11.118,6.833)--(11.222,6.893)--(11.325,6.952)--(11.429,7.011)%
  --(11.532,7.070)--(11.636,7.130)--(11.740,7.189)--(11.843,7.248)--(11.947,7.307);
\gpcolor{color=gp lt color border}
\gpsetlinetype{gp lt border}
\draw[gp path] (1.688,7.825)--(1.688,0.985)--(11.947,0.985)--(11.947,7.825)--cycle;
%% coordinates of the plot area
\gpdefrectangularnode{gp plot 1}{\pgfpoint{1.688cm}{0.985cm}}{\pgfpoint{11.947cm}{7.825cm}}
\end{tikzpicture}
%% gnuplot variables

\caption{Settima serie, decelerazione}
\label{fig:1}
\end{grafico}

\begin{grafico}
    \centering
\begin{tikzpicture}[gnuplot]
%% generated with GNUPLOT 4.6p0 (Lua 5.1; terminal rev. 99, script rev. 100)
%% Tue 15 Apr 2014 09:47:09 PM CEST
\path (0.000,0.000) rectangle (12.500,8.750);
\gpcolor{color=gp lt color border}
\gpsetlinetype{gp lt border}
\gpsetlinewidth{1.00}
\draw[gp path] (1.688,0.985)--(1.868,0.985);
\draw[gp path] (11.947,0.985)--(11.767,0.985);
\node[gp node right] at (1.504,0.985) { 1.1};
\draw[gp path] (1.688,1.607)--(1.868,1.607);
\draw[gp path] (11.947,1.607)--(11.767,1.607);
\node[gp node right] at (1.504,1.607) { 1.12};
\draw[gp path] (1.688,2.229)--(1.868,2.229);
\draw[gp path] (11.947,2.229)--(11.767,2.229);
\node[gp node right] at (1.504,2.229) { 1.14};
\draw[gp path] (1.688,2.850)--(1.868,2.850);
\draw[gp path] (11.947,2.850)--(11.767,2.850);
\node[gp node right] at (1.504,2.850) { 1.16};
\draw[gp path] (1.688,3.472)--(1.868,3.472);
\draw[gp path] (11.947,3.472)--(11.767,3.472);
\node[gp node right] at (1.504,3.472) { 1.18};
\draw[gp path] (1.688,4.094)--(1.868,4.094);
\draw[gp path] (11.947,4.094)--(11.767,4.094);
\node[gp node right] at (1.504,4.094) { 1.2};
\draw[gp path] (1.688,4.716)--(1.868,4.716);
\draw[gp path] (11.947,4.716)--(11.767,4.716);
\node[gp node right] at (1.504,4.716) { 1.22};
\draw[gp path] (1.688,5.338)--(1.868,5.338);
\draw[gp path] (11.947,5.338)--(11.767,5.338);
\node[gp node right] at (1.504,5.338) { 1.24};
\draw[gp path] (1.688,5.960)--(1.868,5.960);
\draw[gp path] (11.947,5.960)--(11.767,5.960);
\node[gp node right] at (1.504,5.960) { 1.26};
\draw[gp path] (1.688,6.581)--(1.868,6.581);
\draw[gp path] (11.947,6.581)--(11.767,6.581);
\node[gp node right] at (1.504,6.581) { 1.28};
\draw[gp path] (1.688,7.203)--(1.868,7.203);
\draw[gp path] (11.947,7.203)--(11.767,7.203);
\node[gp node right] at (1.504,7.203) { 1.3};
\draw[gp path] (1.688,7.825)--(1.868,7.825);
\draw[gp path] (11.947,7.825)--(11.767,7.825);
\node[gp node right] at (1.504,7.825) { 1.32};
\draw[gp path] (1.688,0.985)--(1.688,1.165);
\draw[gp path] (1.688,7.825)--(1.688,7.645);
\node[gp node center] at (1.688,0.677) { 0};
\draw[gp path] (3.154,0.985)--(3.154,1.165);
\draw[gp path] (3.154,7.825)--(3.154,7.645);
\node[gp node center] at (3.154,0.677) { 5};
\draw[gp path] (4.619,0.985)--(4.619,1.165);
\draw[gp path] (4.619,7.825)--(4.619,7.645);
\node[gp node center] at (4.619,0.677) { 10};
\draw[gp path] (6.085,0.985)--(6.085,1.165);
\draw[gp path] (6.085,7.825)--(6.085,7.645);
\node[gp node center] at (6.085,0.677) { 15};
\draw[gp path] (7.550,0.985)--(7.550,1.165);
\draw[gp path] (7.550,7.825)--(7.550,7.645);
\node[gp node center] at (7.550,0.677) { 20};
\draw[gp path] (9.016,0.985)--(9.016,1.165);
\draw[gp path] (9.016,7.825)--(9.016,7.645);
\node[gp node center] at (9.016,0.677) { 25};
\draw[gp path] (10.481,0.985)--(10.481,1.165);
\draw[gp path] (10.481,7.825)--(10.481,7.645);
\node[gp node center] at (10.481,0.677) { 30};
\draw[gp path] (11.947,0.985)--(11.947,1.165);
\draw[gp path] (11.947,7.825)--(11.947,7.645);
\node[gp node center] at (11.947,0.677) { 35};
\draw[gp path] (1.688,7.825)--(1.688,0.985)--(11.947,0.985)--(11.947,7.825)--cycle;
\node[gp node center,rotate=-270] at (0.246,4.405) {Velocità angolare [rad/s]};
\node[gp node center] at (6.817,0.215) {Tempo [s]};
\node[gp node center] at (6.817,8.287) {Velocità angolari in decelerazione [rad/s]};
\node[gp node left] at (7.167,1.627) {Dati};
\gpcolor{color=gp lt color 0}
\gpsetlinetype{gp lt plot 0}
\draw[gp path] (10.663,1.627)--(11.579,1.627);
\draw[gp path] (10.663,1.717)--(10.663,1.537);
\draw[gp path] (11.579,1.717)--(11.579,1.537);
\draw[gp path] (2.069,1.457)--(2.069,7.234);
\draw[gp path] (1.979,1.457)--(2.159,1.457);
\draw[gp path] (1.979,7.234)--(2.159,7.234);
\draw[gp path] (2.450,2.901)--(2.450,5.789);
\draw[gp path] (2.360,2.901)--(2.540,2.901);
\draw[gp path] (2.360,5.789)--(2.540,5.789);
\draw[gp path] (2.846,2.931)--(2.846,4.808);
\draw[gp path] (2.756,2.931)--(2.936,2.931);
\draw[gp path] (2.756,4.808)--(2.936,4.808);
\draw[gp path] (3.227,3.279)--(3.227,4.696);
\draw[gp path] (3.137,3.279)--(3.317,3.279);
\draw[gp path] (3.137,4.696)--(3.317,4.696);
\draw[gp path] (3.608,3.490)--(3.608,4.627);
\draw[gp path] (3.518,3.490)--(3.698,3.490);
\draw[gp path] (3.518,4.627)--(3.698,4.627);
\draw[gp path] (4.018,3.173)--(4.018,4.100);
\draw[gp path] (3.928,3.173)--(4.108,3.173);
\draw[gp path] (3.928,4.100)--(4.108,4.100);
\draw[gp path] (4.399,3.337)--(4.399,4.135);
\draw[gp path] (4.309,3.337)--(4.489,3.337);
\draw[gp path] (4.309,4.135)--(4.489,4.135);
\draw[gp path] (4.854,2.619)--(4.854,3.289);
\draw[gp path] (4.764,2.619)--(4.944,2.619);
\draw[gp path] (4.764,3.289)--(4.944,3.289);
\draw[gp path] (5.220,2.952)--(5.220,3.557);
\draw[gp path] (5.130,2.952)--(5.310,2.952);
\draw[gp path] (5.130,3.557)--(5.310,3.557);
\draw[gp path] (5.630,2.819)--(5.630,3.358);
\draw[gp path] (5.540,2.819)--(5.720,2.819);
\draw[gp path] (5.540,3.358)--(5.720,3.358);
\draw[gp path] (6.041,2.711)--(6.041,3.198);
\draw[gp path] (5.951,2.711)--(6.131,2.711);
\draw[gp path] (5.951,3.198)--(6.131,3.198);
\draw[gp path] (6.451,2.621)--(6.451,3.065);
\draw[gp path] (6.361,2.621)--(6.541,2.621);
\draw[gp path] (6.361,3.065)--(6.541,3.065);
\draw[gp path] (6.861,2.545)--(6.861,2.953);
\draw[gp path] (6.771,2.545)--(6.951,2.545);
\draw[gp path] (6.771,2.953)--(6.951,2.953);
\draw[gp path] (7.272,2.481)--(7.272,2.858);
\draw[gp path] (7.182,2.481)--(7.362,2.481);
\draw[gp path] (7.182,2.858)--(7.362,2.858);
\draw[gp path] (7.668,2.512)--(7.668,2.864);
\draw[gp path] (7.578,2.512)--(7.758,2.512);
\draw[gp path] (7.578,2.864)--(7.758,2.864);
\draw[gp path] (8.122,2.215)--(8.122,2.539);
\draw[gp path] (8.032,2.215)--(8.212,2.215);
\draw[gp path] (8.032,2.539)--(8.212,2.539);
\draw[gp path] (8.532,2.182)--(8.532,2.487);
\draw[gp path] (8.442,2.182)--(8.622,2.182);
\draw[gp path] (8.442,2.487)--(8.622,2.487);
\draw[gp path] (8.957,2.082)--(8.957,2.368);
\draw[gp path] (8.867,2.082)--(9.047,2.082);
\draw[gp path] (8.867,2.368)--(9.047,2.368);
\draw[gp path] (9.382,1.993)--(9.382,2.262);
\draw[gp path] (9.292,1.993)--(9.472,1.993);
\draw[gp path] (9.292,2.262)--(9.472,2.262);
\draw[gp path] (9.807,1.913)--(9.807,2.167);
\draw[gp path] (9.717,1.913)--(9.897,1.913);
\draw[gp path] (9.717,2.167)--(9.897,2.167);
\draw[gp path] (10.247,1.781)--(10.247,2.021);
\draw[gp path] (10.157,1.781)--(10.337,1.781);
\draw[gp path] (10.157,2.021)--(10.337,2.021);
\draw[gp path] (10.687,1.662)--(10.687,1.890);
\draw[gp path] (10.597,1.662)--(10.777,1.662);
\draw[gp path] (10.597,1.890)--(10.777,1.890);
\gpsetpointsize{4.00}
\gppoint{gp mark 1}{(2.069,4.345)}
\gppoint{gp mark 1}{(2.450,4.345)}
\gppoint{gp mark 1}{(2.846,3.870)}
\gppoint{gp mark 1}{(3.227,3.987)}
\gppoint{gp mark 1}{(3.608,4.058)}
\gppoint{gp mark 1}{(4.018,3.637)}
\gppoint{gp mark 1}{(4.399,3.736)}
\gppoint{gp mark 1}{(4.854,2.954)}
\gppoint{gp mark 1}{(5.220,3.254)}
\gppoint{gp mark 1}{(5.630,3.089)}
\gppoint{gp mark 1}{(6.041,2.954)}
\gppoint{gp mark 1}{(6.451,2.843)}
\gppoint{gp mark 1}{(6.861,2.749)}
\gppoint{gp mark 1}{(7.272,2.669)}
\gppoint{gp mark 1}{(7.668,2.688)}
\gppoint{gp mark 1}{(8.122,2.377)}
\gppoint{gp mark 1}{(8.532,2.334)}
\gppoint{gp mark 1}{(8.957,2.225)}
\gppoint{gp mark 1}{(9.382,2.127)}
\gppoint{gp mark 1}{(9.807,2.040)}
\gppoint{gp mark 1}{(10.247,1.901)}
\gppoint{gp mark 1}{(10.687,1.776)}
\gppoint{gp mark 1}{(11.121,1.627)}
\gpcolor{color=gp lt color border}
\node[gp node left] at (7.167,1.319) {Retta interpolante};
\gpcolor{color=gp lt color 1}
\gpsetlinetype{gp lt plot 1}
\draw[gp path] (10.663,1.319)--(11.579,1.319);
\draw[gp path] (2.069,4.057)--(2.156,4.034)--(2.243,4.011)--(2.330,3.988)--(2.417,3.965)%
  --(2.504,3.942)--(2.591,3.919)--(2.678,3.896)--(2.765,3.873)--(2.852,3.850)--(2.940,3.827)%
  --(3.027,3.804)--(3.114,3.781)--(3.201,3.758)--(3.288,3.735)--(3.375,3.712)--(3.462,3.689)%
  --(3.549,3.666)--(3.636,3.643)--(3.723,3.620)--(3.810,3.597)--(3.897,3.574)--(3.984,3.551)%
  --(4.071,3.528)--(4.158,3.505)--(4.245,3.482)--(4.332,3.458)--(4.419,3.435)--(4.506,3.412)%
  --(4.593,3.389)--(4.680,3.366)--(4.767,3.343)--(4.855,3.320)--(4.942,3.297)--(5.029,3.274)%
  --(5.116,3.251)--(5.203,3.228)--(5.290,3.205)--(5.377,3.182)--(5.464,3.159)--(5.551,3.136)%
  --(5.638,3.113)--(5.725,3.090)--(5.812,3.067)--(5.899,3.044)--(5.986,3.021)--(6.073,2.998)%
  --(6.160,2.975)--(6.247,2.952)--(6.334,2.929)--(6.421,2.906)--(6.508,2.883)--(6.595,2.860)%
  --(6.682,2.837)--(6.770,2.813)--(6.857,2.790)--(6.944,2.767)--(7.031,2.744)--(7.118,2.721)%
  --(7.205,2.698)--(7.292,2.675)--(7.379,2.652)--(7.466,2.629)--(7.553,2.606)--(7.640,2.583)%
  --(7.727,2.560)--(7.814,2.537)--(7.901,2.514)--(7.988,2.491)--(8.075,2.468)--(8.162,2.445)%
  --(8.249,2.422)--(8.336,2.399)--(8.423,2.376)--(8.510,2.353)--(8.598,2.330)--(8.685,2.307)%
  --(8.772,2.284)--(8.859,2.261)--(8.946,2.238)--(9.033,2.215)--(9.120,2.191)--(9.207,2.168)%
  --(9.294,2.145)--(9.381,2.122)--(9.468,2.099)--(9.555,2.076)--(9.642,2.053)--(9.729,2.030)%
  --(9.816,2.007)--(9.903,1.984)--(9.990,1.961)--(10.077,1.938)--(10.164,1.915)--(10.251,1.892)%
  --(10.338,1.869)--(10.425,1.846)--(10.513,1.823)--(10.600,1.800)--(10.687,1.777);
\gpcolor{color=gp lt color border}
\gpsetlinetype{gp lt border}
\draw[gp path] (1.688,7.825)--(1.688,0.985)--(11.947,0.985)--(11.947,7.825)--cycle;
%% coordinates of the plot area
\gpdefrectangularnode{gp plot 1}{\pgfpoint{1.688cm}{0.985cm}}{\pgfpoint{11.947cm}{7.825cm}}
\end{tikzpicture}
%% gnuplot variables

\caption{Ottava serie, decelerazione}
\label{fig:1}
\end{grafico}

\begin{grafico}
    \centering
\begin{tikzpicture}[gnuplot]
%% generated with GNUPLOT 4.6p0 (Lua 5.1; terminal rev. 99, script rev. 100)
%% Tue 15 Apr 2014 06:32:35 PM CEST
\path (0.000,0.000) rectangle (12.500,8.750);
\gpcolor{color=gp lt color border}
\gpsetlinetype{gp lt border}
\gpsetlinewidth{1.00}
\draw[gp path] (1.688,0.985)--(1.868,0.985);
\draw[gp path] (11.947,0.985)--(11.767,0.985);
\node[gp node right] at (1.504,0.985) { 1.1};
\draw[gp path] (1.688,1.607)--(1.868,1.607);
\draw[gp path] (11.947,1.607)--(11.767,1.607);
\node[gp node right] at (1.504,1.607) { 1.12};
\draw[gp path] (1.688,2.229)--(1.868,2.229);
\draw[gp path] (11.947,2.229)--(11.767,2.229);
\node[gp node right] at (1.504,2.229) { 1.14};
\draw[gp path] (1.688,2.850)--(1.868,2.850);
\draw[gp path] (11.947,2.850)--(11.767,2.850);
\node[gp node right] at (1.504,2.850) { 1.16};
\draw[gp path] (1.688,3.472)--(1.868,3.472);
\draw[gp path] (11.947,3.472)--(11.767,3.472);
\node[gp node right] at (1.504,3.472) { 1.18};
\draw[gp path] (1.688,4.094)--(1.868,4.094);
\draw[gp path] (11.947,4.094)--(11.767,4.094);
\node[gp node right] at (1.504,4.094) { 1.2};
\draw[gp path] (1.688,4.716)--(1.868,4.716);
\draw[gp path] (11.947,4.716)--(11.767,4.716);
\node[gp node right] at (1.504,4.716) { 1.22};
\draw[gp path] (1.688,5.338)--(1.868,5.338);
\draw[gp path] (11.947,5.338)--(11.767,5.338);
\node[gp node right] at (1.504,5.338) { 1.24};
\draw[gp path] (1.688,5.960)--(1.868,5.960);
\draw[gp path] (11.947,5.960)--(11.767,5.960);
\node[gp node right] at (1.504,5.960) { 1.26};
\draw[gp path] (1.688,6.581)--(1.868,6.581);
\draw[gp path] (11.947,6.581)--(11.767,6.581);
\node[gp node right] at (1.504,6.581) { 1.28};
\draw[gp path] (1.688,7.203)--(1.868,7.203);
\draw[gp path] (11.947,7.203)--(11.767,7.203);
\node[gp node right] at (1.504,7.203) { 1.3};
\draw[gp path] (1.688,7.825)--(1.868,7.825);
\draw[gp path] (11.947,7.825)--(11.767,7.825);
\node[gp node right] at (1.504,7.825) { 1.32};
\draw[gp path] (1.688,0.985)--(1.688,1.165);
\draw[gp path] (1.688,7.825)--(1.688,7.645);
\node[gp node center] at (1.688,0.677) { 0};
\draw[gp path] (2.970,0.985)--(2.970,1.165);
\draw[gp path] (2.970,7.825)--(2.970,7.645);
\node[gp node center] at (2.970,0.677) { 2};
\draw[gp path] (4.253,0.985)--(4.253,1.165);
\draw[gp path] (4.253,7.825)--(4.253,7.645);
\node[gp node center] at (4.253,0.677) { 4};
\draw[gp path] (5.535,0.985)--(5.535,1.165);
\draw[gp path] (5.535,7.825)--(5.535,7.645);
\node[gp node center] at (5.535,0.677) { 6};
\draw[gp path] (6.818,0.985)--(6.818,1.165);
\draw[gp path] (6.818,7.825)--(6.818,7.645);
\node[gp node center] at (6.818,0.677) { 8};
\draw[gp path] (8.100,0.985)--(8.100,1.165);
\draw[gp path] (8.100,7.825)--(8.100,7.645);
\node[gp node center] at (8.100,0.677) { 10};
\draw[gp path] (9.382,0.985)--(9.382,1.165);
\draw[gp path] (9.382,7.825)--(9.382,7.645);
\node[gp node center] at (9.382,0.677) { 12};
\draw[gp path] (10.665,0.985)--(10.665,1.165);
\draw[gp path] (10.665,7.825)--(10.665,7.645);
\node[gp node center] at (10.665,0.677) { 14};
\draw[gp path] (11.947,0.985)--(11.947,1.165);
\draw[gp path] (11.947,7.825)--(11.947,7.645);
\node[gp node center] at (11.947,0.677) { 16};
\draw[gp path] (1.688,7.825)--(1.688,0.985)--(11.947,0.985)--(11.947,7.825)--cycle;
\node[gp node center,rotate=-270] at (0.246,4.405) {Velocità angolare [rad/s]};
\node[gp node center] at (6.817,0.215) {Tempo [s]};
\node[gp node center] at (6.817,8.287) {Velocità angolare, accelerazione [rad/s]};
\node[gp node left] at (7.167,1.627) {Dati};
\gpcolor{color=gp lt color 0}
\gpsetlinetype{gp lt plot 0}
\draw[gp path] (10.663,1.627)--(11.579,1.627);
\draw[gp path] (10.663,1.717)--(10.663,1.537);
\draw[gp path] (11.579,1.717)--(11.579,1.537);
\draw[gp path] (2.522,1.457)--(2.522,7.234);
\draw[gp path] (2.432,1.457)--(2.612,1.457);
\draw[gp path] (2.432,7.234)--(2.612,7.234);
\draw[gp path] (3.323,3.580)--(3.323,6.583);
\draw[gp path] (3.233,3.580)--(3.413,3.580);
\draw[gp path] (3.233,6.583)--(3.413,6.583);
\draw[gp path] (4.157,3.845)--(4.157,5.821);
\draw[gp path] (4.067,3.845)--(4.247,3.845);
\draw[gp path] (4.067,5.821)--(4.247,5.821);
\draw[gp path] (5.022,3.623)--(5.022,5.067);
\draw[gp path] (4.932,3.623)--(5.112,3.623);
\draw[gp path] (4.932,5.067)--(5.112,5.067);
\draw[gp path] (5.888,3.490)--(5.888,4.627);
\draw[gp path] (5.798,3.490)--(5.978,3.490);
\draw[gp path] (5.798,4.627)--(5.978,4.627);
\draw[gp path] (6.721,3.631)--(6.721,4.581);
\draw[gp path] (6.631,3.631)--(6.811,3.631);
\draw[gp path] (6.631,4.581)--(6.811,4.581);
\draw[gp path] (7.555,3.732)--(7.555,4.548);
\draw[gp path] (7.465,3.732)--(7.645,3.732);
\draw[gp path] (7.465,4.548)--(7.645,4.548);
\draw[gp path] (8.453,3.460)--(8.453,4.162);
\draw[gp path] (8.363,3.460)--(8.543,3.460);
\draw[gp path] (8.363,4.162)--(8.543,4.162);
\draw[gp path] (9.254,3.711)--(9.254,4.342);
\draw[gp path] (9.164,3.711)--(9.344,3.711);
\draw[gp path] (9.164,4.342)--(9.344,4.342);
\draw[gp path] (10.152,3.496)--(10.152,4.056);
\draw[gp path] (10.062,3.496)--(10.242,3.496);
\draw[gp path] (10.062,4.056)--(10.242,4.056);
\draw[gp path] (11.017,3.446)--(11.017,3.953);
\draw[gp path] (10.927,3.446)--(11.107,3.446);
\draw[gp path] (10.927,3.953)--(11.107,3.953);
\draw[gp path] (11.915,3.291)--(11.915,3.751);
\draw[gp path] (11.825,3.291)--(12.005,3.291);
\draw[gp path] (11.825,3.751)--(12.005,3.751);
\gpsetpointsize{4.00}
\gppoint{gp mark 1}{(2.522,4.345)}
\gppoint{gp mark 1}{(3.323,5.082)}
\gppoint{gp mark 1}{(4.157,4.833)}
\gppoint{gp mark 1}{(5.022,4.345)}
\gppoint{gp mark 1}{(5.888,4.058)}
\gppoint{gp mark 1}{(6.721,4.106)}
\gppoint{gp mark 1}{(7.555,4.140)}
\gppoint{gp mark 1}{(8.453,3.811)}
\gppoint{gp mark 1}{(9.254,4.027)}
\gppoint{gp mark 1}{(10.152,3.776)}
\gppoint{gp mark 1}{(11.017,3.700)}
\gppoint{gp mark 1}{(11.915,3.521)}
\gppoint{gp mark 1}{(11.121,1.627)}
\gpcolor{color=gp lt color border}
\node[gp node left] at (7.167,1.319) {Retta interpolante};
\gpcolor{color=gp lt color 1}
\gpsetlinetype{gp lt plot 1}
\draw[gp path] (10.663,1.319)--(11.579,1.319);
\draw[gp path] (2.522,4.740)--(2.616,4.728)--(2.711,4.716)--(2.806,4.704)--(2.901,4.692)%
  --(2.996,4.679)--(3.091,4.667)--(3.186,4.655)--(3.281,4.643)--(3.375,4.631)--(3.470,4.619)%
  --(3.565,4.606)--(3.660,4.594)--(3.755,4.582)--(3.850,4.570)--(3.945,4.558)--(4.040,4.546)%
  --(4.135,4.534)--(4.229,4.521)--(4.324,4.509)--(4.419,4.497)--(4.514,4.485)--(4.609,4.473)%
  --(4.704,4.461)--(4.799,4.448)--(4.894,4.436)--(4.988,4.424)--(5.083,4.412)--(5.178,4.400)%
  --(5.273,4.388)--(5.368,4.376)--(5.463,4.363)--(5.558,4.351)--(5.653,4.339)--(5.748,4.327)%
  --(5.842,4.315)--(5.937,4.303)--(6.032,4.290)--(6.127,4.278)--(6.222,4.266)--(6.317,4.254)%
  --(6.412,4.242)--(6.507,4.230)--(6.602,4.218)--(6.696,4.205)--(6.791,4.193)--(6.886,4.181)%
  --(6.981,4.169)--(7.076,4.157)--(7.171,4.145)--(7.266,4.132)--(7.361,4.120)--(7.455,4.108)%
  --(7.550,4.096)--(7.645,4.084)--(7.740,4.072)--(7.835,4.060)--(7.930,4.047)--(8.025,4.035)%
  --(8.120,4.023)--(8.215,4.011)--(8.309,3.999)--(8.404,3.987)--(8.499,3.974)--(8.594,3.962)%
  --(8.689,3.950)--(8.784,3.938)--(8.879,3.926)--(8.974,3.914)--(9.068,3.902)--(9.163,3.889)%
  --(9.258,3.877)--(9.353,3.865)--(9.448,3.853)--(9.543,3.841)--(9.638,3.829)--(9.733,3.816)%
  --(9.828,3.804)--(9.922,3.792)--(10.017,3.780)--(10.112,3.768)--(10.207,3.756)--(10.302,3.744)%
  --(10.397,3.731)--(10.492,3.719)--(10.587,3.707)--(10.681,3.695)--(10.776,3.683)--(10.871,3.671)%
  --(10.966,3.658)--(11.061,3.646)--(11.156,3.634)--(11.251,3.622)--(11.346,3.610)--(11.441,3.598)%
  --(11.535,3.586)--(11.630,3.573)--(11.725,3.561)--(11.820,3.549)--(11.915,3.537);
\gpcolor{color=gp lt color border}
\gpsetlinetype{gp lt border}
\draw[gp path] (1.688,7.825)--(1.688,0.985)--(11.947,0.985)--(11.947,7.825)--cycle;
%% coordinates of the plot area
\gpdefrectangularnode{gp plot 1}{\pgfpoint{1.688cm}{0.985cm}}{\pgfpoint{11.947cm}{7.825cm}}
\end{tikzpicture}
%% gnuplot variables

\caption{Nona serie, decelerazione}
\label{fig:1}
\end{grafico}

\begin{grafico}
    \centering
\begin{tikzpicture}[gnuplot]
%% generated with GNUPLOT 4.6p0 (Lua 5.1; terminal rev. 99, script rev. 100)
%% Tue 15 Apr 2014 09:47:09 PM CEST
\path (0.000,0.000) rectangle (12.500,8.750);
\gpcolor{color=gp lt color border}
\gpsetlinetype{gp lt border}
\gpsetlinewidth{1.00}
\draw[gp path] (1.688,0.985)--(1.868,0.985);
\draw[gp path] (11.947,0.985)--(11.767,0.985);
\node[gp node right] at (1.504,0.985) { 1.05};
\draw[gp path] (1.688,2.125)--(1.868,2.125);
\draw[gp path] (11.947,2.125)--(11.767,2.125);
\node[gp node right] at (1.504,2.125) { 1.1};
\draw[gp path] (1.688,3.265)--(1.868,3.265);
\draw[gp path] (11.947,3.265)--(11.767,3.265);
\node[gp node right] at (1.504,3.265) { 1.15};
\draw[gp path] (1.688,4.405)--(1.868,4.405);
\draw[gp path] (11.947,4.405)--(11.767,4.405);
\node[gp node right] at (1.504,4.405) { 1.2};
\draw[gp path] (1.688,5.545)--(1.868,5.545);
\draw[gp path] (11.947,5.545)--(11.767,5.545);
\node[gp node right] at (1.504,5.545) { 1.25};
\draw[gp path] (1.688,6.685)--(1.868,6.685);
\draw[gp path] (11.947,6.685)--(11.767,6.685);
\node[gp node right] at (1.504,6.685) { 1.3};
\draw[gp path] (1.688,7.825)--(1.868,7.825);
\draw[gp path] (11.947,7.825)--(11.767,7.825);
\node[gp node right] at (1.504,7.825) { 1.35};
\draw[gp path] (1.688,0.985)--(1.688,1.165);
\draw[gp path] (1.688,7.825)--(1.688,7.645);
\node[gp node center] at (1.688,0.677) { 0};
\draw[gp path] (3.154,0.985)--(3.154,1.165);
\draw[gp path] (3.154,7.825)--(3.154,7.645);
\node[gp node center] at (3.154,0.677) { 5};
\draw[gp path] (4.619,0.985)--(4.619,1.165);
\draw[gp path] (4.619,7.825)--(4.619,7.645);
\node[gp node center] at (4.619,0.677) { 10};
\draw[gp path] (6.085,0.985)--(6.085,1.165);
\draw[gp path] (6.085,7.825)--(6.085,7.645);
\node[gp node center] at (6.085,0.677) { 15};
\draw[gp path] (7.550,0.985)--(7.550,1.165);
\draw[gp path] (7.550,7.825)--(7.550,7.645);
\node[gp node center] at (7.550,0.677) { 20};
\draw[gp path] (9.016,0.985)--(9.016,1.165);
\draw[gp path] (9.016,7.825)--(9.016,7.645);
\node[gp node center] at (9.016,0.677) { 25};
\draw[gp path] (10.481,0.985)--(10.481,1.165);
\draw[gp path] (10.481,7.825)--(10.481,7.645);
\node[gp node center] at (10.481,0.677) { 30};
\draw[gp path] (11.947,0.985)--(11.947,1.165);
\draw[gp path] (11.947,7.825)--(11.947,7.645);
\node[gp node center] at (11.947,0.677) { 35};
\draw[gp path] (1.688,7.825)--(1.688,0.985)--(11.947,0.985)--(11.947,7.825)--cycle;
\node[gp node center,rotate=-270] at (0.246,4.405) {Velocità angolare [rad/s]};
\node[gp node center] at (6.817,0.215) {Tempo [s]};
\node[gp node center] at (6.817,8.287) {Velocità angolari in decelerazione [rad/s]};
\node[gp node left] at (7.167,1.627) {Dati};
\gpcolor{color=gp lt color 0}
\gpsetlinetype{gp lt plot 0}
\draw[gp path] (10.663,1.627)--(11.579,1.627);
\draw[gp path] (10.663,1.717)--(10.663,1.537);
\draw[gp path] (11.579,1.717)--(11.579,1.537);
\draw[gp path] (2.069,2.471)--(2.069,6.707);
\draw[gp path] (1.979,2.471)--(2.159,2.471);
\draw[gp path] (1.979,6.707)--(2.159,6.707);
\draw[gp path] (2.435,4.028)--(2.435,6.230);
\draw[gp path] (2.345,4.028)--(2.525,4.028);
\draw[gp path] (2.345,6.230)--(2.525,6.230);
\draw[gp path] (2.816,4.222)--(2.816,5.671);
\draw[gp path] (2.726,4.222)--(2.906,4.222);
\draw[gp path] (2.726,5.671)--(2.906,5.671);
\draw[gp path] (3.183,4.579)--(3.183,5.680);
\draw[gp path] (3.093,4.579)--(3.273,4.579);
\draw[gp path] (3.093,5.680)--(3.273,5.680);
\draw[gp path] (3.564,4.583)--(3.564,5.456);
\draw[gp path] (3.474,4.583)--(3.654,4.583);
\draw[gp path] (3.474,5.456)--(3.654,5.456);
\draw[gp path] (3.945,4.585)--(3.945,5.309);
\draw[gp path] (3.855,4.585)--(4.035,4.585);
\draw[gp path] (3.855,5.309)--(4.035,5.309);
\draw[gp path] (4.341,4.435)--(4.341,5.047);
\draw[gp path] (4.251,4.435)--(4.431,4.435);
\draw[gp path] (4.251,5.047)--(4.431,5.047);
\draw[gp path] (4.736,4.324)--(4.736,4.854);
\draw[gp path] (4.646,4.324)--(4.826,4.324);
\draw[gp path] (4.646,4.854)--(4.826,4.854);
\draw[gp path] (5.117,4.354)--(5.117,4.825);
\draw[gp path] (5.027,4.354)--(5.207,4.354);
\draw[gp path] (5.027,4.825)--(5.207,4.825);
\draw[gp path] (5.484,4.482)--(5.484,4.909);
\draw[gp path] (5.394,4.482)--(5.574,4.482);
\draw[gp path] (5.394,4.909)--(5.574,4.909);
\draw[gp path] (5.880,4.397)--(5.880,4.782);
\draw[gp path] (5.790,4.397)--(5.970,4.397);
\draw[gp path] (5.790,4.782)--(5.970,4.782);
\draw[gp path] (6.290,4.239)--(6.290,4.588);
\draw[gp path] (6.200,4.239)--(6.380,4.239);
\draw[gp path] (6.200,4.588)--(6.380,4.588);
\draw[gp path] (6.671,4.266)--(6.671,4.588);
\draw[gp path] (6.581,4.266)--(6.761,4.266);
\draw[gp path] (6.581,4.588)--(6.761,4.588);
\draw[gp path] (7.081,4.142)--(7.081,4.438);
\draw[gp path] (6.991,4.142)--(7.171,4.142);
\draw[gp path] (6.991,4.438)--(7.171,4.438);
\draw[gp path] (7.492,4.035)--(7.492,4.309);
\draw[gp path] (7.402,4.035)--(7.582,4.035);
\draw[gp path] (7.402,4.309)--(7.582,4.309);
\draw[gp path] (7.873,4.069)--(7.873,4.326);
\draw[gp path] (7.783,4.069)--(7.963,4.069);
\draw[gp path] (7.783,4.326)--(7.963,4.326);
\draw[gp path] (8.298,3.920)--(8.298,4.159);
\draw[gp path] (8.208,3.920)--(8.388,3.920);
\draw[gp path] (8.208,4.159)--(8.388,4.159);
\draw[gp path] (8.708,3.844)--(8.708,4.069);
\draw[gp path] (8.618,3.844)--(8.798,3.844);
\draw[gp path] (8.618,4.069)--(8.798,4.069);
\draw[gp path] (9.118,3.777)--(9.118,3.989);
\draw[gp path] (9.028,3.777)--(9.208,3.777);
\draw[gp path] (9.028,3.989)--(9.208,3.989);
\draw[gp path] (9.646,3.325)--(9.646,3.520);
\draw[gp path] (9.556,3.325)--(9.736,3.325);
\draw[gp path] (9.556,3.520)--(9.736,3.520);
\draw[gp path] (10.350,2.406)--(10.350,2.578);
\draw[gp path] (10.260,2.406)--(10.440,2.406);
\draw[gp path] (10.260,2.578)--(10.440,2.578);
\draw[gp path] (10.980,1.817)--(10.980,1.974);
\draw[gp path] (10.890,1.817)--(11.070,1.817);
\draw[gp path] (10.890,1.974)--(11.070,1.974);
\gpsetpointsize{4.00}
\gppoint{gp mark 1}{(2.069,4.589)}
\gppoint{gp mark 1}{(2.435,5.129)}
\gppoint{gp mark 1}{(2.816,4.947)}
\gppoint{gp mark 1}{(3.183,5.129)}
\gppoint{gp mark 1}{(3.564,5.020)}
\gppoint{gp mark 1}{(3.945,4.947)}
\gppoint{gp mark 1}{(4.341,4.741)}
\gppoint{gp mark 1}{(4.736,4.589)}
\gppoint{gp mark 1}{(5.117,4.589)}
\gppoint{gp mark 1}{(5.484,4.696)}
\gppoint{gp mark 1}{(5.880,4.589)}
\gppoint{gp mark 1}{(6.290,4.414)}
\gppoint{gp mark 1}{(6.671,4.427)}
\gppoint{gp mark 1}{(7.081,4.290)}
\gppoint{gp mark 1}{(7.492,4.172)}
\gppoint{gp mark 1}{(7.873,4.198)}
\gppoint{gp mark 1}{(8.298,4.039)}
\gppoint{gp mark 1}{(8.708,3.957)}
\gppoint{gp mark 1}{(9.118,3.883)}
\gppoint{gp mark 1}{(9.646,3.422)}
\gppoint{gp mark 1}{(10.350,2.492)}
\gppoint{gp mark 1}{(10.980,1.896)}
\gppoint{gp mark 1}{(11.121,1.627)}
\gpcolor{color=gp lt color border}
\node[gp node left] at (7.167,1.319) {Retta interpolante};
\gpcolor{color=gp lt color 1}
\gpsetlinetype{gp lt plot 1}
\draw[gp path] (10.663,1.319)--(11.579,1.319);
\draw[gp path] (2.069,6.515)--(2.159,6.474)--(2.249,6.433)--(2.339,6.392)--(2.429,6.351)%
  --(2.519,6.310)--(2.609,6.269)--(2.699,6.228)--(2.789,6.186)--(2.879,6.145)--(2.969,6.104)%
  --(3.059,6.063)--(3.149,6.022)--(3.239,5.981)--(3.329,5.940)--(3.419,5.899)--(3.509,5.858)%
  --(3.599,5.817)--(3.689,5.776)--(3.779,5.735)--(3.869,5.694)--(3.959,5.653)--(4.049,5.612)%
  --(4.139,5.571)--(4.229,5.529)--(4.319,5.488)--(4.409,5.447)--(4.499,5.406)--(4.589,5.365)%
  --(4.679,5.324)--(4.769,5.283)--(4.859,5.242)--(4.949,5.201)--(5.039,5.160)--(5.129,5.119)%
  --(5.219,5.078)--(5.309,5.037)--(5.399,4.996)--(5.489,4.955)--(5.579,4.913)--(5.669,4.872)%
  --(5.759,4.831)--(5.849,4.790)--(5.939,4.749)--(6.029,4.708)--(6.119,4.667)--(6.209,4.626)%
  --(6.299,4.585)--(6.389,4.544)--(6.479,4.503)--(6.569,4.462)--(6.659,4.421)--(6.749,4.380)%
  --(6.839,4.339)--(6.929,4.297)--(7.019,4.256)--(7.109,4.215)--(7.199,4.174)--(7.289,4.133)%
  --(7.379,4.092)--(7.469,4.051)--(7.559,4.010)--(7.649,3.969)--(7.739,3.928)--(7.829,3.887)%
  --(7.919,3.846)--(8.009,3.805)--(8.100,3.764)--(8.190,3.723)--(8.280,3.682)--(8.370,3.640)%
  --(8.460,3.599)--(8.550,3.558)--(8.640,3.517)--(8.730,3.476)--(8.820,3.435)--(8.910,3.394)%
  --(9.000,3.353)--(9.090,3.312)--(9.180,3.271)--(9.270,3.230)--(9.360,3.189)--(9.450,3.148)%
  --(9.540,3.107)--(9.630,3.066)--(9.720,3.024)--(9.810,2.983)--(9.900,2.942)--(9.990,2.901)%
  --(10.080,2.860)--(10.170,2.819)--(10.260,2.778)--(10.350,2.737)--(10.440,2.696)--(10.530,2.655)%
  --(10.620,2.614)--(10.710,2.573)--(10.800,2.532)--(10.890,2.491)--(10.980,2.450);
\gpcolor{color=gp lt color border}
\gpsetlinetype{gp lt border}
\draw[gp path] (1.688,7.825)--(1.688,0.985)--(11.947,0.985)--(11.947,7.825)--cycle;
%% coordinates of the plot area
\gpdefrectangularnode{gp plot 1}{\pgfpoint{1.688cm}{0.985cm}}{\pgfpoint{11.947cm}{7.825cm}}
\end{tikzpicture}
%% gnuplot variables

\caption{Decima serie, decelerazione}
\label{fig:1}
\end{grafico}

