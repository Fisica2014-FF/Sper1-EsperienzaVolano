\documentclass[12pt,a4paper]{article} % Prepara un documento con un font grande



%----------------------------------------------------------------------------------------
%	PACKAGES AND OTHER DOCUMENT CONFIGURATIONS
%----------------------------------------------------------------------------------------


% Adatta LaTeX alle convenzioni tipografiche italiane,
% e ridefinisce alcuni titoli in italiano, come "Capitolo" al posto di "Chapter",
% se il documento è in italiano
\usepackage[italian]{babel}


\usepackage[utf8]{inputenc} % Consente l'uso caratteri accentati italiani

% Certe cose prese da http://www.latextemplates.com/template/journal-article
\usepackage[sc]{mathpazo} % Use the Palatino font
\linespread{1.05} % Line spacing - Palatino needs more space between lines
\usepackage{microtype} % Slightly tweak font spacing for aesthetics
\usepackage{booktabs} % Horizontal rules in tables
\usepackage[hang, small,labelfont=bf,up,textfont=it,up]{caption} % Custom
% captions under/above floats in tables or figures

\usepackage{amsmath} % Package matematico

\usepackage{graphicx}		% Per le immagini
\usepackage{gnuplot-lua-tikz}
%\usepackage[top=2.5cm, bottom=2cm, left=2cm, right=2cm]{geometry} % Il mio
% geometry originale
\usepackage[hmarginratio=1:1,top=32mm,columnsep=20pt]{geometry} % Document
% margins


%--------------------------------------------------------------------------
% Rinnova i comandi

\usepackage{titlesec} % Allows customization of titles
\renewcommand\thesection{\Roman{section}} % Roman numerals for the sections
\renewcommand\thesubsection{\Roman{subsection}} % Roman numerals for subsections
\titleformat{\section}[block]{\large\scshape\centering}{\thesection.}{1em}{} % Change the look of the section titles
\titleformat{\subsection}[block]{\large}{\thesubsection.}{1em}{} % Change the look of the section titles


%\nonstopmode %non fermarti agli errori

%\usepackage{fancyhdr}
%\setlength{\headheight}{15.2pt}
%\pagestyle{fancy} % Solo le pagine normali, non i titoli nè la pagina iniziale



\usepackage{fancyhdr} % Headers and footers
\pagestyle{fancy} % All pages have headers and footers
	\renewcommand{\sectionmark}[1]{ \markright{#1}{} } % Preso da http://en.wikibooks.org/wiki/LaTeX/Page_Layout#Customizing_with_fancyhdr
    \fancyhf{}% Resetta la formattazione precedente
\fancyhead[C]{Forcher, Piccinin - \emph{Volano} $\bullet$ \thesection.\
\rightmark}
% Custom header text
\fancyfoot[L]{\thepage} % Custom footer text
\fancyfoot[R]{Sec. \thesection} % Custom footer text


\DeclareGraphicsExtensions{.pdf, .png, .jpg} % Se due immagini hanno lo stesso nome sceglile secondo l'ordine di filetype qui
\graphicspath{ {./img/} }					 % Path delle immagini 

\title {Relazione di Laboratorio - Estensimetro}
\author{Francesco F\"{o}rcher\\
Facoltà di Fisica\\
Università di Padova\\
Matricola: \texttt{1073458}\\
\texttt{mailto:francesco.forcher@studenti.unipd.it}\\
\and
Andrea Piccinin\\ 
Facoltà di Fisica\\
Università di Padova\\
Matricola: \texttt{1070620}\\
\texttt{mailto:andrea.piccinin1@studenti.unipd.it}\\
}
\date{\today}

%%%%%%%%%%%%%%%%%%%%%%%%%%%%%%%%%%%%%%5%%%%%%%%%%%%%%%%%%%%%%%%%%%%%%%%%%%%%%%%%%%
\usepackage{float}
\usepackage{caption}
%\usepackage{multirow}
%\usepackage[top=3.6cm, bottom=1.5in, left=0.5in, right=0.5in]{geometry}


% I miei stili di float, con le righe
\floatstyle{ruled}
\newfloat{tabella}{H}{lop}
\floatname{tabella}{Tabella}

\floatstyle{ruled}
\newfloat{grafico}{H}{lop}
\floatname{grafico}{Grafico}
%%%%%%%%%%%%%%%%%%%%%%%%%%%%%%%%%%%%%%%%%%%%%%%%%%%%%%%%%%%%%%%%%%%%%5%%%%%%%%%%%%%



%////////////////////////////////////////////////////////////////////////////////////////////////////////////////////////////
%////////////////////////////////////////////////////////////////////////////////////////////////////////////////////////////
% Fine dei dati iniziali per il latex: il documento finale inizierà da qui
\begin{document}

\maketitle % Produce il titolo a partire dai comandi \title, \author e \date
%\tableofcontents % Prepara l'indice generale

% Le varie sezioni
\section{Obiettivi}
	


\section{Descrizione dell'apparato strumentale}
	In quest'esperienza si è utilizzato, come apparato sperimentale, un volano (volano n° 7F). Esso è composto 


\section{Metodologia di misura}
	\input{./sezioni/metodologia_misura.tex}
	\clearpage
	
\section{Presentazione dei dati}			
	\subsection{Tabelle}
	\begin{tabella}
    \centering
\begin{tabular}{ r r l|l r r }
\hline
\multicolumn{1}{|l|}{\alpha} & \multicolumn{1}{l|}{\sigma \ped{alpha}} &  &  & \multicolumn{1}{l|}{\beta} & \multicolumn{1}{l|}{\sigma \ped{beta}} \\ \hline
0.02093 & 0.00007 &  &  & -0.005 & 0.001 \\ \hline
0.02080 & 0.00003 &  &  & -0.0023 & 0.0002 \\ \hline
0.02026 & 0.00007 &  &  & -0.0020 & 0.0004 \\ \hline
0.02064 & 0.00009 &  &  & -0.0029 & 0.0005 \\ \hline
0.0208 & 0.0001 &  &  & -0.0027 & 0.0005 \\ \hline
0.02063 & 0.00006 &  &  & -0.0020 & 0.0006 \\ \hline
0.0208 & 0.0001 &  &  & -0.0032 & 0.0005 \\ \hline
0.0210 & 0.0004 &  &  & -0.0034 & 0.0004 \\ \hline
0.02064 & 0.00005 &  &  & -0.0026 & 0.0005 \\ \hline
0.02052 & 0.00004 &  &  & -0.0015 & 0.0005 \\ \hline
\end{tabular}


\caption{Accelerazioni e decelerazioni risultanti}
\label{tab:dec}
\end{tabella}

\begin{tabella}
    \centering
\begin{tabular}{r r r r r r r r r r}
\hline
\multicolumn{ 10}{|c|}{Misure ripetute Tempi Accelerazione, \pm 0.1  [s]} \\ \hline
16.8 & 16.9 & 15.7 & 15.9 & 15.6 & 16.0 & 16.8 & 16.1 & 15.8 & 16.3 \\ \hline
24.2 & 24.0 & 22.8 & 23.4 & 23.0 & 23.2 & 23.9 & 23.2 & 23.0 & 23.6 \\ \hline
29.8 & 29.6 & 28.3 & 28.8 & 28.4 & 28.8 & 29.5 & 28.7 & 28.6 & 29.2 \\ \hline
34.6 & 34.2 & 32.9 & 33.5 & 33.1 & 33.4 & 34.1 & 33.3 & 33.3 & 33.9 \\ \hline
38.7 & 38.3 & 37.1 & 37.7 & 37.2 & 37.5 & 38.3 & 37.5 & 37.3 & 38.0 \\ \hline
42.4 & 42.0 & 40.6 & 41.4 & 40.8 & 41.3 & 42.3 & 41.1 & 41.0 & 41.7 \\ \hline
45.9 & 45.5 & 44.1 & 44.8 & 44.3 & 44.7 & 45.4 & 44.3 & 44.5 & 45.1 \\ \hline
49.1 & 48.6 & 47.2 & 47.9 & 47.3 & 47.8 & 48.5 & 46.4 & 47.6 & 48.3 \\ \hline
52.1 & 51.6 & 50.3 & 51.0 & 50.5 & 50.8 & 51.5 & 50.7 & 50.7 & 51.3 \\ \hline
54.8 & 54.5 & 53.0 & 53.6 & 53.3 & 53.8 & 54.3 & 53.5 & 53.4 & 54.2 \\ \hline
57.8 & 57.2 & 55.7 & 56.4 & 56.0 & 56.3 & 57.1 & 56.3 & 56.2 & 56.9 \\ \hline
60.3 & 59.6 & 58.4 & 59.0 & 58.3 & 59.0 & 59.6 & 58.8 & 58.8 & 59.4 \\ \hline
\end{tabular}


\caption{Tempi in accelerazione}
\label{tab:acc}
\end{tabella}

\begin{tabella}
    \centering
\begin{tabular}{r r r r r r r r r r}
\hline
\multicolumn{ 10}{|c|}{Misure ripetute Tempi Decelerazione, \pm 0.1  [s]} \\ \hline
2.4 & 2.5 & 2.6 & 2.8 & 2.5 & 2.6 & 2.5 & 2.6 & 2.6 & 2.6 \\ \hline
5.1 & 5.1 & 5.2 & 5.5 & 5.1 & 5.1 & 5.1 & 5.2 & 5.1 & 5.1 \\ \hline
7.7 & 7.7 & 7.8 & 8.2 & 7.7 & 7.6 & 7.7 & 7.9 & 7.7 & 7.7 \\ \hline
10.2 & 10.3 & 10.6 & 11.1 & 10.2 & 10.2 & 10.3 & 10.5 & 10.4 & 10.2 \\ \hline
12.9 & 12.8 & 13.1 & 13.9 & 12.9 & 12.9 & 13.0 & 13.1 & 13.1 & 12.8 \\ \hline
15.5 & 15.4 & 15.7 & 16.7 & 15.4 & 15.5 & 15.6 & 15.9 & 15.7 & 15.4 \\ \hline
18.1 & 18.1 & 18.4 & 19.5 & 18.1 & 18.1 & 18.2 & 18.5 & 18.3 & 18.1 \\ \hline
20.8 & 20.8 & 21.1 & 22.3 & 20.7 & 20.8 & 20.9 & 21.6 & 21.1 & 20.8 \\ \hline
23.4 & 23.3 & 23.8 & 25.2 & 23.3 & 23.4 & 23.5 & 24.1 & 23.6 & 23.4 \\ \hline
26.0 & 26.0 & 26.4 & 28.1 & 25.9 & 26.1 & 26.2 & 26.9 & 26.4 & 25.9 \\ \hline
28.7 & 28.7 & 29.2 & 30.9 & 28.6 & 28.7 & 28.8 & 29.7 & 29.1 & 28.6 \\ \hline
31.5 & 31.4 & 32.0 & 33.9 & 31.4 & 31.4 & 31.5 & 32.5 & 31.9 & 31.4 \\ \hline
34.3 & 34.2 & 34.8 & 36.9 & 34.0 & 34.2 & 33.6 & 35.3 & 34.5 & 34.0 \\ \hline
36.9 & 36.9 & 37.6 & 39.9 & 36.5 & 36.8 & 35.5 & 38.1 & 37.1 & 36.8 \\ \hline
39.7 & 39.6 & 40.3 & 42.8 & 39.4 & 39.6 & 39.8 & 40.8 & 40.1 & 39.6 \\ \hline
42.4 & 42.3 & 43.1 & 45.9 & 42.1 & 42.3 & 42.6 & 43.9 & 43.1 & 42.2 \\ \hline
45.3 & 45.1 & 45.9 & 48.9 & 44.8 & 45.1 & 45.4 & 46.7 & 45.6 & 45.1 \\ \hline
48.1 & 47.9 & 48.8 & 51.9 & 47.8 & 48.0 & 48.2 & 49.6 & 48.5 & 47.9 \\ \hline
50.9 & 50.7 & 51.8 & 54.9 & 50.5 & 50.7 & 51.0 & 52.5 & 51.3 & 50.7 \\ \hline
53.7 & 53.6 & 54.6 & 58.1 & 53.3 & 53.6 & 53.8 & 55.4 & 54.1 & 54.3 \\ \hline
56.6 & 56.3 & 57.5 & 61.1 & 56.2 & 56.5 & 56.7 & 58.4 & 57.1 & 59.1 \\ \hline
59.4 & 59.2 & 60.3 & 64.3 & 58.9 & 59.2 & 59.6 & 61.4 & 60.0 & 63.4 \\ \hline
\end{tabular}

\caption{Tempi in decelerazione}
\label{tab:dec}
\end{tabella}


	
	\clearpage
	\subsection{Grafici}
	Qui vengono riportati i grafici di ciascuna sessione di presa dati, con la relativa interpolazione lineare per ricavare $\alpha$ e $\beta$.
Per primi i grafici relativi alle accelerazioni. L'errore è abbastanza buono su questi, come si vede anche dagli stessi.

\begin{grafico}
    \centering
\begin{tikzpicture}[gnuplot]
%% generated with GNUPLOT 4.6p0 (Lua 5.1; terminal rev. 99, script rev. 100)
%% Tue 15 Apr 2014 09:47:06 PM CEST
\path (0.000,0.000) rectangle (12.500,8.750);
\gpcolor{color=gp lt color border}
\gpsetlinetype{gp lt border}
\gpsetlinewidth{1.00}
\draw[gp path] (1.688,0.985)--(1.868,0.985);
\draw[gp path] (11.947,0.985)--(11.767,0.985);
\node[gp node right] at (1.504,0.985) { 0.15};
\draw[gp path] (1.688,1.669)--(1.868,1.669);
\draw[gp path] (11.947,1.669)--(11.767,1.669);
\node[gp node right] at (1.504,1.669) { 0.2};
\draw[gp path] (1.688,2.353)--(1.868,2.353);
\draw[gp path] (11.947,2.353)--(11.767,2.353);
\node[gp node right] at (1.504,2.353) { 0.25};
\draw[gp path] (1.688,3.037)--(1.868,3.037);
\draw[gp path] (11.947,3.037)--(11.767,3.037);
\node[gp node right] at (1.504,3.037) { 0.3};
\draw[gp path] (1.688,3.721)--(1.868,3.721);
\draw[gp path] (11.947,3.721)--(11.767,3.721);
\node[gp node right] at (1.504,3.721) { 0.35};
\draw[gp path] (1.688,4.405)--(1.868,4.405);
\draw[gp path] (11.947,4.405)--(11.767,4.405);
\node[gp node right] at (1.504,4.405) { 0.4};
\draw[gp path] (1.688,5.089)--(1.868,5.089);
\draw[gp path] (11.947,5.089)--(11.767,5.089);
\node[gp node right] at (1.504,5.089) { 0.45};
\draw[gp path] (1.688,5.773)--(1.868,5.773);
\draw[gp path] (11.947,5.773)--(11.767,5.773);
\node[gp node right] at (1.504,5.773) { 0.5};
\draw[gp path] (1.688,6.457)--(1.868,6.457);
\draw[gp path] (11.947,6.457)--(11.767,6.457);
\node[gp node right] at (1.504,6.457) { 0.55};
\draw[gp path] (1.688,7.141)--(1.868,7.141);
\draw[gp path] (11.947,7.141)--(11.767,7.141);
\node[gp node right] at (1.504,7.141) { 0.6};
\draw[gp path] (1.688,7.825)--(1.868,7.825);
\draw[gp path] (11.947,7.825)--(11.767,7.825);
\node[gp node right] at (1.504,7.825) { 0.65};
\draw[gp path] (1.688,0.985)--(1.688,1.165);
\draw[gp path] (1.688,7.825)--(1.688,7.645);
\node[gp node center] at (1.688,0.677) { 5};
\draw[gp path] (3.740,0.985)--(3.740,1.165);
\draw[gp path] (3.740,7.825)--(3.740,7.645);
\node[gp node center] at (3.740,0.677) { 10};
\draw[gp path] (5.792,0.985)--(5.792,1.165);
\draw[gp path] (5.792,7.825)--(5.792,7.645);
\node[gp node center] at (5.792,0.677) { 15};
\draw[gp path] (7.843,0.985)--(7.843,1.165);
\draw[gp path] (7.843,7.825)--(7.843,7.645);
\node[gp node center] at (7.843,0.677) { 20};
\draw[gp path] (9.895,0.985)--(9.895,1.165);
\draw[gp path] (9.895,7.825)--(9.895,7.645);
\node[gp node center] at (9.895,0.677) { 25};
\draw[gp path] (11.947,0.985)--(11.947,1.165);
\draw[gp path] (11.947,7.825)--(11.947,7.645);
\node[gp node center] at (11.947,0.677) { 30};
\draw[gp path] (1.688,7.825)--(1.688,0.985)--(11.947,0.985)--(11.947,7.825)--cycle;
\node[gp node center,rotate=-270] at (0.246,4.405) {Velocità angolare [rad/s]};
\node[gp node center] at (6.817,0.215) {Tempo [s]};
\node[gp node center] at (6.817,8.287) {Velocità angolari in accelerazione [rad/s]};
\node[gp node left] at (7.167,1.627) {Dati};
\gpcolor{color=gp lt color 0}
\gpsetlinetype{gp lt plot 0}
\draw[gp path] (10.663,1.627)--(11.579,1.627);
\draw[gp path] (10.663,1.717)--(10.663,1.537);
\draw[gp path] (11.579,1.717)--(11.579,1.537);
\draw[gp path] (2.858,1.635)--(2.858,1.705);
\draw[gp path] (2.768,1.635)--(2.948,1.635);
\draw[gp path] (2.768,1.705)--(2.948,1.705);
\draw[gp path] (4.314,2.669)--(4.314,2.735);
\draw[gp path] (4.224,2.669)--(4.404,2.669);
\draw[gp path] (4.224,2.735)--(4.404,2.735);
\draw[gp path] (5.443,3.456)--(5.443,3.520);
\draw[gp path] (5.353,3.456)--(5.533,3.456);
\draw[gp path] (5.353,3.520)--(5.533,3.520);
\draw[gp path] (6.387,4.125)--(6.387,4.189);
\draw[gp path] (6.297,4.125)--(6.477,4.125);
\draw[gp path] (6.297,4.189)--(6.477,4.189);
\draw[gp path] (7.248,4.693)--(7.248,4.755);
\draw[gp path] (7.158,4.693)--(7.338,4.693);
\draw[gp path] (7.158,4.755)--(7.338,4.755);
\draw[gp path] (7.967,5.252)--(7.967,5.314);
\draw[gp path] (7.877,5.252)--(8.057,5.252);
\draw[gp path] (7.877,5.314)--(8.057,5.314);
\draw[gp path] (8.685,5.723)--(8.685,5.784);
\draw[gp path] (8.595,5.723)--(8.775,5.723);
\draw[gp path] (8.595,5.784)--(8.775,5.784);
\draw[gp path] (9.321,6.185)--(9.321,6.247);
\draw[gp path] (9.231,6.185)--(9.411,6.185);
\draw[gp path] (9.231,6.247)--(9.411,6.247);
\draw[gp path] (9.957,6.591)--(9.957,6.652);
\draw[gp path] (9.867,6.591)--(10.047,6.591);
\draw[gp path] (9.867,6.652)--(10.047,6.652);
\draw[gp path] (10.511,7.010)--(10.511,7.071);
\draw[gp path] (10.421,7.010)--(10.601,7.010);
\draw[gp path] (10.421,7.071)--(10.601,7.071);
\draw[gp path] (11.065,7.388)--(11.065,7.449);
\draw[gp path] (10.975,7.388)--(11.155,7.388);
\draw[gp path] (10.975,7.449)--(11.155,7.449);
\draw[gp path] (11.619,7.732)--(11.619,7.792);
\draw[gp path] (11.529,7.732)--(11.709,7.732);
\draw[gp path] (11.529,7.792)--(11.709,7.792);
\gpsetpointsize{4.00}
\gppoint{gp mark 1}{(2.858,1.670)}
\gppoint{gp mark 1}{(4.314,2.702)}
\gppoint{gp mark 1}{(5.443,3.488)}
\gppoint{gp mark 1}{(6.387,4.157)}
\gppoint{gp mark 1}{(7.248,4.724)}
\gppoint{gp mark 1}{(7.967,5.283)}
\gppoint{gp mark 1}{(8.685,5.753)}
\gppoint{gp mark 1}{(9.321,6.216)}
\gppoint{gp mark 1}{(9.957,6.621)}
\gppoint{gp mark 1}{(10.511,7.040)}
\gppoint{gp mark 1}{(11.065,7.419)}
\gppoint{gp mark 1}{(11.619,7.762)}
\gppoint{gp mark 1}{(11.121,1.627)}
\gpcolor{color=gp lt color border}
\node[gp node left] at (7.167,1.319) {Retta interpolante};
\gpcolor{color=gp lt color 1}
\gpsetlinetype{gp lt plot 1}
\draw[gp path] (10.663,1.319)--(11.579,1.319);
\draw[gp path] (2.858,1.686)--(2.946,1.748)--(3.035,1.809)--(3.123,1.871)--(3.212,1.933)%
  --(3.300,1.994)--(3.389,2.056)--(3.477,2.118)--(3.566,2.180)--(3.654,2.241)--(3.742,2.303)%
  --(3.831,2.365)--(3.919,2.427)--(4.008,2.488)--(4.096,2.550)--(4.185,2.612)--(4.273,2.673)%
  --(4.362,2.735)--(4.450,2.797)--(4.539,2.859)--(4.627,2.920)--(4.716,2.982)--(4.804,3.044)%
  --(4.893,3.106)--(4.981,3.167)--(5.070,3.229)--(5.158,3.291)--(5.247,3.353)--(5.335,3.414)%
  --(5.424,3.476)--(5.512,3.538)--(5.601,3.599)--(5.689,3.661)--(5.778,3.723)--(5.866,3.785)%
  --(5.955,3.846)--(6.043,3.908)--(6.132,3.970)--(6.220,4.032)--(6.309,4.093)--(6.397,4.155)%
  --(6.486,4.217)--(6.574,4.279)--(6.663,4.340)--(6.751,4.402)--(6.840,4.464)--(6.928,4.525)%
  --(7.017,4.587)--(7.105,4.649)--(7.194,4.711)--(7.282,4.772)--(7.371,4.834)--(7.459,4.896)%
  --(7.548,4.958)--(7.636,5.019)--(7.725,5.081)--(7.813,5.143)--(7.902,5.204)--(7.990,5.266)%
  --(8.079,5.328)--(8.167,5.390)--(8.256,5.451)--(8.344,5.513)--(8.433,5.575)--(8.521,5.637)%
  --(8.610,5.698)--(8.698,5.760)--(8.787,5.822)--(8.875,5.884)--(8.964,5.945)--(9.052,6.007)%
  --(9.141,6.069)--(9.229,6.130)--(9.318,6.192)--(9.406,6.254)--(9.495,6.316)--(9.583,6.377)%
  --(9.672,6.439)--(9.760,6.501)--(9.849,6.563)--(9.937,6.624)--(10.026,6.686)--(10.114,6.748)%
  --(10.203,6.809)--(10.291,6.871)--(10.380,6.933)--(10.468,6.995)--(10.557,7.056)--(10.645,7.118)%
  --(10.734,7.180)--(10.822,7.242)--(10.911,7.303)--(10.999,7.365)--(11.088,7.427)--(11.176,7.489)%
  --(11.265,7.550)--(11.353,7.612)--(11.442,7.674)--(11.530,7.735)--(11.619,7.797);
\gpcolor{color=gp lt color border}
\gpsetlinetype{gp lt border}
\draw[gp path] (1.688,7.825)--(1.688,0.985)--(11.947,0.985)--(11.947,7.825)--cycle;
%% coordinates of the plot area
\gpdefrectangularnode{gp plot 1}{\pgfpoint{1.688cm}{0.985cm}}{\pgfpoint{11.947cm}{7.825cm}}
\end{tikzpicture}
%% gnuplot variables

\caption{Prima serie, accelerazione}
\label{fig:1}
\end{grafico}

\begin{grafico}
    \centering
\begin{tikzpicture}[gnuplot]
%% generated with GNUPLOT 4.6p0 (Lua 5.1; terminal rev. 99, script rev. 100)
%% Mon 14 Apr 2014 11:09:57 PM CEST
\path (0.000,0.000) rectangle (12.500,8.750);
\gpcolor{color=gp lt color border}
\gpsetlinetype{gp lt border}
\gpsetlinewidth{1.00}
\draw[gp path] (1.320,0.985)--(1.500,0.985);
\draw[gp path] (11.947,0.985)--(11.767,0.985);
\node[gp node right] at (1.136,0.985) { 10};
\draw[gp path] (1.320,1.745)--(1.500,1.745);
\draw[gp path] (11.947,1.745)--(11.767,1.745);
\node[gp node right] at (1.136,1.745) { 15};
\draw[gp path] (1.320,2.505)--(1.500,2.505);
\draw[gp path] (11.947,2.505)--(11.767,2.505);
\node[gp node right] at (1.136,2.505) { 20};
\draw[gp path] (1.320,3.265)--(1.500,3.265);
\draw[gp path] (11.947,3.265)--(11.767,3.265);
\node[gp node right] at (1.136,3.265) { 25};
\draw[gp path] (1.320,4.025)--(1.500,4.025);
\draw[gp path] (11.947,4.025)--(11.767,4.025);
\node[gp node right] at (1.136,4.025) { 30};
\draw[gp path] (1.320,4.785)--(1.500,4.785);
\draw[gp path] (11.947,4.785)--(11.767,4.785);
\node[gp node right] at (1.136,4.785) { 35};
\draw[gp path] (1.320,5.545)--(1.500,5.545);
\draw[gp path] (11.947,5.545)--(11.767,5.545);
\node[gp node right] at (1.136,5.545) { 40};
\draw[gp path] (1.320,6.305)--(1.500,6.305);
\draw[gp path] (11.947,6.305)--(11.767,6.305);
\node[gp node right] at (1.136,6.305) { 45};
\draw[gp path] (1.320,7.065)--(1.500,7.065);
\draw[gp path] (11.947,7.065)--(11.767,7.065);
\node[gp node right] at (1.136,7.065) { 50};
\draw[gp path] (1.320,7.825)--(1.500,7.825);
\draw[gp path] (11.947,7.825)--(11.767,7.825);
\node[gp node right] at (1.136,7.825) { 55};
\draw[gp path] (1.320,0.985)--(1.320,1.165);
\draw[gp path] (1.320,7.825)--(1.320,7.645);
\node[gp node center] at (1.320,0.677) {-10};
\draw[gp path] (3.977,0.985)--(3.977,1.165);
\draw[gp path] (3.977,7.825)--(3.977,7.645);
\node[gp node center] at (3.977,0.677) {-5};
\draw[gp path] (6.634,0.985)--(6.634,1.165);
\draw[gp path] (6.634,7.825)--(6.634,7.645);
\node[gp node center] at (6.634,0.677) { 0};
\draw[gp path] (9.290,0.985)--(9.290,1.165);
\draw[gp path] (9.290,7.825)--(9.290,7.645);
\node[gp node center] at (9.290,0.677) { 5};
\draw[gp path] (11.947,0.985)--(11.947,1.165);
\draw[gp path] (11.947,7.825)--(11.947,7.645);
\node[gp node center] at (11.947,0.677) { 10};
\draw[gp path] (1.320,7.825)--(1.320,0.985)--(11.947,0.985)--(11.947,7.825)--cycle;
\node[gp node center,rotate=-270] at (0.246,4.405) {Velocità angolare [rad/s]};
\node[gp node center] at (6.633,0.215) {Tempo [s]};
\node[gp node center] at (6.633,8.287) {Velocità angolare, decelerazione [rad/s]};
\node[gp node left] at (7.167,1.627) {Retta interpolante};
\gpcolor{color=gp lt color 1}
\gpsetlinetype{gp lt plot 1}
\draw[gp path] (10.663,1.627)--(11.579,1.627);
\draw[gp path] (1.320,7.251)--(1.427,7.188)--(1.535,7.125)--(1.642,7.062)--(1.749,6.998)%
  --(1.857,6.935)--(1.964,6.872)--(2.071,6.809)--(2.179,6.746)--(2.286,6.682)--(2.393,6.619)%
  --(2.501,6.556)--(2.608,6.493)--(2.715,6.430)--(2.823,6.366)--(2.930,6.303)--(3.037,6.240)%
  --(3.145,6.177)--(3.252,6.114)--(3.360,6.050)--(3.467,5.987)--(3.574,5.924)--(3.682,5.861)%
  --(3.789,5.798)--(3.896,5.734)--(4.004,5.671)--(4.111,5.608)--(4.218,5.545)--(4.326,5.482)%
  --(4.433,5.418)--(4.540,5.355)--(4.648,5.292)--(4.755,5.229)--(4.862,5.166)--(4.970,5.102)%
  --(5.077,5.039)--(5.184,4.976)--(5.292,4.913)--(5.399,4.850)--(5.506,4.786)--(5.614,4.723)%
  --(5.721,4.660)--(5.828,4.597)--(5.936,4.534)--(6.043,4.470)--(6.150,4.407)--(6.258,4.344)%
  --(6.365,4.281)--(6.472,4.217)--(6.580,4.154)--(6.687,4.091)--(6.795,4.028)--(6.902,3.965)%
  --(7.009,3.901)--(7.117,3.838)--(7.224,3.775)--(7.331,3.712)--(7.439,3.649)--(7.546,3.585)%
  --(7.653,3.522)--(7.761,3.459)--(7.868,3.396)--(7.975,3.333)--(8.083,3.269)--(8.190,3.206)%
  --(8.297,3.143)--(8.405,3.080)--(8.512,3.017)--(8.619,2.953)--(8.727,2.890)--(8.834,2.827)%
  --(8.941,2.764)--(9.049,2.701)--(9.156,2.637)--(9.263,2.574)--(9.371,2.511)--(9.478,2.448)%
  --(9.585,2.385)--(9.693,2.321)--(9.800,2.258)--(9.907,2.195)--(10.015,2.132)--(10.122,2.069)%
  --(10.230,2.005)--(10.337,1.942)--(10.444,1.879)--(10.552,1.816)--(10.659,1.753)--(10.766,1.689)%
  --(10.874,1.626)--(10.981,1.563)--(11.088,1.500)--(11.196,1.437)--(11.303,1.373)--(11.410,1.310)%
  --(11.518,1.247)--(11.625,1.184)--(11.732,1.121)--(11.840,1.057)--(11.947,0.994);
\gpcolor{color=gp lt color border}
\gpsetlinetype{gp lt border}
\draw[gp path] (1.320,7.825)--(1.320,0.985)--(11.947,0.985)--(11.947,7.825)--cycle;
%% coordinates of the plot area
\gpdefrectangularnode{gp plot 1}{\pgfpoint{1.320cm}{0.985cm}}{\pgfpoint{11.947cm}{7.825cm}}
\end{tikzpicture}
%% gnuplot variables

\caption{Seconda serie, accelerazione}
\label{fig:1}
\end{grafico}

\begin{grafico}
    \centering
\begin{tikzpicture}[gnuplot]
%% generated with GNUPLOT 4.6p0 (Lua 5.1; terminal rev. 99, script rev. 100)
%% Mon 14 Apr 2014 11:09:57 PM CEST
\path (0.000,0.000) rectangle (12.500,8.750);
\gpcolor{color=gp lt color border}
\gpsetlinetype{gp lt border}
\gpsetlinewidth{1.00}
\draw[gp path] (1.320,0.985)--(1.500,0.985);
\draw[gp path] (11.947,0.985)--(11.767,0.985);
\node[gp node right] at (1.136,0.985) { 10};
\draw[gp path] (1.320,1.745)--(1.500,1.745);
\draw[gp path] (11.947,1.745)--(11.767,1.745);
\node[gp node right] at (1.136,1.745) { 15};
\draw[gp path] (1.320,2.505)--(1.500,2.505);
\draw[gp path] (11.947,2.505)--(11.767,2.505);
\node[gp node right] at (1.136,2.505) { 20};
\draw[gp path] (1.320,3.265)--(1.500,3.265);
\draw[gp path] (11.947,3.265)--(11.767,3.265);
\node[gp node right] at (1.136,3.265) { 25};
\draw[gp path] (1.320,4.025)--(1.500,4.025);
\draw[gp path] (11.947,4.025)--(11.767,4.025);
\node[gp node right] at (1.136,4.025) { 30};
\draw[gp path] (1.320,4.785)--(1.500,4.785);
\draw[gp path] (11.947,4.785)--(11.767,4.785);
\node[gp node right] at (1.136,4.785) { 35};
\draw[gp path] (1.320,5.545)--(1.500,5.545);
\draw[gp path] (11.947,5.545)--(11.767,5.545);
\node[gp node right] at (1.136,5.545) { 40};
\draw[gp path] (1.320,6.305)--(1.500,6.305);
\draw[gp path] (11.947,6.305)--(11.767,6.305);
\node[gp node right] at (1.136,6.305) { 45};
\draw[gp path] (1.320,7.065)--(1.500,7.065);
\draw[gp path] (11.947,7.065)--(11.767,7.065);
\node[gp node right] at (1.136,7.065) { 50};
\draw[gp path] (1.320,7.825)--(1.500,7.825);
\draw[gp path] (11.947,7.825)--(11.767,7.825);
\node[gp node right] at (1.136,7.825) { 55};
\draw[gp path] (1.320,0.985)--(1.320,1.165);
\draw[gp path] (1.320,7.825)--(1.320,7.645);
\node[gp node center] at (1.320,0.677) {-10};
\draw[gp path] (3.977,0.985)--(3.977,1.165);
\draw[gp path] (3.977,7.825)--(3.977,7.645);
\node[gp node center] at (3.977,0.677) {-5};
\draw[gp path] (6.634,0.985)--(6.634,1.165);
\draw[gp path] (6.634,7.825)--(6.634,7.645);
\node[gp node center] at (6.634,0.677) { 0};
\draw[gp path] (9.290,0.985)--(9.290,1.165);
\draw[gp path] (9.290,7.825)--(9.290,7.645);
\node[gp node center] at (9.290,0.677) { 5};
\draw[gp path] (11.947,0.985)--(11.947,1.165);
\draw[gp path] (11.947,7.825)--(11.947,7.645);
\node[gp node center] at (11.947,0.677) { 10};
\draw[gp path] (1.320,7.825)--(1.320,0.985)--(11.947,0.985)--(11.947,7.825)--cycle;
\node[gp node center,rotate=-270] at (0.246,4.405) {Velocità angolare [rad/s]};
\node[gp node center] at (6.633,0.215) {Tempo [s]};
\node[gp node center] at (6.633,8.287) {Velocità angolare, decelerazione [rad/s]};
\node[gp node left] at (7.167,1.627) {Retta interpolante};
\gpcolor{color=gp lt color 1}
\gpsetlinetype{gp lt plot 1}
\draw[gp path] (10.663,1.627)--(11.579,1.627);
\draw[gp path] (1.320,7.303)--(1.427,7.239)--(1.535,7.176)--(1.642,7.112)--(1.749,7.048)%
  --(1.857,6.985)--(1.964,6.921)--(2.071,6.857)--(2.179,6.794)--(2.286,6.730)--(2.393,6.666)%
  --(2.501,6.603)--(2.608,6.539)--(2.715,6.476)--(2.823,6.412)--(2.930,6.348)--(3.037,6.285)%
  --(3.145,6.221)--(3.252,6.157)--(3.360,6.094)--(3.467,6.030)--(3.574,5.966)--(3.682,5.903)%
  --(3.789,5.839)--(3.896,5.775)--(4.004,5.712)--(4.111,5.648)--(4.218,5.584)--(4.326,5.521)%
  --(4.433,5.457)--(4.540,5.393)--(4.648,5.330)--(4.755,5.266)--(4.862,5.202)--(4.970,5.139)%
  --(5.077,5.075)--(5.184,5.011)--(5.292,4.948)--(5.399,4.884)--(5.506,4.820)--(5.614,4.757)%
  --(5.721,4.693)--(5.828,4.629)--(5.936,4.566)--(6.043,4.502)--(6.150,4.438)--(6.258,4.375)%
  --(6.365,4.311)--(6.472,4.247)--(6.580,4.184)--(6.687,4.120)--(6.795,4.056)--(6.902,3.993)%
  --(7.009,3.929)--(7.117,3.865)--(7.224,3.802)--(7.331,3.738)--(7.439,3.674)--(7.546,3.611)%
  --(7.653,3.547)--(7.761,3.483)--(7.868,3.420)--(7.975,3.356)--(8.083,3.292)--(8.190,3.229)%
  --(8.297,3.165)--(8.405,3.101)--(8.512,3.038)--(8.619,2.974)--(8.727,2.910)--(8.834,2.847)%
  --(8.941,2.783)--(9.049,2.719)--(9.156,2.656)--(9.263,2.592)--(9.371,2.528)--(9.478,2.465)%
  --(9.585,2.401)--(9.693,2.337)--(9.800,2.274)--(9.907,2.210)--(10.015,2.146)--(10.122,2.083)%
  --(10.230,2.019)--(10.337,1.955)--(10.444,1.892)--(10.552,1.828)--(10.659,1.764)--(10.766,1.701)%
  --(10.874,1.637)--(10.981,1.574)--(11.088,1.510)--(11.196,1.446)--(11.303,1.383)--(11.410,1.319)%
  --(11.518,1.255)--(11.625,1.192)--(11.732,1.128)--(11.840,1.064)--(11.947,1.001);
\gpcolor{color=gp lt color border}
\gpsetlinetype{gp lt border}
\draw[gp path] (1.320,7.825)--(1.320,0.985)--(11.947,0.985)--(11.947,7.825)--cycle;
%% coordinates of the plot area
\gpdefrectangularnode{gp plot 1}{\pgfpoint{1.320cm}{0.985cm}}{\pgfpoint{11.947cm}{7.825cm}}
\end{tikzpicture}
%% gnuplot variables

\caption{Terza serie, accelerazione}
\label{fig:1}
\end{grafico}

\begin{grafico}
    \centering
\begin{tikzpicture}[gnuplot]
%% generated with GNUPLOT 4.6p0 (Lua 5.1; terminal rev. 99, script rev. 100)
%% Tue 15 Apr 2014 06:32:32 PM CEST
\path (0.000,0.000) rectangle (12.500,8.750);
\gpcolor{color=gp lt color border}
\gpsetlinetype{gp lt border}
\gpsetlinewidth{1.00}
\draw[gp path] (1.688,0.985)--(1.868,0.985);
\draw[gp path] (11.947,0.985)--(11.767,0.985);
\node[gp node right] at (1.504,0.985) { 0.15};
\draw[gp path] (1.688,1.669)--(1.868,1.669);
\draw[gp path] (11.947,1.669)--(11.767,1.669);
\node[gp node right] at (1.504,1.669) { 0.2};
\draw[gp path] (1.688,2.353)--(1.868,2.353);
\draw[gp path] (11.947,2.353)--(11.767,2.353);
\node[gp node right] at (1.504,2.353) { 0.25};
\draw[gp path] (1.688,3.037)--(1.868,3.037);
\draw[gp path] (11.947,3.037)--(11.767,3.037);
\node[gp node right] at (1.504,3.037) { 0.3};
\draw[gp path] (1.688,3.721)--(1.868,3.721);
\draw[gp path] (11.947,3.721)--(11.767,3.721);
\node[gp node right] at (1.504,3.721) { 0.35};
\draw[gp path] (1.688,4.405)--(1.868,4.405);
\draw[gp path] (11.947,4.405)--(11.767,4.405);
\node[gp node right] at (1.504,4.405) { 0.4};
\draw[gp path] (1.688,5.089)--(1.868,5.089);
\draw[gp path] (11.947,5.089)--(11.767,5.089);
\node[gp node right] at (1.504,5.089) { 0.45};
\draw[gp path] (1.688,5.773)--(1.868,5.773);
\draw[gp path] (11.947,5.773)--(11.767,5.773);
\node[gp node right] at (1.504,5.773) { 0.5};
\draw[gp path] (1.688,6.457)--(1.868,6.457);
\draw[gp path] (11.947,6.457)--(11.767,6.457);
\node[gp node right] at (1.504,6.457) { 0.55};
\draw[gp path] (1.688,7.141)--(1.868,7.141);
\draw[gp path] (11.947,7.141)--(11.767,7.141);
\node[gp node right] at (1.504,7.141) { 0.6};
\draw[gp path] (1.688,7.825)--(1.868,7.825);
\draw[gp path] (11.947,7.825)--(11.767,7.825);
\node[gp node right] at (1.504,7.825) { 0.65};
\draw[gp path] (1.688,0.985)--(1.688,1.165);
\draw[gp path] (1.688,7.825)--(1.688,7.645);
\node[gp node center] at (1.688,0.677) { 5};
\draw[gp path] (3.740,0.985)--(3.740,1.165);
\draw[gp path] (3.740,7.825)--(3.740,7.645);
\node[gp node center] at (3.740,0.677) { 10};
\draw[gp path] (5.792,0.985)--(5.792,1.165);
\draw[gp path] (5.792,7.825)--(5.792,7.645);
\node[gp node center] at (5.792,0.677) { 15};
\draw[gp path] (7.843,0.985)--(7.843,1.165);
\draw[gp path] (7.843,7.825)--(7.843,7.645);
\node[gp node center] at (7.843,0.677) { 20};
\draw[gp path] (9.895,0.985)--(9.895,1.165);
\draw[gp path] (9.895,7.825)--(9.895,7.645);
\node[gp node center] at (9.895,0.677) { 25};
\draw[gp path] (11.947,0.985)--(11.947,1.165);
\draw[gp path] (11.947,7.825)--(11.947,7.645);
\node[gp node center] at (11.947,0.677) { 30};
\draw[gp path] (1.688,7.825)--(1.688,0.985)--(11.947,0.985)--(11.947,7.825)--cycle;
\node[gp node center,rotate=-270] at (0.246,4.405) {Velocità angolare [rad/s]};
\node[gp node center] at (6.817,0.215) {Tempo [s]};
\node[gp node center] at (6.817,8.287) {Velocità angolare, decelerazione [rad/s]};
\node[gp node left] at (7.167,1.627) {Dati};
\gpcolor{color=gp lt color 0}
\gpsetlinetype{gp lt plot 0}
\draw[gp path] (10.663,1.627)--(11.579,1.627);
\draw[gp path] (10.663,1.717)--(10.663,1.537);
\draw[gp path] (11.579,1.717)--(11.579,1.537);
\draw[gp path] (2.899,1.601)--(2.899,1.669);
\draw[gp path] (2.809,1.601)--(2.989,1.601);
\draw[gp path] (2.809,1.669)--(2.989,1.669);
\draw[gp path] (4.437,2.574)--(4.437,2.637);
\draw[gp path] (4.347,2.574)--(4.527,2.574);
\draw[gp path] (4.347,2.637)--(4.527,2.637);
\draw[gp path] (5.545,3.378)--(5.545,3.440);
\draw[gp path] (5.455,3.378)--(5.635,3.378);
\draw[gp path] (5.455,3.440)--(5.635,3.440);
\draw[gp path] (6.510,4.033)--(6.510,4.094);
\draw[gp path] (6.420,4.033)--(6.600,4.033);
\draw[gp path] (6.420,4.094)--(6.600,4.094);
\draw[gp path] (7.371,4.602)--(7.371,4.662);
\draw[gp path] (7.281,4.602)--(7.461,4.602);
\draw[gp path] (7.281,4.662)--(7.461,4.662);
\draw[gp path] (8.131,5.130)--(8.131,5.190);
\draw[gp path] (8.041,5.130)--(8.221,5.130);
\draw[gp path] (8.041,5.190)--(8.221,5.190);
\draw[gp path] (8.828,5.617)--(8.828,5.677);
\draw[gp path] (8.738,5.617)--(8.918,5.617);
\draw[gp path] (8.738,5.677)--(8.918,5.677);
\draw[gp path] (9.464,6.079)--(9.464,6.139);
\draw[gp path] (9.374,6.079)--(9.554,6.079);
\draw[gp path] (9.374,6.139)--(9.554,6.139);
\draw[gp path] (10.100,6.486)--(10.100,6.545);
\draw[gp path] (10.010,6.486)--(10.190,6.486);
\draw[gp path] (10.010,6.545)--(10.190,6.545);
\draw[gp path] (10.634,6.920)--(10.634,6.979);
\draw[gp path] (10.544,6.920)--(10.724,6.920);
\draw[gp path] (10.544,6.979)--(10.724,6.979);
\draw[gp path] (11.208,7.284)--(11.208,7.343);
\draw[gp path] (11.118,7.284)--(11.298,7.284);
\draw[gp path] (11.118,7.343)--(11.298,7.343);
\draw[gp path] (11.742,7.643)--(11.742,7.702);
\draw[gp path] (11.652,7.643)--(11.832,7.643);
\draw[gp path] (11.652,7.702)--(11.832,7.702);
\gpsetpointsize{4.00}
\gppoint{gp mark 1}{(2.899,1.635)}
\gppoint{gp mark 1}{(4.437,2.606)}
\gppoint{gp mark 1}{(5.545,3.409)}
\gppoint{gp mark 1}{(6.510,4.064)}
\gppoint{gp mark 1}{(7.371,4.632)}
\gppoint{gp mark 1}{(8.131,5.160)}
\gppoint{gp mark 1}{(8.828,5.647)}
\gppoint{gp mark 1}{(9.464,6.109)}
\gppoint{gp mark 1}{(10.100,6.516)}
\gppoint{gp mark 1}{(10.634,6.950)}
\gppoint{gp mark 1}{(11.208,7.313)}
\gppoint{gp mark 1}{(11.742,7.672)}
\gppoint{gp mark 1}{(11.121,1.627)}
\gpcolor{color=gp lt color border}
\node[gp node left] at (7.167,1.319) {Retta interpolante};
\gpcolor{color=gp lt color 1}
\gpsetlinetype{gp lt plot 1}
\draw[gp path] (10.663,1.319)--(11.579,1.319);
\draw[gp path] (2.899,1.583)--(2.988,1.644)--(3.077,1.706)--(3.167,1.767)--(3.256,1.829)%
  --(3.345,1.890)--(3.435,1.952)--(3.524,2.013)--(3.613,2.075)--(3.702,2.136)--(3.792,2.197)%
  --(3.881,2.259)--(3.970,2.320)--(4.060,2.382)--(4.149,2.443)--(4.238,2.505)--(4.328,2.566)%
  --(4.417,2.628)--(4.506,2.689)--(4.596,2.751)--(4.685,2.812)--(4.774,2.874)--(4.864,2.935)%
  --(4.953,2.996)--(5.042,3.058)--(5.132,3.119)--(5.221,3.181)--(5.310,3.242)--(5.400,3.304)%
  --(5.489,3.365)--(5.578,3.427)--(5.668,3.488)--(5.757,3.550)--(5.846,3.611)--(5.936,3.672)%
  --(6.025,3.734)--(6.114,3.795)--(6.204,3.857)--(6.293,3.918)--(6.382,3.980)--(6.472,4.041)%
  --(6.561,4.103)--(6.650,4.164)--(6.740,4.226)--(6.829,4.287)--(6.918,4.349)--(7.008,4.410)%
  --(7.097,4.471)--(7.186,4.533)--(7.276,4.594)--(7.365,4.656)--(7.454,4.717)--(7.544,4.779)%
  --(7.633,4.840)--(7.722,4.902)--(7.811,4.963)--(7.901,5.025)--(7.990,5.086)--(8.079,5.148)%
  --(8.169,5.209)--(8.258,5.270)--(8.347,5.332)--(8.437,5.393)--(8.526,5.455)--(8.615,5.516)%
  --(8.705,5.578)--(8.794,5.639)--(8.883,5.701)--(8.973,5.762)--(9.062,5.824)--(9.151,5.885)%
  --(9.241,5.947)--(9.330,6.008)--(9.419,6.069)--(9.509,6.131)--(9.598,6.192)--(9.687,6.254)%
  --(9.777,6.315)--(9.866,6.377)--(9.955,6.438)--(10.045,6.500)--(10.134,6.561)--(10.223,6.623)%
  --(10.313,6.684)--(10.402,6.746)--(10.491,6.807)--(10.581,6.868)--(10.670,6.930)--(10.759,6.991)%
  --(10.849,7.053)--(10.938,7.114)--(11.027,7.176)--(11.117,7.237)--(11.206,7.299)--(11.295,7.360)%
  --(11.385,7.422)--(11.474,7.483)--(11.563,7.545)--(11.652,7.606)--(11.742,7.667);
\gpcolor{color=gp lt color border}
\gpsetlinetype{gp lt border}
\draw[gp path] (1.688,7.825)--(1.688,0.985)--(11.947,0.985)--(11.947,7.825)--cycle;
%% coordinates of the plot area
\gpdefrectangularnode{gp plot 1}{\pgfpoint{1.688cm}{0.985cm}}{\pgfpoint{11.947cm}{7.825cm}}
\end{tikzpicture}
%% gnuplot variables

\caption{Quarta serie, accelerazione}
\label{fig:1}
\end{grafico}

\begin{grafico}
    \centering
\begin{tikzpicture}[gnuplot]
%% generated with GNUPLOT 4.6p0 (Lua 5.1; terminal rev. 99, script rev. 100)
%% Tue 15 Apr 2014 09:47:06 PM CEST
\path (0.000,0.000) rectangle (12.500,8.750);
\gpcolor{color=gp lt color border}
\gpsetlinetype{gp lt border}
\gpsetlinewidth{1.00}
\draw[gp path] (1.688,0.985)--(1.868,0.985);
\draw[gp path] (11.947,0.985)--(11.767,0.985);
\node[gp node right] at (1.504,0.985) { 0.15};
\draw[gp path] (1.688,1.669)--(1.868,1.669);
\draw[gp path] (11.947,1.669)--(11.767,1.669);
\node[gp node right] at (1.504,1.669) { 0.2};
\draw[gp path] (1.688,2.353)--(1.868,2.353);
\draw[gp path] (11.947,2.353)--(11.767,2.353);
\node[gp node right] at (1.504,2.353) { 0.25};
\draw[gp path] (1.688,3.037)--(1.868,3.037);
\draw[gp path] (11.947,3.037)--(11.767,3.037);
\node[gp node right] at (1.504,3.037) { 0.3};
\draw[gp path] (1.688,3.721)--(1.868,3.721);
\draw[gp path] (11.947,3.721)--(11.767,3.721);
\node[gp node right] at (1.504,3.721) { 0.35};
\draw[gp path] (1.688,4.405)--(1.868,4.405);
\draw[gp path] (11.947,4.405)--(11.767,4.405);
\node[gp node right] at (1.504,4.405) { 0.4};
\draw[gp path] (1.688,5.089)--(1.868,5.089);
\draw[gp path] (11.947,5.089)--(11.767,5.089);
\node[gp node right] at (1.504,5.089) { 0.45};
\draw[gp path] (1.688,5.773)--(1.868,5.773);
\draw[gp path] (11.947,5.773)--(11.767,5.773);
\node[gp node right] at (1.504,5.773) { 0.5};
\draw[gp path] (1.688,6.457)--(1.868,6.457);
\draw[gp path] (11.947,6.457)--(11.767,6.457);
\node[gp node right] at (1.504,6.457) { 0.55};
\draw[gp path] (1.688,7.141)--(1.868,7.141);
\draw[gp path] (11.947,7.141)--(11.767,7.141);
\node[gp node right] at (1.504,7.141) { 0.6};
\draw[gp path] (1.688,7.825)--(1.868,7.825);
\draw[gp path] (11.947,7.825)--(11.767,7.825);
\node[gp node right] at (1.504,7.825) { 0.65};
\draw[gp path] (1.688,0.985)--(1.688,1.165);
\draw[gp path] (1.688,7.825)--(1.688,7.645);
\node[gp node center] at (1.688,0.677) { 5};
\draw[gp path] (3.740,0.985)--(3.740,1.165);
\draw[gp path] (3.740,7.825)--(3.740,7.645);
\node[gp node center] at (3.740,0.677) { 10};
\draw[gp path] (5.792,0.985)--(5.792,1.165);
\draw[gp path] (5.792,7.825)--(5.792,7.645);
\node[gp node center] at (5.792,0.677) { 15};
\draw[gp path] (7.843,0.985)--(7.843,1.165);
\draw[gp path] (7.843,7.825)--(7.843,7.645);
\node[gp node center] at (7.843,0.677) { 20};
\draw[gp path] (9.895,0.985)--(9.895,1.165);
\draw[gp path] (9.895,7.825)--(9.895,7.645);
\node[gp node center] at (9.895,0.677) { 25};
\draw[gp path] (11.947,0.985)--(11.947,1.165);
\draw[gp path] (11.947,7.825)--(11.947,7.645);
\node[gp node center] at (11.947,0.677) { 30};
\draw[gp path] (1.688,7.825)--(1.688,0.985)--(11.947,0.985)--(11.947,7.825)--cycle;
\node[gp node center,rotate=-270] at (0.246,4.405) {Velocità angolare [rad/s]};
\node[gp node center] at (6.817,0.215) {Tempo [s]};
\node[gp node center] at (6.817,8.287) {Velocità angolari in accelerazione [rad/s]};
\node[gp node left] at (7.167,1.627) {Dati};
\gpcolor{color=gp lt color 0}
\gpsetlinetype{gp lt plot 0}
\draw[gp path] (10.663,1.627)--(11.579,1.627);
\draw[gp path] (10.663,1.717)--(10.663,1.537);
\draw[gp path] (11.579,1.717)--(11.579,1.537);
\draw[gp path] (2.837,1.652)--(2.837,1.723);
\draw[gp path] (2.747,1.652)--(2.927,1.652);
\draw[gp path] (2.747,1.723)--(2.927,1.723);
\draw[gp path] (4.355,2.637)--(4.355,2.702);
\draw[gp path] (4.265,2.637)--(4.445,2.637);
\draw[gp path] (4.265,2.702)--(4.445,2.702);
\draw[gp path] (5.463,3.440)--(5.463,3.504);
\draw[gp path] (5.373,3.440)--(5.553,3.440);
\draw[gp path] (5.373,3.504)--(5.553,3.504);
\draw[gp path] (6.428,4.094)--(6.428,4.157);
\draw[gp path] (6.338,4.094)--(6.518,4.094);
\draw[gp path] (6.338,4.157)--(6.518,4.157);
\draw[gp path] (7.269,4.677)--(7.269,4.739);
\draw[gp path] (7.179,4.677)--(7.359,4.677);
\draw[gp path] (7.179,4.739)--(7.359,4.739);
\draw[gp path] (8.008,5.221)--(8.008,5.283);
\draw[gp path] (7.918,5.221)--(8.098,5.221);
\draw[gp path] (7.918,5.283)--(8.098,5.283);
\draw[gp path] (8.726,5.692)--(8.726,5.753);
\draw[gp path] (8.636,5.692)--(8.816,5.692);
\draw[gp path] (8.636,5.753)--(8.816,5.753);
\draw[gp path] (9.341,6.170)--(9.341,6.231);
\draw[gp path] (9.251,6.170)--(9.431,6.170);
\draw[gp path] (9.251,6.231)--(9.431,6.231);
\draw[gp path] (9.998,6.561)--(9.998,6.621);
\draw[gp path] (9.908,6.561)--(10.088,6.561);
\draw[gp path] (9.908,6.621)--(10.088,6.621);
\draw[gp path] (10.572,6.964)--(10.572,7.025);
\draw[gp path] (10.482,6.964)--(10.662,6.964);
\draw[gp path] (10.482,7.025)--(10.662,7.025);
\draw[gp path] (11.126,7.343)--(11.126,7.403);
\draw[gp path] (11.036,7.343)--(11.216,7.343);
\draw[gp path] (11.036,7.403)--(11.216,7.403);
\draw[gp path] (11.598,7.747)--(11.598,7.808);
\draw[gp path] (11.508,7.747)--(11.688,7.747);
\draw[gp path] (11.508,7.808)--(11.688,7.808);
\gpsetpointsize{4.00}
\gppoint{gp mark 1}{(2.837,1.687)}
\gppoint{gp mark 1}{(4.355,2.669)}
\gppoint{gp mark 1}{(5.463,3.472)}
\gppoint{gp mark 1}{(6.428,4.126)}
\gppoint{gp mark 1}{(7.269,4.708)}
\gppoint{gp mark 1}{(8.008,5.252)}
\gppoint{gp mark 1}{(8.726,5.723)}
\gppoint{gp mark 1}{(9.341,6.200)}
\gppoint{gp mark 1}{(9.998,6.591)}
\gppoint{gp mark 1}{(10.572,6.995)}
\gppoint{gp mark 1}{(11.126,7.373)}
\gppoint{gp mark 1}{(11.598,7.777)}
\gppoint{gp mark 1}{(11.121,1.627)}
\gpcolor{color=gp lt color border}
\node[gp node left] at (7.167,1.319) {Retta interpolante};
\gpcolor{color=gp lt color 1}
\gpsetlinetype{gp lt plot 1}
\draw[gp path] (10.663,1.319)--(11.579,1.319);
\draw[gp path] (2.837,1.647)--(2.926,1.708)--(3.014,1.769)--(3.102,1.831)--(3.191,1.892)%
  --(3.279,1.954)--(3.368,2.015)--(3.456,2.076)--(3.545,2.138)--(3.633,2.199)--(3.722,2.261)%
  --(3.810,2.322)--(3.899,2.384)--(3.987,2.445)--(4.076,2.506)--(4.164,2.568)--(4.253,2.629)%
  --(4.341,2.691)--(4.430,2.752)--(4.518,2.814)--(4.607,2.875)--(4.695,2.936)--(4.784,2.998)%
  --(4.872,3.059)--(4.961,3.121)--(5.049,3.182)--(5.138,3.244)--(5.226,3.305)--(5.315,3.366)%
  --(5.403,3.428)--(5.492,3.489)--(5.580,3.551)--(5.669,3.612)--(5.757,3.673)--(5.846,3.735)%
  --(5.934,3.796)--(6.023,3.858)--(6.111,3.919)--(6.200,3.981)--(6.288,4.042)--(6.377,4.103)%
  --(6.465,4.165)--(6.554,4.226)--(6.642,4.288)--(6.731,4.349)--(6.819,4.411)--(6.908,4.472)%
  --(6.996,4.533)--(7.085,4.595)--(7.173,4.656)--(7.262,4.718)--(7.350,4.779)--(7.439,4.841)%
  --(7.527,4.902)--(7.616,4.963)--(7.704,5.025)--(7.793,5.086)--(7.881,5.148)--(7.970,5.209)%
  --(8.058,5.270)--(8.147,5.332)--(8.235,5.393)--(8.324,5.455)--(8.412,5.516)--(8.501,5.578)%
  --(8.589,5.639)--(8.678,5.700)--(8.766,5.762)--(8.855,5.823)--(8.943,5.885)--(9.032,5.946)%
  --(9.120,6.008)--(9.209,6.069)--(9.297,6.130)--(9.386,6.192)--(9.474,6.253)--(9.563,6.315)%
  --(9.651,6.376)--(9.740,6.438)--(9.828,6.499)--(9.917,6.560)--(10.005,6.622)--(10.094,6.683)%
  --(10.182,6.745)--(10.271,6.806)--(10.359,6.867)--(10.448,6.929)--(10.536,6.990)--(10.625,7.052)%
  --(10.713,7.113)--(10.802,7.175)--(10.890,7.236)--(10.979,7.297)--(11.067,7.359)--(11.156,7.420)%
  --(11.244,7.482)--(11.333,7.543)--(11.421,7.605)--(11.510,7.666)--(11.598,7.727);
\gpcolor{color=gp lt color border}
\gpsetlinetype{gp lt border}
\draw[gp path] (1.688,7.825)--(1.688,0.985)--(11.947,0.985)--(11.947,7.825)--cycle;
%% coordinates of the plot area
\gpdefrectangularnode{gp plot 1}{\pgfpoint{1.688cm}{0.985cm}}{\pgfpoint{11.947cm}{7.825cm}}
\end{tikzpicture}
%% gnuplot variables

\caption{Quinta serie, accelerazione}
\label{fig:1}
\end{grafico}

\begin{grafico}
    \centering
\begin{tikzpicture}[gnuplot]
%% generated with GNUPLOT 4.6p0 (Lua 5.1; terminal rev. 99, script rev. 100)
%% Mon 14 Apr 2014 11:09:58 PM CEST
\path (0.000,0.000) rectangle (12.500,8.750);
\gpcolor{color=gp lt color border}
\gpsetlinetype{gp lt border}
\gpsetlinewidth{1.00}
\draw[gp path] (1.320,0.985)--(1.500,0.985);
\draw[gp path] (11.947,0.985)--(11.767,0.985);
\node[gp node right] at (1.136,0.985) { 5};
\draw[gp path] (1.320,1.669)--(1.500,1.669);
\draw[gp path] (11.947,1.669)--(11.767,1.669);
\node[gp node right] at (1.136,1.669) { 10};
\draw[gp path] (1.320,2.353)--(1.500,2.353);
\draw[gp path] (11.947,2.353)--(11.767,2.353);
\node[gp node right] at (1.136,2.353) { 15};
\draw[gp path] (1.320,3.037)--(1.500,3.037);
\draw[gp path] (11.947,3.037)--(11.767,3.037);
\node[gp node right] at (1.136,3.037) { 20};
\draw[gp path] (1.320,3.721)--(1.500,3.721);
\draw[gp path] (11.947,3.721)--(11.767,3.721);
\node[gp node right] at (1.136,3.721) { 25};
\draw[gp path] (1.320,4.405)--(1.500,4.405);
\draw[gp path] (11.947,4.405)--(11.767,4.405);
\node[gp node right] at (1.136,4.405) { 30};
\draw[gp path] (1.320,5.089)--(1.500,5.089);
\draw[gp path] (11.947,5.089)--(11.767,5.089);
\node[gp node right] at (1.136,5.089) { 35};
\draw[gp path] (1.320,5.773)--(1.500,5.773);
\draw[gp path] (11.947,5.773)--(11.767,5.773);
\node[gp node right] at (1.136,5.773) { 40};
\draw[gp path] (1.320,6.457)--(1.500,6.457);
\draw[gp path] (11.947,6.457)--(11.767,6.457);
\node[gp node right] at (1.136,6.457) { 45};
\draw[gp path] (1.320,7.141)--(1.500,7.141);
\draw[gp path] (11.947,7.141)--(11.767,7.141);
\node[gp node right] at (1.136,7.141) { 50};
\draw[gp path] (1.320,7.825)--(1.500,7.825);
\draw[gp path] (11.947,7.825)--(11.767,7.825);
\node[gp node right] at (1.136,7.825) { 55};
\draw[gp path] (1.320,0.985)--(1.320,1.165);
\draw[gp path] (1.320,7.825)--(1.320,7.645);
\node[gp node center] at (1.320,0.677) {-10};
\draw[gp path] (3.977,0.985)--(3.977,1.165);
\draw[gp path] (3.977,7.825)--(3.977,7.645);
\node[gp node center] at (3.977,0.677) {-5};
\draw[gp path] (6.634,0.985)--(6.634,1.165);
\draw[gp path] (6.634,7.825)--(6.634,7.645);
\node[gp node center] at (6.634,0.677) { 0};
\draw[gp path] (9.290,0.985)--(9.290,1.165);
\draw[gp path] (9.290,7.825)--(9.290,7.645);
\node[gp node center] at (9.290,0.677) { 5};
\draw[gp path] (11.947,0.985)--(11.947,1.165);
\draw[gp path] (11.947,7.825)--(11.947,7.645);
\node[gp node center] at (11.947,0.677) { 10};
\draw[gp path] (1.320,7.825)--(1.320,0.985)--(11.947,0.985)--(11.947,7.825)--cycle;
\node[gp node center,rotate=-270] at (0.246,4.405) {Velocità angolare [rad/s]};
\node[gp node center] at (6.633,0.215) {Tempo [s]};
\node[gp node center] at (6.633,8.287) {Velocità angolare, decelerazione [rad/s]};
\node[gp node left] at (7.167,1.627) {Retta interpolante};
\gpcolor{color=gp lt color 1}
\gpsetlinetype{gp lt plot 1}
\draw[gp path] (10.663,1.627)--(11.579,1.627);
\draw[gp path] (1.320,7.628)--(1.427,7.566)--(1.535,7.505)--(1.642,7.444)--(1.749,7.382)%
  --(1.857,7.321)--(1.964,7.259)--(2.071,7.198)--(2.179,7.137)--(2.286,7.075)--(2.393,7.014)%
  --(2.501,6.952)--(2.608,6.891)--(2.715,6.830)--(2.823,6.768)--(2.930,6.707)--(3.037,6.645)%
  --(3.145,6.584)--(3.252,6.523)--(3.360,6.461)--(3.467,6.400)--(3.574,6.338)--(3.682,6.277)%
  --(3.789,6.216)--(3.896,6.154)--(4.004,6.093)--(4.111,6.031)--(4.218,5.970)--(4.326,5.909)%
  --(4.433,5.847)--(4.540,5.786)--(4.648,5.724)--(4.755,5.663)--(4.862,5.602)--(4.970,5.540)%
  --(5.077,5.479)--(5.184,5.417)--(5.292,5.356)--(5.399,5.294)--(5.506,5.233)--(5.614,5.172)%
  --(5.721,5.110)--(5.828,5.049)--(5.936,4.987)--(6.043,4.926)--(6.150,4.865)--(6.258,4.803)%
  --(6.365,4.742)--(6.472,4.680)--(6.580,4.619)--(6.687,4.558)--(6.795,4.496)--(6.902,4.435)%
  --(7.009,4.373)--(7.117,4.312)--(7.224,4.251)--(7.331,4.189)--(7.439,4.128)--(7.546,4.066)%
  --(7.653,4.005)--(7.761,3.944)--(7.868,3.882)--(7.975,3.821)--(8.083,3.759)--(8.190,3.698)%
  --(8.297,3.637)--(8.405,3.575)--(8.512,3.514)--(8.619,3.452)--(8.727,3.391)--(8.834,3.330)%
  --(8.941,3.268)--(9.049,3.207)--(9.156,3.145)--(9.263,3.084)--(9.371,3.023)--(9.478,2.961)%
  --(9.585,2.900)--(9.693,2.838)--(9.800,2.777)--(9.907,2.716)--(10.015,2.654)--(10.122,2.593)%
  --(10.230,2.531)--(10.337,2.470)--(10.444,2.409)--(10.552,2.347)--(10.659,2.286)--(10.766,2.224)%
  --(10.874,2.163)--(10.981,2.102)--(11.088,2.040)--(11.196,1.979)--(11.303,1.917)--(11.410,1.856)%
  --(11.518,1.795)--(11.625,1.733)--(11.732,1.672)--(11.840,1.610)--(11.947,1.549);
\gpcolor{color=gp lt color border}
\gpsetlinetype{gp lt border}
\draw[gp path] (1.320,7.825)--(1.320,0.985)--(11.947,0.985)--(11.947,7.825)--cycle;
%% coordinates of the plot area
\gpdefrectangularnode{gp plot 1}{\pgfpoint{1.320cm}{0.985cm}}{\pgfpoint{11.947cm}{7.825cm}}
\end{tikzpicture}
%% gnuplot variables

\caption{Sesta serie, accelerazione}
\label{fig:1}
\end{grafico}

\begin{grafico}
    \centering
\begin{tikzpicture}[gnuplot]
%% generated with GNUPLOT 4.6p0 (Lua 5.1; terminal rev. 99, script rev. 100)
%% Mon 14 Apr 2014 11:09:58 PM CEST
\path (0.000,0.000) rectangle (12.500,8.750);
\gpcolor{color=gp lt color border}
\gpsetlinetype{gp lt border}
\gpsetlinewidth{1.00}
\draw[gp path] (1.320,0.985)--(1.500,0.985);
\draw[gp path] (11.947,0.985)--(11.767,0.985);
\node[gp node right] at (1.136,0.985) { 10};
\draw[gp path] (1.320,1.745)--(1.500,1.745);
\draw[gp path] (11.947,1.745)--(11.767,1.745);
\node[gp node right] at (1.136,1.745) { 15};
\draw[gp path] (1.320,2.505)--(1.500,2.505);
\draw[gp path] (11.947,2.505)--(11.767,2.505);
\node[gp node right] at (1.136,2.505) { 20};
\draw[gp path] (1.320,3.265)--(1.500,3.265);
\draw[gp path] (11.947,3.265)--(11.767,3.265);
\node[gp node right] at (1.136,3.265) { 25};
\draw[gp path] (1.320,4.025)--(1.500,4.025);
\draw[gp path] (11.947,4.025)--(11.767,4.025);
\node[gp node right] at (1.136,4.025) { 30};
\draw[gp path] (1.320,4.785)--(1.500,4.785);
\draw[gp path] (11.947,4.785)--(11.767,4.785);
\node[gp node right] at (1.136,4.785) { 35};
\draw[gp path] (1.320,5.545)--(1.500,5.545);
\draw[gp path] (11.947,5.545)--(11.767,5.545);
\node[gp node right] at (1.136,5.545) { 40};
\draw[gp path] (1.320,6.305)--(1.500,6.305);
\draw[gp path] (11.947,6.305)--(11.767,6.305);
\node[gp node right] at (1.136,6.305) { 45};
\draw[gp path] (1.320,7.065)--(1.500,7.065);
\draw[gp path] (11.947,7.065)--(11.767,7.065);
\node[gp node right] at (1.136,7.065) { 50};
\draw[gp path] (1.320,7.825)--(1.500,7.825);
\draw[gp path] (11.947,7.825)--(11.767,7.825);
\node[gp node right] at (1.136,7.825) { 55};
\draw[gp path] (1.320,0.985)--(1.320,1.165);
\draw[gp path] (1.320,7.825)--(1.320,7.645);
\node[gp node center] at (1.320,0.677) {-10};
\draw[gp path] (3.977,0.985)--(3.977,1.165);
\draw[gp path] (3.977,7.825)--(3.977,7.645);
\node[gp node center] at (3.977,0.677) {-5};
\draw[gp path] (6.634,0.985)--(6.634,1.165);
\draw[gp path] (6.634,7.825)--(6.634,7.645);
\node[gp node center] at (6.634,0.677) { 0};
\draw[gp path] (9.290,0.985)--(9.290,1.165);
\draw[gp path] (9.290,7.825)--(9.290,7.645);
\node[gp node center] at (9.290,0.677) { 5};
\draw[gp path] (11.947,0.985)--(11.947,1.165);
\draw[gp path] (11.947,7.825)--(11.947,7.645);
\node[gp node center] at (11.947,0.677) { 10};
\draw[gp path] (1.320,7.825)--(1.320,0.985)--(11.947,0.985)--(11.947,7.825)--cycle;
\node[gp node center,rotate=-270] at (0.246,4.405) {Velocità angolare [rad/s]};
\node[gp node center] at (6.633,0.215) {Tempo [s]};
\node[gp node center] at (6.633,8.287) {Velocità angolare, decelerazione [rad/s]};
\node[gp node left] at (7.167,1.627) {Retta interpolante};
\gpcolor{color=gp lt color 1}
\gpsetlinetype{gp lt plot 1}
\draw[gp path] (10.663,1.627)--(11.579,1.627);
\draw[gp path] (1.320,7.274)--(1.427,7.211)--(1.535,7.147)--(1.642,7.084)--(1.749,7.020)%
  --(1.857,6.957)--(1.964,6.893)--(2.071,6.830)--(2.179,6.766)--(2.286,6.703)--(2.393,6.639)%
  --(2.501,6.576)--(2.608,6.512)--(2.715,6.448)--(2.823,6.385)--(2.930,6.321)--(3.037,6.258)%
  --(3.145,6.194)--(3.252,6.131)--(3.360,6.067)--(3.467,6.004)--(3.574,5.940)--(3.682,5.877)%
  --(3.789,5.813)--(3.896,5.750)--(4.004,5.686)--(4.111,5.623)--(4.218,5.559)--(4.326,5.496)%
  --(4.433,5.432)--(4.540,5.369)--(4.648,5.305)--(4.755,5.242)--(4.862,5.178)--(4.970,5.115)%
  --(5.077,5.051)--(5.184,4.988)--(5.292,4.924)--(5.399,4.860)--(5.506,4.797)--(5.614,4.733)%
  --(5.721,4.670)--(5.828,4.606)--(5.936,4.543)--(6.043,4.479)--(6.150,4.416)--(6.258,4.352)%
  --(6.365,4.289)--(6.472,4.225)--(6.580,4.162)--(6.687,4.098)--(6.795,4.035)--(6.902,3.971)%
  --(7.009,3.908)--(7.117,3.844)--(7.224,3.781)--(7.331,3.717)--(7.439,3.654)--(7.546,3.590)%
  --(7.653,3.527)--(7.761,3.463)--(7.868,3.399)--(7.975,3.336)--(8.083,3.272)--(8.190,3.209)%
  --(8.297,3.145)--(8.405,3.082)--(8.512,3.018)--(8.619,2.955)--(8.727,2.891)--(8.834,2.828)%
  --(8.941,2.764)--(9.049,2.701)--(9.156,2.637)--(9.263,2.574)--(9.371,2.510)--(9.478,2.447)%
  --(9.585,2.383)--(9.693,2.320)--(9.800,2.256)--(9.907,2.193)--(10.015,2.129)--(10.122,2.066)%
  --(10.230,2.002)--(10.337,1.938)--(10.444,1.875)--(10.552,1.811)--(10.659,1.748)--(10.766,1.684)%
  --(10.874,1.621)--(10.981,1.557)--(11.088,1.494)--(11.196,1.430)--(11.303,1.367)--(11.410,1.303)%
  --(11.518,1.240)--(11.625,1.176)--(11.732,1.113)--(11.840,1.049)--(11.947,0.986);
\gpcolor{color=gp lt color border}
\gpsetlinetype{gp lt border}
\draw[gp path] (1.320,7.825)--(1.320,0.985)--(11.947,0.985)--(11.947,7.825)--cycle;
%% coordinates of the plot area
\gpdefrectangularnode{gp plot 1}{\pgfpoint{1.320cm}{0.985cm}}{\pgfpoint{11.947cm}{7.825cm}}
\end{tikzpicture}
%% gnuplot variables

\caption{Settima serie, accelerazione}
\label{fig:1}
\end{grafico}

\begin{grafico}
    \centering
\begin{tikzpicture}[gnuplot]
%% generated with GNUPLOT 4.6p0 (Lua 5.1; terminal rev. 99, script rev. 100)
%% Tue 15 Apr 2014 06:32:33 PM CEST
\path (0.000,0.000) rectangle (12.500,8.750);
\gpcolor{color=gp lt color border}
\gpsetlinetype{gp lt border}
\gpsetlinewidth{1.00}
\draw[gp path] (1.688,0.985)--(1.868,0.985);
\draw[gp path] (11.947,0.985)--(11.767,0.985);
\node[gp node right] at (1.504,0.985) { 0.15};
\draw[gp path] (1.688,1.669)--(1.868,1.669);
\draw[gp path] (11.947,1.669)--(11.767,1.669);
\node[gp node right] at (1.504,1.669) { 0.2};
\draw[gp path] (1.688,2.353)--(1.868,2.353);
\draw[gp path] (11.947,2.353)--(11.767,2.353);
\node[gp node right] at (1.504,2.353) { 0.25};
\draw[gp path] (1.688,3.037)--(1.868,3.037);
\draw[gp path] (11.947,3.037)--(11.767,3.037);
\node[gp node right] at (1.504,3.037) { 0.3};
\draw[gp path] (1.688,3.721)--(1.868,3.721);
\draw[gp path] (11.947,3.721)--(11.767,3.721);
\node[gp node right] at (1.504,3.721) { 0.35};
\draw[gp path] (1.688,4.405)--(1.868,4.405);
\draw[gp path] (11.947,4.405)--(11.767,4.405);
\node[gp node right] at (1.504,4.405) { 0.4};
\draw[gp path] (1.688,5.089)--(1.868,5.089);
\draw[gp path] (11.947,5.089)--(11.767,5.089);
\node[gp node right] at (1.504,5.089) { 0.45};
\draw[gp path] (1.688,5.773)--(1.868,5.773);
\draw[gp path] (11.947,5.773)--(11.767,5.773);
\node[gp node right] at (1.504,5.773) { 0.5};
\draw[gp path] (1.688,6.457)--(1.868,6.457);
\draw[gp path] (11.947,6.457)--(11.767,6.457);
\node[gp node right] at (1.504,6.457) { 0.55};
\draw[gp path] (1.688,7.141)--(1.868,7.141);
\draw[gp path] (11.947,7.141)--(11.767,7.141);
\node[gp node right] at (1.504,7.141) { 0.6};
\draw[gp path] (1.688,7.825)--(1.868,7.825);
\draw[gp path] (11.947,7.825)--(11.767,7.825);
\node[gp node right] at (1.504,7.825) { 0.65};
\draw[gp path] (1.688,0.985)--(1.688,1.165);
\draw[gp path] (1.688,7.825)--(1.688,7.645);
\node[gp node center] at (1.688,0.677) { 5};
\draw[gp path] (3.740,0.985)--(3.740,1.165);
\draw[gp path] (3.740,7.825)--(3.740,7.645);
\node[gp node center] at (3.740,0.677) { 10};
\draw[gp path] (5.792,0.985)--(5.792,1.165);
\draw[gp path] (5.792,7.825)--(5.792,7.645);
\node[gp node center] at (5.792,0.677) { 15};
\draw[gp path] (7.843,0.985)--(7.843,1.165);
\draw[gp path] (7.843,7.825)--(7.843,7.645);
\node[gp node center] at (7.843,0.677) { 20};
\draw[gp path] (9.895,0.985)--(9.895,1.165);
\draw[gp path] (9.895,7.825)--(9.895,7.645);
\node[gp node center] at (9.895,0.677) { 25};
\draw[gp path] (11.947,0.985)--(11.947,1.165);
\draw[gp path] (11.947,7.825)--(11.947,7.645);
\node[gp node center] at (11.947,0.677) { 30};
\draw[gp path] (1.688,7.825)--(1.688,0.985)--(11.947,0.985)--(11.947,7.825)--cycle;
\node[gp node center,rotate=-270] at (0.246,4.405) {Velocità angolare [rad/s]};
\node[gp node center] at (6.817,0.215) {Tempo [s]};
\node[gp node center] at (6.817,8.287) {Velocità angolare, decelerazione [rad/s]};
\node[gp node left] at (7.167,1.627) {Dati};
\gpcolor{color=gp lt color 0}
\gpsetlinetype{gp lt plot 0}
\draw[gp path] (10.663,1.627)--(11.579,1.627);
\draw[gp path] (10.663,1.717)--(10.663,1.537);
\draw[gp path] (11.579,1.717)--(11.579,1.537);
\draw[gp path] (2.940,1.569)--(2.940,1.635);
\draw[gp path] (2.850,1.569)--(3.030,1.569);
\draw[gp path] (2.850,1.635)--(3.030,1.635);
\draw[gp path] (4.396,2.605)--(4.396,2.669);
\draw[gp path] (4.306,2.605)--(4.486,2.605);
\draw[gp path] (4.306,2.669)--(4.486,2.669);
\draw[gp path] (5.525,3.393)--(5.525,3.456);
\draw[gp path] (5.435,3.393)--(5.615,3.393);
\draw[gp path] (5.435,3.456)--(5.615,3.456);
\draw[gp path] (6.469,4.063)--(6.469,4.125);
\draw[gp path] (6.379,4.063)--(6.559,4.063);
\draw[gp path] (6.379,4.125)--(6.559,4.125);
\draw[gp path] (7.330,4.632)--(7.330,4.693);
\draw[gp path] (7.240,4.632)--(7.420,4.632);
\draw[gp path] (7.240,4.693)--(7.420,4.693);
\draw[gp path] (8.069,5.175)--(8.069,5.236);
\draw[gp path] (7.979,5.175)--(8.159,5.175);
\draw[gp path] (7.979,5.236)--(8.159,5.236);
\draw[gp path] (8.726,5.692)--(8.726,5.753);
\draw[gp path] (8.636,5.692)--(8.816,5.692);
\draw[gp path] (8.636,5.753)--(8.816,5.753);
\draw[gp path] (9.157,6.310)--(9.157,6.373);
\draw[gp path] (9.067,6.310)--(9.247,6.310);
\draw[gp path] (9.067,6.373)--(9.247,6.373);
\draw[gp path] (10.039,6.531)--(10.039,6.591);
\draw[gp path] (9.949,6.531)--(10.129,6.531);
\draw[gp path] (9.949,6.591)--(10.129,6.591);
\draw[gp path] (10.613,6.935)--(10.613,6.995);
\draw[gp path] (10.523,6.935)--(10.703,6.935);
\draw[gp path] (10.523,6.995)--(10.703,6.995);
\draw[gp path] (11.188,7.299)--(11.188,7.358);
\draw[gp path] (11.098,7.299)--(11.278,7.299);
\draw[gp path] (11.098,7.358)--(11.278,7.358);
\draw[gp path] (11.701,7.672)--(11.701,7.732);
\draw[gp path] (11.611,7.672)--(11.791,7.672);
\draw[gp path] (11.611,7.732)--(11.791,7.732);
\gpsetpointsize{4.00}
\gppoint{gp mark 1}{(2.940,1.602)}
\gppoint{gp mark 1}{(4.396,2.637)}
\gppoint{gp mark 1}{(5.525,3.425)}
\gppoint{gp mark 1}{(6.469,4.094)}
\gppoint{gp mark 1}{(7.330,4.662)}
\gppoint{gp mark 1}{(8.069,5.206)}
\gppoint{gp mark 1}{(8.726,5.723)}
\gppoint{gp mark 1}{(9.157,6.341)}
\gppoint{gp mark 1}{(10.039,6.561)}
\gppoint{gp mark 1}{(10.613,6.965)}
\gppoint{gp mark 1}{(11.188,7.328)}
\gppoint{gp mark 1}{(11.701,7.702)}
\gppoint{gp mark 1}{(11.121,1.627)}
\gpcolor{color=gp lt color border}
\node[gp node left] at (7.167,1.319) {Retta interpolante};
\gpcolor{color=gp lt color 1}
\gpsetlinetype{gp lt plot 1}
\draw[gp path] (10.663,1.319)--(11.579,1.319);
\draw[gp path] (2.940,1.632)--(3.028,1.694)--(3.117,1.756)--(3.205,1.818)--(3.294,1.880)%
  --(3.382,1.942)--(3.471,2.004)--(3.559,2.066)--(3.648,2.128)--(3.736,2.190)--(3.825,2.252)%
  --(3.913,2.314)--(4.002,2.376)--(4.090,2.438)--(4.179,2.500)--(4.267,2.562)--(4.356,2.624)%
  --(4.444,2.686)--(4.533,2.748)--(4.621,2.810)--(4.710,2.872)--(4.798,2.934)--(4.887,2.996)%
  --(4.975,3.058)--(5.064,3.120)--(5.152,3.182)--(5.241,3.244)--(5.329,3.307)--(5.418,3.369)%
  --(5.506,3.431)--(5.595,3.493)--(5.683,3.555)--(5.771,3.617)--(5.860,3.679)--(5.948,3.741)%
  --(6.037,3.803)--(6.125,3.865)--(6.214,3.927)--(6.302,3.989)--(6.391,4.051)--(6.479,4.113)%
  --(6.568,4.175)--(6.656,4.237)--(6.745,4.299)--(6.833,4.361)--(6.922,4.423)--(7.010,4.485)%
  --(7.099,4.547)--(7.187,4.609)--(7.276,4.671)--(7.364,4.733)--(7.453,4.795)--(7.541,4.857)%
  --(7.630,4.919)--(7.718,4.981)--(7.807,5.043)--(7.895,5.105)--(7.984,5.167)--(8.072,5.229)%
  --(8.161,5.291)--(8.249,5.353)--(8.338,5.415)--(8.426,5.477)--(8.515,5.539)--(8.603,5.601)%
  --(8.692,5.663)--(8.780,5.725)--(8.869,5.787)--(8.957,5.849)--(9.046,5.911)--(9.134,5.973)%
  --(9.223,6.035)--(9.311,6.097)--(9.400,6.159)--(9.488,6.221)--(9.577,6.283)--(9.665,6.345)%
  --(9.754,6.407)--(9.842,6.469)--(9.931,6.531)--(10.019,6.593)--(10.108,6.655)--(10.196,6.717)%
  --(10.285,6.779)--(10.373,6.841)--(10.462,6.903)--(10.550,6.965)--(10.639,7.027)--(10.727,7.090)%
  --(10.816,7.152)--(10.904,7.214)--(10.993,7.276)--(11.081,7.338)--(11.170,7.400)--(11.258,7.462)%
  --(11.347,7.524)--(11.435,7.586)--(11.524,7.648)--(11.612,7.710)--(11.701,7.772);
\gpcolor{color=gp lt color border}
\gpsetlinetype{gp lt border}
\draw[gp path] (1.688,7.825)--(1.688,0.985)--(11.947,0.985)--(11.947,7.825)--cycle;
%% coordinates of the plot area
\gpdefrectangularnode{gp plot 1}{\pgfpoint{1.688cm}{0.985cm}}{\pgfpoint{11.947cm}{7.825cm}}
\end{tikzpicture}
%% gnuplot variables

\caption{Ottava serie, accelerazione}
\label{fig:1}
\end{grafico}

\begin{grafico}
    \centering
\begin{tikzpicture}[gnuplot]
%% generated with GNUPLOT 4.6p0 (Lua 5.1; terminal rev. 99, script rev. 100)
%% Mon 14 Apr 2014 11:09:58 PM CEST
\path (0.000,0.000) rectangle (12.500,8.750);
\gpcolor{color=gp lt color border}
\gpsetlinetype{gp lt border}
\gpsetlinewidth{1.00}
\draw[gp path] (1.320,0.985)--(1.500,0.985);
\draw[gp path] (11.947,0.985)--(11.767,0.985);
\node[gp node right] at (1.136,0.985) { 5};
\draw[gp path] (1.320,1.669)--(1.500,1.669);
\draw[gp path] (11.947,1.669)--(11.767,1.669);
\node[gp node right] at (1.136,1.669) { 10};
\draw[gp path] (1.320,2.353)--(1.500,2.353);
\draw[gp path] (11.947,2.353)--(11.767,2.353);
\node[gp node right] at (1.136,2.353) { 15};
\draw[gp path] (1.320,3.037)--(1.500,3.037);
\draw[gp path] (11.947,3.037)--(11.767,3.037);
\node[gp node right] at (1.136,3.037) { 20};
\draw[gp path] (1.320,3.721)--(1.500,3.721);
\draw[gp path] (11.947,3.721)--(11.767,3.721);
\node[gp node right] at (1.136,3.721) { 25};
\draw[gp path] (1.320,4.405)--(1.500,4.405);
\draw[gp path] (11.947,4.405)--(11.767,4.405);
\node[gp node right] at (1.136,4.405) { 30};
\draw[gp path] (1.320,5.089)--(1.500,5.089);
\draw[gp path] (11.947,5.089)--(11.767,5.089);
\node[gp node right] at (1.136,5.089) { 35};
\draw[gp path] (1.320,5.773)--(1.500,5.773);
\draw[gp path] (11.947,5.773)--(11.767,5.773);
\node[gp node right] at (1.136,5.773) { 40};
\draw[gp path] (1.320,6.457)--(1.500,6.457);
\draw[gp path] (11.947,6.457)--(11.767,6.457);
\node[gp node right] at (1.136,6.457) { 45};
\draw[gp path] (1.320,7.141)--(1.500,7.141);
\draw[gp path] (11.947,7.141)--(11.767,7.141);
\node[gp node right] at (1.136,7.141) { 50};
\draw[gp path] (1.320,7.825)--(1.500,7.825);
\draw[gp path] (11.947,7.825)--(11.767,7.825);
\node[gp node right] at (1.136,7.825) { 55};
\draw[gp path] (1.320,0.985)--(1.320,1.165);
\draw[gp path] (1.320,7.825)--(1.320,7.645);
\node[gp node center] at (1.320,0.677) {-10};
\draw[gp path] (3.977,0.985)--(3.977,1.165);
\draw[gp path] (3.977,7.825)--(3.977,7.645);
\node[gp node center] at (3.977,0.677) {-5};
\draw[gp path] (6.634,0.985)--(6.634,1.165);
\draw[gp path] (6.634,7.825)--(6.634,7.645);
\node[gp node center] at (6.634,0.677) { 0};
\draw[gp path] (9.290,0.985)--(9.290,1.165);
\draw[gp path] (9.290,7.825)--(9.290,7.645);
\node[gp node center] at (9.290,0.677) { 5};
\draw[gp path] (11.947,0.985)--(11.947,1.165);
\draw[gp path] (11.947,7.825)--(11.947,7.645);
\node[gp node center] at (11.947,0.677) { 10};
\draw[gp path] (1.320,7.825)--(1.320,0.985)--(11.947,0.985)--(11.947,7.825)--cycle;
\node[gp node center,rotate=-270] at (0.246,4.405) {Velocità angolare [rad/s]};
\node[gp node center] at (6.633,0.215) {Tempo [s]};
\node[gp node center] at (6.633,8.287) {Velocità angolare, decelerazione [rad/s]};
\node[gp node left] at (7.167,1.627) {Retta interpolante};
\gpcolor{color=gp lt color 1}
\gpsetlinetype{gp lt plot 1}
\draw[gp path] (10.663,1.627)--(11.579,1.627);
\draw[gp path] (1.320,7.708)--(1.427,7.645)--(1.535,7.583)--(1.642,7.520)--(1.749,7.458)%
  --(1.857,7.395)--(1.964,7.333)--(2.071,7.270)--(2.179,7.208)--(2.286,7.145)--(2.393,7.082)%
  --(2.501,7.020)--(2.608,6.957)--(2.715,6.895)--(2.823,6.832)--(2.930,6.770)--(3.037,6.707)%
  --(3.145,6.644)--(3.252,6.582)--(3.360,6.519)--(3.467,6.457)--(3.574,6.394)--(3.682,6.332)%
  --(3.789,6.269)--(3.896,6.207)--(4.004,6.144)--(4.111,6.081)--(4.218,6.019)--(4.326,5.956)%
  --(4.433,5.894)--(4.540,5.831)--(4.648,5.769)--(4.755,5.706)--(4.862,5.644)--(4.970,5.581)%
  --(5.077,5.518)--(5.184,5.456)--(5.292,5.393)--(5.399,5.331)--(5.506,5.268)--(5.614,5.206)%
  --(5.721,5.143)--(5.828,5.081)--(5.936,5.018)--(6.043,4.955)--(6.150,4.893)--(6.258,4.830)%
  --(6.365,4.768)--(6.472,4.705)--(6.580,4.643)--(6.687,4.580)--(6.795,4.517)--(6.902,4.455)%
  --(7.009,4.392)--(7.117,4.330)--(7.224,4.267)--(7.331,4.205)--(7.439,4.142)--(7.546,4.080)%
  --(7.653,4.017)--(7.761,3.954)--(7.868,3.892)--(7.975,3.829)--(8.083,3.767)--(8.190,3.704)%
  --(8.297,3.642)--(8.405,3.579)--(8.512,3.517)--(8.619,3.454)--(8.727,3.391)--(8.834,3.329)%
  --(8.941,3.266)--(9.049,3.204)--(9.156,3.141)--(9.263,3.079)--(9.371,3.016)--(9.478,2.954)%
  --(9.585,2.891)--(9.693,2.828)--(9.800,2.766)--(9.907,2.703)--(10.015,2.641)--(10.122,2.578)%
  --(10.230,2.516)--(10.337,2.453)--(10.444,2.391)--(10.552,2.328)--(10.659,2.265)--(10.766,2.203)%
  --(10.874,2.140)--(10.981,2.078)--(11.088,2.015)--(11.196,1.953)--(11.303,1.890)--(11.410,1.827)%
  --(11.518,1.765)--(11.625,1.702)--(11.732,1.640)--(11.840,1.577)--(11.947,1.515);
\gpcolor{color=gp lt color border}
\gpsetlinetype{gp lt border}
\draw[gp path] (1.320,7.825)--(1.320,0.985)--(11.947,0.985)--(11.947,7.825)--cycle;
%% coordinates of the plot area
\gpdefrectangularnode{gp plot 1}{\pgfpoint{1.320cm}{0.985cm}}{\pgfpoint{11.947cm}{7.825cm}}
\end{tikzpicture}
%% gnuplot variables

\caption{Nona serie, accelerazione}
\label{fig:1}
\end{grafico}

\begin{grafico}
    \centering
\begin{tikzpicture}[gnuplot]
%% generated with GNUPLOT 4.6p0 (Lua 5.1; terminal rev. 99, script rev. 100)
%% Tue 15 Apr 2014 09:47:07 PM CEST
\path (0.000,0.000) rectangle (12.500,8.750);
\gpcolor{color=gp lt color border}
\gpsetlinetype{gp lt border}
\gpsetlinewidth{1.00}
\draw[gp path] (1.688,0.985)--(1.868,0.985);
\draw[gp path] (11.947,0.985)--(11.767,0.985);
\node[gp node right] at (1.504,0.985) { 0.15};
\draw[gp path] (1.688,1.669)--(1.868,1.669);
\draw[gp path] (11.947,1.669)--(11.767,1.669);
\node[gp node right] at (1.504,1.669) { 0.2};
\draw[gp path] (1.688,2.353)--(1.868,2.353);
\draw[gp path] (11.947,2.353)--(11.767,2.353);
\node[gp node right] at (1.504,2.353) { 0.25};
\draw[gp path] (1.688,3.037)--(1.868,3.037);
\draw[gp path] (11.947,3.037)--(11.767,3.037);
\node[gp node right] at (1.504,3.037) { 0.3};
\draw[gp path] (1.688,3.721)--(1.868,3.721);
\draw[gp path] (11.947,3.721)--(11.767,3.721);
\node[gp node right] at (1.504,3.721) { 0.35};
\draw[gp path] (1.688,4.405)--(1.868,4.405);
\draw[gp path] (11.947,4.405)--(11.767,4.405);
\node[gp node right] at (1.504,4.405) { 0.4};
\draw[gp path] (1.688,5.089)--(1.868,5.089);
\draw[gp path] (11.947,5.089)--(11.767,5.089);
\node[gp node right] at (1.504,5.089) { 0.45};
\draw[gp path] (1.688,5.773)--(1.868,5.773);
\draw[gp path] (11.947,5.773)--(11.767,5.773);
\node[gp node right] at (1.504,5.773) { 0.5};
\draw[gp path] (1.688,6.457)--(1.868,6.457);
\draw[gp path] (11.947,6.457)--(11.767,6.457);
\node[gp node right] at (1.504,6.457) { 0.55};
\draw[gp path] (1.688,7.141)--(1.868,7.141);
\draw[gp path] (11.947,7.141)--(11.767,7.141);
\node[gp node right] at (1.504,7.141) { 0.6};
\draw[gp path] (1.688,7.825)--(1.868,7.825);
\draw[gp path] (11.947,7.825)--(11.767,7.825);
\node[gp node right] at (1.504,7.825) { 0.65};
\draw[gp path] (1.688,0.985)--(1.688,1.165);
\draw[gp path] (1.688,7.825)--(1.688,7.645);
\node[gp node center] at (1.688,0.677) { 5};
\draw[gp path] (3.740,0.985)--(3.740,1.165);
\draw[gp path] (3.740,7.825)--(3.740,7.645);
\node[gp node center] at (3.740,0.677) { 10};
\draw[gp path] (5.792,0.985)--(5.792,1.165);
\draw[gp path] (5.792,7.825)--(5.792,7.645);
\node[gp node center] at (5.792,0.677) { 15};
\draw[gp path] (7.843,0.985)--(7.843,1.165);
\draw[gp path] (7.843,7.825)--(7.843,7.645);
\node[gp node center] at (7.843,0.677) { 20};
\draw[gp path] (9.895,0.985)--(9.895,1.165);
\draw[gp path] (9.895,7.825)--(9.895,7.645);
\node[gp node center] at (9.895,0.677) { 25};
\draw[gp path] (11.947,0.985)--(11.947,1.165);
\draw[gp path] (11.947,7.825)--(11.947,7.645);
\node[gp node center] at (11.947,0.677) { 30};
\draw[gp path] (1.688,7.825)--(1.688,0.985)--(11.947,0.985)--(11.947,7.825)--cycle;
\node[gp node center,rotate=-270] at (0.246,4.405) {Velocità angolare [rad/s]};
\node[gp node center] at (6.817,0.215) {Tempo [s]};
\node[gp node center] at (6.817,8.287) {Velocità angolari in accelerazione [rad/s]};
\node[gp node left] at (7.167,1.627) {Dati};
\gpcolor{color=gp lt color 0}
\gpsetlinetype{gp lt plot 0}
\draw[gp path] (10.663,1.627)--(11.579,1.627);
\draw[gp path] (10.663,1.717)--(10.663,1.537);
\draw[gp path] (11.579,1.717)--(11.579,1.537);
\draw[gp path] (2.981,1.537)--(2.981,1.601);
\draw[gp path] (2.891,1.537)--(3.071,1.537);
\draw[gp path] (2.891,1.601)--(3.071,1.601);
\draw[gp path] (4.478,2.544)--(4.478,2.605);
\draw[gp path] (4.388,2.544)--(4.568,2.544);
\draw[gp path] (4.388,2.605)--(4.568,2.605);
\draw[gp path] (5.627,3.317)--(5.627,3.378);
\draw[gp path] (5.537,3.317)--(5.717,3.317);
\draw[gp path] (5.537,3.378)--(5.717,3.378);
\draw[gp path] (6.592,3.973)--(6.592,4.033);
\draw[gp path] (6.502,3.973)--(6.682,3.973);
\draw[gp path] (6.502,4.033)--(6.682,4.033);
\draw[gp path] (7.433,4.557)--(7.433,4.617);
\draw[gp path] (7.343,4.557)--(7.523,4.557);
\draw[gp path] (7.343,4.617)--(7.523,4.617);
\draw[gp path] (8.192,5.086)--(8.192,5.145);
\draw[gp path] (8.102,5.086)--(8.282,5.086);
\draw[gp path] (8.102,5.145)--(8.282,5.145);
\draw[gp path] (8.890,5.573)--(8.890,5.632);
\draw[gp path] (8.800,5.573)--(8.980,5.573);
\draw[gp path] (8.800,5.632)--(8.980,5.632);
\draw[gp path] (9.546,6.021)--(9.546,6.079);
\draw[gp path] (9.456,6.021)--(9.636,6.021);
\draw[gp path] (9.456,6.079)--(9.636,6.079);
\draw[gp path] (10.162,6.442)--(10.162,6.501);
\draw[gp path] (10.072,6.442)--(10.252,6.442);
\draw[gp path] (10.072,6.501)--(10.252,6.501);
\draw[gp path] (10.757,6.832)--(10.757,6.890);
\draw[gp path] (10.667,6.832)--(10.847,6.832);
\draw[gp path] (10.667,6.890)--(10.847,6.890);
\draw[gp path] (11.311,7.211)--(11.311,7.269);
\draw[gp path] (11.221,7.211)--(11.401,7.211);
\draw[gp path] (11.221,7.269)--(11.401,7.269);
\draw[gp path] (11.824,7.584)--(11.824,7.643);
\draw[gp path] (11.734,7.584)--(11.914,7.584);
\draw[gp path] (11.734,7.643)--(11.914,7.643);
\gpsetpointsize{4.00}
\gppoint{gp mark 1}{(2.981,1.569)}
\gppoint{gp mark 1}{(4.478,2.574)}
\gppoint{gp mark 1}{(5.627,3.348)}
\gppoint{gp mark 1}{(6.592,4.003)}
\gppoint{gp mark 1}{(7.433,4.587)}
\gppoint{gp mark 1}{(8.192,5.116)}
\gppoint{gp mark 1}{(8.890,5.602)}
\gppoint{gp mark 1}{(9.546,6.050)}
\gppoint{gp mark 1}{(10.162,6.471)}
\gppoint{gp mark 1}{(10.757,6.861)}
\gppoint{gp mark 1}{(11.311,7.240)}
\gppoint{gp mark 1}{(11.824,7.614)}
\gppoint{gp mark 1}{(11.121,1.627)}
\gpcolor{color=gp lt color border}
\node[gp node left] at (7.167,1.319) {Retta interpolante};
\gpcolor{color=gp lt color 1}
\gpsetlinetype{gp lt plot 1}
\draw[gp path] (10.663,1.319)--(11.579,1.319);
\draw[gp path] (2.981,1.549)--(3.070,1.610)--(3.159,1.671)--(3.249,1.732)--(3.338,1.793)%
  --(3.427,1.854)--(3.517,1.916)--(3.606,1.977)--(3.695,2.038)--(3.785,2.099)--(3.874,2.160)%
  --(3.963,2.221)--(4.053,2.282)--(4.142,2.344)--(4.231,2.405)--(4.321,2.466)--(4.410,2.527)%
  --(4.499,2.588)--(4.588,2.649)--(4.678,2.710)--(4.767,2.771)--(4.856,2.833)--(4.946,2.894)%
  --(5.035,2.955)--(5.124,3.016)--(5.214,3.077)--(5.303,3.138)--(5.392,3.199)--(5.482,3.260)%
  --(5.571,3.322)--(5.660,3.383)--(5.750,3.444)--(5.839,3.505)--(5.928,3.566)--(6.018,3.627)%
  --(6.107,3.688)--(6.196,3.749)--(6.286,3.811)--(6.375,3.872)--(6.464,3.933)--(6.554,3.994)%
  --(6.643,4.055)--(6.732,4.116)--(6.822,4.177)--(6.911,4.238)--(7.000,4.300)--(7.090,4.361)%
  --(7.179,4.422)--(7.268,4.483)--(7.358,4.544)--(7.447,4.605)--(7.536,4.666)--(7.626,4.728)%
  --(7.715,4.789)--(7.804,4.850)--(7.894,4.911)--(7.983,4.972)--(8.072,5.033)--(8.162,5.094)%
  --(8.251,5.155)--(8.340,5.217)--(8.430,5.278)--(8.519,5.339)--(8.608,5.400)--(8.697,5.461)%
  --(8.787,5.522)--(8.876,5.583)--(8.965,5.644)--(9.055,5.706)--(9.144,5.767)--(9.233,5.828)%
  --(9.323,5.889)--(9.412,5.950)--(9.501,6.011)--(9.591,6.072)--(9.680,6.133)--(9.769,6.195)%
  --(9.859,6.256)--(9.948,6.317)--(10.037,6.378)--(10.127,6.439)--(10.216,6.500)--(10.305,6.561)%
  --(10.395,6.622)--(10.484,6.684)--(10.573,6.745)--(10.663,6.806)--(10.752,6.867)--(10.841,6.928)%
  --(10.931,6.989)--(11.020,7.050)--(11.109,7.112)--(11.199,7.173)--(11.288,7.234)--(11.377,7.295)%
  --(11.467,7.356)--(11.556,7.417)--(11.645,7.478)--(11.735,7.539)--(11.824,7.601);
\gpcolor{color=gp lt color border}
\gpsetlinetype{gp lt border}
\draw[gp path] (1.688,7.825)--(1.688,0.985)--(11.947,0.985)--(11.947,7.825)--cycle;
%% coordinates of the plot area
\gpdefrectangularnode{gp plot 1}{\pgfpoint{1.688cm}{0.985cm}}{\pgfpoint{11.947cm}{7.825cm}}
\end{tikzpicture}
%% gnuplot variables

\caption{Decima serie, accelerazione}
\label{fig:1}
\end{grafico}

Qui invece i dati in decelerazione.
L'errore in decelerazione è abbastanza grande, probabilmente perché l'intervallo di tempo era troppo limitato per osservare un effetto significativo (infatti le $\beta$ sono molto piccole rispetto alle $\alpha$)

\begin{grafico}
    \centering
\begin{tikzpicture}[gnuplot]
%% generated with GNUPLOT 4.6p0 (Lua 5.1; terminal rev. 99, script rev. 100)
%% Tue 15 Apr 2014 09:47:08 PM CEST
\path (0.000,0.000) rectangle (12.500,8.750);
\gpcolor{color=gp lt color border}
\gpsetlinetype{gp lt border}
\gpsetlinewidth{1.00}
\draw[gp path] (1.688,0.985)--(1.868,0.985);
\draw[gp path] (11.947,0.985)--(11.767,0.985);
\node[gp node right] at (1.504,0.985) { 1.15};
\draw[gp path] (1.688,2.125)--(1.868,2.125);
\draw[gp path] (11.947,2.125)--(11.767,2.125);
\node[gp node right] at (1.504,2.125) { 1.2};
\draw[gp path] (1.688,3.265)--(1.868,3.265);
\draw[gp path] (11.947,3.265)--(11.767,3.265);
\node[gp node right] at (1.504,3.265) { 1.25};
\draw[gp path] (1.688,4.405)--(1.868,4.405);
\draw[gp path] (11.947,4.405)--(11.767,4.405);
\node[gp node right] at (1.504,4.405) { 1.3};
\draw[gp path] (1.688,5.545)--(1.868,5.545);
\draw[gp path] (11.947,5.545)--(11.767,5.545);
\node[gp node right] at (1.504,5.545) { 1.35};
\draw[gp path] (1.688,6.685)--(1.868,6.685);
\draw[gp path] (11.947,6.685)--(11.767,6.685);
\node[gp node right] at (1.504,6.685) { 1.4};
\draw[gp path] (1.688,7.825)--(1.868,7.825);
\draw[gp path] (11.947,7.825)--(11.767,7.825);
\node[gp node right] at (1.504,7.825) { 1.45};
\draw[gp path] (1.688,0.985)--(1.688,1.165);
\draw[gp path] (1.688,7.825)--(1.688,7.645);
\node[gp node center] at (1.688,0.677) { 0};
\draw[gp path] (3.398,0.985)--(3.398,1.165);
\draw[gp path] (3.398,7.825)--(3.398,7.645);
\node[gp node center] at (3.398,0.677) { 5};
\draw[gp path] (5.108,0.985)--(5.108,1.165);
\draw[gp path] (5.108,7.825)--(5.108,7.645);
\node[gp node center] at (5.108,0.677) { 10};
\draw[gp path] (6.818,0.985)--(6.818,1.165);
\draw[gp path] (6.818,7.825)--(6.818,7.645);
\node[gp node center] at (6.818,0.677) { 15};
\draw[gp path] (8.527,0.985)--(8.527,1.165);
\draw[gp path] (8.527,7.825)--(8.527,7.645);
\node[gp node center] at (8.527,0.677) { 20};
\draw[gp path] (10.237,0.985)--(10.237,1.165);
\draw[gp path] (10.237,7.825)--(10.237,7.645);
\node[gp node center] at (10.237,0.677) { 25};
\draw[gp path] (11.947,0.985)--(11.947,1.165);
\draw[gp path] (11.947,7.825)--(11.947,7.645);
\node[gp node center] at (11.947,0.677) { 30};
\draw[gp path] (1.688,7.825)--(1.688,0.985)--(11.947,0.985)--(11.947,7.825)--cycle;
\node[gp node center,rotate=-270] at (0.246,4.405) {Velocità angolare [rad/s]};
\node[gp node center] at (6.817,0.215) {Tempo [s]};
\node[gp node center] at (6.817,8.287) {Velocità angolari in decelerazione [rad/s]};
\node[gp node left] at (7.167,1.627) {Dati};
\gpcolor{color=gp lt color 0}
\gpsetlinetype{gp lt plot 0}
\draw[gp path] (10.663,1.627)--(11.579,1.627);
\draw[gp path] (10.663,1.717)--(10.663,1.537);
\draw[gp path] (11.579,1.717)--(11.579,1.537);
\draw[gp path] (2.098,2.119)--(2.098,7.090);
\draw[gp path] (2.008,2.119)--(2.188,2.119);
\draw[gp path] (2.008,7.090)--(2.188,7.090);
\draw[gp path] (2.560,1.748)--(2.560,3.950);
\draw[gp path] (2.470,1.748)--(2.650,1.748);
\draw[gp path] (2.470,3.950)--(2.650,3.950);
\draw[gp path] (3.005,1.942)--(3.005,3.391);
\draw[gp path] (2.915,1.942)--(3.095,1.942);
\draw[gp path] (2.915,3.391)--(3.095,3.391);
\draw[gp path] (3.432,2.299)--(3.432,3.400);
\draw[gp path] (3.342,2.299)--(3.522,2.299);
\draw[gp path] (3.342,3.400)--(3.522,3.400);
\draw[gp path] (3.894,2.092)--(3.894,2.953);
\draw[gp path] (3.804,2.092)--(3.984,2.092);
\draw[gp path] (3.804,2.953)--(3.984,2.953);
\draw[gp path] (4.338,2.129)--(4.338,2.844);
\draw[gp path] (4.248,2.129)--(4.428,2.129);
\draw[gp path] (4.248,2.844)--(4.428,2.844);
\draw[gp path] (4.783,2.155)--(4.783,2.767);
\draw[gp path] (4.693,2.155)--(4.873,2.155);
\draw[gp path] (4.693,2.767)--(4.873,2.767);
\draw[gp path] (5.244,2.044)--(5.244,2.574);
\draw[gp path] (5.154,2.044)--(5.334,2.044);
\draw[gp path] (5.154,2.574)--(5.334,2.574);
\draw[gp path] (5.689,2.074)--(5.689,2.545);
\draw[gp path] (5.599,2.074)--(5.779,2.074);
\draw[gp path] (5.599,2.545)--(5.779,2.545);
\draw[gp path] (6.134,2.097)--(6.134,2.521);
\draw[gp path] (6.044,2.097)--(6.224,2.097);
\draw[gp path] (6.044,2.521)--(6.224,2.521);
\draw[gp path] (6.595,2.022)--(6.595,2.404);
\draw[gp path] (6.505,2.022)--(6.685,2.022);
\draw[gp path] (6.505,2.404)--(6.685,2.404);
\draw[gp path] (7.074,1.874)--(7.074,2.220);
\draw[gp path] (6.984,1.874)--(7.164,1.874);
\draw[gp path] (6.984,2.220)--(7.164,2.220);
\draw[gp path] (7.553,1.749)--(7.553,2.066);
\draw[gp path] (7.463,1.749)--(7.643,1.749);
\draw[gp path] (7.463,2.066)--(7.643,2.066);
\draw[gp path] (7.997,1.789)--(7.997,2.083);
\draw[gp path] (7.907,1.789)--(8.087,1.789);
\draw[gp path] (7.907,2.083)--(8.087,2.083);
\draw[gp path] (8.476,1.687)--(8.476,1.960);
\draw[gp path] (8.386,1.687)--(8.566,1.687);
\draw[gp path] (8.386,1.960)--(8.566,1.960);
\draw[gp path] (8.938,1.662)--(8.938,1.917);
\draw[gp path] (8.848,1.662)--(9.028,1.662);
\draw[gp path] (8.848,1.917)--(9.028,1.917);
\draw[gp path] (9.434,1.522)--(9.434,1.759);
\draw[gp path] (9.344,1.522)--(9.524,1.522);
\draw[gp path] (9.344,1.759)--(9.524,1.759);
\draw[gp path] (9.912,1.453)--(9.912,1.676);
\draw[gp path] (9.822,1.453)--(10.002,1.453);
\draw[gp path] (9.822,1.676)--(10.002,1.676);
\draw[gp path] (10.391,1.392)--(10.391,1.602);
\draw[gp path] (10.301,1.392)--(10.481,1.392);
\draw[gp path] (10.301,1.602)--(10.481,1.602);
\draw[gp path] (10.870,1.338)--(10.870,1.536);
\draw[gp path] (10.780,1.338)--(10.960,1.338);
\draw[gp path] (10.780,1.536)--(10.960,1.536);
\draw[gp path] (11.366,1.242)--(11.366,1.430);
\draw[gp path] (11.276,1.242)--(11.456,1.242);
\draw[gp path] (11.276,1.430)--(11.456,1.430);
\draw[gp path] (11.844,1.200)--(11.844,1.378);
\draw[gp path] (11.754,1.200)--(11.934,1.200);
\draw[gp path] (11.754,1.378)--(11.934,1.378);
\gpsetpointsize{4.00}
\gppoint{gp mark 1}{(2.098,4.604)}
\gppoint{gp mark 1}{(2.560,2.849)}
\gppoint{gp mark 1}{(3.005,2.667)}
\gppoint{gp mark 1}{(3.432,2.849)}
\gppoint{gp mark 1}{(3.894,2.523)}
\gppoint{gp mark 1}{(4.338,2.487)}
\gppoint{gp mark 1}{(4.783,2.461)}
\gppoint{gp mark 1}{(5.244,2.309)}
\gppoint{gp mark 1}{(5.689,2.309)}
\gppoint{gp mark 1}{(6.134,2.309)}
\gppoint{gp mark 1}{(6.595,2.213)}
\gppoint{gp mark 1}{(7.074,2.047)}
\gppoint{gp mark 1}{(7.553,1.908)}
\gppoint{gp mark 1}{(7.997,1.936)}
\gppoint{gp mark 1}{(8.476,1.823)}
\gppoint{gp mark 1}{(8.938,1.789)}
\gppoint{gp mark 1}{(9.434,1.640)}
\gppoint{gp mark 1}{(9.912,1.565)}
\gppoint{gp mark 1}{(10.391,1.497)}
\gppoint{gp mark 1}{(10.870,1.437)}
\gppoint{gp mark 1}{(11.366,1.336)}
\gppoint{gp mark 1}{(11.844,1.289)}
\gppoint{gp mark 1}{(11.121,1.627)}
\gpcolor{color=gp lt color border}
\node[gp node left] at (7.167,1.319) {Retta interpolante};
\gpcolor{color=gp lt color 1}
\gpsetlinetype{gp lt plot 1}
\draw[gp path] (10.663,1.319)--(11.579,1.319);
\draw[gp path] (2.098,2.902)--(2.197,2.885)--(2.295,2.869)--(2.394,2.852)--(2.492,2.836)%
  --(2.591,2.819)--(2.689,2.803)--(2.787,2.786)--(2.886,2.770)--(2.984,2.753)--(3.083,2.736)%
  --(3.181,2.720)--(3.280,2.703)--(3.378,2.687)--(3.477,2.670)--(3.575,2.654)--(3.673,2.637)%
  --(3.772,2.621)--(3.870,2.604)--(3.969,2.588)--(4.067,2.571)--(4.166,2.555)--(4.264,2.538)%
  --(4.363,2.522)--(4.461,2.505)--(4.559,2.488)--(4.658,2.472)--(4.756,2.455)--(4.855,2.439)%
  --(4.953,2.422)--(5.052,2.406)--(5.150,2.389)--(5.249,2.373)--(5.347,2.356)--(5.445,2.340)%
  --(5.544,2.323)--(5.642,2.307)--(5.741,2.290)--(5.839,2.273)--(5.938,2.257)--(6.036,2.240)%
  --(6.135,2.224)--(6.233,2.207)--(6.331,2.191)--(6.430,2.174)--(6.528,2.158)--(6.627,2.141)%
  --(6.725,2.125)--(6.824,2.108)--(6.922,2.092)--(7.021,2.075)--(7.119,2.059)--(7.217,2.042)%
  --(7.316,2.025)--(7.414,2.009)--(7.513,1.992)--(7.611,1.976)--(7.710,1.959)--(7.808,1.943)%
  --(7.907,1.926)--(8.005,1.910)--(8.104,1.893)--(8.202,1.877)--(8.300,1.860)--(8.399,1.844)%
  --(8.497,1.827)--(8.596,1.811)--(8.694,1.794)--(8.793,1.777)--(8.891,1.761)--(8.990,1.744)%
  --(9.088,1.728)--(9.186,1.711)--(9.285,1.695)--(9.383,1.678)--(9.482,1.662)--(9.580,1.645)%
  --(9.679,1.629)--(9.777,1.612)--(9.876,1.596)--(9.974,1.579)--(10.072,1.563)--(10.171,1.546)%
  --(10.269,1.529)--(10.368,1.513)--(10.466,1.496)--(10.565,1.480)--(10.663,1.463)--(10.762,1.447)%
  --(10.860,1.430)--(10.958,1.414)--(11.057,1.397)--(11.155,1.381)--(11.254,1.364)--(11.352,1.348)%
  --(11.451,1.331)--(11.549,1.315)--(11.648,1.298)--(11.746,1.281)--(11.844,1.265);
\gpcolor{color=gp lt color border}
\gpsetlinetype{gp lt border}
\draw[gp path] (1.688,7.825)--(1.688,0.985)--(11.947,0.985)--(11.947,7.825)--cycle;
%% coordinates of the plot area
\gpdefrectangularnode{gp plot 1}{\pgfpoint{1.688cm}{0.985cm}}{\pgfpoint{11.947cm}{7.825cm}}
\end{tikzpicture}
%% gnuplot variables

\caption{Prima serie, decelerazione}
\label{fig:1}
\end{grafico}

\begin{grafico}
    \centering
\begin{tikzpicture}[gnuplot]
%% generated with GNUPLOT 4.6p0 (Lua 5.1; terminal rev. 99, script rev. 100)
%% Tue 15 Apr 2014 06:32:33 PM CEST
\path (0.000,0.000) rectangle (12.500,8.750);
\gpcolor{color=gp lt color border}
\gpsetlinetype{gp lt border}
\gpsetlinewidth{1.00}
\draw[gp path] (1.688,0.985)--(1.868,0.985);
\draw[gp path] (11.947,0.985)--(11.767,0.985);
\node[gp node right] at (1.504,0.985) { 1.04};
\draw[gp path] (1.688,1.669)--(1.868,1.669);
\draw[gp path] (11.947,1.669)--(11.767,1.669);
\node[gp node right] at (1.504,1.669) { 1.06};
\draw[gp path] (1.688,2.353)--(1.868,2.353);
\draw[gp path] (11.947,2.353)--(11.767,2.353);
\node[gp node right] at (1.504,2.353) { 1.08};
\draw[gp path] (1.688,3.037)--(1.868,3.037);
\draw[gp path] (11.947,3.037)--(11.767,3.037);
\node[gp node right] at (1.504,3.037) { 1.1};
\draw[gp path] (1.688,3.721)--(1.868,3.721);
\draw[gp path] (11.947,3.721)--(11.767,3.721);
\node[gp node right] at (1.504,3.721) { 1.12};
\draw[gp path] (1.688,4.405)--(1.868,4.405);
\draw[gp path] (11.947,4.405)--(11.767,4.405);
\node[gp node right] at (1.504,4.405) { 1.14};
\draw[gp path] (1.688,5.089)--(1.868,5.089);
\draw[gp path] (11.947,5.089)--(11.767,5.089);
\node[gp node right] at (1.504,5.089) { 1.16};
\draw[gp path] (1.688,5.773)--(1.868,5.773);
\draw[gp path] (11.947,5.773)--(11.767,5.773);
\node[gp node right] at (1.504,5.773) { 1.18};
\draw[gp path] (1.688,6.457)--(1.868,6.457);
\draw[gp path] (11.947,6.457)--(11.767,6.457);
\node[gp node right] at (1.504,6.457) { 1.2};
\draw[gp path] (1.688,7.141)--(1.868,7.141);
\draw[gp path] (11.947,7.141)--(11.767,7.141);
\node[gp node right] at (1.504,7.141) { 1.22};
\draw[gp path] (1.688,7.825)--(1.868,7.825);
\draw[gp path] (11.947,7.825)--(11.767,7.825);
\node[gp node right] at (1.504,7.825) { 1.24};
\draw[gp path] (1.688,0.985)--(1.688,1.165);
\draw[gp path] (1.688,7.825)--(1.688,7.645);
\node[gp node center] at (1.688,0.677) { 0};
\draw[gp path] (2.828,0.985)--(2.828,1.165);
\draw[gp path] (2.828,7.825)--(2.828,7.645);
\node[gp node center] at (2.828,0.677) { 2};
\draw[gp path] (3.968,0.985)--(3.968,1.165);
\draw[gp path] (3.968,7.825)--(3.968,7.645);
\node[gp node center] at (3.968,0.677) { 4};
\draw[gp path] (5.108,0.985)--(5.108,1.165);
\draw[gp path] (5.108,7.825)--(5.108,7.645);
\node[gp node center] at (5.108,0.677) { 6};
\draw[gp path] (6.248,0.985)--(6.248,1.165);
\draw[gp path] (6.248,7.825)--(6.248,7.645);
\node[gp node center] at (6.248,0.677) { 8};
\draw[gp path] (7.387,0.985)--(7.387,1.165);
\draw[gp path] (7.387,7.825)--(7.387,7.645);
\node[gp node center] at (7.387,0.677) { 10};
\draw[gp path] (8.527,0.985)--(8.527,1.165);
\draw[gp path] (8.527,7.825)--(8.527,7.645);
\node[gp node center] at (8.527,0.677) { 12};
\draw[gp path] (9.667,0.985)--(9.667,1.165);
\draw[gp path] (9.667,7.825)--(9.667,7.645);
\node[gp node center] at (9.667,0.677) { 14};
\draw[gp path] (10.807,0.985)--(10.807,1.165);
\draw[gp path] (10.807,7.825)--(10.807,7.645);
\node[gp node center] at (10.807,0.677) { 16};
\draw[gp path] (11.947,0.985)--(11.947,1.165);
\draw[gp path] (11.947,7.825)--(11.947,7.645);
\node[gp node center] at (11.947,0.677) { 18};
\draw[gp path] (1.688,7.825)--(1.688,0.985)--(11.947,0.985)--(11.947,7.825)--cycle;
\node[gp node center,rotate=-270] at (0.246,4.405) {Velocità angolare [rad/s]};
\node[gp node center] at (6.817,0.215) {Tempo [s]};
\node[gp node center] at (6.817,8.287) {Velocità angolare, accelerazione [rad/s]};
\node[gp node left] at (7.167,1.627) {Dati};
\gpcolor{color=gp lt color 0}
\gpsetlinetype{gp lt plot 0}
\draw[gp path] (10.663,1.627)--(11.579,1.627);
\draw[gp path] (10.663,1.717)--(10.663,1.537);
\draw[gp path] (11.579,1.717)--(11.579,1.537);
\draw[gp path] (2.472,1.640)--(2.472,7.320);
\draw[gp path] (2.382,1.640)--(2.562,1.640);
\draw[gp path] (2.382,7.320)--(2.562,7.320);
\draw[gp path] (3.244,3.325)--(3.244,6.207);
\draw[gp path] (3.154,3.325)--(3.334,3.325);
\draw[gp path] (3.154,6.207)--(3.334,6.207);
\draw[gp path] (4.030,3.669)--(4.030,5.576);
\draw[gp path] (3.940,3.669)--(4.120,3.669);
\draw[gp path] (3.940,5.576)--(4.120,5.576);
\draw[gp path] (4.843,3.532)--(4.843,4.934);
\draw[gp path] (4.753,3.532)--(4.933,3.532);
\draw[gp path] (4.753,4.934)--(4.933,4.934);
\draw[gp path] (5.643,3.556)--(5.643,4.671);
\draw[gp path] (5.553,3.556)--(5.733,3.556);
\draw[gp path] (5.553,4.671)--(5.733,4.671);
\draw[gp path] (6.447,3.550)--(6.447,4.474);
\draw[gp path] (6.357,3.550)--(6.537,3.550);
\draw[gp path] (6.357,4.474)--(6.537,4.474);
\draw[gp path] (7.236,3.642)--(7.236,4.435);
\draw[gp path] (7.146,3.642)--(7.326,3.642);
\draw[gp path] (7.146,4.435)--(7.326,4.435);
\draw[gp path] (8.051,3.558)--(8.051,4.247);
\draw[gp path] (7.961,3.558)--(8.141,3.558);
\draw[gp path] (7.961,4.247)--(8.141,4.247);
\draw[gp path] (8.869,3.478)--(8.869,4.086);
\draw[gp path] (8.779,3.478)--(8.959,3.478);
\draw[gp path] (8.779,4.086)--(8.959,4.086);
\draw[gp path] (9.707,3.320)--(9.707,3.862);
\draw[gp path] (9.617,3.320)--(9.797,3.320);
\draw[gp path] (9.617,3.862)--(9.797,3.862);
\draw[gp path] (10.491,3.423)--(10.491,3.918);
\draw[gp path] (10.401,3.423)--(10.581,3.423);
\draw[gp path] (10.401,3.918)--(10.581,3.918);
\draw[gp path] (11.349,3.218)--(11.349,3.667);
\draw[gp path] (11.259,3.218)--(11.439,3.218);
\draw[gp path] (11.259,3.667)--(11.439,3.667);
\gpsetpointsize{4.00}
\gppoint{gp mark 1}{(2.472,4.480)}
\gppoint{gp mark 1}{(3.244,4.766)}
\gppoint{gp mark 1}{(4.030,4.622)}
\gppoint{gp mark 1}{(4.843,4.233)}
\gppoint{gp mark 1}{(5.643,4.114)}
\gppoint{gp mark 1}{(6.447,4.012)}
\gppoint{gp mark 1}{(7.236,4.038)}
\gppoint{gp mark 1}{(8.051,3.902)}
\gppoint{gp mark 1}{(8.869,3.782)}
\gppoint{gp mark 1}{(9.707,3.591)}
\gppoint{gp mark 1}{(10.491,3.670)}
\gppoint{gp mark 1}{(11.349,3.443)}
\gppoint{gp mark 1}{(11.121,1.627)}
\gpcolor{color=gp lt color border}
\node[gp node left] at (7.167,1.319) {Retta interpolante};
\gpcolor{color=gp lt color 1}
\gpsetlinetype{gp lt plot 1}
\draw[gp path] (10.663,1.319)--(11.579,1.319);
\draw[gp path] (2.472,4.652)--(2.561,4.639)--(2.651,4.627)--(2.741,4.615)--(2.830,4.603)%
  --(2.920,4.591)--(3.010,4.578)--(3.099,4.566)--(3.189,4.554)--(3.279,4.542)--(3.368,4.530)%
  --(3.458,4.517)--(3.548,4.505)--(3.637,4.493)--(3.727,4.481)--(3.817,4.469)--(3.906,4.457)%
  --(3.996,4.444)--(4.086,4.432)--(4.175,4.420)--(4.265,4.408)--(4.355,4.396)--(4.444,4.383)%
  --(4.534,4.371)--(4.624,4.359)--(4.713,4.347)--(4.803,4.335)--(4.893,4.322)--(4.982,4.310)%
  --(5.072,4.298)--(5.162,4.286)--(5.251,4.274)--(5.341,4.262)--(5.431,4.249)--(5.520,4.237)%
  --(5.610,4.225)--(5.700,4.213)--(5.789,4.201)--(5.879,4.188)--(5.969,4.176)--(6.058,4.164)%
  --(6.148,4.152)--(6.238,4.140)--(6.327,4.127)--(6.417,4.115)--(6.507,4.103)--(6.596,4.091)%
  --(6.686,4.079)--(6.776,4.067)--(6.865,4.054)--(6.955,4.042)--(7.045,4.030)--(7.134,4.018)%
  --(7.224,4.006)--(7.314,3.993)--(7.403,3.981)--(7.493,3.969)--(7.583,3.957)--(7.672,3.945)%
  --(7.762,3.932)--(7.852,3.920)--(7.941,3.908)--(8.031,3.896)--(8.121,3.884)--(8.210,3.872)%
  --(8.300,3.859)--(8.390,3.847)--(8.479,3.835)--(8.569,3.823)--(8.659,3.811)--(8.748,3.798)%
  --(8.838,3.786)--(8.928,3.774)--(9.017,3.762)--(9.107,3.750)--(9.197,3.737)--(9.286,3.725)%
  --(9.376,3.713)--(9.466,3.701)--(9.555,3.689)--(9.645,3.677)--(9.735,3.664)--(9.824,3.652)%
  --(9.914,3.640)--(10.004,3.628)--(10.093,3.616)--(10.183,3.603)--(10.273,3.591)--(10.362,3.579)%
  --(10.452,3.567)--(10.542,3.555)--(10.631,3.542)--(10.721,3.530)--(10.811,3.518)--(10.900,3.506)%
  --(10.990,3.494)--(11.080,3.482)--(11.169,3.469)--(11.259,3.457)--(11.349,3.445);
\gpcolor{color=gp lt color border}
\gpsetlinetype{gp lt border}
\draw[gp path] (1.688,7.825)--(1.688,0.985)--(11.947,0.985)--(11.947,7.825)--cycle;
%% coordinates of the plot area
\gpdefrectangularnode{gp plot 1}{\pgfpoint{1.688cm}{0.985cm}}{\pgfpoint{11.947cm}{7.825cm}}
\end{tikzpicture}
%% gnuplot variables

\caption{Seconda serie, decelerazione}
\label{fig:1}
\end{grafico}

\begin{grafico}
    \centering
\begin{tikzpicture}[gnuplot]
%% generated with GNUPLOT 4.6p0 (Lua 5.1; terminal rev. 99, script rev. 100)
%% Tue 15 Apr 2014 09:47:08 PM CEST
\path (0.000,0.000) rectangle (12.500,8.750);
\gpcolor{color=gp lt color border}
\gpsetlinetype{gp lt border}
\gpsetlinewidth{1.00}
\draw[gp path] (1.688,0.985)--(1.868,0.985);
\draw[gp path] (11.947,0.985)--(11.767,0.985);
\node[gp node right] at (1.504,0.985) { 1.1};
\draw[gp path] (1.688,1.669)--(1.868,1.669);
\draw[gp path] (11.947,1.669)--(11.767,1.669);
\node[gp node right] at (1.504,1.669) { 1.12};
\draw[gp path] (1.688,2.353)--(1.868,2.353);
\draw[gp path] (11.947,2.353)--(11.767,2.353);
\node[gp node right] at (1.504,2.353) { 1.14};
\draw[gp path] (1.688,3.037)--(1.868,3.037);
\draw[gp path] (11.947,3.037)--(11.767,3.037);
\node[gp node right] at (1.504,3.037) { 1.16};
\draw[gp path] (1.688,3.721)--(1.868,3.721);
\draw[gp path] (11.947,3.721)--(11.767,3.721);
\node[gp node right] at (1.504,3.721) { 1.18};
\draw[gp path] (1.688,4.405)--(1.868,4.405);
\draw[gp path] (11.947,4.405)--(11.767,4.405);
\node[gp node right] at (1.504,4.405) { 1.2};
\draw[gp path] (1.688,5.089)--(1.868,5.089);
\draw[gp path] (11.947,5.089)--(11.767,5.089);
\node[gp node right] at (1.504,5.089) { 1.22};
\draw[gp path] (1.688,5.773)--(1.868,5.773);
\draw[gp path] (11.947,5.773)--(11.767,5.773);
\node[gp node right] at (1.504,5.773) { 1.24};
\draw[gp path] (1.688,6.457)--(1.868,6.457);
\draw[gp path] (11.947,6.457)--(11.767,6.457);
\node[gp node right] at (1.504,6.457) { 1.26};
\draw[gp path] (1.688,7.141)--(1.868,7.141);
\draw[gp path] (11.947,7.141)--(11.767,7.141);
\node[gp node right] at (1.504,7.141) { 1.28};
\draw[gp path] (1.688,7.825)--(1.868,7.825);
\draw[gp path] (11.947,7.825)--(11.767,7.825);
\node[gp node right] at (1.504,7.825) { 1.3};
\draw[gp path] (1.688,0.985)--(1.688,1.165);
\draw[gp path] (1.688,7.825)--(1.688,7.645);
\node[gp node center] at (1.688,0.677) { 0};
\draw[gp path] (3.154,0.985)--(3.154,1.165);
\draw[gp path] (3.154,7.825)--(3.154,7.645);
\node[gp node center] at (3.154,0.677) { 5};
\draw[gp path] (4.619,0.985)--(4.619,1.165);
\draw[gp path] (4.619,7.825)--(4.619,7.645);
\node[gp node center] at (4.619,0.677) { 10};
\draw[gp path] (6.085,0.985)--(6.085,1.165);
\draw[gp path] (6.085,7.825)--(6.085,7.645);
\node[gp node center] at (6.085,0.677) { 15};
\draw[gp path] (7.550,0.985)--(7.550,1.165);
\draw[gp path] (7.550,7.825)--(7.550,7.645);
\node[gp node center] at (7.550,0.677) { 20};
\draw[gp path] (9.016,0.985)--(9.016,1.165);
\draw[gp path] (9.016,7.825)--(9.016,7.645);
\node[gp node center] at (9.016,0.677) { 25};
\draw[gp path] (10.481,0.985)--(10.481,1.165);
\draw[gp path] (10.481,7.825)--(10.481,7.645);
\node[gp node center] at (10.481,0.677) { 30};
\draw[gp path] (11.947,0.985)--(11.947,1.165);
\draw[gp path] (11.947,7.825)--(11.947,7.645);
\node[gp node center] at (11.947,0.677) { 35};
\draw[gp path] (1.688,7.825)--(1.688,0.985)--(11.947,0.985)--(11.947,7.825)--cycle;
\node[gp node center,rotate=-270] at (0.246,4.405) {Velocità angolare [rad/s]};
\node[gp node center] at (6.817,0.215) {Tempo [s]};
\node[gp node center] at (6.817,8.287) {Velocità angolari in decelerazione [rad/s]};
\node[gp node left] at (7.167,1.627) {Dati};
\gpcolor{color=gp lt color 0}
\gpsetlinetype{gp lt plot 0}
\draw[gp path] (10.663,1.627)--(11.579,1.627);
\draw[gp path] (10.663,1.717)--(10.663,1.537);
\draw[gp path] (11.579,1.717)--(11.579,1.537);
\draw[gp path] (2.071,1.370)--(2.071,7.676);
\draw[gp path] (1.981,1.370)--(2.161,1.370);
\draw[gp path] (1.981,7.676)--(2.161,7.676);
\draw[gp path] (2.444,3.388)--(2.444,6.615);
\draw[gp path] (2.354,3.388)--(2.534,3.388);
\draw[gp path] (2.354,6.615)--(2.534,6.615);
\draw[gp path] (2.828,3.723)--(2.828,5.852);
\draw[gp path] (2.738,3.723)--(2.918,3.723);
\draw[gp path] (2.738,5.852)--(2.918,5.852);
\draw[gp path] (3.237,3.248)--(3.237,4.786);
\draw[gp path] (3.147,3.248)--(3.327,3.248);
\draw[gp path] (3.147,4.786)--(3.327,4.786);
\draw[gp path] (3.605,3.801)--(3.605,5.056);
\draw[gp path] (3.515,3.801)--(3.695,3.801);
\draw[gp path] (3.515,5.056)--(3.695,5.056);
\draw[gp path] (3.989,3.895)--(3.989,4.941);
\draw[gp path] (3.899,3.895)--(4.079,3.895);
\draw[gp path] (3.899,4.941)--(4.079,4.941);
\draw[gp path] (4.382,3.832)--(4.382,4.722);
\draw[gp path] (4.292,3.832)--(4.472,3.832);
\draw[gp path] (4.292,4.722)--(4.472,4.722);
\draw[gp path] (4.773,3.803)--(4.773,4.578);
\draw[gp path] (4.683,3.803)--(4.863,3.803);
\draw[gp path] (4.683,4.578)--(4.863,4.578);
\draw[gp path] (5.170,3.713)--(5.170,4.398);
\draw[gp path] (5.080,3.713)--(5.260,3.713);
\draw[gp path] (5.080,4.398)--(5.260,4.398);
\draw[gp path] (5.557,3.747)--(5.557,4.363);
\draw[gp path] (5.467,3.747)--(5.647,3.747);
\draw[gp path] (5.467,4.363)--(5.647,4.363);
\draw[gp path] (5.967,3.555)--(5.967,4.109);
\draw[gp path] (5.877,3.555)--(6.057,3.555);
\draw[gp path] (5.877,4.109)--(6.057,4.109);
\draw[gp path] (6.378,3.397)--(6.378,3.900);
\draw[gp path] (6.288,3.397)--(6.468,3.397);
\draw[gp path] (6.288,3.900)--(6.468,3.900);
\draw[gp path] (6.781,3.321)--(6.781,3.783);
\draw[gp path] (6.691,3.321)--(6.871,3.321);
\draw[gp path] (6.691,3.783)--(6.871,3.783);
\draw[gp path] (7.194,3.182)--(7.194,3.608);
\draw[gp path] (7.104,3.182)--(7.284,3.182);
\draw[gp path] (7.104,3.608)--(7.284,3.608);
\draw[gp path] (7.600,3.111)--(7.600,3.507);
\draw[gp path] (7.510,3.111)--(7.690,3.111);
\draw[gp path] (7.510,3.507)--(7.690,3.507);
\draw[gp path] (8.010,3.022)--(8.010,3.391);
\draw[gp path] (7.920,3.022)--(8.100,3.022);
\draw[gp path] (7.920,3.391)--(8.100,3.391);
\draw[gp path] (8.418,2.961)--(8.418,3.307);
\draw[gp path] (8.328,2.961)--(8.508,2.961);
\draw[gp path] (8.328,3.307)--(8.508,3.307);
\draw[gp path] (8.844,2.801)--(8.844,3.126);
\draw[gp path] (8.754,2.801)--(8.934,2.801);
\draw[gp path] (8.754,3.126)--(8.934,3.126);
\draw[gp path] (9.275,2.638)--(9.275,2.942);
\draw[gp path] (9.185,2.638)--(9.365,2.638);
\draw[gp path] (9.185,2.942)--(9.365,2.942);
\draw[gp path] (9.683,2.606)--(9.683,2.894);
\draw[gp path] (9.593,2.606)--(9.773,2.606);
\draw[gp path] (9.593,2.894)--(9.773,2.894);
\draw[gp path] (10.114,2.468)--(10.114,2.741);
\draw[gp path] (10.024,2.468)--(10.204,2.468);
\draw[gp path] (10.024,2.741)--(10.204,2.741);
\draw[gp path] (10.525,2.427)--(10.525,2.687);
\draw[gp path] (10.435,2.427)--(10.615,2.427);
\draw[gp path] (10.435,2.687)--(10.615,2.687);
\gpsetpointsize{4.00}
\gppoint{gp mark 1}{(2.071,4.523)}
\gppoint{gp mark 1}{(2.444,5.002)}
\gppoint{gp mark 1}{(2.828,4.787)}
\gppoint{gp mark 1}{(3.237,4.017)}
\gppoint{gp mark 1}{(3.605,4.429)}
\gppoint{gp mark 1}{(3.989,4.418)}
\gppoint{gp mark 1}{(4.382,4.277)}
\gppoint{gp mark 1}{(4.773,4.191)}
\gppoint{gp mark 1}{(5.170,4.055)}
\gppoint{gp mark 1}{(5.557,4.055)}
\gppoint{gp mark 1}{(5.967,3.832)}
\gppoint{gp mark 1}{(6.378,3.648)}
\gppoint{gp mark 1}{(6.781,3.552)}
\gppoint{gp mark 1}{(7.194,3.395)}
\gppoint{gp mark 1}{(7.600,3.309)}
\gppoint{gp mark 1}{(8.010,3.206)}
\gppoint{gp mark 1}{(8.418,3.134)}
\gppoint{gp mark 1}{(8.844,2.964)}
\gppoint{gp mark 1}{(9.275,2.790)}
\gppoint{gp mark 1}{(9.683,2.750)}
\gppoint{gp mark 1}{(10.114,2.604)}
\gppoint{gp mark 1}{(10.525,2.557)}
\gppoint{gp mark 1}{(11.121,1.627)}
\gpcolor{color=gp lt color border}
\node[gp node left] at (7.167,1.319) {Retta interpolante};
\gpcolor{color=gp lt color 1}
\gpsetlinetype{gp lt plot 1}
\draw[gp path] (10.663,1.319)--(11.579,1.319);
\draw[gp path] (2.071,4.892)--(2.156,4.868)--(2.241,4.844)--(2.327,4.820)--(2.412,4.795)%
  --(2.498,4.771)--(2.583,4.747)--(2.668,4.723)--(2.754,4.699)--(2.839,4.675)--(2.925,4.651)%
  --(3.010,4.627)--(3.095,4.603)--(3.181,4.578)--(3.266,4.554)--(3.352,4.530)--(3.437,4.506)%
  --(3.522,4.482)--(3.608,4.458)--(3.693,4.434)--(3.779,4.410)--(3.864,4.386)--(3.949,4.361)%
  --(4.035,4.337)--(4.120,4.313)--(4.206,4.289)--(4.291,4.265)--(4.376,4.241)--(4.462,4.217)%
  --(4.547,4.193)--(4.633,4.169)--(4.718,4.144)--(4.803,4.120)--(4.889,4.096)--(4.974,4.072)%
  --(5.060,4.048)--(5.145,4.024)--(5.230,4.000)--(5.316,3.976)--(5.401,3.952)--(5.487,3.927)%
  --(5.572,3.903)--(5.657,3.879)--(5.743,3.855)--(5.828,3.831)--(5.914,3.807)--(5.999,3.783)%
  --(6.084,3.759)--(6.170,3.735)--(6.255,3.710)--(6.341,3.686)--(6.426,3.662)--(6.511,3.638)%
  --(6.597,3.614)--(6.682,3.590)--(6.768,3.566)--(6.853,3.542)--(6.938,3.517)--(7.024,3.493)%
  --(7.109,3.469)--(7.195,3.445)--(7.280,3.421)--(7.365,3.397)--(7.451,3.373)--(7.536,3.349)%
  --(7.622,3.325)--(7.707,3.300)--(7.793,3.276)--(7.878,3.252)--(7.963,3.228)--(8.049,3.204)%
  --(8.134,3.180)--(8.220,3.156)--(8.305,3.132)--(8.390,3.108)--(8.476,3.083)--(8.561,3.059)%
  --(8.647,3.035)--(8.732,3.011)--(8.817,2.987)--(8.903,2.963)--(8.988,2.939)--(9.074,2.915)%
  --(9.159,2.891)--(9.244,2.866)--(9.330,2.842)--(9.415,2.818)--(9.501,2.794)--(9.586,2.770)%
  --(9.671,2.746)--(9.757,2.722)--(9.842,2.698)--(9.928,2.674)--(10.013,2.649)--(10.098,2.625)%
  --(10.184,2.601)--(10.269,2.577)--(10.355,2.553)--(10.440,2.529)--(10.525,2.505);
\gpcolor{color=gp lt color border}
\gpsetlinetype{gp lt border}
\draw[gp path] (1.688,7.825)--(1.688,0.985)--(11.947,0.985)--(11.947,7.825)--cycle;
%% coordinates of the plot area
\gpdefrectangularnode{gp plot 1}{\pgfpoint{1.688cm}{0.985cm}}{\pgfpoint{11.947cm}{7.825cm}}
\end{tikzpicture}
%% gnuplot variables

\caption{Terza serie, decelerazione}
\label{fig:1}
\end{grafico}

\begin{grafico}
    \centering
\begin{tikzpicture}[gnuplot]
%% generated with GNUPLOT 4.6p0 (Lua 5.1; terminal rev. 99, script rev. 100)
%% Tue 15 Apr 2014 06:32:34 PM CEST
\path (0.000,0.000) rectangle (12.500,8.750);
\gpcolor{color=gp lt color border}
\gpsetlinetype{gp lt border}
\gpsetlinewidth{1.00}
\draw[gp path] (1.688,0.985)--(1.868,0.985);
\draw[gp path] (11.947,0.985)--(11.767,0.985);
\node[gp node right] at (1.504,0.985) { 1.15};
\draw[gp path] (1.688,2.353)--(1.868,2.353);
\draw[gp path] (11.947,2.353)--(11.767,2.353);
\node[gp node right] at (1.504,2.353) { 1.2};
\draw[gp path] (1.688,3.721)--(1.868,3.721);
\draw[gp path] (11.947,3.721)--(11.767,3.721);
\node[gp node right] at (1.504,3.721) { 1.25};
\draw[gp path] (1.688,5.089)--(1.868,5.089);
\draw[gp path] (11.947,5.089)--(11.767,5.089);
\node[gp node right] at (1.504,5.089) { 1.3};
\draw[gp path] (1.688,6.457)--(1.868,6.457);
\draw[gp path] (11.947,6.457)--(11.767,6.457);
\node[gp node right] at (1.504,6.457) { 1.35};
\draw[gp path] (1.688,7.825)--(1.868,7.825);
\draw[gp path] (11.947,7.825)--(11.767,7.825);
\node[gp node right] at (1.504,7.825) { 1.4};
\draw[gp path] (1.688,0.985)--(1.688,1.165);
\draw[gp path] (1.688,7.825)--(1.688,7.645);
\node[gp node center] at (1.688,0.677) { 0};
\draw[gp path] (2.970,0.985)--(2.970,1.165);
\draw[gp path] (2.970,7.825)--(2.970,7.645);
\node[gp node center] at (2.970,0.677) { 2};
\draw[gp path] (4.253,0.985)--(4.253,1.165);
\draw[gp path] (4.253,7.825)--(4.253,7.645);
\node[gp node center] at (4.253,0.677) { 4};
\draw[gp path] (5.535,0.985)--(5.535,1.165);
\draw[gp path] (5.535,7.825)--(5.535,7.645);
\node[gp node center] at (5.535,0.677) { 6};
\draw[gp path] (6.818,0.985)--(6.818,1.165);
\draw[gp path] (6.818,7.825)--(6.818,7.645);
\node[gp node center] at (6.818,0.677) { 8};
\draw[gp path] (8.100,0.985)--(8.100,1.165);
\draw[gp path] (8.100,7.825)--(8.100,7.645);
\node[gp node center] at (8.100,0.677) { 10};
\draw[gp path] (9.382,0.985)--(9.382,1.165);
\draw[gp path] (9.382,7.825)--(9.382,7.645);
\node[gp node center] at (9.382,0.677) { 12};
\draw[gp path] (10.665,0.985)--(10.665,1.165);
\draw[gp path] (10.665,7.825)--(10.665,7.645);
\node[gp node center] at (10.665,0.677) { 14};
\draw[gp path] (11.947,0.985)--(11.947,1.165);
\draw[gp path] (11.947,7.825)--(11.947,7.645);
\node[gp node center] at (11.947,0.677) { 16};
\draw[gp path] (1.688,7.825)--(1.688,0.985)--(11.947,0.985)--(11.947,7.825)--cycle;
\node[gp node center,rotate=-270] at (0.246,4.405) {Velocità angolare [rad/s]};
\node[gp node center] at (6.817,0.215) {Tempo [s]};
\node[gp node center] at (6.817,8.287) {Velocità angolare, accelerazione [rad/s]};
\node[gp node left] at (7.167,1.627) {Dati};
\gpcolor{color=gp lt color 0}
\gpsetlinetype{gp lt plot 0}
\draw[gp path] (10.663,1.627)--(11.579,1.627);
\draw[gp path] (10.663,1.717)--(10.663,1.537);
\draw[gp path] (11.579,1.717)--(11.579,1.537);
\draw[gp path] (2.489,1.147)--(2.489,6.645);
\draw[gp path] (2.399,1.147)--(2.579,1.147);
\draw[gp path] (2.399,6.645)--(2.579,6.645);
\draw[gp path] (3.323,1.901)--(3.323,4.543);
\draw[gp path] (3.233,1.901)--(3.413,1.901);
\draw[gp path] (3.233,4.543)--(3.413,4.543);
\draw[gp path] (4.157,2.134)--(4.157,3.873);
\draw[gp path] (4.067,2.134)--(4.247,2.134);
\draw[gp path] (4.067,3.873)--(4.247,3.873);
\draw[gp path] (4.990,2.247)--(4.990,3.543);
\draw[gp path] (4.900,2.247)--(5.080,2.247);
\draw[gp path] (4.900,3.543)--(5.080,3.543);
\draw[gp path] (5.792,2.566)--(5.792,3.615);
\draw[gp path] (5.702,2.566)--(5.882,2.566);
\draw[gp path] (5.702,3.615)--(5.882,3.615);
\draw[gp path] (6.625,2.569)--(6.625,3.438);
\draw[gp path] (6.535,2.569)--(6.715,2.569);
\draw[gp path] (6.535,3.438)--(6.715,3.438);
\draw[gp path] (7.491,2.389)--(7.491,3.124);
\draw[gp path] (7.401,2.389)--(7.581,2.389);
\draw[gp path] (7.401,3.124)--(7.581,3.124);
\draw[gp path] (8.356,2.256)--(8.356,2.892);
\draw[gp path] (8.266,2.256)--(8.446,2.256);
\draw[gp path] (8.266,2.892)--(8.446,2.892);
\draw[gp path] (9.158,2.431)--(9.158,3.001);
\draw[gp path] (9.068,2.431)--(9.248,2.431);
\draw[gp path] (9.068,3.001)--(9.248,3.001);
\draw[gp path] (10.023,2.320)--(10.023,2.828);
\draw[gp path] (9.933,2.320)--(10.113,2.320);
\draw[gp path] (9.933,2.828)--(10.113,2.828);
\draw[gp path] (10.889,2.229)--(10.889,2.688);
\draw[gp path] (10.799,2.229)--(10.979,2.229);
\draw[gp path] (10.799,2.688)--(10.979,2.688);
\draw[gp path] (11.755,2.154)--(11.755,2.573);
\draw[gp path] (11.665,2.154)--(11.845,2.154);
\draw[gp path] (11.665,2.573)--(11.845,2.573);
\gpsetpointsize{4.00}
\gppoint{gp mark 1}{(2.489,3.896)}
\gppoint{gp mark 1}{(3.323,3.222)}
\gppoint{gp mark 1}{(4.157,3.003)}
\gppoint{gp mark 1}{(4.990,2.895)}
\gppoint{gp mark 1}{(5.792,3.090)}
\gppoint{gp mark 1}{(6.625,3.003)}
\gppoint{gp mark 1}{(7.491,2.757)}
\gppoint{gp mark 1}{(8.356,2.574)}
\gppoint{gp mark 1}{(9.158,2.716)}
\gppoint{gp mark 1}{(10.023,2.574)}
\gppoint{gp mark 1}{(10.889,2.459)}
\gppoint{gp mark 1}{(11.755,2.363)}
\gppoint{gp mark 1}{(11.121,1.627)}
\gpcolor{color=gp lt color border}
\node[gp node left] at (7.167,1.319) {Retta interpolante};
\gpcolor{color=gp lt color 1}
\gpsetlinetype{gp lt plot 1}
\draw[gp path] (10.663,1.319)--(11.579,1.319);
\draw[gp path] (2.489,3.442)--(2.583,3.430)--(2.677,3.419)--(2.770,3.408)--(2.864,3.396)%
  --(2.957,3.385)--(3.051,3.373)--(3.145,3.362)--(3.238,3.350)--(3.332,3.339)--(3.425,3.327)%
  --(3.519,3.316)--(3.613,3.305)--(3.706,3.293)--(3.800,3.282)--(3.893,3.270)--(3.987,3.259)%
  --(4.080,3.247)--(4.174,3.236)--(4.268,3.224)--(4.361,3.213)--(4.455,3.201)--(4.548,3.190)%
  --(4.642,3.179)--(4.736,3.167)--(4.829,3.156)--(4.923,3.144)--(5.016,3.133)--(5.110,3.121)%
  --(5.204,3.110)--(5.297,3.098)--(5.391,3.087)--(5.484,3.076)--(5.578,3.064)--(5.671,3.053)%
  --(5.765,3.041)--(5.859,3.030)--(5.952,3.018)--(6.046,3.007)--(6.139,2.995)--(6.233,2.984)%
  --(6.327,2.972)--(6.420,2.961)--(6.514,2.950)--(6.607,2.938)--(6.701,2.927)--(6.795,2.915)%
  --(6.888,2.904)--(6.982,2.892)--(7.075,2.881)--(7.169,2.869)--(7.262,2.858)--(7.356,2.847)%
  --(7.450,2.835)--(7.543,2.824)--(7.637,2.812)--(7.730,2.801)--(7.824,2.789)--(7.918,2.778)%
  --(8.011,2.766)--(8.105,2.755)--(8.198,2.743)--(8.292,2.732)--(8.385,2.721)--(8.479,2.709)%
  --(8.573,2.698)--(8.666,2.686)--(8.760,2.675)--(8.853,2.663)--(8.947,2.652)--(9.041,2.640)%
  --(9.134,2.629)--(9.228,2.618)--(9.321,2.606)--(9.415,2.595)--(9.509,2.583)--(9.602,2.572)%
  --(9.696,2.560)--(9.789,2.549)--(9.883,2.537)--(9.976,2.526)--(10.070,2.514)--(10.164,2.503)%
  --(10.257,2.492)--(10.351,2.480)--(10.444,2.469)--(10.538,2.457)--(10.632,2.446)--(10.725,2.434)%
  --(10.819,2.423)--(10.912,2.411)--(11.006,2.400)--(11.100,2.389)--(11.193,2.377)--(11.287,2.366)%
  --(11.380,2.354)--(11.474,2.343)--(11.567,2.331)--(11.661,2.320)--(11.755,2.308);
\gpcolor{color=gp lt color border}
\gpsetlinetype{gp lt border}
\draw[gp path] (1.688,7.825)--(1.688,0.985)--(11.947,0.985)--(11.947,7.825)--cycle;
%% coordinates of the plot area
\gpdefrectangularnode{gp plot 1}{\pgfpoint{1.688cm}{0.985cm}}{\pgfpoint{11.947cm}{7.825cm}}
\end{tikzpicture}
%% gnuplot variables

\caption{Quarta serie, decelerazione}
\label{fig:1}
\end{grafico}

\begin{grafico}
    \centering
\begin{tikzpicture}[gnuplot]
%% generated with GNUPLOT 4.6p0 (Lua 5.1; terminal rev. 99, script rev. 100)
%% Tue 15 Apr 2014 06:32:34 PM CEST
\path (0.000,0.000) rectangle (12.500,8.750);
\gpcolor{color=gp lt color border}
\gpsetlinetype{gp lt border}
\gpsetlinewidth{1.00}
\draw[gp path] (1.688,0.985)--(1.868,0.985);
\draw[gp path] (11.947,0.985)--(11.767,0.985);
\node[gp node right] at (1.504,0.985) { 1.15};
\draw[gp path] (1.688,2.353)--(1.868,2.353);
\draw[gp path] (11.947,2.353)--(11.767,2.353);
\node[gp node right] at (1.504,2.353) { 1.2};
\draw[gp path] (1.688,3.721)--(1.868,3.721);
\draw[gp path] (11.947,3.721)--(11.767,3.721);
\node[gp node right] at (1.504,3.721) { 1.25};
\draw[gp path] (1.688,5.089)--(1.868,5.089);
\draw[gp path] (11.947,5.089)--(11.767,5.089);
\node[gp node right] at (1.504,5.089) { 1.3};
\draw[gp path] (1.688,6.457)--(1.868,6.457);
\draw[gp path] (11.947,6.457)--(11.767,6.457);
\node[gp node right] at (1.504,6.457) { 1.35};
\draw[gp path] (1.688,7.825)--(1.868,7.825);
\draw[gp path] (11.947,7.825)--(11.767,7.825);
\node[gp node right] at (1.504,7.825) { 1.4};
\draw[gp path] (1.688,0.985)--(1.688,1.165);
\draw[gp path] (1.688,7.825)--(1.688,7.645);
\node[gp node center] at (1.688,0.677) { 0};
\draw[gp path] (2.970,0.985)--(2.970,1.165);
\draw[gp path] (2.970,7.825)--(2.970,7.645);
\node[gp node center] at (2.970,0.677) { 2};
\draw[gp path] (4.253,0.985)--(4.253,1.165);
\draw[gp path] (4.253,7.825)--(4.253,7.645);
\node[gp node center] at (4.253,0.677) { 4};
\draw[gp path] (5.535,0.985)--(5.535,1.165);
\draw[gp path] (5.535,7.825)--(5.535,7.645);
\node[gp node center] at (5.535,0.677) { 6};
\draw[gp path] (6.818,0.985)--(6.818,1.165);
\draw[gp path] (6.818,7.825)--(6.818,7.645);
\node[gp node center] at (6.818,0.677) { 8};
\draw[gp path] (8.100,0.985)--(8.100,1.165);
\draw[gp path] (8.100,7.825)--(8.100,7.645);
\node[gp node center] at (8.100,0.677) { 10};
\draw[gp path] (9.382,0.985)--(9.382,1.165);
\draw[gp path] (9.382,7.825)--(9.382,7.645);
\node[gp node center] at (9.382,0.677) { 12};
\draw[gp path] (10.665,0.985)--(10.665,1.165);
\draw[gp path] (10.665,7.825)--(10.665,7.645);
\node[gp node center] at (10.665,0.677) { 14};
\draw[gp path] (11.947,0.985)--(11.947,1.165);
\draw[gp path] (11.947,7.825)--(11.947,7.645);
\node[gp node center] at (11.947,0.677) { 16};
\draw[gp path] (1.688,7.825)--(1.688,0.985)--(11.947,0.985)--(11.947,7.825)--cycle;
\node[gp node center,rotate=-270] at (0.246,4.405) {Velocità angolare [rad/s]};
\node[gp node center] at (6.817,0.215) {Tempo [s]};
\node[gp node center] at (6.817,8.287) {Velocità angolare, accelerazione [rad/s]};
\node[gp node left] at (7.167,1.627) {Dati};
\gpcolor{color=gp lt color 0}
\gpsetlinetype{gp lt plot 0}
\draw[gp path] (10.663,1.627)--(11.579,1.627);
\draw[gp path] (10.663,1.717)--(10.663,1.537);
\draw[gp path] (11.579,1.717)--(11.579,1.537);
\draw[gp path] (2.489,1.147)--(2.489,6.645);
\draw[gp path] (2.399,1.147)--(2.579,1.147);
\draw[gp path] (2.399,6.645)--(2.579,6.645);
\draw[gp path] (3.323,1.901)--(3.323,4.543);
\draw[gp path] (3.233,1.901)--(3.413,1.901);
\draw[gp path] (3.233,4.543)--(3.413,4.543);
\draw[gp path] (4.157,2.134)--(4.157,3.873);
\draw[gp path] (4.067,2.134)--(4.247,2.134);
\draw[gp path] (4.067,3.873)--(4.247,3.873);
\draw[gp path] (4.958,2.561)--(4.958,3.883);
\draw[gp path] (4.868,2.561)--(5.048,2.561);
\draw[gp path] (4.868,3.883)--(5.048,3.883);
\draw[gp path] (5.824,2.314)--(5.824,3.346);
\draw[gp path] (5.734,2.314)--(5.914,2.314);
\draw[gp path] (5.734,3.346)--(5.914,3.346);
\draw[gp path] (6.625,2.569)--(6.625,3.438);
\draw[gp path] (6.535,2.569)--(6.715,2.569);
\draw[gp path] (6.535,3.438)--(6.715,3.438);
\draw[gp path] (7.491,2.389)--(7.491,3.124);
\draw[gp path] (7.401,2.389)--(7.581,2.389);
\draw[gp path] (7.401,3.124)--(7.581,3.124);
\draw[gp path] (8.324,2.413)--(8.324,3.054);
\draw[gp path] (8.234,2.413)--(8.414,2.413);
\draw[gp path] (8.234,3.054)--(8.414,3.054);
\draw[gp path] (9.158,2.431)--(9.158,3.001);
\draw[gp path] (9.068,2.431)--(9.248,2.431);
\draw[gp path] (9.068,3.001)--(9.248,3.001);
\draw[gp path] (9.991,2.445)--(9.991,2.958);
\draw[gp path] (9.901,2.445)--(10.081,2.445);
\draw[gp path] (9.901,2.958)--(10.081,2.958);
\draw[gp path] (10.857,2.343)--(10.857,2.805);
\draw[gp path] (10.767,2.343)--(10.947,2.343);
\draw[gp path] (10.767,2.805)--(10.947,2.805);
\draw[gp path] (11.755,2.154)--(11.755,2.573);
\draw[gp path] (11.665,2.154)--(11.845,2.154);
\draw[gp path] (11.665,2.573)--(11.845,2.573);
\gpsetpointsize{4.00}
\gppoint{gp mark 1}{(2.489,3.896)}
\gppoint{gp mark 1}{(3.323,3.222)}
\gppoint{gp mark 1}{(4.157,3.003)}
\gppoint{gp mark 1}{(4.958,3.222)}
\gppoint{gp mark 1}{(5.824,2.830)}
\gppoint{gp mark 1}{(6.625,3.003)}
\gppoint{gp mark 1}{(7.491,2.757)}
\gppoint{gp mark 1}{(8.324,2.734)}
\gppoint{gp mark 1}{(9.158,2.716)}
\gppoint{gp mark 1}{(9.991,2.702)}
\gppoint{gp mark 1}{(10.857,2.574)}
\gppoint{gp mark 1}{(11.755,2.363)}
\gppoint{gp mark 1}{(11.121,1.627)}
\gpcolor{color=gp lt color border}
\node[gp node left] at (7.167,1.319) {Retta interpolante};
\gpcolor{color=gp lt color 1}
\gpsetlinetype{gp lt plot 1}
\draw[gp path] (10.663,1.319)--(11.579,1.319);
\draw[gp path] (2.489,3.452)--(2.583,3.441)--(2.677,3.430)--(2.770,3.419)--(2.864,3.408)%
  --(2.957,3.397)--(3.051,3.386)--(3.145,3.376)--(3.238,3.365)--(3.332,3.354)--(3.425,3.343)%
  --(3.519,3.332)--(3.613,3.321)--(3.706,3.310)--(3.800,3.300)--(3.893,3.289)--(3.987,3.278)%
  --(4.080,3.267)--(4.174,3.256)--(4.268,3.245)--(4.361,3.234)--(4.455,3.223)--(4.548,3.213)%
  --(4.642,3.202)--(4.736,3.191)--(4.829,3.180)--(4.923,3.169)--(5.016,3.158)--(5.110,3.147)%
  --(5.204,3.136)--(5.297,3.126)--(5.391,3.115)--(5.484,3.104)--(5.578,3.093)--(5.671,3.082)%
  --(5.765,3.071)--(5.859,3.060)--(5.952,3.049)--(6.046,3.039)--(6.139,3.028)--(6.233,3.017)%
  --(6.327,3.006)--(6.420,2.995)--(6.514,2.984)--(6.607,2.973)--(6.701,2.962)--(6.795,2.952)%
  --(6.888,2.941)--(6.982,2.930)--(7.075,2.919)--(7.169,2.908)--(7.262,2.897)--(7.356,2.886)%
  --(7.450,2.875)--(7.543,2.865)--(7.637,2.854)--(7.730,2.843)--(7.824,2.832)--(7.918,2.821)%
  --(8.011,2.810)--(8.105,2.799)--(8.198,2.789)--(8.292,2.778)--(8.385,2.767)--(8.479,2.756)%
  --(8.573,2.745)--(8.666,2.734)--(8.760,2.723)--(8.853,2.712)--(8.947,2.702)--(9.041,2.691)%
  --(9.134,2.680)--(9.228,2.669)--(9.321,2.658)--(9.415,2.647)--(9.509,2.636)--(9.602,2.625)%
  --(9.696,2.615)--(9.789,2.604)--(9.883,2.593)--(9.976,2.582)--(10.070,2.571)--(10.164,2.560)%
  --(10.257,2.549)--(10.351,2.538)--(10.444,2.528)--(10.538,2.517)--(10.632,2.506)--(10.725,2.495)%
  --(10.819,2.484)--(10.912,2.473)--(11.006,2.462)--(11.100,2.451)--(11.193,2.441)--(11.287,2.430)%
  --(11.380,2.419)--(11.474,2.408)--(11.567,2.397)--(11.661,2.386)--(11.755,2.375);
\gpcolor{color=gp lt color border}
\gpsetlinetype{gp lt border}
\draw[gp path] (1.688,7.825)--(1.688,0.985)--(11.947,0.985)--(11.947,7.825)--cycle;
%% coordinates of the plot area
\gpdefrectangularnode{gp plot 1}{\pgfpoint{1.688cm}{0.985cm}}{\pgfpoint{11.947cm}{7.825cm}}
\end{tikzpicture}
%% gnuplot variables

\caption{Quinta serie, decelerazione}
\label{fig:1}
\end{grafico}

\begin{grafico}
    \centering
\begin{tikzpicture}[gnuplot]
%% generated with GNUPLOT 4.6p0 (Lua 5.1; terminal rev. 99, script rev. 100)
%% Tue 15 Apr 2014 09:47:09 PM CEST
\path (0.000,0.000) rectangle (12.500,8.750);
\gpcolor{color=gp lt color border}
\gpsetlinetype{gp lt border}
\gpsetlinewidth{1.00}
\draw[gp path] (1.688,0.985)--(1.868,0.985);
\draw[gp path] (11.947,0.985)--(11.767,0.985);
\node[gp node right] at (1.504,0.985) { 1.1};
\draw[gp path] (1.688,1.607)--(1.868,1.607);
\draw[gp path] (11.947,1.607)--(11.767,1.607);
\node[gp node right] at (1.504,1.607) { 1.12};
\draw[gp path] (1.688,2.229)--(1.868,2.229);
\draw[gp path] (11.947,2.229)--(11.767,2.229);
\node[gp node right] at (1.504,2.229) { 1.14};
\draw[gp path] (1.688,2.850)--(1.868,2.850);
\draw[gp path] (11.947,2.850)--(11.767,2.850);
\node[gp node right] at (1.504,2.850) { 1.16};
\draw[gp path] (1.688,3.472)--(1.868,3.472);
\draw[gp path] (11.947,3.472)--(11.767,3.472);
\node[gp node right] at (1.504,3.472) { 1.18};
\draw[gp path] (1.688,4.094)--(1.868,4.094);
\draw[gp path] (11.947,4.094)--(11.767,4.094);
\node[gp node right] at (1.504,4.094) { 1.2};
\draw[gp path] (1.688,4.716)--(1.868,4.716);
\draw[gp path] (11.947,4.716)--(11.767,4.716);
\node[gp node right] at (1.504,4.716) { 1.22};
\draw[gp path] (1.688,5.338)--(1.868,5.338);
\draw[gp path] (11.947,5.338)--(11.767,5.338);
\node[gp node right] at (1.504,5.338) { 1.24};
\draw[gp path] (1.688,5.960)--(1.868,5.960);
\draw[gp path] (11.947,5.960)--(11.767,5.960);
\node[gp node right] at (1.504,5.960) { 1.26};
\draw[gp path] (1.688,6.581)--(1.868,6.581);
\draw[gp path] (11.947,6.581)--(11.767,6.581);
\node[gp node right] at (1.504,6.581) { 1.28};
\draw[gp path] (1.688,7.203)--(1.868,7.203);
\draw[gp path] (11.947,7.203)--(11.767,7.203);
\node[gp node right] at (1.504,7.203) { 1.3};
\draw[gp path] (1.688,7.825)--(1.868,7.825);
\draw[gp path] (11.947,7.825)--(11.767,7.825);
\node[gp node right] at (1.504,7.825) { 1.32};
\draw[gp path] (1.688,0.985)--(1.688,1.165);
\draw[gp path] (1.688,7.825)--(1.688,7.645);
\node[gp node center] at (1.688,0.677) { 0};
\draw[gp path] (3.398,0.985)--(3.398,1.165);
\draw[gp path] (3.398,7.825)--(3.398,7.645);
\node[gp node center] at (3.398,0.677) { 5};
\draw[gp path] (5.108,0.985)--(5.108,1.165);
\draw[gp path] (5.108,7.825)--(5.108,7.645);
\node[gp node center] at (5.108,0.677) { 10};
\draw[gp path] (6.818,0.985)--(6.818,1.165);
\draw[gp path] (6.818,7.825)--(6.818,7.645);
\node[gp node center] at (6.818,0.677) { 15};
\draw[gp path] (8.527,0.985)--(8.527,1.165);
\draw[gp path] (8.527,7.825)--(8.527,7.645);
\node[gp node center] at (8.527,0.677) { 20};
\draw[gp path] (10.237,0.985)--(10.237,1.165);
\draw[gp path] (10.237,7.825)--(10.237,7.645);
\node[gp node center] at (10.237,0.677) { 25};
\draw[gp path] (11.947,0.985)--(11.947,1.165);
\draw[gp path] (11.947,7.825)--(11.947,7.645);
\node[gp node center] at (11.947,0.677) { 30};
\draw[gp path] (1.688,7.825)--(1.688,0.985)--(11.947,0.985)--(11.947,7.825)--cycle;
\node[gp node center,rotate=-270] at (0.246,4.405) {Velocità angolare [rad/s]};
\node[gp node center] at (6.817,0.215) {Tempo [s]};
\node[gp node center] at (6.817,8.287) {Velocità angolari in decelerazione [rad/s]};
\node[gp node left] at (7.167,1.627) {Dati};
\gpcolor{color=gp lt color 0}
\gpsetlinetype{gp lt plot 0}
\draw[gp path] (10.663,1.627)--(11.579,1.627);
\draw[gp path] (10.663,1.717)--(10.663,1.537);
\draw[gp path] (11.579,1.717)--(11.579,1.537);
\draw[gp path] (2.133,1.457)--(2.133,7.234);
\draw[gp path] (2.043,1.457)--(2.223,1.457);
\draw[gp path] (2.043,7.234)--(2.223,7.234);
\draw[gp path] (2.560,3.580)--(2.560,6.583);
\draw[gp path] (2.470,3.580)--(2.650,3.580);
\draw[gp path] (2.470,6.583)--(2.650,6.583);
\draw[gp path] (2.987,4.320)--(2.987,6.348);
\draw[gp path] (2.897,4.320)--(3.077,4.320);
\draw[gp path] (2.897,6.348)--(3.077,6.348);
\draw[gp path] (3.432,4.331)--(3.432,5.832);
\draw[gp path] (3.342,4.331)--(3.522,4.331);
\draw[gp path] (3.342,5.832)--(3.522,5.832);
\draw[gp path] (3.894,4.050)--(3.894,5.223);
\draw[gp path] (3.804,4.050)--(3.984,4.050);
\draw[gp path] (3.804,5.223)--(3.984,5.223);
\draw[gp path] (4.338,4.100)--(4.338,5.075);
\draw[gp path] (4.248,4.100)--(4.428,4.100);
\draw[gp path] (4.248,5.075)--(4.428,5.075);
\draw[gp path] (4.783,4.136)--(4.783,4.970);
\draw[gp path] (4.693,4.136)--(4.873,4.136);
\draw[gp path] (4.693,4.970)--(4.873,4.970);
\draw[gp path] (5.244,3.984)--(5.244,4.706);
\draw[gp path] (5.154,3.984)--(5.334,3.984);
\draw[gp path] (5.154,4.706)--(5.334,4.706);
\draw[gp path] (5.689,4.024)--(5.689,4.666);
\draw[gp path] (5.599,4.024)--(5.779,4.024);
\draw[gp path] (5.599,4.666)--(5.779,4.666);
\draw[gp path] (6.151,3.915)--(6.151,4.488);
\draw[gp path] (6.061,3.915)--(6.241,3.915);
\draw[gp path] (6.061,4.488)--(6.241,4.488);
\draw[gp path] (6.595,3.954)--(6.595,4.475);
\draw[gp path] (6.505,3.954)--(6.685,3.954);
\draw[gp path] (6.505,4.475)--(6.685,4.475);
\draw[gp path] (7.057,3.868)--(7.057,4.344);
\draw[gp path] (6.967,3.868)--(7.147,3.868);
\draw[gp path] (6.967,4.344)--(7.147,4.344);
\draw[gp path] (7.536,3.689)--(7.536,4.123);
\draw[gp path] (7.446,3.689)--(7.626,3.689);
\draw[gp path] (7.446,4.123)--(7.626,4.123);
\draw[gp path] (7.980,3.735)--(7.980,4.139);
\draw[gp path] (7.890,3.735)--(8.070,3.735);
\draw[gp path] (7.890,4.139)--(8.070,4.139);
\draw[gp path] (8.464,3.562)--(8.464,3.935);
\draw[gp path] (8.374,3.562)--(8.554,3.562);
\draw[gp path] (8.374,3.935)--(8.554,3.935);
\draw[gp path] (8.921,3.549)--(8.921,3.898);
\draw[gp path] (8.831,3.549)--(9.011,3.549);
\draw[gp path] (8.831,3.898)--(9.011,3.898);
\draw[gp path] (9.399,3.432)--(9.399,3.759);
\draw[gp path] (9.309,3.432)--(9.489,3.432);
\draw[gp path] (9.309,3.759)--(9.489,3.759);
\draw[gp path] (9.895,3.254)--(9.895,3.559);
\draw[gp path] (9.805,3.254)--(9.985,3.254);
\draw[gp path] (9.805,3.559)--(9.985,3.559);
\draw[gp path] (10.357,3.238)--(10.357,3.526);
\draw[gp path] (10.267,3.238)--(10.447,3.238);
\draw[gp path] (10.267,3.526)--(10.447,3.526);
\draw[gp path] (10.853,3.088)--(10.853,3.360);
\draw[gp path] (10.763,3.088)--(10.943,3.088);
\draw[gp path] (10.763,3.360)--(10.943,3.360);
\draw[gp path] (11.349,2.954)--(11.349,3.211);
\draw[gp path] (11.259,2.954)--(11.439,2.954);
\draw[gp path] (11.259,3.211)--(11.439,3.211);
\draw[gp path] (11.810,2.954)--(11.810,3.199);
\draw[gp path] (11.720,2.954)--(11.900,2.954);
\draw[gp path] (11.720,3.199)--(11.900,3.199);
\gpsetpointsize{4.00}
\gppoint{gp mark 1}{(2.133,4.345)}
\gppoint{gp mark 1}{(2.560,5.082)}
\gppoint{gp mark 1}{(2.987,5.334)}
\gppoint{gp mark 1}{(3.432,5.082)}
\gppoint{gp mark 1}{(3.894,4.636)}
\gppoint{gp mark 1}{(4.338,4.588)}
\gppoint{gp mark 1}{(4.783,4.553)}
\gppoint{gp mark 1}{(5.244,4.345)}
\gppoint{gp mark 1}{(5.689,4.345)}
\gppoint{gp mark 1}{(6.151,4.201)}
\gppoint{gp mark 1}{(6.595,4.214)}
\gppoint{gp mark 1}{(7.057,4.106)}
\gppoint{gp mark 1}{(7.536,3.906)}
\gppoint{gp mark 1}{(7.980,3.937)}
\gppoint{gp mark 1}{(8.464,3.748)}
\gppoint{gp mark 1}{(8.921,3.724)}
\gppoint{gp mark 1}{(9.399,3.596)}
\gppoint{gp mark 1}{(9.895,3.406)}
\gppoint{gp mark 1}{(10.357,3.382)}
\gppoint{gp mark 1}{(10.853,3.224)}
\gppoint{gp mark 1}{(11.349,3.082)}
\gppoint{gp mark 1}{(11.810,3.076)}
\gppoint{gp mark 1}{(11.121,1.627)}
\gpcolor{color=gp lt color border}
\node[gp node left] at (7.167,1.319) {Retta interpolante};
\gpcolor{color=gp lt color 1}
\gpsetlinetype{gp lt plot 1}
\draw[gp path] (10.663,1.319)--(11.579,1.319);
\draw[gp path] (2.133,5.135)--(2.230,5.113)--(2.328,5.092)--(2.426,5.071)--(2.524,5.050)%
  --(2.621,5.029)--(2.719,5.008)--(2.817,4.986)--(2.915,4.965)--(3.012,4.944)--(3.110,4.923)%
  --(3.208,4.902)--(3.306,4.881)--(3.403,4.859)--(3.501,4.838)--(3.599,4.817)--(3.697,4.796)%
  --(3.794,4.775)--(3.892,4.753)--(3.990,4.732)--(4.088,4.711)--(4.185,4.690)--(4.283,4.669)%
  --(4.381,4.648)--(4.479,4.626)--(4.576,4.605)--(4.674,4.584)--(4.772,4.563)--(4.870,4.542)%
  --(4.967,4.521)--(5.065,4.499)--(5.163,4.478)--(5.261,4.457)--(5.358,4.436)--(5.456,4.415)%
  --(5.554,4.394)--(5.652,4.372)--(5.749,4.351)--(5.847,4.330)--(5.945,4.309)--(6.043,4.288)%
  --(6.140,4.267)--(6.238,4.245)--(6.336,4.224)--(6.434,4.203)--(6.531,4.182)--(6.629,4.161)%
  --(6.727,4.140)--(6.825,4.118)--(6.923,4.097)--(7.020,4.076)--(7.118,4.055)--(7.216,4.034)%
  --(7.314,4.013)--(7.411,3.991)--(7.509,3.970)--(7.607,3.949)--(7.705,3.928)--(7.802,3.907)%
  --(7.900,3.886)--(7.998,3.864)--(8.096,3.843)--(8.193,3.822)--(8.291,3.801)--(8.389,3.780)%
  --(8.487,3.759)--(8.584,3.737)--(8.682,3.716)--(8.780,3.695)--(8.878,3.674)--(8.975,3.653)%
  --(9.073,3.632)--(9.171,3.610)--(9.269,3.589)--(9.366,3.568)--(9.464,3.547)--(9.562,3.526)%
  --(9.660,3.504)--(9.757,3.483)--(9.855,3.462)--(9.953,3.441)--(10.051,3.420)--(10.148,3.399)%
  --(10.246,3.377)--(10.344,3.356)--(10.442,3.335)--(10.539,3.314)--(10.637,3.293)--(10.735,3.272)%
  --(10.833,3.250)--(10.930,3.229)--(11.028,3.208)--(11.126,3.187)--(11.224,3.166)--(11.321,3.145)%
  --(11.419,3.123)--(11.517,3.102)--(11.615,3.081)--(11.712,3.060)--(11.810,3.039);
\gpcolor{color=gp lt color border}
\gpsetlinetype{gp lt border}
\draw[gp path] (1.688,7.825)--(1.688,0.985)--(11.947,0.985)--(11.947,7.825)--cycle;
%% coordinates of the plot area
\gpdefrectangularnode{gp plot 1}{\pgfpoint{1.688cm}{0.985cm}}{\pgfpoint{11.947cm}{7.825cm}}
\end{tikzpicture}
%% gnuplot variables

\caption{Sesta serie, decelerazione}
\label{fig:1}
\end{grafico}

\begin{grafico}
    \centering
\begin{tikzpicture}[gnuplot]
%% generated with GNUPLOT 4.6p0 (Lua 5.1; terminal rev. 99, script rev. 100)
%% Tue 15 Apr 2014 06:32:34 PM CEST
\path (0.000,0.000) rectangle (12.500,8.750);
\gpcolor{color=gp lt color border}
\gpsetlinetype{gp lt border}
\gpsetlinewidth{1.00}
\draw[gp path] (1.688,0.985)--(1.868,0.985);
\draw[gp path] (11.947,0.985)--(11.767,0.985);
\node[gp node right] at (1.504,0.985) { 1.15};
\draw[gp path] (1.688,2.353)--(1.868,2.353);
\draw[gp path] (11.947,2.353)--(11.767,2.353);
\node[gp node right] at (1.504,2.353) { 1.2};
\draw[gp path] (1.688,3.721)--(1.868,3.721);
\draw[gp path] (11.947,3.721)--(11.767,3.721);
\node[gp node right] at (1.504,3.721) { 1.25};
\draw[gp path] (1.688,5.089)--(1.868,5.089);
\draw[gp path] (11.947,5.089)--(11.767,5.089);
\node[gp node right] at (1.504,5.089) { 1.3};
\draw[gp path] (1.688,6.457)--(1.868,6.457);
\draw[gp path] (11.947,6.457)--(11.767,6.457);
\node[gp node right] at (1.504,6.457) { 1.35};
\draw[gp path] (1.688,7.825)--(1.868,7.825);
\draw[gp path] (11.947,7.825)--(11.767,7.825);
\node[gp node right] at (1.504,7.825) { 1.4};
\draw[gp path] (1.688,0.985)--(1.688,1.165);
\draw[gp path] (1.688,7.825)--(1.688,7.645);
\node[gp node center] at (1.688,0.677) { 0};
\draw[gp path] (2.970,0.985)--(2.970,1.165);
\draw[gp path] (2.970,7.825)--(2.970,7.645);
\node[gp node center] at (2.970,0.677) { 2};
\draw[gp path] (4.253,0.985)--(4.253,1.165);
\draw[gp path] (4.253,7.825)--(4.253,7.645);
\node[gp node center] at (4.253,0.677) { 4};
\draw[gp path] (5.535,0.985)--(5.535,1.165);
\draw[gp path] (5.535,7.825)--(5.535,7.645);
\node[gp node center] at (5.535,0.677) { 6};
\draw[gp path] (6.818,0.985)--(6.818,1.165);
\draw[gp path] (6.818,7.825)--(6.818,7.645);
\node[gp node center] at (6.818,0.677) { 8};
\draw[gp path] (8.100,0.985)--(8.100,1.165);
\draw[gp path] (8.100,7.825)--(8.100,7.645);
\node[gp node center] at (8.100,0.677) { 10};
\draw[gp path] (9.382,0.985)--(9.382,1.165);
\draw[gp path] (9.382,7.825)--(9.382,7.645);
\node[gp node center] at (9.382,0.677) { 12};
\draw[gp path] (10.665,0.985)--(10.665,1.165);
\draw[gp path] (10.665,7.825)--(10.665,7.645);
\node[gp node center] at (10.665,0.677) { 14};
\draw[gp path] (11.947,0.985)--(11.947,1.165);
\draw[gp path] (11.947,7.825)--(11.947,7.645);
\node[gp node center] at (11.947,0.677) { 16};
\draw[gp path] (1.688,7.825)--(1.688,0.985)--(11.947,0.985)--(11.947,7.825)--cycle;
\node[gp node center,rotate=-270] at (0.246,4.405) {Velocità angolare [rad/s]};
\node[gp node center] at (6.817,0.215) {Tempo [s]};
\node[gp node center] at (6.817,8.287) {Velocità angolare, accelerazione [rad/s]};
\node[gp node left] at (7.167,1.627) {Dati};
\gpcolor{color=gp lt color 0}
\gpsetlinetype{gp lt plot 0}
\draw[gp path] (10.663,1.627)--(11.579,1.627);
\draw[gp path] (10.663,1.717)--(10.663,1.537);
\draw[gp path] (11.579,1.717)--(11.579,1.537);
\draw[gp path] (2.489,1.147)--(2.489,6.645);
\draw[gp path] (2.399,1.147)--(2.579,1.147);
\draw[gp path] (2.399,6.645)--(2.579,6.645);
\draw[gp path] (3.323,1.901)--(3.323,4.543);
\draw[gp path] (3.233,1.901)--(3.413,1.901);
\draw[gp path] (3.233,4.543)--(3.413,4.543);
\draw[gp path] (4.157,2.134)--(4.157,3.873);
\draw[gp path] (4.067,2.134)--(4.247,2.134);
\draw[gp path] (4.067,3.873)--(4.247,3.873);
\draw[gp path] (4.990,2.247)--(4.990,3.543);
\draw[gp path] (4.900,2.247)--(5.080,2.247);
\draw[gp path] (4.900,3.543)--(5.080,3.543);
\draw[gp path] (5.856,2.066)--(5.856,3.082);
\draw[gp path] (5.766,2.066)--(5.946,2.066);
\draw[gp path] (5.766,3.082)--(5.946,3.082);
\draw[gp path] (6.689,2.150)--(6.689,2.998);
\draw[gp path] (6.599,2.150)--(6.779,2.150);
\draw[gp path] (6.599,2.998)--(6.779,2.998);
\draw[gp path] (7.523,2.211)--(7.523,2.937);
\draw[gp path] (7.433,2.211)--(7.613,2.211);
\draw[gp path] (7.433,2.937)--(7.613,2.937);
\draw[gp path] (8.388,2.101)--(8.388,2.731);
\draw[gp path] (8.298,2.101)--(8.478,2.101);
\draw[gp path] (8.298,2.731)--(8.478,2.731);
\draw[gp path] (9.222,2.153)--(9.222,2.713);
\draw[gp path] (9.132,2.153)--(9.312,2.153);
\draw[gp path] (9.132,2.713)--(9.312,2.713);
\draw[gp path] (10.088,2.071)--(10.088,2.572);
\draw[gp path] (9.998,2.071)--(10.178,2.071);
\draw[gp path] (9.998,2.572)--(10.178,2.572);
\draw[gp path] (10.921,2.117)--(10.921,2.572);
\draw[gp path] (10.831,2.117)--(11.011,2.117);
\draw[gp path] (10.831,2.572)--(11.011,2.572);
\draw[gp path] (11.787,2.051)--(11.787,2.467);
\draw[gp path] (11.697,2.051)--(11.877,2.051);
\draw[gp path] (11.697,2.467)--(11.877,2.467);
\gpsetpointsize{4.00}
\gppoint{gp mark 1}{(2.489,3.896)}
\gppoint{gp mark 1}{(3.323,3.222)}
\gppoint{gp mark 1}{(4.157,3.003)}
\gppoint{gp mark 1}{(4.990,2.895)}
\gppoint{gp mark 1}{(5.856,2.574)}
\gppoint{gp mark 1}{(6.689,2.574)}
\gppoint{gp mark 1}{(7.523,2.574)}
\gppoint{gp mark 1}{(8.388,2.416)}
\gppoint{gp mark 1}{(9.222,2.433)}
\gppoint{gp mark 1}{(10.088,2.322)}
\gppoint{gp mark 1}{(10.921,2.344)}
\gppoint{gp mark 1}{(11.787,2.259)}
\gppoint{gp mark 1}{(11.121,1.627)}
\gpcolor{color=gp lt color border}
\node[gp node left] at (7.167,1.319) {Retta interpolante};
\gpcolor{color=gp lt color 1}
\gpsetlinetype{gp lt plot 1}
\draw[gp path] (10.663,1.319)--(11.579,1.319);
\draw[gp path] (2.489,3.349)--(2.583,3.336)--(2.677,3.323)--(2.771,3.310)--(2.865,3.297)%
  --(2.959,3.284)--(3.053,3.271)--(3.147,3.258)--(3.241,3.246)--(3.335,3.233)--(3.429,3.220)%
  --(3.523,3.207)--(3.616,3.194)--(3.710,3.181)--(3.804,3.168)--(3.898,3.155)--(3.992,3.142)%
  --(4.086,3.129)--(4.180,3.116)--(4.274,3.103)--(4.368,3.090)--(4.462,3.077)--(4.556,3.064)%
  --(4.649,3.051)--(4.743,3.038)--(4.837,3.025)--(4.931,3.012)--(5.025,2.999)--(5.119,2.986)%
  --(5.213,2.973)--(5.307,2.960)--(5.401,2.947)--(5.495,2.934)--(5.589,2.921)--(5.682,2.908)%
  --(5.776,2.895)--(5.870,2.882)--(5.964,2.869)--(6.058,2.856)--(6.152,2.843)--(6.246,2.830)%
  --(6.340,2.817)--(6.434,2.804)--(6.528,2.791)--(6.622,2.778)--(6.715,2.765)--(6.809,2.752)%
  --(6.903,2.739)--(6.997,2.726)--(7.091,2.713)--(7.185,2.700)--(7.279,2.687)--(7.373,2.674)%
  --(7.467,2.661)--(7.561,2.648)--(7.655,2.635)--(7.749,2.622)--(7.842,2.609)--(7.936,2.596)%
  --(8.030,2.583)--(8.124,2.571)--(8.218,2.558)--(8.312,2.545)--(8.406,2.532)--(8.500,2.519)%
  --(8.594,2.506)--(8.688,2.493)--(8.782,2.480)--(8.875,2.467)--(8.969,2.454)--(9.063,2.441)%
  --(9.157,2.428)--(9.251,2.415)--(9.345,2.402)--(9.439,2.389)--(9.533,2.376)--(9.627,2.363)%
  --(9.721,2.350)--(9.815,2.337)--(9.908,2.324)--(10.002,2.311)--(10.096,2.298)--(10.190,2.285)%
  --(10.284,2.272)--(10.378,2.259)--(10.472,2.246)--(10.566,2.233)--(10.660,2.220)--(10.754,2.207)%
  --(10.848,2.194)--(10.942,2.181)--(11.035,2.168)--(11.129,2.155)--(11.223,2.142)--(11.317,2.129)%
  --(11.411,2.116)--(11.505,2.103)--(11.599,2.090)--(11.693,2.077)--(11.787,2.064);
\gpcolor{color=gp lt color border}
\gpsetlinetype{gp lt border}
\draw[gp path] (1.688,7.825)--(1.688,0.985)--(11.947,0.985)--(11.947,7.825)--cycle;
%% coordinates of the plot area
\gpdefrectangularnode{gp plot 1}{\pgfpoint{1.688cm}{0.985cm}}{\pgfpoint{11.947cm}{7.825cm}}
\end{tikzpicture}
%% gnuplot variables

\caption{Settima serie, decelerazione}
\label{fig:1}
\end{grafico}

\begin{grafico}
    \centering
\begin{tikzpicture}[gnuplot]
%% generated with GNUPLOT 4.6p0 (Lua 5.1; terminal rev. 99, script rev. 100)
%% Tue 15 Apr 2014 06:32:34 PM CEST
\path (0.000,0.000) rectangle (12.500,8.750);
\gpcolor{color=gp lt color border}
\gpsetlinetype{gp lt border}
\gpsetlinewidth{1.00}
\draw[gp path] (1.688,0.985)--(1.868,0.985);
\draw[gp path] (11.947,0.985)--(11.767,0.985);
\node[gp node right] at (1.504,0.985) { 1.1};
\draw[gp path] (1.688,1.607)--(1.868,1.607);
\draw[gp path] (11.947,1.607)--(11.767,1.607);
\node[gp node right] at (1.504,1.607) { 1.12};
\draw[gp path] (1.688,2.229)--(1.868,2.229);
\draw[gp path] (11.947,2.229)--(11.767,2.229);
\node[gp node right] at (1.504,2.229) { 1.14};
\draw[gp path] (1.688,2.850)--(1.868,2.850);
\draw[gp path] (11.947,2.850)--(11.767,2.850);
\node[gp node right] at (1.504,2.850) { 1.16};
\draw[gp path] (1.688,3.472)--(1.868,3.472);
\draw[gp path] (11.947,3.472)--(11.767,3.472);
\node[gp node right] at (1.504,3.472) { 1.18};
\draw[gp path] (1.688,4.094)--(1.868,4.094);
\draw[gp path] (11.947,4.094)--(11.767,4.094);
\node[gp node right] at (1.504,4.094) { 1.2};
\draw[gp path] (1.688,4.716)--(1.868,4.716);
\draw[gp path] (11.947,4.716)--(11.767,4.716);
\node[gp node right] at (1.504,4.716) { 1.22};
\draw[gp path] (1.688,5.338)--(1.868,5.338);
\draw[gp path] (11.947,5.338)--(11.767,5.338);
\node[gp node right] at (1.504,5.338) { 1.24};
\draw[gp path] (1.688,5.960)--(1.868,5.960);
\draw[gp path] (11.947,5.960)--(11.767,5.960);
\node[gp node right] at (1.504,5.960) { 1.26};
\draw[gp path] (1.688,6.581)--(1.868,6.581);
\draw[gp path] (11.947,6.581)--(11.767,6.581);
\node[gp node right] at (1.504,6.581) { 1.28};
\draw[gp path] (1.688,7.203)--(1.868,7.203);
\draw[gp path] (11.947,7.203)--(11.767,7.203);
\node[gp node right] at (1.504,7.203) { 1.3};
\draw[gp path] (1.688,7.825)--(1.868,7.825);
\draw[gp path] (11.947,7.825)--(11.767,7.825);
\node[gp node right] at (1.504,7.825) { 1.32};
\draw[gp path] (1.688,0.985)--(1.688,1.165);
\draw[gp path] (1.688,7.825)--(1.688,7.645);
\node[gp node center] at (1.688,0.677) { 0};
\draw[gp path] (2.828,0.985)--(2.828,1.165);
\draw[gp path] (2.828,7.825)--(2.828,7.645);
\node[gp node center] at (2.828,0.677) { 2};
\draw[gp path] (3.968,0.985)--(3.968,1.165);
\draw[gp path] (3.968,7.825)--(3.968,7.645);
\node[gp node center] at (3.968,0.677) { 4};
\draw[gp path] (5.108,0.985)--(5.108,1.165);
\draw[gp path] (5.108,7.825)--(5.108,7.645);
\node[gp node center] at (5.108,0.677) { 6};
\draw[gp path] (6.248,0.985)--(6.248,1.165);
\draw[gp path] (6.248,7.825)--(6.248,7.645);
\node[gp node center] at (6.248,0.677) { 8};
\draw[gp path] (7.387,0.985)--(7.387,1.165);
\draw[gp path] (7.387,7.825)--(7.387,7.645);
\node[gp node center] at (7.387,0.677) { 10};
\draw[gp path] (8.527,0.985)--(8.527,1.165);
\draw[gp path] (8.527,7.825)--(8.527,7.645);
\node[gp node center] at (8.527,0.677) { 12};
\draw[gp path] (9.667,0.985)--(9.667,1.165);
\draw[gp path] (9.667,7.825)--(9.667,7.645);
\node[gp node center] at (9.667,0.677) { 14};
\draw[gp path] (10.807,0.985)--(10.807,1.165);
\draw[gp path] (10.807,7.825)--(10.807,7.645);
\node[gp node center] at (10.807,0.677) { 16};
\draw[gp path] (11.947,0.985)--(11.947,1.165);
\draw[gp path] (11.947,7.825)--(11.947,7.645);
\node[gp node center] at (11.947,0.677) { 18};
\draw[gp path] (1.688,7.825)--(1.688,0.985)--(11.947,0.985)--(11.947,7.825)--cycle;
\node[gp node center,rotate=-270] at (0.246,4.405) {Velocità angolare [rad/s]};
\node[gp node center] at (6.817,0.215) {Tempo [s]};
\node[gp node center] at (6.817,8.287) {Velocità angolare, accelerazione [rad/s]};
\node[gp node left] at (7.167,1.627) {Dati};
\gpcolor{color=gp lt color 0}
\gpsetlinetype{gp lt plot 0}
\draw[gp path] (10.663,1.627)--(11.579,1.627);
\draw[gp path] (10.663,1.717)--(10.663,1.537);
\draw[gp path] (11.579,1.717)--(11.579,1.537);
\draw[gp path] (2.429,1.457)--(2.429,7.234);
\draw[gp path] (2.339,1.457)--(2.519,1.457);
\draw[gp path] (2.339,7.234)--(2.519,7.234);
\draw[gp path] (3.170,2.901)--(3.170,5.789);
\draw[gp path] (3.080,2.901)--(3.260,2.901);
\draw[gp path] (3.080,5.789)--(3.260,5.789);
\draw[gp path] (3.939,2.931)--(3.939,4.808);
\draw[gp path] (3.849,2.931)--(4.029,2.931);
\draw[gp path] (3.849,4.808)--(4.029,4.808);
\draw[gp path] (4.680,3.279)--(4.680,4.696);
\draw[gp path] (4.590,3.279)--(4.770,3.279);
\draw[gp path] (4.590,4.696)--(4.770,4.696);
\draw[gp path] (5.421,3.490)--(5.421,4.627);
\draw[gp path] (5.331,3.490)--(5.511,3.490);
\draw[gp path] (5.331,4.627)--(5.511,4.627);
\draw[gp path] (6.219,3.173)--(6.219,4.100);
\draw[gp path] (6.129,3.173)--(6.309,3.173);
\draw[gp path] (6.129,4.100)--(6.309,4.100);
\draw[gp path] (6.960,3.337)--(6.960,4.135);
\draw[gp path] (6.870,3.337)--(7.050,3.337);
\draw[gp path] (6.870,4.135)--(7.050,4.135);
\draw[gp path] (7.843,2.619)--(7.843,3.289);
\draw[gp path] (7.753,2.619)--(7.933,2.619);
\draw[gp path] (7.753,3.289)--(7.933,3.289);
\draw[gp path] (8.556,2.952)--(8.556,3.557);
\draw[gp path] (8.466,2.952)--(8.646,2.952);
\draw[gp path] (8.466,3.557)--(8.646,3.557);
\draw[gp path] (9.354,2.819)--(9.354,3.358);
\draw[gp path] (9.264,2.819)--(9.444,2.819);
\draw[gp path] (9.264,3.358)--(9.444,3.358);
\draw[gp path] (10.152,2.711)--(10.152,3.198);
\draw[gp path] (10.062,2.711)--(10.242,2.711);
\draw[gp path] (10.062,3.198)--(10.242,3.198);
\draw[gp path] (10.950,2.621)--(10.950,3.065);
\draw[gp path] (10.860,2.621)--(11.040,2.621);
\draw[gp path] (10.860,3.065)--(11.040,3.065);
\gpsetpointsize{4.00}
\gppoint{gp mark 1}{(2.429,4.345)}
\gppoint{gp mark 1}{(3.170,4.345)}
\gppoint{gp mark 1}{(3.939,3.870)}
\gppoint{gp mark 1}{(4.680,3.987)}
\gppoint{gp mark 1}{(5.421,4.058)}
\gppoint{gp mark 1}{(6.219,3.637)}
\gppoint{gp mark 1}{(6.960,3.736)}
\gppoint{gp mark 1}{(7.843,2.954)}
\gppoint{gp mark 1}{(8.556,3.254)}
\gppoint{gp mark 1}{(9.354,3.089)}
\gppoint{gp mark 1}{(10.152,2.954)}
\gppoint{gp mark 1}{(10.950,2.843)}
\gppoint{gp mark 1}{(11.121,1.627)}
\gpcolor{color=gp lt color border}
\node[gp node left] at (7.167,1.319) {Retta interpolante};
\gpcolor{color=gp lt color 1}
\gpsetlinetype{gp lt plot 1}
\draw[gp path] (10.663,1.319)--(11.579,1.319);
\draw[gp path] (2.429,4.377)--(2.515,4.361)--(2.601,4.345)--(2.687,4.329)--(2.773,4.313)%
  --(2.859,4.297)--(2.945,4.281)--(3.031,4.264)--(3.117,4.248)--(3.204,4.232)--(3.290,4.216)%
  --(3.376,4.200)--(3.462,4.184)--(3.548,4.168)--(3.634,4.152)--(3.720,4.136)--(3.806,4.119)%
  --(3.892,4.103)--(3.978,4.087)--(4.064,4.071)--(4.150,4.055)--(4.236,4.039)--(4.322,4.023)%
  --(4.408,4.007)--(4.495,3.991)--(4.581,3.975)--(4.667,3.958)--(4.753,3.942)--(4.839,3.926)%
  --(4.925,3.910)--(5.011,3.894)--(5.097,3.878)--(5.183,3.862)--(5.269,3.846)--(5.355,3.830)%
  --(5.441,3.814)--(5.527,3.797)--(5.613,3.781)--(5.699,3.765)--(5.786,3.749)--(5.872,3.733)%
  --(5.958,3.717)--(6.044,3.701)--(6.130,3.685)--(6.216,3.669)--(6.302,3.653)--(6.388,3.636)%
  --(6.474,3.620)--(6.560,3.604)--(6.646,3.588)--(6.732,3.572)--(6.818,3.556)--(6.904,3.540)%
  --(6.990,3.524)--(7.077,3.508)--(7.163,3.491)--(7.249,3.475)--(7.335,3.459)--(7.421,3.443)%
  --(7.507,3.427)--(7.593,3.411)--(7.679,3.395)--(7.765,3.379)--(7.851,3.363)--(7.937,3.347)%
  --(8.023,3.330)--(8.109,3.314)--(8.195,3.298)--(8.282,3.282)--(8.368,3.266)--(8.454,3.250)%
  --(8.540,3.234)--(8.626,3.218)--(8.712,3.202)--(8.798,3.186)--(8.884,3.169)--(8.970,3.153)%
  --(9.056,3.137)--(9.142,3.121)--(9.228,3.105)--(9.314,3.089)--(9.400,3.073)--(9.486,3.057)%
  --(9.573,3.041)--(9.659,3.025)--(9.745,3.008)--(9.831,2.992)--(9.917,2.976)--(10.003,2.960)%
  --(10.089,2.944)--(10.175,2.928)--(10.261,2.912)--(10.347,2.896)--(10.433,2.880)--(10.519,2.863)%
  --(10.605,2.847)--(10.691,2.831)--(10.777,2.815)--(10.864,2.799)--(10.950,2.783);
\gpcolor{color=gp lt color border}
\gpsetlinetype{gp lt border}
\draw[gp path] (1.688,7.825)--(1.688,0.985)--(11.947,0.985)--(11.947,7.825)--cycle;
%% coordinates of the plot area
\gpdefrectangularnode{gp plot 1}{\pgfpoint{1.688cm}{0.985cm}}{\pgfpoint{11.947cm}{7.825cm}}
\end{tikzpicture}
%% gnuplot variables

\caption{Ottava serie, decelerazione}
\label{fig:1}
\end{grafico}

\begin{grafico}
    \centering
\begin{tikzpicture}[gnuplot]
%% generated with GNUPLOT 4.6p0 (Lua 5.1; terminal rev. 99, script rev. 100)
%% Mon 14 Apr 2014 11:10:00 PM CEST
\path (0.000,0.000) rectangle (12.500,8.750);
\gpcolor{color=gp lt color border}
\gpsetlinetype{gp lt border}
\gpsetlinewidth{1.00}
\draw[gp path] (1.504,0.985)--(1.684,0.985);
\draw[gp path] (11.947,0.985)--(11.767,0.985);
\node[gp node right] at (1.320,0.985) {-800};
\draw[gp path] (1.504,1.962)--(1.684,1.962);
\draw[gp path] (11.947,1.962)--(11.767,1.962);
\node[gp node right] at (1.320,1.962) {-600};
\draw[gp path] (1.504,2.939)--(1.684,2.939);
\draw[gp path] (11.947,2.939)--(11.767,2.939);
\node[gp node right] at (1.320,2.939) {-400};
\draw[gp path] (1.504,3.916)--(1.684,3.916);
\draw[gp path] (11.947,3.916)--(11.767,3.916);
\node[gp node right] at (1.320,3.916) {-200};
\draw[gp path] (1.504,4.894)--(1.684,4.894);
\draw[gp path] (11.947,4.894)--(11.767,4.894);
\node[gp node right] at (1.320,4.894) { 0};
\draw[gp path] (1.504,5.871)--(1.684,5.871);
\draw[gp path] (11.947,5.871)--(11.767,5.871);
\node[gp node right] at (1.320,5.871) { 200};
\draw[gp path] (1.504,6.848)--(1.684,6.848);
\draw[gp path] (11.947,6.848)--(11.767,6.848);
\node[gp node right] at (1.320,6.848) { 400};
\draw[gp path] (1.504,7.825)--(1.684,7.825);
\draw[gp path] (11.947,7.825)--(11.767,7.825);
\node[gp node right] at (1.320,7.825) { 600};
\draw[gp path] (1.504,0.985)--(1.504,1.165);
\draw[gp path] (1.504,7.825)--(1.504,7.645);
\node[gp node center] at (1.504,0.677) {-10};
\draw[gp path] (4.115,0.985)--(4.115,1.165);
\draw[gp path] (4.115,7.825)--(4.115,7.645);
\node[gp node center] at (4.115,0.677) {-5};
\draw[gp path] (6.726,0.985)--(6.726,1.165);
\draw[gp path] (6.726,7.825)--(6.726,7.645);
\node[gp node center] at (6.726,0.677) { 0};
\draw[gp path] (9.336,0.985)--(9.336,1.165);
\draw[gp path] (9.336,7.825)--(9.336,7.645);
\node[gp node center] at (9.336,0.677) { 5};
\draw[gp path] (11.947,0.985)--(11.947,1.165);
\draw[gp path] (11.947,7.825)--(11.947,7.645);
\node[gp node center] at (11.947,0.677) { 10};
\draw[gp path] (1.504,7.825)--(1.504,0.985)--(11.947,0.985)--(11.947,7.825)--cycle;
\node[gp node center,rotate=-270] at (0.246,4.405) {Velocità angolare [rad/s]};
\node[gp node center] at (6.725,0.215) {Tempo [s]};
\node[gp node center] at (6.725,8.287) {Velocità angolare, decelerazione [rad/s]};
\node[gp node left] at (7.167,1.627) {Retta interpolante};
\gpcolor{color=gp lt color 1}
\gpsetlinetype{gp lt plot 1}
\draw[gp path] (10.663,1.627)--(11.579,1.627);
\draw[gp path] (1.504,1.459)--(1.609,1.516)--(1.715,1.572)--(1.820,1.628)--(1.926,1.685)%
  --(2.031,1.741)--(2.137,1.798)--(2.242,1.854)--(2.348,1.910)--(2.453,1.967)--(2.559,2.023)%
  --(2.664,2.080)--(2.770,2.136)--(2.875,2.192)--(2.981,2.249)--(3.086,2.305)--(3.192,2.362)%
  --(3.297,2.418)--(3.403,2.474)--(3.508,2.531)--(3.614,2.587)--(3.719,2.644)--(3.825,2.700)%
  --(3.930,2.756)--(4.036,2.813)--(4.141,2.869)--(4.247,2.926)--(4.352,2.982)--(4.458,3.038)%
  --(4.563,3.095)--(4.669,3.151)--(4.774,3.208)--(4.880,3.264)--(4.985,3.320)--(5.090,3.377)%
  --(5.196,3.433)--(5.301,3.490)--(5.407,3.546)--(5.512,3.602)--(5.618,3.659)--(5.723,3.715)%
  --(5.829,3.772)--(5.934,3.828)--(6.040,3.884)--(6.145,3.941)--(6.251,3.997)--(6.356,4.054)%
  --(6.462,4.110)--(6.567,4.167)--(6.673,4.223)--(6.778,4.279)--(6.884,4.336)--(6.989,4.392)%
  --(7.095,4.449)--(7.200,4.505)--(7.306,4.561)--(7.411,4.618)--(7.517,4.674)--(7.622,4.731)%
  --(7.728,4.787)--(7.833,4.843)--(7.939,4.900)--(8.044,4.956)--(8.150,5.013)--(8.255,5.069)%
  --(8.361,5.125)--(8.466,5.182)--(8.571,5.238)--(8.677,5.295)--(8.782,5.351)--(8.888,5.407)%
  --(8.993,5.464)--(9.099,5.520)--(9.204,5.577)--(9.310,5.633)--(9.415,5.689)--(9.521,5.746)%
  --(9.626,5.802)--(9.732,5.859)--(9.837,5.915)--(9.943,5.971)--(10.048,6.028)--(10.154,6.084)%
  --(10.259,6.141)--(10.365,6.197)--(10.470,6.253)--(10.576,6.310)--(10.681,6.366)--(10.787,6.423)%
  --(10.892,6.479)--(10.998,6.535)--(11.103,6.592)--(11.209,6.648)--(11.314,6.705)--(11.420,6.761)%
  --(11.525,6.817)--(11.631,6.874)--(11.736,6.930)--(11.842,6.987)--(11.947,7.043);
\gpcolor{color=gp lt color border}
\gpsetlinetype{gp lt border}
\draw[gp path] (1.504,7.825)--(1.504,0.985)--(11.947,0.985)--(11.947,7.825)--cycle;
%% coordinates of the plot area
\gpdefrectangularnode{gp plot 1}{\pgfpoint{1.504cm}{0.985cm}}{\pgfpoint{11.947cm}{7.825cm}}
\end{tikzpicture}
%% gnuplot variables

\caption{Nona serie, decelerazione}
\label{fig:1}
\end{grafico}

\begin{grafico}
    \centering
\begin{tikzpicture}[gnuplot]
%% generated with GNUPLOT 4.6p0 (Lua 5.1; terminal rev. 99, script rev. 100)
%% Mon 14 Apr 2014 11:10:00 PM CEST
\path (0.000,0.000) rectangle (12.500,8.750);
\gpcolor{color=gp lt color border}
\gpsetlinetype{gp lt border}
\gpsetlinewidth{1.00}
\draw[gp path] (1.688,0.985)--(1.868,0.985);
\draw[gp path] (11.947,0.985)--(11.767,0.985);
\node[gp node right] at (1.504,0.985) {-1000};
\draw[gp path] (1.688,1.840)--(1.868,1.840);
\draw[gp path] (11.947,1.840)--(11.767,1.840);
\node[gp node right] at (1.504,1.840) {-800};
\draw[gp path] (1.688,2.695)--(1.868,2.695);
\draw[gp path] (11.947,2.695)--(11.767,2.695);
\node[gp node right] at (1.504,2.695) {-600};
\draw[gp path] (1.688,3.550)--(1.868,3.550);
\draw[gp path] (11.947,3.550)--(11.767,3.550);
\node[gp node right] at (1.504,3.550) {-400};
\draw[gp path] (1.688,4.405)--(1.868,4.405);
\draw[gp path] (11.947,4.405)--(11.767,4.405);
\node[gp node right] at (1.504,4.405) {-200};
\draw[gp path] (1.688,5.260)--(1.868,5.260);
\draw[gp path] (11.947,5.260)--(11.767,5.260);
\node[gp node right] at (1.504,5.260) { 0};
\draw[gp path] (1.688,6.115)--(1.868,6.115);
\draw[gp path] (11.947,6.115)--(11.767,6.115);
\node[gp node right] at (1.504,6.115) { 200};
\draw[gp path] (1.688,6.970)--(1.868,6.970);
\draw[gp path] (11.947,6.970)--(11.767,6.970);
\node[gp node right] at (1.504,6.970) { 400};
\draw[gp path] (1.688,7.825)--(1.868,7.825);
\draw[gp path] (11.947,7.825)--(11.767,7.825);
\node[gp node right] at (1.504,7.825) { 600};
\draw[gp path] (1.688,0.985)--(1.688,1.165);
\draw[gp path] (1.688,7.825)--(1.688,7.645);
\node[gp node center] at (1.688,0.677) {-10};
\draw[gp path] (4.253,0.985)--(4.253,1.165);
\draw[gp path] (4.253,7.825)--(4.253,7.645);
\node[gp node center] at (4.253,0.677) {-5};
\draw[gp path] (6.818,0.985)--(6.818,1.165);
\draw[gp path] (6.818,7.825)--(6.818,7.645);
\node[gp node center] at (6.818,0.677) { 0};
\draw[gp path] (9.382,0.985)--(9.382,1.165);
\draw[gp path] (9.382,7.825)--(9.382,7.645);
\node[gp node center] at (9.382,0.677) { 5};
\draw[gp path] (11.947,0.985)--(11.947,1.165);
\draw[gp path] (11.947,7.825)--(11.947,7.645);
\node[gp node center] at (11.947,0.677) { 10};
\draw[gp path] (1.688,7.825)--(1.688,0.985)--(11.947,0.985)--(11.947,7.825)--cycle;
\node[gp node center,rotate=-270] at (0.246,4.405) {Velocità angolare [rad/s]};
\node[gp node center] at (6.817,0.215) {Tempo [s]};
\node[gp node center] at (6.817,8.287) {Velocità angolare, decelerazione [rad/s]};
\node[gp node left] at (7.167,1.627) {Retta interpolante};
\gpcolor{color=gp lt color 1}
\gpsetlinetype{gp lt plot 1}
\draw[gp path] (10.663,1.627)--(11.579,1.627);
\draw[gp path] (1.688,1.667)--(1.792,1.726)--(1.895,1.785)--(1.999,1.844)--(2.103,1.903)%
  --(2.206,1.962)--(2.310,2.020)--(2.413,2.079)--(2.517,2.138)--(2.621,2.197)--(2.724,2.256)%
  --(2.828,2.315)--(2.932,2.374)--(3.035,2.433)--(3.139,2.492)--(3.242,2.551)--(3.346,2.610)%
  --(3.450,2.669)--(3.553,2.728)--(3.657,2.787)--(3.761,2.846)--(3.864,2.904)--(3.968,2.963)%
  --(4.071,3.022)--(4.175,3.081)--(4.279,3.140)--(4.382,3.199)--(4.486,3.258)--(4.590,3.317)%
  --(4.693,3.376)--(4.797,3.435)--(4.900,3.494)--(5.004,3.553)--(5.108,3.612)--(5.211,3.671)%
  --(5.315,3.729)--(5.419,3.788)--(5.522,3.847)--(5.626,3.906)--(5.729,3.965)--(5.833,4.024)%
  --(5.937,4.083)--(6.040,4.142)--(6.144,4.201)--(6.248,4.260)--(6.351,4.319)--(6.455,4.378)%
  --(6.558,4.437)--(6.662,4.496)--(6.766,4.555)--(6.869,4.613)--(6.973,4.672)--(7.077,4.731)%
  --(7.180,4.790)--(7.284,4.849)--(7.387,4.908)--(7.491,4.967)--(7.595,5.026)--(7.698,5.085)%
  --(7.802,5.144)--(7.906,5.203)--(8.009,5.262)--(8.113,5.321)--(8.216,5.380)--(8.320,5.438)%
  --(8.424,5.497)--(8.527,5.556)--(8.631,5.615)--(8.735,5.674)--(8.838,5.733)--(8.942,5.792)%
  --(9.045,5.851)--(9.149,5.910)--(9.253,5.969)--(9.356,6.028)--(9.460,6.087)--(9.564,6.146)%
  --(9.667,6.205)--(9.771,6.264)--(9.874,6.322)--(9.978,6.381)--(10.082,6.440)--(10.185,6.499)%
  --(10.289,6.558)--(10.393,6.617)--(10.496,6.676)--(10.600,6.735)--(10.703,6.794)--(10.807,6.853)%
  --(10.911,6.912)--(11.014,6.971)--(11.118,7.030)--(11.222,7.089)--(11.325,7.147)--(11.429,7.206)%
  --(11.532,7.265)--(11.636,7.324)--(11.740,7.383)--(11.843,7.442)--(11.947,7.501);
\gpcolor{color=gp lt color border}
\gpsetlinetype{gp lt border}
\draw[gp path] (1.688,7.825)--(1.688,0.985)--(11.947,0.985)--(11.947,7.825)--cycle;
%% coordinates of the plot area
\gpdefrectangularnode{gp plot 1}{\pgfpoint{1.688cm}{0.985cm}}{\pgfpoint{11.947cm}{7.825cm}}
\end{tikzpicture}
%% gnuplot variables

\caption{Decima serie, decelerazione}
\label{fig:1}
\end{grafico}


		
\section{Conclusioni}
	IL momento d'inerzia, calcolato con FINIRE risultato è $0.27 \pm 0.01$

	
\section{Codice}
	\input{./sezioni/codice.tex}
	
%\subsection{Esempio immagini}
%\begin{figure}[p]
% \centering
% \includegraphics[width=0.8\textwidth]{spazio1}
% \caption{Spazio!}
% \label{fig:spazio1}
%\end{figure}


\end{document}
